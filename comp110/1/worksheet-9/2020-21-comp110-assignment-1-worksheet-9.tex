\documentclass{../../../fal_assignment}
\graphicspath{ {../../../} }

\usepackage{enumitem}
\setlist{nosep} % Make enumerate / itemize lists more closely spaced
\usepackage[T1]{fontenc} % http://tex.stackexchange.com/a/17858
\usepackage{url}
\usepackage{todonotes}

\title{COMP110 Worksheet 9: TIS-100}
\module{COMP110}
\author{Ed Powley}
\version{4.0}

\begin{document}

\maketitle

\section*{Introduction}

\begin{marginquote}
``The Tesselated Intelligence System is a massively parallel computer architecture comprised of non-uniformly
interconnected heterogeneous nodes. The Tesselated Intelligence System is ideal for applications requiring complex
data stream processing, such as automated financial trading, bulk data collection, and civilian behavioural analysis.''

--- TIS-100 Reference Manual
\end{marginquote}
\marginpicture{flavour_pic}{
    \emph{TIS-100} is a puzzle game in which players must write assembly code to process data streams
    in increasingly complex ways.
}

\textbf{TIS-100} is a puzzle game released in 2015 by independent developers Zachtronics,
who you will remember from Worksheet 1 as the developers of SpaceChem.
The gameplay consists of writing assembly code for a fictional parallel computer architecture,
and as such requires skills in low-level programming.

To complete this worksheet:
\begin{enumerate}[label=(\alph*)]
	\item \textbf{Play} TIS-100 --- it is available on the PCs in the Games Academy, or can be purchased from popular game stores (Steam, GOG, Humble, itch.io) for around \textsterling 5;
 	\item \textbf{Complete} the first thirteen levels of the game, upto and including ``Signal Multiplier'';
 	\item \textbf{Complete} the optional stretch goals.
\end{enumerate}

\section*{Submission instructions}

There is \textbf{no formative submission} for this worksheet --- you will submit your work as part of your summative submission of the nine COMP110 worksheets.

Play TIS-100.
When you are ready to prepare your work for submission, click the ``Open Save Directory'' button within the game (on the level select screen),
and navigate up one directory.
You should now be in a directory whose name is a long string of numbers.
Add the contents of this directory, including the \texttt{save.dat} file and the \texttt{saves} directory, to your summative submission.
Please see the instructions and video on LearningSpace for further details on preparing your summative submission.

\section*{Marking criteria and stretch goals}

The basic task for this worksheet is to complete the first 13 non-sandbox levels of the game, upto and including ``Signal Multiplier''.
Partial credit will be awarded for completing some portion of these levels --- if you are stuck on a particular level, skip it and carry on to solve the rest.

You are \textbf{not} marked on the quality of your solutions, just on how many levels you are able to complete.

The in-game manual is essential reading to understand how to solve the puzzles.
You may look for help online, but resist the temptation to copy solutions ---
the markers reserve the right to deduct marks if it is clear that you do not understand your solutions
or if they are suspiciously similar to solutions that can be found online.

Extra marks are available for completing one or more of the following \textbf{stretch goals}:

	\begin{itemize}
		\item Solve ``Differential Converter'' in 210 cycles or fewer;
		\item Solve ``Sequence Counter'' in 4 nodes or fewer;
		\item Continue progressing through the game to unlock the bottom row of the level select screen, and solve ``Sequence Sorter''. (I was stuck on this level for several weeks when I first played TIS-100!)
	\end{itemize}

\end{document}
