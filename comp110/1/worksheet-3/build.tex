\documentclass{../../../fal_assignment}
\graphicspath{ {../../../} }

\usepackage{enumitem}
\setlist{nosep} % Make enumerate / itemize lists more closely spaced
\usepackage[T1]{fontenc} % http://tex.stackexchange.com/a/17858
\usepackage{url}
\usepackage{todonotes}

\title{COMP110 Worksheet B: Flowcharts and Pseudocode}
\module{COMP110}
\author{Ed Powley}
\version{2.0}

\begin{document}

\maketitle

\section*{Introduction}

\begin{marginquote}
``I'm gonna run some diagnostics while you're tinkering. Take your time.''

--- Nick Valentine, Fallout 4
\end{marginquote}
\marginpicture{flavour_pic}{
    The terminal hacking minigame in \emph{Fallout 4}.
}

The open-world RPG \textbf{Fallout~4} contains a terminal hacking minigame, in which
the player must guess a secret $n$-letter word.
In this worksheet, you will model this minigame using a flowchart and a piece of pseudocode.

In the minigame, each guess receives a \textbf{likeness} score,
defined as the number of letters which match the secret word (i.e.\ the same letter in the same position).
For example if the secret word is \texttt{HOUSE} and the guess is \texttt{MOUSE}, the likeness is $4$ out of $5$.
If the guess is \texttt{HOPES}, the likeness is $2$ out of $5$
(the letters \texttt{S} and \texttt{E} do not count as they are in the wrong positions).
The minigame ends when the player guesses correctly, or after four incorrect guesses.
(In Fallout~4 there is a way to replenish the number of available guesses,
but this is disregarded in this worksheet for simplicity.)

To complete this worksheet:
\begin{enumerate}[label=(\alph*)]
	\item \textbf{Write} a piece of pseudocode which, given the secret word and the guessed word,
		calculates and displays the similarity score.
	\item \textbf{Draw} a flowchart for the overall minigame.
\end{enumerate}

\section*{Submission instructions}

Begin by \textbf{forking} the GitHub repository at the following URL:

\url{https://github.com/Falmouth-Games-Academy/comp110-worksheet-B}

Write your \textbf{pseudocode} in the \texttt{README.md} file.
Upload your \textbf{flowchart} as an image, and embed it in the \texttt{README.md} file also.
Open a \textbf{pull request}.

You may use any tool you wish to produce your flowchart,
be it a software tool or pen and paper.
If you use pen and paper, upload a scan or a photograph of your flowchart,
ensuring that the resolution and image quality are sufficient for the flowchart to be easily legible.

Attend the timetabled worksheet review session in \textbf{Week 4},
ensuring that you have uploaded all material to GitHub and opened a pull request before this time.

\section*{Marking criteria}

Remember that \textbf{it is better to submit incomplete work than to submit nothing at all}.
Any attempt, even unfinished, at producing a flowchart and a piece of pseudocode will receive a passing grade.

Your work will be marked according to the following criteria:
\begin{itemize}
	\item Are your flowchart and pseudocode \textbf{clear} and \textbf{comprehensive}?
	\item Is your formatting \textbf{readable} and \textbf{consistent}?
	\item Have you chosen an appropriate level of \textbf{abstraction}?
	\item In your pseudocode, have you used appropriate \textbf{identifier names} and \textbf{comments}?
\end{itemize}

\end{document}
