\documentclass{../../../fal_assignment}
\graphicspath{ {../../../} }

\usepackage{amsmath}
\usepackage{enumitem}
\setlist{nosep} % Make enumerate / itemize lists more closely spaced
\usepackage[T1]{fontenc} % http://tex.stackexchange.com/a/17858
\usepackage{url}
\usepackage{todonotes}
\usepackage{algpseudocode}
\usepackage{listings}
\lstset{
	basicstyle=\ttfamily,
	frame=single,
	showstringspaces=false,
	breaklines=true,
	prebreak={\space\hbox{\textcolor{gray}{$\hookleftarrow$}}}
}
\lstset{
	commentstyle=\ttfamily\textit,
	keywordstyle=\ttfamily\textbf,
	stringstyle=\ttfamily,
	rulecolor=\color{black}
}
\lstset{language=Python}

\title{Worksheet 4: LaTeX Typesetting}
\module{COMP110}
\author{Ed Powley}
\version{2.0}

\begin{document}

\maketitle
\marginpicture{flavour_pic}{
	The TeX typesetting system, on which LaTeX is based, was originally developed by Donald Knuth
	to typeset his seminal series of textbooks, \emph{The Art of Computer Programming}.
}

\section*{Introduction}

This worksheet gives you practice at typesetting documents in LaTeX.
You will be using LaTeX to typeset much of your written work during your studies, beginning with your Research Journal
assignment for this module.
LaTeX has a steeper learning curve than WYSIWYG systems such as Microsoft Word,
but when used properly can produce much more professional-looking documents,
as well as easing the task of managing complex documents with many cross-references and large bibliographies.

This worksheet tasks you with reproducing a given reference document using LaTeX.
The document in question is based on the paper ``Semi-automated Level Design via Auto-Playtesting for Handheld Casual 
Game Creation'', which was written by research staff in the Games Academy at Falmouth University in 2016
and was published in the IEEE Conference on Computational Intelligence in Games (CIG) that year.

To complete this worksheet:

\begin{enumerate}[label=(\alph*)]
	\item \textbf{Fork} the base project for this worksheet, via Overleaf or GitHub
	\item \textbf{Examine} the reference document included in the repository
	\item \textbf{Edit} the LaTeX source code provided as a template, to produce a document similar in content and layout to the reference document
\end{enumerate}

\section*{Submission instructions}

Begin by \textbf{forking} the base repository for this worksheet, either on Overleaf or on GitHub.
Please see LearningSpace for a link to the repository.
You will find the following files:

\begin{itemize}
	\item \textbf{reference.pdf}: the reference document
	\item \textbf{content.md}: a Markdown file containing the text of the reference document, for you to copy and paste
	\item \textbf{images}: a folder containing all the images used in the reference document
	\item \textbf{worksheet.bib}: a BibTeX file, from which the bibliography can be generated
	\item \textbf{worksheet.tex}: a template LaTeX document for you to edit
\end{itemize}

Edit \textbf{worksheet.tex} to produce a document similar in content and layout to \textbf{reference.pdf}.
You may also add other files to the repository as required, for example if you decide to split the document into multiple .tex files.

The \textbf{content.md} file contains the text of the paper formatted in Markdown.
Pay particular attention to the following notation:

\begin{itemize}
	\item Section headings are underlined with \lstinline{====}, and subsection headings with \lstinline{----}
	\item Text which should be \emph{emphasised with italics} is surrounded by single asterisks, \lstinline{*like this*}
	\item Text which should be \textbf{emboldened} is surrounded by double asterisks, \lstinline{**like this**}
	\item Bulleted lists are formatted with an asterisk \lstinline{*} at the beginning of each line
	\item Some mathematical notation is already formatted in LaTeX, and is surrounded by \lstinline{$} symbols
	\item \textbf{***REF***} denotes a cross-reference to a section, figure or table within the document,
		and \textbf{***CITE***} denotes a citation to a source from the bibliography.
		You will need to check the \textbf{reference.pdf} file to find out what these references and citations should point to.
	\item Images are omitted from the Markdown file; you will need to include the appropriate images at the correct points in the document, referring to \textbf{reference.pdf} and the images in the \textbf{images} folder
\end{itemize}

Note that you are \textbf{not} required to replicate the layout of the reference document exactly --- minor differences are fine,
and improvements will earn you higher marks.
However you should not alter the content of the document itself.
Note also that the text of the document is provided for you to copy and paste --- please do not waste time re-typing the text.

Once you have finished, post a \textbf{link} to your project in the formative submission area on LearningSpace.

\rubrichead{To \textbf{pass} this assignment (achieve 40\% or more), you must submit a reasonable attempt at the worksheet by the formative deadline stated on LearningSpace.}
\begin{markingrubric}
	\firstcriterion{Basic competency threshold}{30\%}
        \grade\fail	A reasonable attempt at the worksheet was not submitted by the formative deadline.
        \par Breach of academic integrity.
    \criterion{Text presentation}{20\%}
        \grade\fail Document is corrupted or incomplete.
        \grade There are obvious errors in text formatting, e.g.\ incorrect paragraph breaks.
			\par Boldface, italics etc are missing.
			\par Section headings are missing, or are hard-coded using non-standard markup.
		\grade There are no obvious errors in text formatting.
			\par Boldface, italics etc are used somewhat correctly.
			\par Section headings are used somewhat correctly.
		\grade There are no obvious errors in text formatting.
			\par Boldface, italics etc are used mostly correctly.
			\par Section headings are used mostly correctly.
			\par Mathematical notation is used in some places.
		\grade Text formatting matches the standard of the reference document.
			\par Boldface, italics etc are used correctly.
			\par Section headings are used correctly.
			\par Mathematical notation is used where appropriate.
		\grade Text presentation exceeds the standard of the reference document.
			\par Boldface, italics etc are used correctly.
			\par Section headings are used correctly.
			\par Mathematical notation is used where appropriate.
	\criterion{Cross-referencing and citing}{15\%}
		\grade\fail Cross-references are entirely missing.
			\par Citations are entirely missing.
			\par The bibliography is missing.
		\grade Cross-references are incorrect, or are hard-coded using non-standard markup.
			\par Citations are incorrect, or are hard-coded using non-standard markup.
			\par The bibliography is present but has errors.
		\grade Cross-references are somewhat correct.
			\par Citations are somewhat correct.
			\par The bibliography is present and somewhat correct.
		\grade Cross-references are mostly correct.
			\par Citations are mostly correct.
			\par The bibliography is present and mostly correct.
		\grade Cross-references are correct.
			\par Citations are correct.
			\par The bibliography is present and correct.
		\grade Cross-references are correct.
			\par Citations are correct.
			\par The bibliography is present, correct, and formatted to a higher standard than the reference document.
	\criterion{Figures and tables}{15\%}
		\grade\fail One or more figures or tables are missing.
		\grade Figures are present but are poorly formatted in terms of positioning and sizing.
			\par The table is present but poorly formatted.
		\grade Positioning and sizing of figures and tables is good.
			\par The table is formatted reasonably well.
		\grade Positioning and sizing of figures and tables is very good.
			\par The table is formatted well.
		\grade Positioning and sizing of figures and tables is excellent.
			\par The table is formatted to the same standard as the reference document.
		\grade Positioning and sizing of figures and tables is outstanding.
			\par The table is formatted to a higher standard than the reference document.
	\criterion{Source code formatting}{20\%}
		\grade\fail Source code is missing or compiles with errors.
		\grade Source code compiles without errors.
			\par Source code is structured and formatted poorly.
			\par Identifier names for labels etc are poorly chosen.
		\grade Source code compiles without errors.
			\par Source code is structured and formatted reasonably well.
			\par Identifier names for labels etc are reasonably appropriate.
		\grade Source code compiles without errors and with only minor warnings.
			\par Source code is structured and formatted well.
			\par Identifier names for labels etc are appropriate.
		\grade Source code compiles without errors or warnings (except for underfull/overfull box warnings).
			\par Source code is structured and formatted well, with appropriate use of multiple files.
			\par Identifier names for labels etc are descriptive.
		\grade Source code compiles without errors or warnings (except for underfull/overfull box warnings).
			\par Source code is structured and formatted extremely well, with appropriate use of multiple files.
			\par Identifier names for labels etc are very descriptive.
\end{markingrubric}

\end{document}
