\documentclass{../../../fal_assignment}
\graphicspath{ {../../../} }

\usepackage{enumitem}
\setlist{nosep} % Make enumerate / itemize lists more closely spaced
\usepackage[T1]{fontenc} % http://tex.stackexchange.com/a/17858
\usepackage{url}
\usepackage{todonotes}

\usepackage{listings}
\lstset{
    language=Python,
    basicstyle=\footnotesize\ttfamily,
    commentstyle=\ttfamily\textit,
    keywordstyle=\ttfamily\textbf,
    stringstyle=\ttfamily,
    frame=single,
    tabsize=4,
    showstringspaces=false
}

\title{COMP110 Worksheet 4: Logic Gates}
\module{COMP110}
\author{Ed Powley}
\version{2.0}

\newcommand{\TT}{\textsc{True}}
\newcommand{\FF}{\textsc{False}}
\newcommand{\OP}[1]{\ \textsc{#1}\ }
\newcommand{\OPand}{\OP{and}}
\newcommand{\OPor}{\OP{or}}
\newcommand{\OPxor}{\OP{xor}}
\newcommand{\OPnand}{\OP{nand}}
\newcommand{\OPnor}{\OP{nor}}
\newcommand{\OPxnor}{\OP{xnor}}
\newcommand{\OPnot}{\textsc{not}\ }

\begin{document}

\maketitle

\section*{Submission instructions}

Begin by \textbf{forking} the GitHub repository at the following URL:

\url{https://github.com/Falmouth-Games-Academy/comp110-worksheet-4}

\textbf{Edit} the \texttt{README.md} file with your solutions to the following exercises,
making appropriate use of Markdown formatting to ensure that it displays properly when viewed through the GitHub web interface.
Upload your circuit diagrams for question~2 as images, and embed them in the \texttt{README.md} file.

You may use any tool you wish to produce your circuit diagrams,
be it a software tool or pen and paper.
If you use pen and paper, upload a scan or a photograph of your flowchart,
ensuring that the resolution and image quality are sufficient for the circuit diagrams to be easily legible.

Open a \textbf{pull request} by \textbf{5pm on Friday 25th October}.

\section*{Exercises}

\begin{enumerate}
    \setlength\itemsep{1em}
    \item \label{q:expressions} Write out the truth tables for the following boolean expressions, for all possible values of boolean variables
        $A, B, C, \dots$:
        \begin{enumerate}
            \item $A \OPand B \OPand \OPnot C$
            \item $A \OPand \OPnot(B \OPand \OPnot C)$
            \item $(A \OPor \OPnot B) \OPand (A \OPor C)$
            \item $A \OPand \OPnot (B \OPor \OPnot C) \OPand (\OPnot A \OPand D)$
        \end{enumerate}
    \item Draw logic circuits for each of the expressions in Question~\ref{q:expressions}.
    \item \label{q:identities} Use truth tables to show that the following identities hold:
        \begin{enumerate}
            \item $\OPnot (A \OPor B) = \OPnot A \OPand \OPnot B$
            \item $\OPnot (A \OPand B) = \OPnot A \OPor \OPnot B$
            \item $(A \OPand B) \OPor (A \OPand C) = A \OPand (B \OPor C)$
            \item $(A \OPor B) \OPand (A \OPor C) = A \OPor (B \OPand C)$
        \end{enumerate}
    \item Explain, using the identities in Question~\ref{q:identities} and/or truth tables,    
        why each of the following pairs of programs is equivalent:
        \begin{enumerate}
            \item {\ }
                \begin{lstlisting}
if not(file_exists("a.txt") and file_exists("b.txt")):
    print("A required file is missing")
                \end{lstlisting}
                \begin{lstlisting}
if not file_exists("a.txt") or not file_exists("b.txt"):
    print("A required file is missing")
                \end{lstlisting}
            \item {\ }
                \begin{lstlisting}
if (type(x) == int and x > 7) or (type(x) == float and x > 7):
    print("Hello")
                \end{lstlisting}
                \begin{lstlisting}
if (type(x) == int or type(x) == float) and x > 7:
    print("Hello")
                \end{lstlisting}
            \item {\ }
                \begin{lstlisting}
if x == 0 and y == 0:
    do_something()
else:
    print("Do nothing")
                \end{lstlisting}
                \begin{lstlisting}
if x != 0 or y != 0:
    print("Do nothing")
else:
    do_something()
                \end{lstlisting}
            \item {\ }
                \begin{lstlisting}
if x > 10 or (x > 0 and y > 0):
    do_something()
                \end{lstlisting}
                \begin{lstlisting}
if x > 0 and (x > 10 or y > 0):
    do_something()
                \end{lstlisting}
        \end{enumerate}
\end{enumerate}

\end{document}
