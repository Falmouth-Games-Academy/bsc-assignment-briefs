\documentclass{../../../fal_assignment}
\graphicspath{ {../../../} }

\usepackage{enumitem}
\setlist{nosep} % Make enumerate / itemize lists more closely spaced
\usepackage[T1]{fontenc} % http://tex.stackexchange.com/a/17858
\usepackage{url}
\usepackage{todonotes}

\title{COMP110 Exercise Sheet I: Title}
\module{COMP110}
\author{Ed Powley}
\version{1.0}

\begin{document}

\maketitle

\section*{Introduction}

\textbf{SpaceChem} is a puzzle game released in 2011 by independent developers Zachtronics. The gameplay consists of solving problems in a visual programming environment, and as such requires many of the skills required of a programmer. This worksheet allows you to begin developing these skills, even if you are not yet comfortable with the syntax of more conventional programming languages such as Python.

To complete this worksheet:
\begin{enumerate}[label=(\alph*)]
	\item \textbf{Play} SpaceChem --- it is available on the PCs in the Teaching Space,
		or it can be purchased from Steam for around \textsterling 7;
 	\item \textbf{Complete} all of the ``research'' and ``production'' levels on the planets of Sernimir II, Sernimir IV, and Danopth.
\end{enumerate}

\section*{Submission instructions}

Begin by \textbf{forking} the GitHub repository at the following URL:

\url{https://github.com/Falmouth-Games-Academy/comp110-worksheet-A}

Once each level has been completed, upload a \textbf{video} of your solution to YouTube using the in-game upload feature.
Create a YouTube playlist for your videos, and add a link to it to the \texttt{README.md} file within the forked repository.
Note that SpaceChem disables its YouTube recording feature for ``defense'' levels; you do \textbf{not} need to submit videos for these.

In addition, locate your SpaceChem save file on your hard disk, for example:

\url{C:\Users\<your name>\AppData\Local\Zachtronics Industries\SpaceChem\save\000.user}

\textbf{Upload} this file to your forked repository.
An automated Travis~CI script will be used to scan your save file and determine your game progress,
therefore please do \textbf{not} rename the file or upload it into a subdirectory.

Attend the timetabled worksheet review session in \textbf{Week 2},
ensuring that you have uploaded all material to GitHub before this time.

\section*{Marking criteria}

Remember that \textbf{it is better to submit incomplete work than to submit nothing at all}. If you do not manage to finish all assigned levels by the deadline, upload videos and screenshots of whatever levels you have completed and you will receive a passing grade.

To demonstrate \textbf{basic competency}, upload \textbf{at least one} completed level video to YouTube, and upload your save file to GitHub, as instructed above.

To demonstrate \textbf{basic proficiency}, upload videos for \textbf{all} ``Research'' and ``Production'' levels on Sernimir II, Sernimir IV, and Danopth, except for the optional challenge level ``In-Place Swap''.

To demonstrate \textbf{novice competency}, achieve basic proficiency plus \textbf{one} of the following:
	\begin{itemize}
		\item Solve ``Everyday is the First Day'' in 300 cycles or fewer;
		\item Solve ``It Takes Three'' in 20 symbols or fewer;
		\item Solve ``Challenge: In-Place Swap'' from Danopth, and ``An Introduction to Sensing'', ``Prelude to a Migraine'' and ``Random Oxides'' from Alkonost.
	\end{itemize}

To demonstrate \textbf{novice proficiency}, achieve basic proficiency plus \textbf{two} of the criteria above.

To demonstrate \textbf{professional competency}, achieve basic proficiency plus \textbf{all three} of the criteria above.

\end{document}
