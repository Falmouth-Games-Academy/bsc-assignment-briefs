\documentclass{../../../fal_assignment}
\graphicspath{ {../../../} }

\usepackage{amsmath}
\usepackage{enumitem}
\setlist{nosep} % Make enumerate / itemize lists more closely spaced
\usepackage[T1]{fontenc} % http://tex.stackexchange.com/a/17858
\usepackage{url}
\usepackage{todonotes}
\usepackage{algpseudocode}
\usepackage{listings}
\lstset{
	basicstyle=\ttfamily,
	frame=single,
	showstringspaces=false,
	breaklines=true,
	prebreak={\space\hbox{\textcolor{gray}{$\hookleftarrow$}}}
}
\lstset{
	commentstyle=\ttfamily\textit,
	keywordstyle=\ttfamily\textbf,
	stringstyle=\ttfamily,
	rulecolor=\color{black}
}
\lstset{language=Python}

\title{Worksheet D: Data Structures}
\author{Dr Ed Powley}
\module{COMP110}
\version{2.0}

\begin{document}

\maketitle
\marginpicture{flavour_pic}{
    OXO on the EDSAC computer, one of the earliest examples of a computer game.
}

\section*{Introduction}

In this worksheet, you will complete a Python implementation of the pen-and-paper game Noughts and Crosses
(also known as OXO or Tic-Tac-Toe).

Noughts and Crosses is a two-player game played on a $3 \times 3$ grid.
Players take turns to place their mark in an empty square of their choosing; usually
player~1 marks \texttt{O} and player~2 marks \texttt{X}.
The winner is the first player to get three marks in a row horizontally, vertically or diagonally.

To complete this worksheet:

\begin{enumerate}[label=(\alph*)]
	\item \textbf{Fork} the skeleton project and \textbf{open} \texttt{oxo.py} in your favourite Python IDE.
	\item \textbf{Choose} an appropriate data structure to represent the state of the board.
	\item \textbf{Implement} the following methods of the \lstinline{OxoBoard} class:
		\begin{enumerate}[label=(\roman*)]
			\item \lstinline{__init__()}, which should initialise the data structure and any other fields that are required.
			\item \lstinline{get_square(x, y)}, which should return the current contents of the square at coordinates $x,y$.
				For this and other functions, $x$ and $y$ have values of $0$, $1$ or $2$:
				$0,0$ is the top left corner, $1,0$ is the top middle, and so on.
				Cell contents are integers: $0$ for an empty square,
				$1$ for a player~1 mark, and $2$ for a player~2 mark.
			\item \lstinline{set_square(x, y, mark)}, which should check if the square at coordinates $x,y$ is empty.
				If it is empty, fill it with the value of \lstinline{mark} and return \lstinline{True};
				if the square is not empty, leave it alone and return \lstinline{False}.
			\item \lstinline{is_board_full()}, which should return a boolean indicating whether all spaces on the board are occupied.
			\item \lstinline{get_winner()}, which should check if either player has made three in a row.
				If they have, return the player number (1 or 2).
				If neither player has made three in a row, return 0.
				If the board state is such that both players have made three in a row (which cannot occur in a normal game),
				behaviour is undefined (i.e.\ your function does not need to handle this case).
		\end{enumerate}
\end{enumerate}

It is anticipated that \lstinline{get_winner()} will be the most challenging of these,
so please plan your time accordingly.

The skeleton project contains a file \texttt{play.py}, which imports your \lstinline{OxoBoard} class
and uses it to graphically play a game of Noughts and Crosses. You may find this useful when testing your code.

\section*{Submission instructions}

Begin by \textbf{forking} the GitHub repository at the following URL:

\url{https://github.com/Falmouth-Games-Academy/comp110-worksheet-D}

Edit \texttt{oxo.py}, implementing the required functions.
When you have finished, open a \textbf{pull request}.

\textbf{Do not move or rename \texttt{oxo.py}, and do not edit or delete any of the other files in the repository.}
Doing so will interfere with the automated testing scripts used to check your submission for correctness,
and as a result may lead to you losing marks.

Attend the scheduled worksheet feedback session in \textbf{week 11},
ensuring that you have uploaded all material to GitHub and opened a pull request before this time.

\section*{Marking criteria}

Remember that \textbf{it is better to submit incomplete work than to submit nothing at all}.
Any attempt, even unfinished, will receive a passing grade.

Your work will be marked according to the following criteria:
\begin{itemize}
	\item \textbf{Functional coherence}. Is your implementation correct?
		Your code will be run through TravisCI to verify that it gives the correct results for a large sample of input values.
	\item \textbf{Sophistication}. Have you made use of appropriate code structures and data structures?
		Note the emphasis is on \textbf{appropriate}; extra credit will \textbf{not} be given for unnecessarily complex solutions.
	\item \textbf{Maintainability: readability}. Is your code well commented? Are your identifier names appropriate and descriptive?
		Have you adhered to appropriate coding standards (e.g. PEP-8)?
	\item \textbf{Maintainability: expandability}. Suppose that we wanted to implement an $n \times n$ variant of Noughts and Crosses
		that works for any positive integer $n$.
		How easily could your code be adapted to this change in requirements?
		How about an $m \times n$ variant, where the objective is to get $k$ in a row,
		for any positive integers $m,n,k$?
\end{itemize}

\end{document}
