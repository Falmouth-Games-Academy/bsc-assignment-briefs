\documentclass{../../../fal_assignment}
\graphicspath{ {../../../} }

\usepackage{amsmath}
\usepackage{enumitem}
\setlist{nosep} % Make enumerate / itemize lists more closely spaced
\usepackage[T1]{fontenc} % http://tex.stackexchange.com/a/17858
\usepackage{url}
\usepackage{todonotes}
\usepackage{algpseudocode}
\usepackage{listings}
\lstset{
	basicstyle=\ttfamily,
	frame=single,
	showstringspaces=false,
	breaklines=true,
	prebreak={\space\hbox{\textcolor{gray}{$\hookleftarrow$}}}
}
\lstset{
	commentstyle=\ttfamily\textit,
	keywordstyle=\ttfamily\textbf,
	stringstyle=\ttfamily,
	rulecolor=\color{black}
}
\lstset{language=Python}

\title{Worksheet 5: Computational Complexity}
\module{COMP110}
\author{Ed Powley}
\version{2.1}

\begin{document}

\maketitle
\marginpicture{flavour_pic}{
    Considering every pair of elements in a data structure often leads to quadratic complexity.
}

\section*{Tasks}

This worksheet tests your understanding of the concept of computational complexity,
as well as your ability to communicate this understanding in writing.

Consider the following algorithm:

\begin{algorithmic}[1]
	\Procedure{HasDuplicate}{list}
		\State let $n$ be the length of list
		\For{$i = 0, 1, \dots, n - 1$}
			\For{$j = 0, 1, \dots, n - 1$} \label{line:forj}
				\If{$i \neq j$ and $\text{list}[i] = \text{list}[j]$}
					\State \textbf{return} true
				\EndIf
			\EndFor
		\EndFor
		\State \textbf{return} false
	\EndProcedure
\end{algorithmic}

\begin{enumerate}[label=(\alph*)]
	\item\label{q:first} \textbf{State} what task this algorithm performs.
	\item \textbf{Explain} why the worst case running time of the algorithm is quadratic, i.e.\ $O(n^2)$.
\end{enumerate}

Now suppose that the \textbf{for} loop on line~\ref{line:forj} is changed so that $j$ ranges from $0$ to $i-1$.

\begin{enumerate}[resume,label=(\alph*)]
	\item \textbf{Explain} why the algorithm is still correct.
	\item \textbf{Explain} why the algorithm will run approximately twice as fast.
	\item Is the time complexity of the algorithm still quadratic? \textbf{Explain} your answer.
\end{enumerate}

Now consider the following algorithm, which performs the same task as the first:

\begin{algorithmic}[1]
	\Procedure{HasDuplicate}{list}
		\State let $n$ be the length of list
		\State sortedList $\gets \Call{sort}{\text{list}}$
		\For{$i = 1, 2, \dots, n - 1$}
			\If{$\text{list}[i-1] = \text{list}[i]$}
				\State \textbf{return} true
			\EndIf
		\EndFor
		\State \textbf{return} false
	\EndProcedure
\end{algorithmic}

\begin{enumerate}[resume,label=(\alph*)]
	\item With reference to an appropriate source, \textbf{write down} the time complexity of Python's built-in
		\lstinline{sort} function, in big-$O$ notation.
	\item Thus \textbf{write down} the time complexity of the above algorithm in big-$O$ notation.
		\textbf{Explain} your answer.
	\item If the size of the input list is large, which of these two algorithms is likely to run faster?
		\textbf{Explain} your answer.
	\item \textbf{Suggest one} reason why a programmer might choose the ``slower'' algorithm
		over the ``faster'' one.
	\label{q:last}
\end{enumerate}

\section*{Submission instructions}

Begin by \textbf{forking} the GitHub repository at the following URL:

\url{https://github.com/Falmouth-Games-Academy/comp110-worksheet-5}

Write your answers to questions \ref{q:first}--\ref{q:last} in the \texttt{README.md} file.
Open a \textbf{pull request}.

\section*{Marking criteria}

Remember that \textbf{it is better to submit incomplete work than to submit nothing at all}.
Any attempt, even unfinished, will receive a passing grade.

Your work will be marked according to the following criteria:
\begin{itemize}
	\item Where appropriate, are your answers \textbf{correct}?
	\item Are your explanations \textbf{clear}, \textbf{concise} and \textbf{accurate}?
	\item Where you have obtained information from external sources, are they \textbf{properly cited}?
\end{itemize}

\end{document}
