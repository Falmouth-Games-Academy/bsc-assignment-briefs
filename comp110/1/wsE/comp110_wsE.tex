\documentclass{../../../fal_assignment}
\graphicspath{ {../../../} }

\usepackage{enumitem}
\setlist{nosep} % Make enumerate / itemize lists more closely spaced
\usepackage[T1]{fontenc} % http://tex.stackexchange.com/a/17858
\usepackage{url}
\usepackage{todonotes}

\title{COMP110 Worksheet E: TIS-100}
\module{COMP110}
\author{Ed Powley}
\version{2.0}

\begin{document}

\maketitle

\section*{Introduction}

\begin{marginquote}
``The Tesselated Intelligence System is a massively parallel computer architecture comprised of non-uniformly
interconnected heterogeneous nodes. The Tesselated Intelligence System is ideal for applications requiring complex
data stream processing, such as automated financial trading, bulk data collection, and civilian behavioural analysis.''

--- TIS-100 Reference Manual
\end{marginquote}
\marginpicture{flavour_pic}{
    \emph{TIS-100} is a puzzle game in which players must write assembly code to process data streams
    in increasingly complex ways.
}

\textbf{TIS-100} is a puzzle game released in 2015 by independent developers Zachtronics.
The gameplay consists of writing assembly code for a fictional parallel computer architecture,
and as such requires skills in low-level programming.

To complete this worksheet:
\begin{enumerate}[label=(\alph*)]
	\item \textbf{Play} TIS-100 --- it is available to purchase from Steam for around \textsterling 5;
 	\item \textbf{Complete} the first thirteen levels of the game, upto and including ``Signal Multiplier'';
 	\item \textbf{Annotate} your solutions with descriptive comments.
\end{enumerate}

\section*{Submission instructions}

Begin by \textbf{forking} the GitHub repository at the following URL:

\url{https://github.com/Falmouth-Games-Academy/comp110-worksheet-E}

Once each level has been completed, you must add \textbf{comments} to explain your solution.
TIS-100 does not give you sufficient space to do this effectively.
Instead, click the ``Open Save Directory'' button within TIS-100 and locate
the text file containing your solution.
It will be named according to the ``Segment'' number in the level select screen, for example your solution to
``Self-Test Diagnostic'' will be saved as \texttt{00150.0.txt}.
Open the text file in your preferred text editor.
Add comments to your code, prefixing your comments with the \texttt{\#} character.

\textbf{Warning}: after adding comments, if you load your solution into TIS-100 you will see the text is truncated.
If you then edit the solution in TIS-100, the truncated text will be lost!
Therefore it is recommended that you do not edit your solutions after annotating them,
and that you make regular backups of your work.

\textbf{Upload} your annotated solutions to your forked repository on GitHub.
Attend the timetabled worksheet review session in \textbf{Week 13},
ensuring that you have uploaded all material to GitHub before this time.

\section*{Marking criteria}

Remember that \textbf{it is better to submit incomplete work than to submit nothing at all}.
If you do not manage to finish all assigned levels by the deadline, upload solutions for whatever levels you have completed
and you will receive a passing grade.

Your work will be assessed on the \textbf{number of levels} you have completed from the first thirteen.
All thirteen levels must be solved to demonstrate \textbf{professional competency},
however partial credit will be awarded if some levels have been skipped.

Your work will equally be assessed on the \textbf{quality of your comments}.
Your comments should be clear, comprehensive and demonstrate a good standard of written English.
Remember that comments should ideally focus on \textbf{why} rather than \textbf{what} or \textbf{how},
but use your judgement as to what is already evident from the code and what warrants explanation in comments.

\end{document}
