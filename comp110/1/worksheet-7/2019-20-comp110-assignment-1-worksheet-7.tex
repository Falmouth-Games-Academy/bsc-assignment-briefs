\documentclass{../../../fal_assignment}
\graphicspath{ {../../../} }

\usepackage{enumitem}
\setlist{nosep} % Make enumerate / itemize lists more closely spaced
\usepackage[T1]{fontenc} % http://tex.stackexchange.com/a/17858
\usepackage{url}
\usepackage{todonotes}
\usepackage{amsmath}

\usepackage{listings}
\lstset{
    language=Python,
    basicstyle=\footnotesize\ttfamily,
    commentstyle=\ttfamily\textit,
    keywordstyle=\ttfamily\textbf,
    stringstyle=\ttfamily,
    frame=single,
    tabsize=4,
    showstringspaces=false
}

\newcommand{\TT}{\textsc{True}}
\newcommand{\FF}{\textsc{False}}
\newcommand{\OP}[1]{\ \textsc{#1}\ }
\newcommand{\OPand}{\wedge}
\newcommand{\OPor}{\vee}
\newcommand{\OPnot}{\neg}

\newcommand{\fib}{\operatorname{fib}}

\title{COMP110 Exercise Sheet III: Recursion and Induction}
\module{COMP110}
\author{Ed Powley}
\version{1.0}

\begin{document}

\maketitle

To complete this exercise sheet:
\begin{itemize}
    \item \textbf{complete} the following problems using \textbf{pen and paper}; and
    \item \textbf{hand in} your solutions in the COMP110 workshop session in week 8.
\end{itemize}

\begin{enumerate}
    \item Prove the following identities by induction. In all cases, $X_1, \dots, X_n$ and $Y$ are
		boolean variables.
        \begin{enumerate}
            \item $\OPnot (X_1 \OPor X_2 \OPor \dots \OPor X_n) = \OPnot X_1 \OPand \OPnot X_2 \OPand \dots \OPand \OPnot X_n$
            \item $\OPnot (X_1 \OPand X_2 \OPand \dots \OPand X_n) = \OPnot X_1 \OPor \OPnot X_2 \OPor \dots \OPor \OPnot X_n$
            \item $(Y \OPand X_1) \OPor (Y \OPand X_2) \OPor \dots \OPor (Y \OPand X_n) = Y \OPand (X_1 \OPor X_2 \OPor \dots \OPor X_n)$
            \item $(Y \OPor X_1) \OPand (Y \OPor X_2) \OPand \dots \OPand (Y \OPor X_n) = Y \OPor (X_1 \OPand X_2 \OPand \dots \OPand X_n)$
        \end{enumerate}
		\vspace{1ex}
	\item Prove by induction that, for all positive integers $n$,
		$$ \sum_{i=1}^n i^2 = \frac{n(n+1)(2n+1)}{6} $$
	\item The Fibonacci numbers are defined recursively as follows:
		\begin{align*}
			\fib(0) &= 1 \\
			\fib(1) &= 1 \\
			\fib(n) &= \fib(n-1) + \fib(n-2) \quad \text{ for $n \geq 2$}
		\end{align*}
        \begin{enumerate}
            \item Write down the first $10$ Fibonacci numbers (i.e.\ $\fib(0), \dots, \fib(9)$).
			\item Consider the following Python function:
\begin{lstlisting}
def fib(n):
    if n <= 1:
	    return 1
	else:
	    return fib(n-1) + fib(n-2)
\end{lstlisting}
				Prove, by induction, that the total number of calls to \lstinline{fib} required to compute
				\lstinline{fib(n)} is $2 \times \fib(n) - 1$.
        \end{enumerate}
\end{enumerate}

\end{document}
