\documentclass{../../fal_assignment}
\graphicspath{ {../../} }

\usepackage{enumitem}
\setlist{nosep} % Make enumerate / itemize lists more closely spaced
\usepackage[T1]{fontenc} % http://tex.stackexchange.com/a/17858
\usepackage{url}
\usepackage{todonotes}

\title{Research Journal --- Legacy Game Systems}
\author{Ed Powley}
\module{COMP310}

\begin{document}

\maketitle

\section*{Introduction}

\begin{marginquote}
  ``There are two ways of constructing a software design: One way is to make it so simple that there are obviously no deficiencies and the other way is to make it so complicated that there are no obvious deficiencies.''
    
    --- C.A.R. Hoare
    
    \marginquoterule
    
    ``The computing scientist's main challenge is not to get confused by the complexities of [their] own making.''
    
    --- E. W. Dijkstra
    
    \marginquoterule

        ``Controlling complexity is the essence of computer programming.''
    
    --- Brian Kernighan
            
\end{marginquote}
\marginpicture{flavour_pic}{
    \textit{ACM SIGGRAPH} is the premier conference on computer graphics, with contributions to it highly respected by the games industry.
}

In this assignment, you will collaborate to create a wiki focusing on 6502 assembly programming and homebrew NES game development.

Bla bla bla motivation.

This assignment is formed of several parts:

\begin{enumerate}[label=(\Alph*)]
    \item \textbf{Write} a draft plan in the wiki for the library, that will:
    	\begin{enumerate}[label=(\roman*)]
    		\item \textbf{write} User Stories for the library
    		\item \textbf{list} the key sources to support the library.
    		\item \textbf{list} a series of Unit Tests to be carried out during the development of the library
    	\end{enumerate}
    \item \textbf{Develop} the first draft of the library, this should include the key classes/functions for the library
    	\begin{enumerate}[label=(\roman*)]
    		\item \textbf{summarise} the key features of the library;
    		\item \textbf{write} entries in the wiki to support these key components of the library
    		\item \textbf{revise} any issues raised by your tutor and/or your peers.
    		\item \textbf{detail} the results of the Unit tests
    	\end{enumerate}
    \item \textbf{Develop} the final draft of the library
        \begin{enumerate}[label=(\roman*)]
    		\item \textbf{summarise} the key features of the library;
    		\item \textbf{write} entries in the wiki to support these key components of the library
    		\item \textbf{revise} any issues raised by your tutor and/or your peers.
    		\item \textbf{complete} a peer assessment
    	\end{enumerate}
    \item \textbf{Discuss} your contribution to the wiki \& library with your tutor in the viva session in class.
\end{enumerate}

\subsection*{Assignment Setup}

This assignment is an \textbf{development} and \textbf{wiki task}. Fork the GitHub repository at the following URL:

\indent \url{https://github.com/Falmouth-Games-Academy/comp220-maths-lib}

Use the existing directory structure and, as required, extend this structure with sub-directories.

Modify the \texttt{.gitignore} to the defaults for \textbf{Visual Studio}. Please, also ensure that you add editor-specific files and folders to \texttt{.gitignore}. 

A Trello board has also been setup, please ensure that you visit the following link to become a member - \url{https://trello.com/invite/b/fAFs8G9h/f3f7671ef0511c139dc00715e229e002/comp220-maths-library-2017-18}

\subsection*{Part A}

Part A consists of a \textbf{single formative submissions}. This work is \textbf{collaborative}.

Please ensure that you have added all User Stories to the Trello board. Any discussion on the generation of this should be evident on the history of the wiki page for the features of the library.

The key sources should be added to a \textbf{sources page} on the wiki, again this should be a collaborative effort and it should be evident from the history of the page (and discussion) that everyone has contributed to this.

The Unit tests should be detailed on a \textbf{testing card} on the Trello board and these should be given more detail on a wiki page called \textbf{testing}.

Please ensure that all changes have been made to wiki and pull request has been made by week 3. 

You will receive \textbf{immediate informal feedback}.

\subsection*{Part B}

Part B consists of a \textbf{single formative submissions}. This work is \textbf{collaborative}.

To complete Part B, you should carry out a first pass implementation of the library. At this point key classes such as Vectors and Matrices should be created, in addition to these classes then some of the key operations should also be implemented such as addition, multiplication, dot product, cross product etc. All code written at this point should be fully commented in the Doxygen style, these doscs should also be uploaded to a wikipage called \textbf{docs} 

The results of Unit tests should be detailed on the \textbf{testing} wiki page.

Please ensure that all changes have been made to wiki and pull request has been made by week 6. 

You will receive \textbf{immediate informal feedback}.

\subsection*{Part C}

Part C consists of a \textbf{single formative submissions}. This work is \textbf{collaborative}.

To complete Part C, you should update the maths library to include more functionality such as a class to support Quaternions.

The results of Unit tests should be detailed on the \textbf{testing} wiki page.

Please ensure that all changes have been made to wiki and pull request has been made by week 10. 

Finally, you should fill out a peer assessment for each student in the class.

You will receive \textbf{immediate informal feedback}.

\subsection*{Part D}

Part D consists of a \textbf{single formative submission}. This work is \textbf{individual} and will be assessed on a \textbf{threshold} basis.

To complete Part D, bring the final version of the library to the viva session in class.
Be ready to discuss your work with your tutor.

You will receive \textbf{immediate informal feedback}.

\section*{Additional Guidance}


\section*{FAQ}

\begin{itemize}
	\item 	\textbf{What is the deadline for this assignment?} \\ 
    		Falmouth University policy states that deadlines must only be specified on the MyFalmouth system.
    		
	\item 	\textbf{What should I do to seek help?} \\ 
    		You can email your tutor for informal clarifications. For informal feedback, make a pull request on GitHub. 
    		
    	\item 	\textbf{Is this a mistake?} \\ 	
    		If you have discovered an issue with the brief itself, the source files are available at: \\
    		\url{https://github.com/Falmouth-Games-Academy/bsc-assignment-briefs}.\\
    		 Please make a pull request and comment accordingly.
\end{itemize}

\rubrichead{Criteria marked with a $\ddagger$ are shared by the group. All other criteria are individual.}
\rubricyearthree
\begin{markingrubric}
	\firstcriterion{Basic Competency Threshold}{40\%}
		\gradespan{1}{\fail At least one part is missing or is unsatisfactory.}
		\gradespan{5}{The student participated in the peer review activity,
			with enough work available to allow a meaningful review.
			\par The student gave a meaningful review of at least one peer's work.
			\par The student participated in the viva, with enough work available to hold a meaningful discussion.
			\par Sources have been cited in an appropriate manner, without any obvious errors.
			\par There are no breaches of academic integrity.
		}
%
    \criterion{Structure}{5\% $\ddagger$}
        \grade\fail 	There is little structure to the wiki.
        \grade 		There is some structure to the wiki.
        \grade 		The wiki is structured somewhat sensibly.
        \grade 		The wiki is structured somewhat effectively.
        \grade 		The wiki is structured effectively.
        \grade 		The wiki is structured highly effectively.
%
    \criterion{Coverage}{5\% $\ddagger$}
        \grade\fail		The wiki gives a superficial coverage of the topic.
				\par			There is much superfluous material.
        \grade 		The wiki gives an incomplete coverage of the topic.
				\par			There is some superfluous material.
        \grade 		The wiki gives an adequate coverage of the topic.
				\par			There is some superfluous material.
        \grade 		The wiki gives a somewhat comprehensive coverage of the topic.
				\par			There is little superfluous material.
        \grade 		The wiki gives a comprehensive coverage of the topic.
				\par			There is very little superfluous material.
        \grade 		The wiki gives an extremely comprehensive coverage of the topic.
				\par			There is no superfluous material.
%
    \criterion{Specificity, verifiability \& accuracy of claims}{10\%}
        \grade\fail 	No citations to evidence to claims.
        \par 		Substantial errors and/or misinterpretations.
        \grade 		Few claims have a clear source of evidence.
        \par 		Significant errors and/or misinterpretations.
        \grade 		Some claims have a clear source of evidence.
        \par 		Many errors and/or misinterpretations.
        \grade 		Many claims have a clear source of evidence.
        \par 		Some errors and/or misinterpretations.
        \grade 		Most claims have a clear source of evidence.
        \par 		Few errors and/or misinterpretations.
        \grade 		All claims have a clear source of evidence.
        \par 		Almost no errors and/or misinterpretations.
%
    \criterion{Depth of insight}{15\%}
        \grade\fail Little insight is demonstrated.
        \par		Material is summarised in the student's own words.
        \grade		Some insight is demonstrated.
        \par		Attempts are made at discussion beyond summary.
        \grade		Much insight is demonstrated.
        \par		Discussion is inferential in nature.
        \grade		Considerable insight is demonstrated.
        \par		Discussion is analytical in nature.
        \grade		Significant insight is demonstrated.
        \par		Discussion is analytical and evaluative in nature.
        \grade		Extensive insight is demonstrated.
        \par		Discussion is highly analytical and evaluative in nature.
%
    \criterion{Synthesis}{15\%}
        \grade\fail No connections are made between different sources.
        \grade		Basic connections are made between different sources.
        \grade		Reasonable connections are made between different sources.
        \par		Connections go beyond mere description.
        \grade		Strong connections are made between different sources.
        \par		Connections are analytical in nature.
        \grade		Sources are synthesised into a coherent narrative.
        \par		Connections are analytical and evaluative in nature.
        \grade		Sources are synthesised into a highly coherent narrative.
        \par		Connections are analytical and evaluative in nature.
%
    \criterion{Community engagement}{10\%}
        \grade\fail 	No contribution has been made to the wiki.
        \grade 		A few contributions have been made to the wiki.
				\par		Contributions are isolated, with little to no engagement in community discourse.
        \grade 		Some contributions have been made to the wiki.
        \par		The student has made some attempt to engage in community discourse.
        \grade 		Many contributions have been made to the wiki.
        \par		The student has actively engaged in the community discourse.
        \grade 		A significant number of contributions have been made to the wiki.
        \par		The student has participated in steering the community discourse.
        \grade 		An extensive number of contributions have been made to the wiki.
        \par		The student has played a key role in driving the community discourse.
%
    \criterion{Spelling \& grammar}{5\%}
        \grade\fail 	Substantial spelling and/or grammatical errors.
        \grade 		Many spelling and/or grammatical errors.
        \grade 		Some spelling and/or grammatical errors.  
        \grade 		Few spelling and/or grammatical errors.
        \grade 		Almost no spelling and/or grammatical errors.
        \grade 		No spelling or grammatical errors.
\end{markingrubric}

\end{document}