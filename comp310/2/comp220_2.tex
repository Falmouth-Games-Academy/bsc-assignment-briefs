\documentclass{../../fal_assignment}
\graphicspath{ {../../} }

\usepackage{enumitem}
\setlist{nosep} % Make enumerate / itemize lists more closely spaced
\usepackage[T1]{fontenc} % http://tex.stackexchange.com/a/17858
\usepackage{url}
\usepackage{todonotes}

\title{Research Journal --- Graphics}
\author{Brian McDonald}
\module{COMP220}

\begin{document}

\maketitle

\section*{Introduction}

\begin{marginquote}
  ``There are two ways of constructing a software design: One way is to make it so simple that there are obviously no deficiencies and the other way is to make it so complicated that there are no obvious deficiencies.''
    
    --- C.A.R. Hoare
    
    \marginquoterule
    
    ``The computing scientist's main challenge is not to get confused by the complexities of [their] own making.''
    
    --- E. W. Dijkstra
    
    \marginquoterule

        ``Controlling complexity is the essence of computer programming.''
    
    --- Brian Kernighan
            
\end{marginquote}
\marginpicture{flavour_pic}{
    \textit{ACM SIGGRAPH} is the premier conference on computer graphics, with contributions to it highly respected by the games industry.
}

In this assignment, you will produce a Mathematics library which can be used by a Computer Graphics API such as OpenGL. You will work as a class to contribute to the creation of this library and build on it over the course of the module. To aid in the development of the library you will also create a wiki and also produce code documentation in the doxygen format - \url{http://www.stack.nl/~dimitri/doxygen/}.

You will also need to read and understand academic and non-academic sources in order to build this library. This research should be documented in the wiki and pages developed to support the synthesis of the source material. 

This assignment is formed of several parts:

\begin{enumerate}[label=(\Alph*)]
    \item \textbf{Write} a draft plan in the wiki for the library, that will:
    	\begin{enumerate}[label=(\roman*)]
    		\item \textbf{write} User Stories for the library
    		\item \textbf{list} the key sources to support the library.
    		\item \textbf{list} a series of Unit Tests to be carried out during the development of the library
    	\end{enumerate}
    \item \textbf{Develop} the first draft of the library, this should include the key classes/functions for the library
    	\begin{enumerate}[label=(\roman*)]
    		\item \textbf{summarise} the key features of the library;
    		\item \textbf{write} entries in the wiki to support these key components of the library
    		\item \textbf{revise} any issues raised by your tutor and/or your peers.
    		\item \textbf{detail} the results of the Unit tests
    	\end{enumerate}
    \item \textbf{Develop} the final draft of the library
        \begin{enumerate}[label=(\roman*)]
    		\item \textbf{summarise} the key features of the library;
    		\item \textbf{write} entries in the wiki to support these key components of the library
    		\item \textbf{revise} any issues raised by your tutor and/or your peers.
    		\item \textbf{complete} a peer assessment
    	\end{enumerate}
    \item \textbf{Discuss} your contribution to the wiki \& library with your tutor in the viva session in class.
\end{enumerate}

\subsection*{Assignment Setup}

This assignment is an \textbf{development} and \textbf{wiki task}. Fork the GitHub repository at the following URL:

\indent \url{https://github.com/Falmouth-Games-Academy/comp220-maths-lib}

Use the existing directory structure and, as required, extend this structure with sub-directories.

Modify the \texttt{.gitignore} to the defaults for \textbf{Visual Studio}. Please, also ensure that you add editor-specific files and folders to \texttt{.gitignore}. 

A Trello board has also been setup, please ensure that you visit the following link to become a member - \url{https://trello.com/invite/b/fAFs8G9h/f3f7671ef0511c139dc00715e229e002/comp220-maths-library-2017-18}

\subsection*{Part A}

Part A consists of a \textbf{single formative submissions}. This work is \textbf{collaborative}.

Please ensure that you have added all User Stories to the Trello board. Any discussion on the generation of this should be evident on the history of the wiki page for the features of the library.

The key sources should be added to a \textbf{sources page} on the wiki, again this should be a collaborative effort and it should be evident from the history of the page (and discussion) that everyone has contributed to this.

The Unit tests should be detailed on a \textbf{testing card} on the Trello board and these should be given more detail on a wiki page called \textbf{testing}.

Please ensure that all changes have been made to wiki and pull request has been made by week 3. 

You will receive \textbf{immediate informal feedback}.

\subsection*{Part B}

Part B consists of a \textbf{single formative submissions}. This work is \textbf{collaborative}.

To complete Part B, you should carry out a first pass implementation of the library. At this point key classes such as Vectors and Matrices should be created, in addition to these classes then some of the key operations should also be implemented such as addition, multiplication, dot product, cross product etc. All code written at this point should be fully commented in the Doxygen style, these doscs should also be uploaded to a wikipage called \textbf{docs} 

The results of Unit tests should be detailed on the \textbf{testing} wiki page.

Please ensure that all changes have been made to wiki and pull request has been made by week 6. 

You will receive \textbf{immediate informal feedback}.

\subsection*{Part C}

Part C consists of a \textbf{single formative submissions}. This work is \textbf{collaborative}.

To complete Part C, you should update the maths library to include more functionality such as a class to support Quaternions.

The results of Unit tests should be detailed on the \textbf{testing} wiki page.

Please ensure that all changes have been made to wiki and pull request has been made by week 10. 

Finally, you should fill out a peer assessment for each student in the class.

You will receive \textbf{immediate informal feedback}.

\subsection*{Part D}

Part D consists of a \textbf{single formative submission}. This work is \textbf{individual} and will be assessed on a \textbf{threshold} basis.

To complete Part D, bring the final version of the library to the viva session in class.
Be ready to discuss your work with your tutor.

You will receive \textbf{immediate informal feedback}.

\section*{Additional Guidance}


\section*{FAQ}

\begin{itemize}
	\item 	\textbf{What is the deadline for this assignment?} \\ 
    		Falmouth University policy states that deadlines must only be specified on the MyFalmouth system.
    		
	\item 	\textbf{What should I do to seek help?} \\ 
    		You can email your tutor for informal clarifications. For informal feedback, make a pull request on GitHub. 
    		
    	\item 	\textbf{Is this a mistake?} \\ 	
    		If you have discovered an issue with the brief itself, the source files are available at: \\
    		\url{https://github.com/Falmouth-Games-Academy/bsc-assignment-briefs}.\\
    		 Please make a pull request and comment accordingly.
\end{itemize}

\rubrichead{Criteria marked with a $\ddagger$ are shared by the group. All other criteria are individual.}
\rubricyeartwo
\begin{markingrubric}
%
	\firstcriterion{Basic Competency Threshold}{40\%}
		\gradespan{1}{\fail At least one part is missing or is unsatisfactory.}
		\gradespan{5}{Formative submissions have been signed off
			\par The student participated in the viva, with enough work available to hold a meaningful discussion.
			\par Contributed to the codebase and the wiki
			\par No breaches of academic integrity.
		}
%
	\criterion{User Stories}{5\% $\ddagger$}
	\grade \fail Few user stories are distinguishable and easily measured.
	\grade Some user stories are distinguishable and easily measured.
	\grade Most user stories are distinguishable and easily measured.
	\par User stories correspond to the product design.
	\grade Nearly all user stories are distinguishable and easily measured.
	\par User stories clearly correspond to the product design.
	\grade All user stories are distinguishable and easily measured.
	\par User stories clearly and comprehensively correspond to the product design.
	\grade All user stories are distinguishable and easily measured.
	\par User stories clearly and comprehensively correspond to the product design.
%
	\criterion{Code documentation}{10\% $\ddagger$}
	\grade \fail No or very few comments
	\grade Some very high level comments
	\grade Most code commented but only using the \textbf{brief} description
	\grade Most code commented using \textbf{brief} \& \textbf{detailed} description
	\grade Most code commented to a good level of detail using \textbf{brief} \& \textbf{detailed} description
	\grade All code commented, including member variables and functions 
%
	\criterion{Unit Tests}{10\% $\ddagger$}
	\grade \fail No Unit Tests used
	\grade Some Unit Tests created, but results not presented in the wiki 
	\grade Some Unit Tests created and results documented in the wiki
	\grade A good level Unit Tests created and results documented in the wiki
	\grade A good level of Unit Tests created throughout the project and all results documented in the wiki
	\grade Excellent level of Unit Tests and results well presented in the wiki	
%
	\criterion{Wiki}{10\% $\ddagger$}
	\grade \fail No entries added to the wiki throughout the project
	\grade Some entries added to the wiki, but most pages aren't written in a wiki style
	\grade Some entries added to the wiki and each page is in the wiki style
	\grade A good level of entries in the wiki and each page is in the wiki style
	\grade A good level of entries in the wiki, each page is in the wiki style and some evidence of discussion of entries
	\grade Excellent level of entries, all written in the wiki style which also show evidence of editing based on discussion amongst the peer group
%
	\criterion{Peer Assessment}{10\%}
		\gradespan{1}{\fail No peer assessment mark}
		\gradespan{5}{Peer Assessment Mark, scaled to 10}
% 
    \criterion{Use of Version Control}{5\%}
	\grade \fail Material has been checked into GitHub less frequently than once per sprint.
	\grade Code has been checked into GitHub at least once per sprint.
	\grade Code has been checked into GitHub several times per sprint.
		\par Commit messages are clear, concise and relevant.
		\par There is some evidence of engagement with peers (e.g.\ code review).
	\grade Code has been checked into GitHub several times per sprint.
		\par Commit messages are clear, concise and relevant.
		\par There is much evidence of engagement with peers (e.g.\ code review).
	\grade Code has been checked into GitHub several times per sprint.
		\par Commit messages are clear, concise and relevant.
		\par There is significant evidence of engagement with peers (e.g.\ code review).
	\grade Code has been checked into GitHub several times per week.
		\par Commit messages are clear, concise and relevant.
		\par There is extensive evidence of engagement with peers (e.g.\ code review).
%
    \criterion{Community engagement}{10\%}
        \grade\fail 	Few or no contributions have been made to the wiki.
        \grade 		Some contributions have been made to the wiki.
        \par		Contributions are non-trivial.
        \par		The student has made some attempt to engage in community discourse.
        \grade 		Many contributions have been made to the wiki.
        \par		Contributions are well-reasoned.
        \par		The student has engaged in the community discourse.
        \grade 		A considerable number of contributions have been made to the wiki.
        \par		Contributions are well-reasoned.
        \par		The student has actively engaged in the community discourse.
        \grade 		A significant number of contributions have been made to the wiki.
        \par		Contributions are well-reasoned and academically sound.
        \par		The student has participated in steering the community discourse.
        \grade 		An extensive number of contributions have been made to the wiki.
        \par		Contributions are well-reasoned and academically sound.
        \par		The student has played a key role in driving the community discourse.
\end{markingrubric}

\end{document}