\documentclass{../../fal_assignment}
\graphicspath{ {../../} }

\usepackage{enumitem}
\setlist{nosep} % Make enumerate / itemize lists more closely spaced
\usepackage[T1]{fontenc} % http://tex.stackexchange.com/a/17858
\usepackage[hyphens]{url}
\usepackage{todonotes}

\title{Constrained development task}
\author{Dr Ed Powley}
\module{COMP310}

\begin{document}

\maketitle

\section*{Introduction}

\begin{marginquote}
``Because of the nature of Moore's law, anything that an extremely clever graphics programmer can do at one point can be replicated by a merely competent programmer some number of years later.''

--- John Carmack

    \marginquoterule

``Programming is like this amazing puzzle game where the puzzles are created by your own stupidity.''

--- Brad O'Farrell
\end{marginquote}
\marginpicture{flavour_pic}{
    A demo of fluid simulation with NVIDIA's PhysX.
    Recent advances in GPU technology have enabled a wide range of high-fidelity real-time rendering and simulation effects.
}

In this assignment, you are required to \textbf{design} and \textbf{implement} a ``de-make''
of a well-known game. Your de-make will be implemented in assembly language and will run on the 
Nintendo Entertainment System (NES).

Bla bla bla motivation.

This assignment is formed of several parts:
\begin{enumerate}[label=(\Alph*)]
	\item \textbf{Prepare} a 10-minute presentation that will:
		\begin{enumerate}[label=(\roman*)]
			\item \textbf{outline} the concept of your de-make, including the game it is based upon;
			\item \textbf{explain} the key mechanic around which your de-make will focus;
			\item \textbf{justify} the technical feasibility of your concept.
		\end{enumerate}
	\item \textbf{Write} a computer program in 6502 assembly language that will
		\begin{enumerate}[label=(\roman*)]
			\item \textbf{implement} your de-make concept.
		\end{enumerate}
\end{enumerate}

\subsection*{Assignment Setup}

This assignment is a \textbf{programming} task. Fork the GitHub repository at the following URL:

\indent \url{https://github.com/Falmouth-Games-Academy/comp310-portfolio}

Use the existing directory structure and, as required, extend this structure with sub-directories.
Ensure that you maintain the \texttt{readme.md} file.
If necessary, please ensure that you add any editor-specific files and folders to \texttt{.gitignore}. 

\subsection*{Part A}

Part A consists of a \textbf{single formative submission}. This work is \textbf{individual} and will be assessed on a \textbf{threshold} basis. Answer the following questions to pass:

\begin{itemize}
	\item What is the title and high concept of the demo?
	\item What is the intended aesthetic?
	\item For each of two graphical or simulation effects:
	\begin{itemize}
		\item What is the effect?
		\item How will the effect contribute to the aesthetic?
	\end{itemize}
	\item Is the scope appropriate for the product development time-frame?
\end{itemize}

To complete Part A, prepare the handout using any word processing tool.
To help illustrate your intended aesthetic, your handout may include images and/or links to online videos.

Show the handout to your \textbf{tutor} for immediate \textbf{informal feedback}.

Once you have completed Part A, you should move onto the Worksheets. See the worksheets on the Learning Space for more details.  

\section*{Additional Guidance}

As always, avoid underestimating the effort required to implement even simple software; always consider scope.
From the proposal stage, you should consider very carefully what is feasible.

Your code will be assessed on \textbf{functional coherence}:
how well the finished product corresponds to the user stories,
and whether it has any obvious bugs.
Correspondence to user stories runs both ways:
implementing features that were not present in the design (``feature creep'')
is just as bad as neglecting to implement features.

Unlike your previous assignments,
you will be assessed on the \textbf{performance} of your solution.
Real-time graphics and simulation are not just about creating aesthetically pleasing effects,
but doing so whilst maintaining a smooth and consistent framerate free of any lag or glitches that might frustrate the player.
It may be necessary to trade-off the complexity or fidelity of an effect
in order to achieve acceptable performance.

Your code will also be assessed on \textbf{sophistication}.
To succeed on a project of this size and complexity,
you will need to make use of appropriate algorithms, data structures, libraries, and object oriented programming concepts.
Appropriateness to the task at hand is key:
you will \textbf{not} receive credit for complexity  
where something simpler would have sufficed.

\textbf{Maintainability} is important in all programming projects,
but doubly so when working in a team.
Use \textbf{comments} liberally to improve code comprehension,
and carefully choose the \textbf{names} for your files, classes, functions and variables.
Use a well-established commenting convention
for \textbf{high-level documentation}.
The open-source tool Doxygen supports several such conventions.
Also ensure that all code corresponds to a sensible and consistent \textbf{formatting style}:
indentation, whitespace, placement of curly braces, etc.
Hard-coded \textbf{literals} (numbers and strings) within the source should be avoided,
with values instead defined as constants together in a single place.
Consider allowing some literal values, where appropriate, to be ``tinkered'' without changing the source code,
e.g.\ by defining them in an external file read at startup.

As with all assignments on this course, you are expected to display a level of
\textbf{innovation and creative flair} befitting Falmouth University's reputation as a world-leading
arts institution.
One approach to promoting creativity is
\textbf{divergent thinking}: generating ideas by exploring many possible solutions.
Often the most interesting ideas are \textbf{subversive}: they deliberately go against
convention or obvious solutions.

You will \textbf{not} be judged on the quality of your art assets.
It is fine to use meshes and textures found online,
as long as they are available under an appropriate license and are properly attributed.

\section*{FAQ}

\begin{itemize}
	\item 	\textbf{What is the deadline for this assignment?} \\ 
    		Falmouth University policy states that deadlines must only be specified on the MyFalmouth system.
    		
	\item 	\textbf{What should I do to seek help?} \\ 
    		You can email your tutor for informal clarifications. For informal feedback, make a pull request on GitHub. 
    		
    	\item 	\textbf{Is this a mistake?} \\ 	
    		If you have discovered an issue with the brief itself, the source files are available at: \\
    		\url{https://github.com/Falmouth-Games-Academy/bsc-assignment-briefs}.\\
    		 Please make a pull request and comment accordingly.
\end{itemize}

\section*{Additional Resources}

\begin{itemize}
    \item \url{http://www.opengl-tutorial.org}
    \item \url{http://gamedev.stackexchange.com/questions/32876/good-resources-for-learning-modern-opengl-3-0-or-later}
    \item \url{https://google.github.io/styleguide/cppguide.html}
\end{itemize}

\rubricyeartwo
\begin{markingrubric}
	%
    \firstcriterion{Basic competency threshold}{40\%}
		\grade\fail Bla
		\gradespan{5}{Bla}
	%
\end{markingrubric}

\end{document}