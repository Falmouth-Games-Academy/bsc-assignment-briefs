\documentclass{../../fal_assignment}
\graphicspath{ {../../} }

\usepackage{enumitem}
\setlist{nosep} % Make enumerate / itemize lists more closely spaced
\usepackage[T1]{fontenc} % http://tex.stackexchange.com/a/17858
\usepackage{url}
\usepackage{todonotes}

\title{Creative Computing Project}
\author{Brian McDonald and Matt Watkins}
\module{COMP140}
\version{3.1}

\begin{document}
	
	\maketitle
	
	\section*{Introduction}
	
	\begin{marginquote}
		``As soon as we started programming, we found out to our surprise that it 
		wasn't as easy to get programs right as we had thought. 
		Debugging had to be discovered. I can remember the exact instant when 
		I realized that a large part of my life from then on was going to be spent 
		in finding mistakes in my own programs.''
		\par --- Maurice Wilkes
		\marginquoterule
		\par ``C++ is history repeated as tragedy. Java is history repeated as farce.''
		\par --- Scott McKay
	\end{marginquote}
	\marginpicture{pancake}{
		'Pancake' game and kitchen utensil controllers by Falmouth student Ashley Sands
	}
	
	In this assignment, you are required to \textbf{create} a playful experience 
	which \textbf{interfaces} with a custom control system.
	
	Experimentation, ingenuity, and creativity are at the heart of everything that 
	creative developers do. To this end, building your own custom interface is the perfect 
	place to exercise these characteristics. However, you will also gain invaluable exposure to 
	working with computer hardware and embedded systems. In recent years, there has been considerable growth in the 
	development of new fabrication technologies, such as 3D printers. In addition, electronics, 
	from primitive transistors to complex computer chips, have all become much cheaper. Accessibility 
	to these tools has, therefore, unveiled an unprecedented opportunity to invent and innovate 
	in this space. Increasingly, app developers are augmenting mobile software with new wearable 
	devices. With the advent of VR, AR and XR and the demand for augmenting reality through new sensing technology, developers are increasingly expected to consider new interfaces in the design of entertainment products
	
	This assignment is formed of several parts:
	
	\begin{enumerate}[label=(\Alph*)]
		\item \textbf{Write}, as an \textbf{individual}, prototype game or other playful experience (see the contracts below) which interfaces with a custom controller or sensing system
		\item \textbf{Write}, as an \textbf{individual}, a final version of your project
		\item \textbf{Present}, as an \textbf{individual}, a practical demo of the computer program to your tutor that will:
		\begin{enumerate}[label=\roman*.]
			\item \textbf{demonstrate} your academic integrity;
			\item as well as \textbf{demonstrate} your \textbf{individual} programming \& hardware knowledge.
		\end{enumerate}
	\end{enumerate}
	\subsection*{Contracts}
	To give you a wide brief and accommodate a broad range of applications you will \textbf{choose one} of two contracts. Please refer to the \textbf{separate contracts document} in the \textbf{learning space}.

	
	\subsection*{Assignment Setup}
	Fork the GitHub repository at:
	\indent \url{https://gamesgit.falmouth.ac.uk/projects/COMP140}
	
	Use the existing directory structure, the Unity Project should be placed inside the \textbf{Unity Project} folder and the Arduino project files should be placed inside \textbf{Arduino Project} folder. Ensure that you maintain the \texttt{readme.md} file.
	
	\subsection*{Part A}
	
	Part A consists of a \textbf{single formative submission}. You should demonstrate your progress to a tutor in the timetabled session by \textbf{Week 6}
	
	You will receive \textbf{immediate informal} feedback from your \textbf{tutor}.
	
	\subsection*{Part B}
	
	Part B consists of a \textbf{single summative submissions}. You should download your project from GitHub, and submit a \textbf{zip file} which contains the following
	
	\begin{enumerate}
		\item The Unity Project including all source code and assets
		\item The Arduino Project
		\item Two images of the controller: one of the wiring and another with case/housing
		\item Video footage of the controller and game/experience being played. This should be uploaded to \textbf{Microsoft Streams}.
		\item readme.md with references to all sources and assets used in the project
	\end{enumerate}
	
	
	\subsection*{Part C}
	To complete Part C, implement the final changes to your project. Prepare a practical demonstration of the project. Ensure that the source code and related assets are pushed to GitHub and a pull request is made prior to the scheduled viva session. Then, attend the scheduled viva session via Mircosoft Teams. 
	
	You will receive \textbf{immediate informal} feedback from your \textbf{tutor}.
	
	\section*{Additional Guidance}

	Nobody learns in a vacuum: you are allowed, and indeed encouraged, to discuss your work with your peers. 
	However you must be very careful to avoid falling into \textbf{academic misconduct}, 
	in particular \textbf{plagiarism}. If any part of your solution is \textbf{not your own individual work}, you must make this as clear as possible in your submission, for example in source code comments.
	
	\section*{FAQ}
	
	\begin{itemize}
		\item 	\textbf{What is the deadline for this assignment?} \\ 
		Each worksheet has its own formative deadline, specified on that worksheet and also communicated in class.
		Falmouth University policy states that summative deadlines must only be specified on the MyFalmouth system.
		
		\item 	\textbf{What should I do to seek help?} \\ 
		You can email your tutor for informal clarifications. For informal feedback, make a pull request on GitHub. 
		
		\item 	\textbf{How will I receive feedback on my work?} \\ 
		You will be given verbal feedback on your work during the session in which it is marked.
		If you require more in-depth feedback or discussion, please book an appointment with your tutor.
		
		\item 	\textbf{Is this a mistake?} \\ 	
		If you have discovered an issue with the brief itself, the source files are available at: \\
		\url{https://github.com/Falmouth-Games-Academy/bsc-assignment-briefs}.\\
		Please make a pull request and comment accordingly.
		
		\item \textbf{What coding standards are we using on this assignment?}\\
		We are using the Microsoft's coding styles for C\# \\
		\url{https://docs.microsoft.com/en-us/dotnet/csharp/programming-guide/inside-a-program/coding-conventions}
	\end{itemize}
	
	\section*{Additional Resources}
	\begin{itemize}
		\item Wilkinson, K. and Petrich, M. (2014) The Art of Tinkering: Meet 150 Markers Working at the Intersection of Art, Science \& Technology. Weldon Owen: London.
		\item Alicia Gibb. Building Open Source Hardware: DIY Manufacturing for Hackers and Makers. Addison Wesley, 2014. 
		\item Jeremy Blum. Exploring Arduino: Tools and Techniques for Engineering Wizardry. John Wiley, 2013. 
		\item Kelly, K. (2014) Cool Tools: A Catalogue of Possibilities. Cool Tools.
		\item \url{https://www.sitepoint.com/heuristic-evaluation-guide/}
		\item \url{https://www.usability.gov/how-to-and-tools/methods/heuristic-evaluation.html}
		\item \url{https://github.com/arduino/Arduino/blob/master/.gitignore}
		\item \url{https://gitignore.io/}
	\end{itemize}
	
\rubrichead{All submissions and assessment criteria for this assignment are individual.}
\begin{markingrubric}
	%
	\firstcriterion{Basic Competency Threshold}{30\%}
	\gradespan{1}{\fail At least one part is missing or is inadequate.}
	\gradespan{5}{Adequate ability to generate ideas, problem solving, concepts, technical competency and proposals in response to set briefs and/or self-initiated activity.
		\par The work demonstrates an adequate, ethically informed, real-world experience of industry/business environments and markets.
		\par Enough work is available to hold a meaningful discussion.
		\par Clear evidence of programming knowledge.
		\par Constraints followed
		\par No breaches of academic integrity.
}
	%
	\criterion{PROCESS: \\Sophistication of Code}{15\%}
	\grade\fail No insight into the appropriate use of programming constructs is evident from the source code.
	\par No attempt to structure the program (e.g. one monolithic function).
	\grade Little insight into the appropriate use of programming constructs is evident from the source code.
	\par The program structure is poor.
	\grade Some insight into the appropriate use of programming constructs is evident from the source code.
	\par The program structure is adequate.
	\grade Much insight into the appropriate use of programming constructs is evident from the source code.
	\par The program structure is appropriate.
	\grade Considerable insight into the appropriate use of programming constructs is evident from the source code.
	\par The program structure is effective. There is high cohesion and low coupling.
	\grade Significant insight into the appropriate use of programming constructs is evident from the source code.
	\par The program structure is very effective. There is high cohesion and low coupling.
	%
	\criterion{PROCESS: \\Maintainability of Code}{10\%}
	\grade\fail There are no comments in the source code, or comments are misleading.
	\par Most variable names are unclear or inappropriate.
	\par Code formatting hinders readability.
	\grade The source code is only sporadically commented, or comments are unclear.
	\par Some identifier names are unclear or inappropriate.
	\par Code formatting is inconsistent or does not aid readability.
	\grade The source code is somewhat well commented.
	\par Some identifier names are descriptive and appropriate.
	\par An attempt has been made to adhere to Microsoft's formatting style.
	\par There is little obvious duplication of code or of literal values.           
	\grade The source code is reasonably well commented.
	\par Most identifier names are descriptive and appropriate.
	\par Most code adheres to the Microsoft's formatting style.
	\par There is almost no obvious duplication of code or of literal values.   
	\grade The source code is reasonably well commented in the Microsoft's style
	\par Almost all identifier names are descriptive and appropriate.
	\par Almost all code adheres to the Microsoft's formatting style.
	\par There is no obvious duplication of code or of literal values. Some literal values can be easily changed in the Unity Editor.
	\grade The source code is very well commented, with Python doc-strings.
	\par All identifier names are descriptive and appropriate.
	\par All source code adheres to the Microsoft's formatting style.
	\par There is no obvious duplication of code or of literal values. Most literal values are, where appropriate, easily changed in the Unity Editor.  
	%            
	\criterion{PROCESS: \\Functionality of Physical Prototype}{10\%}
	\grade\fail No physical prototype, or the prototype is not functional.
	\grade The physical prototype has a little functionality.
	\par There are critical technical and/or constructional flaws.
	\grade The physical prototype has some functionality.
	\par There are major technical and/or constructional flaws.
	\grade The physical prototype has much functionality.
	\par If any, only minor technical and/or constructional flaws.
	\grade The physical prototype has considerable functionality.
	\par If any, only minor technical and/or constructional flaws.
	\grade The physical prototype has significant functionality.
	\par If any, only superficial technical and/or constructional flaws.
	%
	\criterion{PROCESS: Electronics Sophistication}{5\%}
	\grade\fail Solution lacks even basic use of electronic components.
	\grade Solution has some sophistication in terms of electronics.
	\par Little insight insight into electronic circuits.
	\grade Solution has some sophistication  in terms of electronics.
	\par Some insight into electronic circuits.
	\grade Solution leverages electronic components with much effectiveness.
	\par Much insight into electronic circuits.
	\grade Solution leverages electronic components with much effectiveness.
	\par Considerable insight into electronic circuits.
	\grade Solution leverages electronic components with much effectiveness.
	\par Significant insight into electronic circuits.
	%
	\criterion{PROCESS:Physical Form Factor Sophistication}{10\%}
	\grade\fail No physical prototype, or it is limited to a breadboard without housing.
	\grade Physical form factor has a little sophistication.
	\par Little insight into human-computer interaction.
	\grade Physical form factor has some sophistication.
	\par Some insight into human-computer interaction.
	\grade Physical form factor has much sophistication and sturdiness.
	\par Much insight into human-computer interaction.
	\grade Physical form factor has much sophistication  and sturdiness.
	\par Considerable insight into human-computer interaction.
	\grade Physical form factor has considerable sophistication and sturdiness.
	\par Significant insight into human-computer interaction.
	\par Controller has both practical and aesthetic value.
	%
	\criterion{INDUSTRY: \\Creative Response to Brief}{10\%}
	\grade\fail No creativity.
	\par The work is a clone of an existing work with mere cosmetic alterations.
	\grade Little creativity.
	\par The work is derivative of existing works, with only minor alterations.
	\grade Some creativity.
	\par The work is derivative of existing works, demonstrating little divergent and/or subversive thinking.
	\grade Much creativity.
	\par The work is somewhat novel, demonstrating some divergent and/or subversive thinking.
	\grade Considerable creativity.
	\par The work is novel, demonstrating significant divergent and/or subversive thinking.
	\grade Significant creativity.
	\par The work is highly original, with strong evidence of divergent and/or subversive thinking.
	%
	\criterion{INDUSTRY: \\Use of Version Control}{10\%}
	\grade\fail Version control (e.g. GitHub) has not been used.
	\grade Source code has been checked into version control (e.g. GitHub).
	\grade Source code has been checked into version control (e.g. GitHub) at least once per week.
	\par Sensible commit messages are present.
	\grade Source code  has been checked into version control (e.g. GitHub) several times per week.
	\par Commit messages are clear, concise and relevant.
	\par There is evidence of somewhat meaningful engagement with peers (e.g.\ code review).
	\par Comments to peers are somewhat constructive and provide some insight.
	\grade Source code has been checked into version control (e.g. GitHub) several times per week.
	\par Commit messages are clear, concise and relevant.
	\par There is evidence of meaningful engagement with peers (e.g.\ code review).
	\par Comments to peers are reasonably constructive and provide much insight.
	\grade Source code has been checked into version control (e.g. GitHub) many times per week.
	\par Commit messages are clear, concise and relevant.
	\par There is evidence of effective engagement with peers (e.g.\ code review).
	\par Comments to peers are reasonably constructive and provide considerable insight.
	%
\end{markingrubric}
	
\end{document}