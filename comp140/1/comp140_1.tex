\documentclass{../../fal_assignment}
\graphicspath{ {../../} }

\usepackage{enumitem}
\setlist{nosep} % Make enumerate / itemize lists more closely spaced
\usepackage[T1]{fontenc} % http://tex.stackexchange.com/a/17858
\usepackage{url}
\usepackage{todonotes}

\title{Creative Computing Project}
\author{Brian McDonald, Alcwyn Parker, Dr Michael Scott \& Dr Ed Powley}
\module{COMP140}

\begin{document}

\maketitle

\section*{Introduction}

\begin{marginquote}
Hacker definition: ``A person who enjoys exploring the details of programmable systems and stretching their capabilities, as opposed to most users, who prefer to learn only the minimum necessary.''

--- Jargon File

\end{marginquote}
\marginpicture{flavour_pic}{
    Arduino is an open-source prototyping platform based on easy-to-use hardware and software.
}

In this assignment, you will work as an individual to design a novel game controller and implement a physical prototype. Your prototype should function as an input device the game developed for the first coursework. Your prototype should use a hardware platform, such as Arduino, to convert player actions into game inputs. 

Experimentation, ingenuity, and creativity are at the heart of everything that professional game developers do. To this end, building your own custom game controller is the perfect place to exercise these characteristics. However, you will also gain invaluable exposure to working with computer hardware. In recent years, there has been considerable growth in the development of new fabrication technologies, such as 3D printers. In addition, electronics, from primitive transistors to complex computer chips, have all become much cheaper. Accessibility to these tools has, therefore, unveiled an unprecedented opportunity to invent and innovate in this space. Increasingly, app developers are augmenting mobile software with new wearable devices, and so too will game developers with the advent and increasing popularity of virtual, augmented, and mixed reality games.


This assignment is formed of several parts:

\begin{enumerate}[label=(\Alph*)]
    \item \textbf{Write} a proposal for a novel game \textbf{and} controller that will:
    	\begin{enumerate}[label=\roman*.]
    		\item \textbf{describe} the game design that will form the basis for your interface;
    		\item \textbf{illustrate} basic  research into electronic component and physical form factors for controllers;
		\item \textbf{analyse} the design of the controller in detail;
		\item and \textbf{list} the key user stories.
	\end{enumerate}
    \item \textbf{Implement} a final game \textbf{and} physical controller prototype that will:
    	\begin{enumerate}[label=\roman*.]
    		\item \textbf{integrate} a physical controller with a digital game;
		\item as well as \textbf{illustrate} your innovation and creative flair.
	\end{enumerate}
    \item \textbf{Present} a practical demo of your work that will:
    	\begin{enumerate}[label=\roman*.]
    		\item \textbf{highlight} both your technical knowledge \textbf{and} insight;
    		\item as well as \textbf{demonstrate} your academic integrity.
	\end{enumerate}
\end{enumerate}

	\subsection*{Assignment Setup}
	
	Fork the GitHub repository at:
	
	\indent \url{https://github.com/Falmouth-Games-Academy/comp140-gam160-game}
	
	Use the existing directory structure and, as required, extend this structure with sub-directories. Ensure that you maintain the \texttt{readme.md} file.
	
	Modify the \texttt{.gitignore} to the defaults for \textbf{Visual Studio}, \textbf{C}, \textbf{C++}, and \textbf{Arduino}. Please, also ensure that you add editor-specific files and folders to \texttt{.gitignore}. You may need to review the sources in the additional guidance section in order to generate the \texttt{.gitignore} file most appropriate to your repository. 

\subsection*{Part A}

Part A consists of a \textbf{single formative submission}. This work will be assessed on a \textbf{threshold} basis. The following criteria are used to determine a pass or fail:

\begin{enumerate}[label=(\alph*)]
	\item Submission is timely;
	\item Game design is feasible;
	\item Range of controllers assessed is somewhat comprehensive;
	\item Controller design is feasible;
	\item Designs are somewhat distinctive and have at least some creative merit.
\end{enumerate}

To complete part A, write your proposal in the \texttt{readme.md} document. The proposal must include some market research, visualisations of the design, user stories and a rough bill of materials (BOM). You are not allowed to use thumb sticks anywhere on the interface. Following on from this, the proposed design should bare \textbf{no resemblance to traditional arcade or console controllers}. Show this to one of your tutors before the end of the studio practice period. If acceptable, this will be signed-off.

You will receive immediate \textbf{informal feedback} from your \textbf{tutor}.

\subsection*{Part B}

Part B is a \textbf{single summative submission}. This is \textbf{individual} work and will be assessed on a \textbf{criterion-references} basis. Please refer to the marking rubric at the end of this assignment brief for details.

To complete Part B, prepare a practical demonstration of the physical game controller and record this as a 1-2 minute video with at least 3 accompanying photographs (2 views of the controller and 1 of the controller being used with the digital game in-shot). Package these photographs and video together into a single \texttt{.zip} file. Then, upload these to the summative submission area on the LearningSpace. Also prepare the source code and assets for your controller and digital game into a second \texttt{.zip} file. The source code for your digital game and your Arduino code should be included together in this file. Then, upload these to summative submission area on the LearningSpace. 

You will receive \textbf{formal feedback} from your \textbf{tutor} within 3 working days after the viva in Part C.

\subsection*{Part C}

Part C is a \textbf{single summative submission}. This is \textbf{individual} work will be assessed on a \textbf{threshold} basis. The following criteria are used to determine a pass or fail: 

\begin{enumerate}[label=(\alph*)]
	\item Enough work is available to hold a meaningful discussion; 
	\item Clear evidence of programming knowledge and communication skills; 
	\item No breaches of academic integrity. 
\end{enumerate}

To complete Part C, prepare a practical demonstration of the game controller. Ensure that the final build and physical hardware are bought to the viva session. Then, attend the scheduled viva session. 

You will receive \textbf{immediate informal} feedback from your \textbf{tutor}.

\section*{Additional Guidance}
Falmouth University is internationally renowned for the creative industries. Despite the fact that you are studying for a science degree in a technical discipline, you are still expected to strive for the same level of innovation and creative flair as your peers. This assignment is more heavily weighted towards the creative than the assignments you have encountered thus far. On this assignment, a competent execution of an unimaginative idea is unlikely to achieve a high grade, as opposed to an imperfect execution of a unique and ambitious concept, which will be seen favourably by examiners. Consider this when working on your design. One approach to promote creativity is divergent thinking: generate ideas by exploring many possible solutions. Often the most interesting ideas are subversive: they deliberately go against the accepted or most obvious solution.

The history of video games is littered with failed peripherals; often, perceived as expensive gimmicks rather than legitimate enhancements to gameplay. Balance your creative insight with commercial awareness: your design should be informed by research into products that have succeeded and failed in the past, and those underexploited niches in the present. A great project will be highly divergent, but one that has clear commercial viability. Do not be discouraged if you fall short: professionals find this difficult! 

We have given you some of the materials you need: an Arduino, wires, and a few electronic components. However, you will need to incorporate your own materials. For example: LEDs; buzzers; photoresistors; housings; etc. You are expected to spend up to \pounds~40 on such components (as outlined in your welcome letter). However you are discouraged from spending above this. Doing so is not required to achieve a high mark! A \textit{Blue Peter} style prototype made from household items is fine, as is modelling clay. Choose your materials carefully, though, as overly flimsy construction may impair functionality. 

Strive to demonstrate a high level of sophistication in the technical execution of your prototype. Having the insight to choose the right tool for the job is critical: if simpler techniques fulfill the requirements, use them! Unnecessary complexity, serving to solely showcase one's own cleverness, is a bad habit. 

You are not assessed on your skills in photography! The sole purpose of making recordings and photos is archival: to aid moderators and external examiners. Photos and videos do not need not be highly polished. 

\section*{FAQ}

\begin{itemize}
	\item 	\textbf{What is the deadline for this assignment?} \\ 
    		Falmouth University policy states that deadlines must only be specified on the MyFalmouth system.
    		
	\item 	\textbf{What should I do to seek help?} \\ 
    		You can email your tutor for informal clarifications. For informal feedback, make a pull request on GitHub. 
    		
    	\item 	\textbf{Is this a mistake?} \\ 	
    		If you have discovered an issue with the brief itself, the source files are available at: \\
    		\url{https://github.com/Falmouth-Games-Academy/bsc-assignment-briefs}.\\
    		 Please raise an issue and comment accordingly.
\end{itemize}

\section*{Additional Resources}

\begin{itemize}
     \item Wilkinson, K. and Petrich, M. (2014) The Art of Tinkering: Meet 150 Markers Working at the Intersection of Art, Science \& Technology. Weldon Owen: London.
    \item Alicia Gibb. Building Open Source Hardware: DIY Manufacturing for Hackers and Makers. Addison Wesley, 2014. 
    \item Jeremy Blum. Exploring Arduino: Tools and Techniques for Engineering Wizardry. John Wiley, 2013. 
    \item Kelly, K. (2014) Cool Tools: A Catalogue of Possibilities. Cool Tools.
    \item \url{https://www.sitepoint.com/heuristic-evaluation-guide/}
    \item \url{https://www.usability.gov/how-to-and-tools/methods/heuristic-evaluation.html}
    \item \url{https://github.com/arduino/Arduino/blob/master/.gitignore}
     \item \url{https://gitignore.io/}
    
\end{itemize}

\begin{markingrubric}
    \firstcriterion{Basic Competency Threshold}{40\%}
        \gradespan{1}{\fail At least one part is missing or is unsatisfactory. 
        
        There is little or no evidence of an iterative development process and no improvement over time in regards to the quality of the design and build of the prototype.}
        \gradespan{5}{Submission is timely.
          \par A working, playable game is submitted.
          \par Video of an accompanying and appropriately designed game controller is submitted.
        	\par Enough work is available to hold a meaningful discussion.
	\par Clear evidence of programming knowledge and ability to discuss technical aspects of the project at viva.
	\par No breaches of academic integrity.}
%
    \criterion{Design of the Solution}{5\%}
        \grade\fail No evidence of upfront design and/or user stories, or significantly diverges from implementation.
        \grade There is a little correspondence between the design and the implementation, even if somewhat tenuous.
        \grade There is some correspondence between the design and the implementation.
        \grade Design is appropriate for most of the given requirements.
        \par     There is much correspondence between the user stories, design work, and implementation.
        \par     Proposed design is free, or mostly free, of critical flaws.
        \grade Design is aligned quite well with most, if not all, of the given requirements.
        \par     There is much, clear and explicit, correspondence between the user stories, design work, and implementation.
        \par     Proposed design is somewhat fit-for-purpose and free of critical flaws.
        \grade Design is comprehensively aligned to the stated requirements.
        \par     There is considerable, clear and explicit, correspondence between the user stories, design work, and implementation.
        \par     Proposed design is fit-for-purpose and free of major flaws.
%
    \criterion{Innovation \& Creative Flair}{15\%}
        \grade\fail No evidence of innovation and/or creativity.
        \grade Little evidence of emerging innovation and/or creativity.
            \par No evidence of divergent thinking.
        \grade Some evidence of emerging innovation and/or creativity.
            \par Solution goes beyond mere derivation.
        \grade Much evidence of emerging innovation and/or creativity.
            \par The solution is an interesting and somewhat innovative product.
            \par Some evidence of divergent thinking.
        \grade Considerable evidence of emerging innovative and/or creativity.
            \par The solution is a novel and innovative product.
            \par Much evidence of divergent thinking.
        \grade Significant evidence of emerging innovative and/or creativity.
            \par The solution is a unique and innovative product.
            \par Considerable evidence of divergent thinking.
            \par Target audience will potentially find the game fun to play.
%            
    \criterion{Functionality of Physical Prototype}{10\%}
        \grade\fail No physical prototype, or the prototype is not functional.
        \grade The physical prototype has a little functionality.
            \par There are critical technical and/or constructional flaws.
        \grade The physical prototype has some functionality.
            \par There are major technical and/or constructional flaws.
        \grade The physical prototype has much functionality.
            \par If any, only minor technical and/or constructional flaws.
        \grade The physical prototype has considerable functionality.
            \par If any, only minor technical and/or constructional flaws.
        \grade The physical prototype has significant functionality.
            \par If any, only superficial technical and/or constructional flaws.
%
    \criterion{Sophistication: Electronics}{5\%}
        \grade\fail Solution lacks even basic use of electronic components.
        \grade Solution has some sophistication in terms of electronics.
            \par Little insight insight into electronic circuits.
        \grade Solution has some sophistication  in terms of electronics.
            \par Some insight insight into electronic circuits.
        \grade Solution leverages electronic components with much effectiveness.
            \par Much insight insight into electronic circuits.
        \grade Solution leverages electronic components with much effectiveness.
            \par Considerable insight insight into electronic circuits.
        \grade Solution leverages electronic components with much effectiveness.
            \par Significant insight insight into electronic circuits.
%
    \criterion{Sophistication: Physical Form Factor}{10\%}
        \grade\fail No physical prototype, or it is limited to a breadboard without housing.
        \grade Physical form factor has a little sophistication.
            \par Little insight into human-computer interaction.
        \grade Physical form factor has some sophistication.
            \par Some insight into human-computer interaction.
        \grade Physical form factor has much sophistication and sturdiness.
            \par Much insight into human-computer interaction.
        \grade Physical form factor has much sophistication  and sturdiness.
            \par Considerable insight into human-computer interaction.
        \grade Physical form factor has considerable sophistication and sturdiness.
            \par Significant insight into human-computer interaction.
            \par Controller has both practical and aesthetic value.
%
    \criterion{Sophistication \& Maintainability: Software}{15\%}
        \grade\fail Solution lacks even basic levels of sophistication and/or maintainability.
        \grade Solution has a little sophistication in terms of programming constructs.
            \par Little insight insight into maintainable software design.
        \grade Solution has some sophistication in terms of programming constructs.
            \par Some insight insight into maintainable software design.
        \grade Solution leverages OO programming constructs with much sophistication.
            \par Much insight into maintainable software design.
        \grade Solution leverages OO programming constructs with much sophistication.
            \par Considerable insight has been demonstrated into maintainable software design.
        \grade Solution leverages OO programming constructs with considerable sophistication.
            \par Significant insight into maintainable software design.
\end{markingrubric}

\end{document}