\documentclass{../../..//fal_assignment}
\graphicspath{ {../../../} }

\usepackage{enumitem}
\setlist{nosep} % Make enumerate / itemize lists more closely spaced
\usepackage[T1]{fontenc} % http://tex.stackexchange.com/a/17858
\usepackage{url}
\usepackage{todonotes}

\title{Hacking Hardware - Referral 2}
\author{Brian McDonald}
\module{COMP140-GAM160}

\begin{document}

\maketitle

\section*{Introduction}

\begin{marginquote}
Hacker definition: ``A person who enjoys exploring the details of programmable systems and stretching their capabilities, as opposed to most users, who prefer to learn only the minimum necessary.''

--- Jargon File

\end{marginquote}
\marginpicture{flavour_pic}{
    Arduino is an open-source prototyping platform based on easy-to-use hardware and software.
}

In this assignment, you will work as an individual to develop a small game and then integrate an Ardunio to control some elements of the game (e.g. a button to toggle lights, a potentiometer to control strafing etc).

The game developed for this assignment could be the output from Referral Assignment 1 or a simple arcade game such as Breakout, Space Invaders etc.

\textbf{Please note that you can't use Pong for the game submission}

This assignment is formed of several parts:

\begin{enumerate}[label=(\Alph*)]
    \item \textbf{Implement}, a small prototype game of your choice, that will:
    	\begin{enumerate}[label=\roman*.]
    		\item \textbf{demonstrate} some gameplay
			\item \textbf{contains} elements that can be controlled by an Arduino
	\end{enumerate}
    \item \textbf{Implement} an Arduino controller:
    	\begin{enumerate}[label=\roman*.]
		\item \textbf{Reads} data from sensors
		    \item \textbf{Interfaces} with your game in some manner
	\end{enumerate}
\end{enumerate}

\subsection*{Assignment Setup}

Fork the GitHub repository at:

\indent \url{https://github.com/Falmouth-Games-Academy/comp140-gam160-game}

Use the existing directory structure and, as required, extend this structure with sub-directories. Ensure that you maintain the \texttt{readme.md} file.

Modify the \texttt{.gitignore} to the defaults for \textbf{Unity} or \textbf{Visual Studio/C++}. Please, also ensure that you add editor-specific files and folders to \texttt{.gitignore}.  


\subsection*{Coursework Submission}

Part B is the final submission of the code base to the learning space, this should be a zipped version of your GitHub repository that contains the following

\begin{itemize}
	\item Code for your game
	\item Code for your controller
	\item Images of the Arduino Controller
	\item 'Shakey cam' video footage of your Controller working with the game
\end{itemize}

The submission will be assessed on a threshold basis:

\begin{enumerate}[label=(\alph*)]
	\item Submission is timely;
	\item Game functions correctly
	\item Arduino interfaces with game 
	\item There is an attempt to write well engineered software
\end{enumerate}


\section*{Additional Guidance}
Falmouth University is internationally renowned for the arts. Despite the fact that you are studying for a BSc degree in a technical discipline, you are still expected to strive for the same level of innovation and creative flair as your peers. This assignment is more heavily weighted towards the creative than the assignments you have completed thus far. On this assignment, a competent execution of an unimaginative idea is unlikely to achieve a high grade, as opposed to an imperfect execution of a unique and ambitious concept, which will be seen favourably by examiners. Consider this when working on your design. One approach to promote creativity is divergent thinking: generate ideas by exploring many possible solutions. Often the most interesting ideas are subversive: they deliberately go against the accepted or most obvious solution.

The history of video games is littered with failed peripherals. They were perceived as expensive gimmicks rather than legitimate enhancements to gameplay. Your creativity should be balanced by commercial awareness: your design should be informed by research into products that have succeeded and failed in the past, and what underexploited niches exist in the present. A great project will be highly divergent, but one that has clear commercial viability. Do not be discouraged if you fall short: professionals find it difficult! 

We have given you some of the materials you need: an Arduino and other useful components. You will need to add your own materials to produce a functional physical prototype. A \textit{Blue Peter} style prototype made from household items is fine, as is something made out of modelling clay. However you should still choose your materials carefully, as overly flimsy construction may impair functionality. 

You may also wish to connect electronic components such as LEDs, buzzers, photoresistors etc to the Arduino, or even use a different, more flexible hard- ware platform such as RaspberryPi. However you are discouraged from spending large sums of money on extra hardware, and doing so is not required to achieve a high mark. If you choose to go down this route, it is possible to purchase a RaspberryPi and other useful peripheral online for around the price of a textbook (up to \pounds40). 

You should aim to demonstrate a high level of sophistication in the technical execution of your prototype. An important part of sophistication is having the insight to choose the right tool for the job: if a simpler technique fulfils all the requirements, use it. The use of unnecessarily complicated techniques, serving only to showcase one's own cleverness, is a dangerous habit. 

The sole purpose of the recorded demonstration is to aid the external moderators and examiners. Furthermore, any photos and/or videos submitted do not need to be entertaining or highly polished.

\section*{FAQ}

\begin{itemize}
	\item 	\textbf{What is the deadline for this assignment?} \\ 
    		Falmouth University policy states that deadlines must only be specified on the MyFalmouth system.
    		
	\item 	\textbf{What should I do to seek help?} \\ 
    		You can email your tutor for informal clarifications. For informal feedback, make a pull request on GitHub. 
    		
    	\item 	\textbf{Is this a mistake?} \\ 	
    		If you have discovered an issue with the brief itself, the source files are available at: \\
    		\url{https://github.com/Falmouth-Games-Academy/bsc-assignment-briefs}.\\
    		 Please raise an issue and comment accordingly.
\end{itemize}

\section*{Additional Resources}

\begin{itemize}
     \item Wilkinson, K. and Petrich, M. (2014) The Art of Tinkering: Meet 150 Markers Working at the Intersection of Art, Science \& Technology. Weldon Owen: London.
    \item Alicia Gibb. Building Open Source Hardware: DIY Manufacturing for Hackers and Makers. Addison Wesley, 2014. 
    \item Jeremy Blum. Exploring Arduino: Tools and Techniques for Engineering Wizardry. John Wiley, 2013. 
    \item Kelly, K. (2014) Cool Tools: A Catalogue of Possibilities. Cool Tools.
    \item Hatch, M. (2013) The Maker Movement Manifesto: Rules for Innovation in the New World of Creators, Hackers, and Tinkerers. McGraw Hill: New York.
    \item https://www.sitepoint.com/heuristic-evaluation-guide/
    \item https://www.usability.gov/how-to-and-tools/methods/heuristic-evaluation.html
    
\end{itemize}

\begin{markingrubric}
    \firstcriterion{Basic Competency Threshold}{40\%}
        \gradespan{1}{\fail At least one part is missing or is unsatisfactory. 
        
        There is little or no evidence of an iterative development process and no improvement over time in regards to the quality of the design and build of the prototype.}
        \gradespan{5}{Submission is timely.
	\par Clear evidence of a `reasonable' iterative development process
	\par Clear evidence of programming knowledge and communication skills.
	\par Clear evidence of reflection on own performance and contribution.
	\par No breaches of academic integrity.}
%
    \criterion{Design of the solution}{15\%}
        \grade\fail No evidence of upfront design
        \grade The correspondence between design and implementation is tenuous.
        \grade The design somewhat corresponds to the final implementation.
        \grade The design corresponds to the final implementation.
        \grade The design clearly corresponds to the implementation.
        \grade The design clearly and comprehensively corresponds to the implementation.
%
    \criterion{Innovation and creative flair}{10\%}
        \grade\fail No evidence of innovation and/or creativity.
        \grade Some evidence of emerging innovation and/or creativity.
            \par The solution is purely derivative of existing products.
            \par There is no evidence of divergent thinking.
        \grade Little evidence of emerging innovation and/or creativity.
            \par The solution is mostly derivative, with some attempts at innovation.
            \par There is evidence of an attempt at divergent thinking.
        \grade Much evidence of emerging innovation and/or creativity.
            \par The solution is an interesting and somewhat innovative product.
            \par There is some evidence of divergent thinking.
        \grade Considerable evidence of mastery of innovative and creative practice.
            \par The solution is a novel and innovative product.
            \par There is much evidence of divergent thinking.
        \grade Significant evidence of mastery of innovative and creative practice.
            \par The solution is a unique and innovative product.
            \par There is significant evidence of divergent thinking.
%            
    \criterion{Functionality of physical prototype}{15\%}
        \grade\fail A physical prototype is not produced, or the prototype is completely non-functional.
        \grade The physical prototype has no functionality.
            \par There are serious technical and/or constructional flaws.
        \grade The physical prototype has some functionality.
            \par There are obvious technical and/or constructional flaws.
        \grade The physical prototype has much functionality.
            \par There are minor technical and/or constructional flaws.
        \grade The physical prototype has considerable functionality.
            \par There are superficial technical and/or constructional flaws.
        \grade The physical prototype has significant functionality.
            \par The technical execution and physical construction are flawless.
%
    \criterion{Sophistication: \par Software \par Electronics \par Physical construction}{15\%}
        \grade\fail The solution lacks even a basic level of sophistication in any of the three areas.
        \grade The solution evidences some sophistication in one or more of the three areas.
            \par Some insight has been demonstrated in any area.
        \grade The solution evidences little sophistication in one or more of the three areas.
            \par Little insight has been demonstrated in at least one of the areas.
        \grade The solution evidences much sophistication in two or more of the three areas.
            \par Much insight has been demonstrated in at least one of the areas.
        \grade The solution evidences considerable sophistication in all three areas.
            \par Considerable insight has been demonstrated in at least two of these areas.
        \grade The solution evidences significant sophistication in all three areas..
            \par Significant insight has been demonstrated in all three areas.
%            
    \criterion{Use of Version Control}{5\%}
        \grade\fail GitHub has not been used.
        \grade Source code has rarely been checked into GitHub.
        \grade Source code  has been checked into GitHub at least once per week.
            \par Commit messages are present.
            \par There is evidence of engagement with peers (e.g.\ code review).
        \grade Source code  has been checked into GitHub several times per week.
            \par Commit messages are clear, concise and relevant.
            \par There is evidence of somewhat meaningful engagement with peers (e.g.\ code review).
        \grade Source code has been checked into GitHub several times per week.
            \par Commit messages are clear, concise and relevant.
            \par There is evidence of meaningful engagement with peers (e.g.\ code review).
        \grade Source code has been checked into GitHub several times per week.
            \par Commit messages are clear, concise and relevant.
            \par There is evidence of effective engagement with peers (e.g.\ code review).
%
\end{markingrubric}

\end{document}