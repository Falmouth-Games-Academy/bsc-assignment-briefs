\documentclass{../../fal_assignment}
\graphicspath{ {../../} }

\usepackage{enumitem}
\setlist{nosep} % Make enumerate / itemize lists more closely spaced
\usepackage[T1]{fontenc} % http://tex.stackexchange.com/a/17858
\usepackage{url}
\usepackage{todonotes}

\title{Development Report}
\author{Brian McDonald}
\module{COMP140}

\begin{document}

\maketitle

\section*{Introduction}

\begin{marginquote}
Hacker definition: ``A person who enjoys exploring the details of programmable systems and stretching their capabilities, as opposed to most users, who prefer to learn only the minimum necessary.''

--- Jargon File

\end{marginquote}
\marginpicture{flavour_pic}{
    Arduino is an open-source prototyping platform based on easy-to-use hardware and software.
}

In this assignment, you will write a report and poster. 

This report will be used to capture the development of your project from ideation, development and completion of the project. 
It will contain some elements of reflection, were you will evaluate the lessons you have learned 
about the development of the project. This should be written sing \textbf{LaTeX} format.

The poster will summarise the report, and will capture some of the key design decisions in the 
hardware and software side of things. This should be written in the provided \textbf{Powerpoint} template.
This assignment is formed of several parts:

\begin{enumerate}[label=(\Alph*)]
    \item \textbf{Write} a \textbf{1 page} proposal for a novel game/experience \textbf{and} controller, this is due in the \textbf{proposal review} sessions in \textbf{Week 3}. 
	\item \textbf{Write} a \textbf{draft poster} which provides an overview of your game and controller, this is due in the \textbf{Project Presentations} sessions in \textbf{Week 8}. 
	\item \textbf{Write} a draft \textbf{4 page} report which details the development of the project, this is due for the \textbf{Report Peer Review} in \textbf{Week 10}. 
	\item \textbf{Revise} the \textbf{4 page} report and poster based on feedback and submit on the  \textbf{hand-in} date
	\item \textbf{Present}, as an \textbf{individual} your final poster at the scheduled viva session
\end{enumerate}


	\subsection*{Assignment Setup}
	
	Fork the GitHub repository at:
	
	\indent \url{https://github.com/Falmouth-Games-Academy/comp140-Development-Report}
	
	Use the existing directory structure and, as required, extend this structure with sub-directories. Ensure that you maintain the \texttt{readme.md} file.
	

\subsection*{Part A}

Part A is a \textbf{single formative submission} of a \textbf{1 page} proposal for a novel game/experience \textbf{and} controller.  

    \begin{enumerate}[label=\roman*.]
		\item \textbf{describe} the game design that will form the basis for your interface;
		\item \textbf{illustrate} basic  research into electronic component and physical form factors for controllers;
		\item \textbf{analyse} the design of the controller in detail;
		\item \textbf{list} the key electronic components of your controller
		\item and \textbf{list} the key user stories.
	\end{enumerate}

To complete part A, write your proposal and then show this to a tutor at the \textbf{proposal review} sessions in \textbf{Week 3}

You will receive immediate \textbf{informal feedback} from your \textbf{tutor}.

\subsection*{Part B}

Part B is a \textbf{single formative submission} of a \textbf{draft digital poster}.

The poster must contain the following

\begin{enumerate}[label=\roman*.]
	\item Detail the hardware of the controller
	\item Detail the design of the controller
	\item Detail the elements of the game/experience
	\item UML Diagram of the software architecture
\end{enumerate}

To complete part B, write your draft poster and show this to a tutor at the \textbf{Project Presentation} session in \textbf{Week 8}

You will receive immediate \textbf{informal feedback} from your \textbf{tutor}.

\subsection*{Part C}

Part C is a \textbf{single formative submission}. Write a draft report which must include the following:

\begin{enumerate}[label=\roman*.]
	\item The project proposal
	\item Hardware of the controller, including components
	\item Design of the the controller, including wiring, casing and attachments
	\item The game/experience concept
	\item Software design with UML Diagrams
	\item Reflection on the project
\end{enumerate}

To completed part C, write your draft report and show this to your peers in the \textbf{Report Peer Review} session in \textbf{Week 10}. 

You will receive \textbf{immediate informal} feedback from your \textbf{peers}.

\subsection*{Part D}

Part D is a \textbf{single summative submission}. You should download your project from GitHub, and submit a \textbf{zip file} which contains the following

	\begin{enumerate}
	\item The final version of your report
	\item The final version of your poster
	\end{enumerate}

\subsection*{Part E}

	To complete Part E, please print out your poster and present it along with your alternative controller and game/ experience.

	You will receive \textbf{immediate informal} feedback from your \textbf{tutor}.

\section*{Additional Guidance}
It is important to keep up with the formative deadlines as receiving and then acting on the feedback 
will not only enable you to achieve a high grade, it will enable you to make a better project. 

When it comes to the poster, you should endevour to limit the amount of text, 
you always should favour images over text. If you use text, please ensure the 
font size is legible on the printed out poster.

\section*{FAQ}

\begin{itemize}
	\item 	\textbf{What is the deadline for this assignment?} \\ 
    		Falmouth University policy states that deadlines must only be specified on the MyFalmouth system.
    		
	\item 	\textbf{What should I do to seek help?} \\ 
    		You can email your tutor for informal clarifications. For informal feedback, make a pull request on GitHub. 
    		
    	\item 	\textbf{Is this a mistake?} \\ 	
    		If you have discovered an issue with the brief itself, the source files are available at: \\
    		\url{https://github.com/Falmouth-Games-Academy/bsc-assignment-briefs}.\\
    		 Please raise an issue and comment accordingly.
\end{itemize}

\section*{Additional Resources}

\begin{itemize}
     \item Wilkinson, K. and Petrich, M. (2014) The Art of Tinkering: Meet 150 Markers Working at the Intersection of Art, Science \& Technology. Weldon Owen: London.
    \item Alicia Gibb. Building Open Source Hardware: DIY Manufacturing for Hackers and Makers. Addison Wesley, 2014. 
    \item Jeremy Blum. Exploring Arduino: Tools and Techniques for Engineering Wizardry. John Wiley, 2013. 
    \item Kelly, K. (2014) Cool Tools: A Catalogue of Possibilities. Cool Tools.
    \item \url{https://www.sitepoint.com/heuristic-evaluation-guide/}
    \item \url{https://www.usability.gov/how-to-and-tools/methods/heuristic-evaluation.html}
    \item \url{https://github.com/arduino/Arduino/blob/master/.gitignore}
     \item \url{https://gitignore.io/}
\end{itemize}

\rubrichead{All submissions and assessment criteria for this assignment are individual.}
\begin{markingrubric}
	%
	\firstcriterion{Basic Competency Threshold}{30\%}
	\gradespan{1}{\fail At least one part is missing or is inadequate.}
	\gradespan{5}{Adequate ability to generate ideas, problem solving, concepts, technical competency and proposals in response to set briefs and/or self-initiated activity.
		\par The work demonstrates an adequate, ethically informed, real-world experience of industry/business environments and markets.
		\par Enough work is available to hold a meaningful discussion.
		\par Adequate participation in-class peer-review activities at least at the level of basic competency.
		\par Clear evidence of programming knowledge.
		\par A report and poster has been completed.
		\par No breaches of academic integrity.
	}
	%
	\criterion{ARCHITECT: \\UML Diagrams}{10\%}
	\grade\fail No UML Diagrams.
	\grade UML Diagrams are incomplete.
	\par Non-standard UML notation has been used.
	\par Incorrect use of diagram type.
	\grade UML Diagrams are mostly complete.
	\par The UML notation are mostly correct.
	\grade UML Diagrams are appropriate.
	\par The choice of diagrams are appropriate.
	\par There are only minor issues with the notation.
	\grade UML Diagrams is directly linked to the software architecture.
	\par There are only minor errors in the diagrams.
	\grade UML Diagrams is of a professional standard.
	\par UML notation is correct.
	\par There are no errors in the diagrams.
	%
	\criterion{ARCHITECT: \\Description of Software Architecture}{15\%}
	\grade\fail No description of the software architecture
	\par No attempt to describe the design of the software.
	\grade Little insight into the software architecture.
	\par There is a description of the key classes but no insight into the data structures or design patterns used.
	\grade Some insight into the software architecture.
	\par There is a description of the key classes.
	\par The data structures and design patterns have been described with no context
	\grade Much insight into the software architecture.
	\par There is a description of the key classes with reference to their functionality.
	\par The data structures and design patterns have been described with context to their application.
	\grade Considerable insight into the software architecture.
	\par There is a description of the key classes with reference to their functionality.
	\par The data structures and design patterns have been described with context to their application.
	\par There is some justification for the selection of data structures and design patterns.
	\grade Significant insight into the software architecture.
	\par There is significant evidence of software design
	\par There is synergy between data structure and design pattern selection.
	\par The selection of data structures and design patterns are well justified.
	%
	\criterion{ARCHITECT: \\Description of hardware design}{15\%}
	\grade\fail No description of the hardware design.
	\par No attempt to describe the design of the hardware, basically a list of components.
	\grade Little insight into the hardware design.
	\par There is little evidence of design, no regard to the selection of components and form factor.
	\grade Some insight into the hardware design.
	\par There is some evidence of deliberate practice from the student.
	\par The selection of components and form factor is justified.
	\grade Much insight into the hardware design.
	\par There is some evidence of deliberate practice from the student.
	\par The selection of components and hardware complement each other
	\grade Considerable insight into the hardware design.
	\par There is considerable evidence of deliberate practice from the student.
	\par The selection of components and hardware complement is justified.
	\par The controller design is linked to the digital project
	\grade Significant insight into the hardware design.
	\par There is significant evidence of deliberate practice from the student.
	\par The selection of hardware is complemented and justified in terms of the game.
	%
	\criterion{ANALYSIS: \\Reflection}{15\%}
	\grade\fail 	No reflection.
	\grade 		Very limited reflection.
	\par Very few areas considered and/or no depth of consideration.
	\grade 		Generally broad analysis.  
	\par limited reflection across areas of importance.
	\par Missing areas of true significance. 
	\grade 		Reasonably clear and insightful analysis.
	\par Reflection tends to be broad rather than narrow.
	\par Missing some appropriate depth. 
	\grade 		Clear and insightful analysis and reflections.
	\par Reflection is narrow and deep rather than trying to address all issues.
	\grade 		Reflection is extremely insightful.
	%
	\criterion{ANALYSIS: \\Appropriateness of Spelling \& Grammar}{5\%}
	\grade\fail 	Substantial spelling and/or grammar errors.
	\grade 		Many spelling and/or grammar errors.
	\grade 		Some spelling and/or grammar errors.  
	\grade 		Few spelling and/or grammar errors.
	\grade 		Almost no spelling and/or grammar errors.
	\grade 		No spelling or grammar errors.
	%
	\criterion{ANALYSIS: \\Appropriateness of Report Structure}{5\%}
    \grade\fail 	There is no structure, or the structure is unclear.
	\grade 		There is little structure.
	\grade 		There is some structure.
	\par 		A few sentences and paragraphs are well constructed.
	\grade 		There is much structure.
	\par 		Some sentences and paragraphs are well constructed.
	\par 		There is a clear introduction and conclusion.
	\grade 		There is much structure, highlighting the argument.
	\par 		Most sentences and paragraphs are well constructed.
	\par 		There is a clear and well-constructed introduction and conclusion.
	\grade 		There is much structure, highlighting the argument.
	\par 		All sentences and paragraphs are well constructed.
	\par 		There is a clear and well-constructed introduction and conclusion.
	%
	\criterion{ANALYSIS: \\Appropriateness of Poster Structure}{5\%}
	\grade\fail 	There is no structure, or the structure is unclear.
	\grade 		There is little structure.
	\grade 		There is some structure.
	\par 		The balance is towards text rather than images.
	\par		The text is difficult to read on the poster.
	\grade 		There is much structure.
	\par 		There is a good balance of images and text in the poster.
	\grade 		There is much structure.
	\par 		With appropriate images and text.
	\grade 		There is much structure.
	\par 		The poster could be published at a conference.
	%
\end{markingrubric}
\end{document}