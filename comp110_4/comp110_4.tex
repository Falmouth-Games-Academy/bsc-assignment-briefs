\documentclass{../fal_assignment}
\graphicspath{ {../} }

\usepackage{enumitem}
\usepackage[T1]{fontenc} % http://tex.stackexchange.com/a/17858
\usepackage{url}
\usepackage{todonotes}

\title{Coding Task II: Game Component}
\author{Dr Michael Scott}

\begin{document}

\maketitle
\begin{marginquote}
    ``Engage with the community and support each other. This is important.
    Upload your code to GitHub and receive feedback from experienced peers.
    Review your peers' work yourself and really consider what `quality' actually means.
    Debate, argue, and question others about it---
    an open and sustained discourse is an excellent way for all to learn!''
\end{marginquote}
\marginpicture{MakeyMakey.jpg}{
    The \emph{MaKey~MaKey} allows a multitude of materials to be used to create videogame controllers.
}
\section*{Introduction}

In this assignment, you are required to propose and write a game component for an existing game. You will then present a demonstration of your work.

Games are often comprised of a rich architecture incorporating many components. As such, coding tasks in the games industry may require you to develop bespoke code on a particular aspect of a game. Examples include: level generators; event triggers; conflict mechanics; character behaviours; physics; and so on. Through this project, you will become acquainted with techniques and methods that help you to work effectively to write such components; thereby, gaining knowledge of coding and its various aspects.

This assignment is formed of several parts:

\begin{enumerate}[label=(\alph*)]
    \item \textbf{Write} a brief proposal that will:
    	\begin{enumerate}[label=\roman*.]
    		\item \textbf{identify} a component of game architecture;
    		\item \textbf{describe} what will be created;
    		\item \textbf{illustrate} the key user stories;
    		\item and \textbf{identify} the existing game the component will integrate into;
	\end{enumerate}
    \item \textbf{Write} a draft computer program that will:
        	\begin{enumerate}[label=\roman*.]
    		\item \textbf{implement} the game component;
    		\item and \textbf{address} the requirements highlighted in the proposal;
	\end{enumerate}
    \item \textbf{Write} a final computer program that will:
    	\begin{enumerate}[label=\roman*.]
    		\item \textbf{revise} any issues raised by your tutor or your peers;
	\end{enumerate}
    \item \textbf{Present} an executable version of your final program.
\end{enumerate}

\todo[inline]{\textbf{Note:} All submissions must be clearly distinctive. Members of the same development group must \textbf{not} target the same component.}

\subsection*{Part A}

Part A consists of a \textbf{single formative submission}. This work is \textbf{individual} and will be assessed on a \textbf{threshold} basis. The following criteria are used to determine a pass or fail:

\begin{enumerate}[label=(\alph*)]
	\item Submission is timely;
	\item Proposed game component is relevant to the existing game;
	\item Coding task is non-trivial and achievable.
\end{enumerate}

To complete Part A, prepare a short \texttt{readme.md} document and a Trello board. Show your proposal to your tutor.  If acceptable, this will be signed-off. 

You will receive immediate informal feedback.

\subsection*{Part B}

Part B is a \textbf{single formative submission}. This work is \textbf{individual} and will be assessed on a \textbf{threshold} basis. The following criteria are used to determine a pass or fail:

\begin{enumerate}[label=(\alph*)]
	\item Submission is timely;
	\item Enough work is available to conduct a meaningful review;
	\item A broadly appropriate review of a peer's work is submitted.
\end{enumerate}

To complete Part B, prepare a draft version of your source code. Use the programming language and IDE of your choice. Then, attend the scheduled code review session. Ensure that the source code and related assets are pushed to GitHub and a pull request is made prior to this session. 

You will receive peer feedback within 3 working days.

\subsection*{Part C}

Part C is a \textbf{single summative submission}. This work is \textbf{individual} and will be assessed on a \textbf{criterion-referenced} basis. The following criteria are used to allocate marks:

\begin{enumerate}[label=(\alph*)]
	\item Appropriateness of Requirements;
	\item Appropriateness of Design;
	\item Functional Coherence of Executable Solution;
	\item Maintainability of Source Code;
	\item Sophistication of Source Code;
	\item Professional Practice;
\end{enumerate}

To complete Part C, revise the source code based on the feedback you have received. Then, upload the source code and related assets to the LearningSpace. Please note, the LearningSpace will only accept a single \texttt{.zip} file (the \texttt{.rar} format must not be used).

You will receive formal feedback three weeks after the final deadline.

\subsection*{Part D}

Part D is \textbf{not grade-bearing}; however, \textbf{participation is mandatory}. Failure to attend will result in a grade capped at 40\% (D-).

To complete Part D, attend the scheduled demo session. Ensure that a working executable demo of your work has been pushed into GitHub prior to this session. You will receive informal feedback immediately.

\begin{marginquote}
    ``Remember, learning to program can take a surprising amount of time \& effort --- students may get there at different rates, but all students who put in the time \& effort get there eventually. Making good use of [reflection and deliberate practice] are an essential part of this process.''
    
    --- Professor Quintin Cutts
\end{marginquote}
\marginpicture{guitarhero}{
    Rhythm games such as \emph{Guitar Hero} and \emph{Rock Band} are excellent examples of games
    which make use of unique input devices to enhance gameplay.
}

\newpage

\section*{Additional Guidance}

Fork the GitHub repository at the following URL:

\indent \url{https://github.com/Falmouth-Games-Academy/comp110-coding-task-2}

Write your proposal in the \texttt{readme.md} file within this repository. Also use this repository for any other digital assets you create (e.g. diagrams and pseudocode), checking them in regularly as you work on your projects. Such assets should be embedded directly in the \texttt{readme.md} file.

Do not begin programming the game component until your tutor has reviewed your proposal. It is important that the main requirements are firmly specified in order to avoid over-scoping the task or otherwise falling into a related pitfall.

You may use the IDE and programming language of your choice. Remember to commit frequently and push your source code and related assets to the GitHub repository.

Poor planning and poor time management can have a significant impact on this assignment. It is very easy to underestimate how much work is involved in first learning programming concepts and then actually applying them in order to write a computer program. As some of you may have already discovered, programming is quite unlike other subjects in that it cannot be ``crammed''. Sustain a steady pace across the duration of the course. Do a little programming every day, if you can!

\section*{FAQ}

\begin{itemize}
	\item 	\textbf{What is the deadline for this assignment?} \\ 
    		Falmouth University policy states that deadlines must only be specified on LearningSpace. Please examine the assignment area where you located this document.
    		
    	\item 	\textbf{How can I create a \texttt{.zip} of my source code?} \\ 
    		See: \url{http://i.stack.imgur.com/NhsQe.png}. 
    		
	\item 	\textbf{What should I do to seek help?} \\ 
    		You can email your tutor for informal clarifications. If you have discovered an issue with the brief itself, the source files are available at: 
    		\url{https://github.com/Falmouth-Games-Academy/bsc-assignment-briefs}. Please make a pull request and comment accordingly.
\end{itemize}

\section*{Additional Resources}

\begin{itemize}
    \item To be advised by your tutor after submission of the proposal.
\end{itemize}

\begin{marginquote}
    ``The first 90 percent of the code accounts for the first 90 percent of the development time.
    
    ``The remaining 10 percent of the code accounts for the other 90 percent of the development time.''
    
    --- Tom Cargill
    
    \marginquoterule
    
    ``Hofstadter's Law:
    
    ``It always takes longer than you expect, even when you take into account Hofstadter's Law.''
    
    --- Douglas Hofstadter
\end{marginquote}
\marginpicture{fishing}{
    The \emph{Dreamcast Fishing Controller}, released as a peripheral for the game \emph{Sega Bass Fishing}.
    Even peripherals which appeal to only a small audience can enjoy moderate commercial success.
}

\begin{markingrubric}
%
    \firstcriterion{Satisfactory Preparation of Proposal}{5\%}
        \gradespan{5}{\fail The proposal is inappropriate and/or is late.}
        \grade 		The proposal has been signed-off by your tutor by the deadline.
%
    \criterion{Satisfactory Completion of Peer-Review Tasks}{5\%}
        \gradespan{5}{\fail No work was submitted for peer-review and/or no peer-review has been submitted and/or either is late.}
        \grade 		Work submitted for peer-review on time and reviews of peers' work submitted on time.
%
    \criterion{Appropriateness of Requirements}{5\%}
        \grade\fail 	No user stories are provided.
        \grade 		Few user stories are appropriately formatted, distinguishable, and easily measured.
        \grade 		Some user stories are appropriately formatted, distinguishable, and easily measured.
        \grade 		Most user stories are appropriately formatted, distinguishable, and easily measured.
        \par 		The scope and relevance of all requirements is appropriate.
        \grade 		Nearly all user stories are appropriately formatted, distinguishable, and easily measured.
        \par 		The scope and relevance of all requirements is appropriate.
        \grade 		All user stories are appropriately formatted, distinguishable, and easily measured.
        \par 		The scope and relevance of all requirements is appropriate.
%
    \criterion{Appropriateness of Design}{10\%}
        \grade\fail 	No design is presented.
        \grade 		The design is very flawed and/or very poorly described.
        \grade 		The design is flawed and/or poorly described.
        \grade 		The design is acceptable and adequately described.
        \grade 		The design is sound and well described.
        \grade 		The design is exceptional and very well described.
%
    \criterion{Functional Coherence}{15\%}
        \grade\fail 	The component is non-functional.
        \grade 		Few requirements have been met.
        \par 		There are many obvious bugs.
        \grade 		Some requirements have been met.
        \par 		There are some obvious bugs.
        \grade 		Many requirements have been met.
        \par 		There are few obvious bugs.
        \grade 		The game component is fit-for-purpose.
        \par 		There are almost no obvious bugs.
        \grade 		The game component is fit-for-purpose.
        \par 		There are no obvious bugs.
%
    \criterion{Sophistication}{25\%}
        \grade\fail 	No insight into the appropriate use of programming constructs is evident from the source code.
        \grade 		Little insight into the appropriate use of programming constructs is evident from the source code.
        \grade 		Some insight into the appropriate use of programming constructs is evident from the source code.
        \grade 		Much insight into the appropriate use of programming constructs is evident from the source code.
        \par 		The program is structured appropriately.
        \grade 		Significant insight into the appropriate use of programming constructs is evident from the source code.
        \par 		The program is structured effectively, such that there is high cohesion and low coupling.
        \grade 		Exemplary insight into the appropriate use of programming constructs is evident from the source code.
        \par 		The program is structured very effectively, such that there is very high cohesion and very low coupling.
%
    \criterion{Maintainability}{25\%}
        \grade\fail 	The source code cannot be maintained.
        \grade 		There are many problems which affect the maintainability of the source code.
        \grade 		There are some problems which affect the maintainability of the source code. 
        \par 		Some clear and appropriate comments are present.
        \grade 		There are few problems which affect the maintainability of the source code. 
        \par 		Many clear and appropriate comments are present.
        \grade 		There are almost no problems which affect the maintainability of the source code. 
        \par 		Source code is well commented.
        \par 		Doc strings (or equivalent) are provided.       
        \grade 		There are no problems which affect the maintainability of the source code.
        \par 		Source code is exceptionally well commented.
        \par 		Appropriate doc strings (or equivalent) are provided. 
%
    \criterion{Professional Practice}{10\%}
        \grade\fail 	GitHub has not been used.
        \grade 		Source code and assets have been checked into the repository only just before a deadline.
        \grade 		Source code and assets have seldom been checked into the repository.
        \grade 		Source code and assets have regularly been checked into the repository.
        \par 		An attempt has been made to document the project using \texttt{readme.md} and \texttt{changelog.md}.
        \grade 		Source code and assets have regularly been checked into the repository.
        \par  		The first check-in to the repository is in the first half of the semester.
        \par 		The project is appropriately documented using \texttt{readme.md} and \texttt{changelog.md}.
        \par 		There is evidence of some engagement with the Falmouth Games Academy community (e.g. reviewing peers' pull requests).
        \grade 		Source code and assets have regularly been checked into the repository.
        \par  		The first check-in to the repository is in the first quarter of the semester.
        \par 		The project is exemplary documented using \texttt{readme.md} and \texttt{changelog.md}.
        \par 		There is evidence of much engagement with the Falmouth Games Academy community (e.g. reviewing peers' pull requests).
\end{markingrubric}

\end{document}