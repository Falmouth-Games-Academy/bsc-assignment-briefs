\documentclass{../../fal_assignment}
\graphicspath{ {../../} }

\usepackage{enumitem}
\setlist{nosep} % Make enumerate / itemize lists more closely spaced
\usepackage[T1]{fontenc} % http://tex.stackexchange.com/a/17858
\usepackage[hyphens]{url}
\usepackage{todonotes}

\usepackage{xcolor}

\title{Individual Specialist Computing Project - Technical Report}
\author{Dr Michael Scott}
\module{COMP2**}
\version{0.1}

\newcommand\ytl[2]{
\parbox[b]{8em}{\hfill{\color{cyan}\bfseries\sffamily #1}~$\cdots\cdots$~}\makebox[0pt][c]{$\bullet$}\vrule\quad \parbox[c]{4.5cm}{\vspace{7pt}\color{red!40!black!80}\raggedright\sffamily #2.\\[7pt]}\\[-3pt]}

\begin{document}

\maketitle

\section*{Introduction}

\begin{marginquote}
``A good portfolio is essential for getting your first role in the games industry, and not just for artists! You will be expected to show examples of your work when applying for roles at a games studio, whether you’re a game artist, programmer, designer, or any other dev role!'\thinspace''

--- Aardvark Swift
\end{marginquote}
\marginpicture{flavour_pic_1}{
    Poster demonstrations are a common way of presenting computing projects and will help
    you to develop communication skills which are useful in interview contexts, especially when
    showcasing your porfolio work to potential employers.
}

For this assignment, you will create and present both a poster and web page in which you report the outcome of practice-based research related to the technical architecture of your individual specialist computing project.

Whether you are going to apply for a role in the industry or for a role further afield, employers will want to know what you can do and what value will you bring
to their company. People care more about concrete practical examples of your work experience and your skills, and so it is important to develop a comfort with demonstrating
and discussing technical projects. Hence, this `technical report' assignment is framed around how you communicate a potential portfolio piece to an employer.
 
There are \textbf{TWO} key learning outcomes which you need to address in your technical report. These are:

\begin{itemize}
	\setlength{\itemindent}{1em}
	\item[LO3. ] \textbf{Integrate} appropriate data structures and interoperating components into software, with reference to their merits and flaws.
	\item[LO5. ] \textbf{Develop} an argument on a topic using appropriate research methods, primary and secondary sources, and academic conventions.
\end{itemize}

As such, there needs to be an element of reflection in your technical report where you analyse the advantages and disadvantages of the shape of 
your computing artefact. There also needs to be an element of research in which you develop an \textit{argument} which you support through appropriate methods. By argument,
we mean ``an evidence-based defense of a non-obvious position on a complex issue''. This is \textbf{not} a traditional academic essay, and so please do not take a formal approach and do not restrict yourself to academic sources. Though you need to follow academic conventions, it is not a literature review! Citing the grey literature is not only acceptable but is encouraged. The best way of achieving both of these learning outcomes is through a process of practice-based research. This means we anticipate that you conduct ``an original investigation in order to gain new knowledge by means of practice and the outcomes of that practice''. Typically, tinkering with your artefact, experimenting with ways of doing things, and comparing outcomes to see what is effective.

This assignment is directly linked to the `computing artefact' assignment in the specialist module that you have selected. Accordingly, the assignment is formed of these parts:

\begin{enumerate}[label=(\Alph*)]
	\item \textbf{Prepare} a brief proposal which will:
		\begin{enumerate}[label=(\roman*)]
			\item \textbf{outline} the computing artefact you intend to create;
			\item \textbf{align} the computing artefact with the specialism you have chosen and its structure of worksheet activities;
			\item \textbf{identify} the broader context and potential application of your computing artefact;
			\item \textbf{describe} the work required;
			\item and \textbf{justify} that the computing artefact is feasible in scope.
		\end{enumerate}
	\item \textbf{Implement} a draft of your computing artefact, ensuring that you:
		\begin{enumerate}[label=(\roman*)]
			\item \textbf{refer} to the assignment brief that details requirements for your specific module (i.e., choice of specialism);
			\item and \textbf{conduct} some form of practice-based research as you implement the artefact, which you could report on.
		\end{enumerate}
	\item \textbf{Prepare} a poster that will:
		\begin{enumerate}[label=(\roman*)]
			\item \textbf{describe} your computing artefact;
			\item \textbf{illustrate} the architecture of your computing artefact;
			\item and \textbf{justify} key development choices and design decisions.
		\end{enumerate}
	\item \textbf{Present}, as an individual, the poster:
		\begin{enumerate}[label=(\roman*)]
			\item \textbf{articulating} your key design decisions \textbf{and} approaches to developing your computing artefact;
			\item \textbf{analysing} the merits and flaws of these decisions and approaches;
			\item as well as \textbf{explaining} your practice-based research.
		\end{enumerate}
	\item \textbf{Prepare} a report as a web page that will:
		\begin{enumerate}[label=(\roman*)]
			\item \textbf{describe} your computing artefact;
			\item \textbf{illustrate} the architecture of your computing artefact;
			\item \textbf{embed} a video of your computing artefact;
			\item \textbf{communicate} your practice-based research;
			\item and \textbf{defend} an argument related to the outcome of this research.
		\end{enumerate}		
	\item \textbf{Present}, as an individual, the web page to your tutor, ensuring that you:
		\begin{enumerate}[label=(\roman*)]
			\item \textbf{demonstrate} your academic integrity;
			\item as well as \textbf{showcase} your knowledge of system architecture/research.
		\end{enumerate}			
\end{enumerate}

You should discuss what constitutes appropriate practice-based research with your specialist tutor. An example of practice-based research that you could report on is comparing different solutions and techniques. See:

\url{https://snowhydra.wordpress.com/2015/06/01/unity-performance-testing-getcomponent-fields-tags/} 

\url{https://www.gamasutra.com/blogs/JakobRasmussen/20160427/271188/Are_Behavior_Trees_a_Thing_of_the_Past.php} 

This will vary by area of specialism. You will meet with your tutor every two weeks to support this aspect of your individual specialist project.

\subsection*{Assignment Setup}

This assignment is a \textbf{reporting} task. There is no template repository for this assignment; create a new repository for your work. It should contain source to produce a \textbf{.pdf} for your poster and \textbf{.html} documents for your web page; eventually to submit together as a single \textbf{.zip}.
Remember to modify the \texttt{.gitignore}.

\subsection*{Part A}

Part A consists of a \textbf{single formative submission}. This work is \textbf{individual} and will be assessed on a \textbf{threshold} basis. Answer the following questions to pass:

\begin{itemize}
	\item What is the high concept of your computing artefact?
	\item What functionality will your component include?
	\item How does your component fit into your chosen specialism?
	\item Why is this artefact needed?
	\item What are the key requirements?
	\item Is the scope appropriate for the product development time-frame?
	\item How will you address the architect and research requirement?
\end{itemize}

To complete Part A, prepare the handout using any word processing tool.
Your handout may include images and/or links to online videos.

Show the handout to the \textbf{tutor} in the timetabled proposal review session in Week 2 for immediate \textbf{informal feedback}.

\subsection*{Part B}

Part B is a \textbf{single formative submission}. This work is \textbf{individual} and will be assessed on a \textbf{threshold} basis. The following criteria are used to determine a pass or fail:

\begin{enumerate}[label=(\alph*)]
	\item Submission is timely;
	\item The computing artefact fits within the scope of the chosen specialism;
	\item The computing artefact is appropriate according to the constraints in the assignment brieffor the chosen specialism;
	\item There is sufficient original computing content to be developed to deliver proposed computing artefact (i.e., it is non-trivial);
	\item The proposed computing artefact can serve as a platform for practice-based research;
	\item Enough work is available to conduct a meaningful review.
\end{enumerate}

To complete Part B, prepare a draft version of your computing component. Ensure that you refer to the specific assignment brief for your choice of module.
Bring the artefact to the supervision session in Week 4 and the mid-term review session in Week 6. Ensure that you use version control for your work, but in the event of connectivity problems, a backup on a USB storage device is advisable.

Include a \texttt{readme.md} file detailing which parts of the project constitute your component for this assignment.

You will receive immediate \textbf{informal feedback} from your \textbf{tutor}.

\subsection*{Part C}

Part C is a \textbf{single formative submission}. This work is \textbf{individual} and will be assessed on a \textbf{threshold} basis.  The following criteria are used to determine a pass or fail:

\begin{enumerate}[label=(\alph*)]
	\item Submission is timely;
	\item The poster follows the required template  (i.e., is a portrait A3 document with all specified sections);
	\item There is identifying information on the poster (i.e., your name and the project title);
	\item There is at least one UML diagram;
	\item The poster clearly illustrates the key system components, patterns, and/or data structures;
	\item Enough work is available to conduct a meaningful review.
\end{enumerate}

To complete Part C, produce an A3 portrait poster. You may use any software to prepare the poster, but you should follow the layout in the provided \textbf{Powerpoint} template.  Bring the digital version of the poster to the supervision session in Week 8.

You will receive immediate \textbf{informal feedback} from your \textbf{tutor}.

\subsection*{Part D}

Part D is a \textbf{single summative submission}. This work is \textbf{individual} and will be assessed on a \textbf{criterion-referenced} basis.  Please refer to the marking rubric at the end of this document for further detail.

Update the poster based on the feedback that you received in Part C. Then, print the poster on white A3 paper. Bring a physical copy of the poster to the poster demonstration session scheduled in Week 8.

You will receive immediate \textbf{informal feedback} from your \textbf{tutor}.

\subsection*{Part E}

Part E is a \textbf{single formative submission}. This work is \textbf{individual} and will be assessed on a \textbf{threshold} basis.  The following criteria are used to determine a pass or fail:

\begin{enumerate}[label=(\alph*)]
	\item Submission is timely;
	\item The web page is suitable for submission in \texttt{.html} format (i.e., custom source, or using a tool like SiteSucker);
	\item Enough work is available to conduct a meaningful review.
\end{enumerate}

To complete Part E, produce a web page containing your report. You may use any framework to prepare the website, for example WordPress, but you should follow the layout suggested in the portfolio development workshop sessions. It is likely that the technical report represents part of the portfolio website that you have created.  Bring the digital version of the website to the peer review session in Week 9 and the supervision session in Week 10.

You will receive immediate \textbf{informal feedback} from your \textbf{tutor}.

\subsection*{Part F}

Part F is a \textbf{single summative submission}. This work is \textbf{individual} and will be assessed on a \textbf{criterion-referenced} basis.  Please refer to the marking rubric at the end of this document for further detail.

Update the website based on the feedback that you received in Part E. Upload to \texttt{LearningSpace} by the time/date specified in the list of deadlines available on \url{myfalmouth.falmouth.ac.uk} (typically, shortly following the supervision meetings in Week 10) and ensure that you include your website, poster in your submission to the `technical report' submission queue. Please note that you should upload the video of your artefact to the appropriate submission queue for the `computing artefact` assignment. 

Then, attend the timetabled viva session.

You will receive \textbf{formal feedback} from your \textbf{tutor} three weeks after the viva.

\section*{Additional Guidance}

It is important to keep up with the formative deadlines as receiving and then
acting on the feedback will not only enable you to achieve a high grade, it
will enable you to make a better project.

When it comes to the poster, you should endavour to limit the amount of text,
you always should favour images over text. If you use text, please ensure the
font size is legible on the printed out poster.

\section*{FAQ}

\begin{itemize}
	\item 	\textbf{What is the deadline for this assignment?} \\ 
    		Falmouth University policy states that deadlines must only be specified on the MyFalmouth system.
    		
	\item 	\textbf{What should I do to seek help?} \\ 
    		You can email your tutor for informal clarifications. For informal feedback, make a pull request on BitBucket. 
    		
    	\item 	\textbf{Is this a mistake?} \\ 	
    		If you have discovered an issue with the brief itself, the source files are available at: \\
    		\url{https://github.com/Falmouth-Games-Academy/bsc-assignment-briefs}.\\
    		 Please make a pull request and comment accordingly.
\end{itemize}

\begin{table}[!h]
\caption{Indicative Assignment Timeline}
\centering
\begin{minipage}[t]{.7\linewidth}
\color{gray}
\rule{\linewidth}{1pt}
\ytl{Week 2}{Show Proposal to Supervisor (Part A)}
\ytl{Week 4}{Show Computing Artefact Work-in-Progress to Supervisor (Part B)}
\ytl{Week 6}{Show Computing Artefact Work-in-Progress to Supervisor (Part B)}
\ytl{Week 8}{Show Draft Poster to Supervisor (Part C)}
\ytl{Week 8}{Present Poster to Peers (Part D)}
\ytl{Week 9}{Peer Review Web Page (Part E)}
\ytl{Week 10}{Show Web Page to Supervisor (Part E)}
\ytl{Week 10}{Submit Poster and Web Page to LearningSpace (Part F)}
\ytl{Week 13}{Present Web Page at Viva (Part F)}
\bigskip
\rule{\linewidth}{1pt}%
\end{minipage}%
\end{table}

\rubricyeartwo
\rubrichead{This rubric applies to the assessment of the \textbf{poster} in Part D.}
\begin{markingrubric}
	\firstcriterion{Completion of Poster}{15\%}
		\grade\fail At least one part is missing or is unsatisfactory.
		\gradespan{5}{Submission is timely and satisfactory.
			\par Enough work is available to hold a meaningful discussion.
			\par No breaches of academic integrity.
			\par The student demonstrates adequate research and information skills.
			\par The student demonstrates adequate ability to critically engage with and analyse information and formulate reasoned arguments. }
%
    \criterion{ARCHITECT: Description of Software Architecture}{10\%}
	\grade\fail No description of the software architecture
	\par No attempt to describe the design of the software.
	\grade Little insight into the software architecture.
	\par There is a description of the key classes but no insight into the data structures or design patterns used.
	\grade Some insight into the software architecture.
	\par There is a description of the key classes.
	\par The data structures and design patterns have been described with no context
	\grade Much insight into the software architecture.
	\par There is a description of the key classes with reference to their functionality.
	\par The data structures and design patterns have been described with context to their application.
	\grade Considerable insight into the software architecture.
	\par There is a description of the key classes with reference to their functionality.
	\par The data structures and design patterns have been described with context to their application.
	\par There is some justification for the selection of data structures and design patterns.
	\grade Significant insight into the software architecture.
	\par There is significant evidence of software design
	\par There is synergy between data structure and design pattern selection.
	\par The selection of data structures and design patterns are well justified.
%
    \criterion{ARCHITECT: UML Diagram}{15\%}
	\grade\fail No UML Diagrams.
	\grade UML Diagrams are incomplete.
	\par Non-standard UML notation has been used.
	\par Incorrect use of diagram type.
	\grade UML Diagrams are mostly complete.
	\par The UML notation are mostly correct.
	\grade UML Diagrams are appropriate.
	\par The choice of diagrams are appropriate.
	\par There are only minor issues with the notation.
	\grade UML Diagrams are directly linked to the software architecture.
	\par There are only minor errors in the diagrams.
	\grade UML Diagrams are of a professional standard.
	\par All notation is correct.
	\par There are no errors in the diagrams.
%
    \criterion{RESEARCH: Context for Practice-Based Research}{10\%}
        \grade\fail No proposed context for doing practice-based research.
        \grade The proposed context for doing practice-based research is broadly appropriate.
        \grade The proposed context for doing practice-based research is appropriate.
        \par The proposed research context strives to go beyond analysis of the artefact.
        \grade The proposed context for doing practice-based research is appropriate, and leverages the computing artefact well.
        \par The proposed research context strives to go beyond analysis of the computing artefact and/or benchmarking.
        \grade The proposed context for doing practice-based research is sound, and is well suited to the computing artefact being developed.
        \par The proposed research context is somewhat ambitious.
        \grade The proposed context for doing practice-based research is sound, and the computing artefact being developed served as an ideal platform for its timely completion. 
        \par The proposed research context is ambitious.
\end{markingrubric}

\rubricyeartwo
\rubrichead{This rubric applies to the assessment of the \textbf{web page} in Part F.}
\begin{markingrubric}
	\firstcriterion{Completion of \\Web Page}{15\%}
		\grade\fail At least one part is missing or is unsatisfactory.
		\gradespan{5}{Submission of all parts is timely and satisfactory.
			\par Enough work is available to hold a meaningful discussion.
			\par No breaches of academic integrity.
			\par The student demonstrates adequate research and information skills.
			\par The student demonstrates adequate ability to critically engage with and analyse information and formulate reasoned arguments. }
%
    \criterion{ARCHITECT: Reflection on Computing Artefact}{10\%}
	\grade\fail 	There is no reflection on the merits and flaws of the use of different data structures and interoperating system components in the computing artefact.
	\grade There is limited reflection on the merits and flaws of data structures, patterns, and interoperating components.
	\par No depth of consideration.
	\grade There is a general, broad, and high-level reflection on the merits and flaws of different data structures, patterns, and/or interoperating system components.
	\par Some depth of consideration, demonstrating a little analysis.
	\par Little to no generalisation to a significant wider issue. 
	\grade There is a focused reflection on the merits and flaws of the use of different data structures, patterns, and/or interoperating system components applied to the design of the computing artefact.
	\par Much depth of consideration, demonstrating some critical thinking and analytical skills.
	\par Allusions to significant issues of interest to other developers, suggesting the findings can be generalised. 
	\grade There is a narrow and focused reflection on the merits and flaws of the use of different data structures, patterns, and interoperating system components applied to the design of the computing artefact.
	\par Considerable depth of reflection, demonstrating critical thinking and evaluation.
	\par Links are made to significant issues of interest to other developers, showing how findings can be generalised. 
	\grade There is a very narrow and highly pertinent reflection on the merits and flaws of the use of different data structures, patterns, and interoperating system components applied to the design of the computing artefact, motivated by wider issues in the field.
	\par Significant depth of reflection, demonstrating strong critical thinking and evaluation.
	\par Clear links are made to significant issues of interest to other developers, showing how findings can be broadly generalised. 
%
    \criterion{RESEARCH: \\Defense of Argument}{10\%}
        \grade\fail There is no argument, or it is not defended.
        \grade A claim is made on an issue.
            \par There is a little evidence to support the argument.
        \grade There is a defense of a non-obvious position on an issue, deriving from the practice.
            \par There is some evidence to support the argument.
        \grade There is a defense of a non-obvious position on a complex issue, deriving from practice.
            \par There is much evidence to support the argument.
        \grade There is a strong defense of a non-obvious position on a complex issue, deriving from practice-based research.
            \par There is much evidence to support the argument, and it is convincing.
        \grade There is a very strong defense of a non-obvious position on a complex issue, deriving from practice-based research.
            \par There is much evidence to support the argument, and it is very convincing.
%
    \criterion{RESEARCH: Appropriateness of Practice-Based Research Methods}{10\%}
        \grade\fail There is no practice-based research.
        \grade The work implies the application of practice-based research methods, but they are not articulated with sufficient clarity.
        \grade The work somewhat desecribes the application of practice-based research methods.
            \par The choice of methods is broadly appropriate.
        \grade The work desecribes the application of several practice-based research methods with clarity.
            \par The choice of methods is appropriate.
            \par There is a combination of primary and secondary methods.
        \grade The work reflects upon the application of several practice-based research methods.
            \par The choice of methods is appropriate and has some justification.
            \par There is a combination of primary and secondary methods.
        \grade The work critically reflects upon the application of several practice-based research methods 
            \par The choice of methods is appropriate and has much justification.
            \par There is a combination of primary and secondary methods.
%
    \criterion{RESEARCH: Application of Academic Conventions}{5\%}
        \grade\fail No evidence for partial-mastery of academic conventions.
        \par The reference section is missing.
        \par There is no structure.
        \grade 	Some evidence for partial-mastery of academic conventions.
            \par There is a little structure.
            \par A few sentences and paragraphs are well constructed.
            \par 	The reference section is incomplete and/or malformed.
        \grade Much evidence for partial-mastery of academic conventions.
        \par There is some structure.
            \par Some sentences and paragraphs are well constructed.
        \par Most in-text citations and quotations are correct.
        \par The reference section is complete and mostly well-formed in Harvard, ACM or IEEE format.
        \grade Some evidence for mastery of academic conventions.
            \par There is much structure, highlighting the argument.
            \par Most sentences and paragraphs are well constructed, clearly articulating points in the argument.
        \par All in-text citations and quotations are correct.
        \par The reference section is complete and entirely well-formed in Harvard, ACM or IEEE format.
        \grade 	Much evidence for mastery of academic conventions.
            \par There is considerable structure, bolstering the argument.
            \par Most sentences and paragraphs are well constructed, clearly articulating and signposting points in the argument.
        \par All in-text citations and quotations are correct.
        \par The reference section is complete and entirely well-formed in Harvard, ACM or IEEE format.
        \grade 	Considerable evidence for mastery of academic conventions.
            \par There is significant structure, supporting the logical flow of the argument.
            \par Nearly all sentences and paragraphs are well constructed, emphasising points in the argument.
        \par All in-text citations and quotations are correct.
        \par The reference section is complete and entirely well-formed in Harvard, ACM or IEEE format.
\end{markingrubric}

\end{document}
