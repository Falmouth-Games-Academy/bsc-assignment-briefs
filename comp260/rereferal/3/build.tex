\documentclass{../../fal_assignment}
\graphicspath{ {../../} }

\usepackage{enumitem}
\setlist{nosep} % Make enumerate / itemize lists more closely spaced
\usepackage[T1]{fontenc} % http://tex.stackexchange.com/a/17858
\usepackage{url}
\usepackage{todonotes}

\title{Research Journal --- Security}
\author{Dr Michael Scott}
\module{COMP260}

\begin{document}

\maketitle

\section*{Introduction}

\begin{marginquote}
  ``The methods that will most effectively minimize the ability of intruders to compromise information security are comprehensive user training and education. Enacting policies and procedures simply won't suffice.''
    
    --- Kevin Mitnick
    
    \marginquoterule
    
    ``One person's paranoia is another person's engineering redundancy.''
    
    --- Marcus J. Ranum
    
    \marginquoterule

        ``Intercepting and modifying the packets, such that handicaps were applied to our opponents, was quite trivial. The game client did not encrypt anything, nor did it require a checksum.''
    
    --- Anonymous
            
\end{marginquote}
\marginpicture{flavour_pic}{
    It is critical to appreciate the implications of network security when making multiplayer games.
}

In this assignment, you will produce a journal detailing your research on network protocols and network security. In addition to your individual journal, as a group you will create a community wiki to collect and discuss your findings.

Familiarity with the scientific literature is extremely helpful for the computing professional,
both to understand the seminal works that lay the foundations of the field,
and to keep abreast of recent developments at the cutting edge.
Games technology is a fast-moving field, and keeping up is important.
However scientific papers are written in a way that is sometimes daunting to newcomers,
so it is essential to practice the skill of reading and comprehending such papers.
Keeping a research journal is a useful way to record your thoughts
(questions, hypotheses, connections, ideas, ...) as you explore the literature.

Cybersecurity  is not only a core area of knowledge for all computing professionals, 
but also presents key challenges to game developers. Many games are multiplayer and operate over
computer networks. To prevent cheating, rigorous defences are needed. However, this is fast-moving area, so an ability to keep up with the literature on
computer network security is a must.

This assignment is formed of several parts:

\begin{enumerate}[label=(\Alph*)]
    \item \textbf{Write} a draft 1000-word research journal:
    	\begin{enumerate}[label=(\roman*)]
    		\item \textbf{reviewing} at least \textbf{FOUR} papers from the academic literature on the network security challenge you have been \textbf{allocated};
    		\item \textbf{synthesising} your findings into a cohesive analysis;
    		\item and \textbf{recommending} a strategy to overcome the security issue.
    	\end{enumerate}
    \item \textbf{Incorporate} your work into a community wiki, such that you:
    	\begin{enumerate}[label=(\roman*)]
    		\item \textbf{add} your research journal as a wiki page;
    		\item \textbf{extend} at least \textbf{ONE} wiki page (authored by a peer) \textbf{inside} your field of interest with \textbf{ONE} reference not already in the wiki.
    		\item and \textbf{extend} at least \textbf{ONE} wiki page (authored by a peer) \textbf{outside} your field of interest with \textbf{ONE} reference not already in the wiki.
    	\end{enumerate}
    \item \textbf{Finalise} your wiki page, such that you:
    	\begin{enumerate}[label=(\roman*)]
    		\item \textbf{incorporate} the work of your peers and \textbf{tidy} their writing;
    		\item \textbf{revise} any issues raised by your tutor and/or your peers.
    	\end{enumerate}
\end{enumerate}

\subsection*{Assignment Setup}

This assignment is a \textbf{wiki task}. Fork the following GitHub repository:

\indent \url{https://github.com/Falmouth-Games-Academy/comp260-journal}

Form your own page structure and writing conventions as a community. The entire class will be penalised if the wiki is of poor quality and/or difficult to navigate. 

Ensure that you use a consistent username on GitHub (for tracking).

\subsection*{Part A}

Part A consists of \textbf{single formative submissions}. This work is \textbf{individual} and will \textbf{not} be assessed.

To complete Part A, review at least \textbf{FOUR} papers in your allocated area of study.
You must find these yourself using the IEEE and ACM digital libraries. Also consider reviewing other
libraries such as Springer LNCS and ScienceDirect for relevant computing journals and proceedings.
Grey sources are permissible to reference, but will not count toward your target and so will only serve as supplementary sources for deeper analysis and synthesis.
For each paper you read, write a short description of the article and its findings. This should be about 200 words.
This initial writing will set the theme and tone for your subsequent wiki entry;
however, you are expected to also provide some concrete advice to game developers who aim to avoid or mitigate the security challenge. This should also be about 200 words.

There will be no feedback for this initial part.

\subsection*{Part B}

Part B is a \textbf{single formative submission}. This work is \textbf{individual} and will be assessed on a \textbf{threshold} basis. The following criteria are used to determine a pass or fail:

\begin{enumerate}[label=(\alph*)]
	\item Wiki post is timely;
	\item Contribution is within the scope of the allocated security challenge;
	\item A broadly appropriate description and synthesis of \textbf{FOUR} or more papers from the academic literature is present;
	\item Concrete and relevant recommendations are made;
	\item Enough work is available to conduct a meaningful review;
	\item A broadly appropriate review of a peer's work is submitted.
\end{enumerate}

To complete Part B, upload the draft version of the research journal to the community wiki. Continue to make amendments to the wiki as you refine your research findings. 
Please feel welcome to go beyond the four-paper target to strive for higher marks.

Please ensure that is clear which page in the wiki you have authored.
 
Furthermore, add at least \textbf{ONE} additional reference to a page of a peer \textbf{inside} your allocated research field. Then, add yet another \textbf{ONE} or more reference(s) to a page of a peer \textbf{outside} your allocated research field. 

Make your amendments to the wiki ahead of the peer-review workshop in Week 5. Then, attend the scheduled session and contribute to the peer review.

You will receive immediate \textbf{informal feedback} from your peers.

\subsection*{Part C}

Part C consists of \textbf{single summative submission}. This work is \textbf{individual} but with a \textbf{collaborative} component, and will be assessed on a \textbf{criterion-referenced} basis using the criteria listed in the marking rubric at the end of this document.

To complete Part C, finalise your page on the wiki. Please ensure that you are editing the wiki for the \texttt{Falmouth-Games-Academy} repository,
and \textbf{not} the wiki for your fork of the repository. Then, clone the wiki to your local machine: 

\url{https://github.com/Falmouth-Games-Academy/comp260-journal.wiki.git}

Package this as a single \texttt{.zip} file and upload to the LearningSpace.

You will receive \textbf{formal feedback} three weeks after the final deadline.

\section*{Additional Guidance}

Aim to find and review one paper every week; this is the bare minimum that you should read to successfully complete the assignment.
You should also follow up some of the prior work referenced in these papers,
as well as subsequent work that references them.
As much as possible you should focus your reading on peer-reviewed scholarly sources reporting primary research:
articles in scientific workshops, conferences, journals, and some books or book chapters.
Other sources tend to be less rigorous, and should be used only for background information
or in cases where their use can be convincingly justified.

A common pitfall is to focus too much on summarising the content of the papers you have read.
For higher marks you need to demonstrate \textbf{insight} into what you have read:
forming inferences and analyses beyond what is written in the paper.
Some questions you might ask yourself are:
Why is the paper significant and/or influential?
Why did the researchers choose the approach that they did?
Is there anything counterintuitive or surprising in the paper?
Do you disagree with any of the assumptions or claims it makes?
Does the paper suggest any further research questions?

A related pitfall is to structure the journal as a sequence of disconnected entries.
Instead aim to \textbf{synthesise} multiple papers into a cohesive whole,
drawing connections between works by different authors.
Forming a holistic picture of a field is much more valuable than simply understanding individual works.

The wiki is primarily intended to become a useful shared resource for the cohort.
As such, students are expected to direct themselves and their peers in populating, structuring and editing the wiki.
If appropriate, you may copy and paste material from your individual journal into the wiki.
However note that a wiki is not a piece of academic writing
and thus will tend to have a less formal tone than you should be aiming for in your journal.
Edits may be required to ensure a consistent tone for the wiki.

You can, and indeed should, take inspiration from others' wiki contributions when working on your own journal.
However you \textbf{must not} copy verbatim material written by others;
doing so will be considered academic misconduct.

Most researchers write scientific papers for the intended audience of their fellow researchers.
Thus some papers can seem impenetrable to the novice reader.
Don't lose heart!
Discuss the paper with your peers.
Follow up the papers it cites to find alternative explanations.
If all else fails, continue reading the paper --- often a difficult paragraph is clarified by something which appears later.

\section*{FAQ}

\begin{itemize}
	\item 	\textbf{What is the deadline for this assignment?} \\ 
    		Falmouth University policy states that deadlines must only be specified on the MyFalmouth system.
    		
	\item 	\textbf{What should I do to seek help?} \\ 
    		You can email your tutor for informal clarifications. For informal feedback, make a pull request on GitHub. 
    		
    	\item 	\textbf{Is this a mistake?} \\ 	
    		If you have discovered an issue with the brief itself, the source files are available at: \\
    		\url{https://github.com/Falmouth-Games-Academy/bsc-assignment-briefs}.\\
    		 Please make a pull request and comment accordingly.
\end{itemize}

%\section*{Additional Resources}
%
%\begin{itemize}
%    \item Keith, C. (2010) Agile Game Development with Scrum. Pearson Education.
%    \item http://agilemanifesto.org/
%\end{itemize}

\rubricyeartwo

\begin{markingrubric}
%
	\firstcriterion{Basic Competency Threshold}{40\%}
		\gradespan{1}{\fail At least one part is missing or is unsatisfactory or academic misconduct.}
		\gradespan{5}{Parts A and B signed off via participation in and suitable contribution to the peer-review .
			\par Enough work available to hold a meaningful discussion during review.
			\par Most claims are specific, verifiable (with a clear source of evidence), and accurate.
			\par Sources have been cited in an appropriate manner, without any obvious errors.
			\par No breaches of academic integrity.
		}
%
%    \criterion{Formative submissions}{Threshold 5\%}
%        \gradespan{5}{\fail Formative submissions for Part A have been signed off less frequently than once every three weeks.}
%        \grade 		Formative submissions for Part A have been signed off at least once every three weeks.
%
%    \criterion{Participation in viva}{Threshold 5\%}
%        \gradespan{5}{\fail The student did not participate in the viva, or did not submit sufficient work to discuss in the viva.}
%        \grade 		The student participated in the viva.
%
    \criterion{Breadth of Literature Review}{10\%}
        \grade\fail	Fewer than four academic articles are appropriately referenced.
        \grade		At least four academic peer-reviewed articles are appropriately referenced.
        \grade		At least six academic peer-reviewed articles are appropriately referenced.
        \par 		Some primary sources are suitably leveraged to form the analysis.
        \grade		At least eight academic peer-reviewed articles are appropriately referenced.
        \par 		Many primary sources are suitably leveraged to form the analysis.
        \grade		At least ten academic peer-reviewed articles are appropriately referenced.
        \par 		Mostly primary sources are suitably leveraged to form the analysis.
        \par 		Highly regarded and/or seminal work has been incorporated into the work.
        \grade		At least twelve academic peer-reviewed articles are appropriately referenced.
        \par 		Where relevant, nearly all sources leveraged to form the analysis report primary research.
        \par 		Highly regarded and/or seminal work has been incorporated into the work.
%
    \criterion{Depth of Insight}{15\%}
        \grade\fail Little to no insight is demonstrated.
        \par		Papers are merely paraphrased.
        \grade		Some insight is demonstrated.
        \par		Papers are summarised in the student's own words.
        \grade		Modest insight is demonstrated.
        \par		Attempts are made at discussion beyond summary.
        \grade		Much insight is demonstrated.
        \par		Discussion is inferential in nature.
        \grade		Considerable insight is demonstrated.
        \par		Discussion is analytical in nature.
        \grade		Significant insight is demonstrated.
        \par		Discussion is analytical and evaluative in nature.
%
    \criterion{Synthesis}{15\%}
        \grade\fail No connections are made between different sources.
        \grade		Superficial connections are made between different sources.
        \grade		Basic connections are made between different sources.
        \grade		Reasonable connections are made between different sources.
        \par		Connections go beyond mere description.
        \grade		Strong connections are made between different sources.
        \par		Connections are analytical in nature.
        \grade		Strong connections are made between different sources.
        \par		Connections are analytical and evaluative in nature.
%
    \criterion{Community Engagement}{10\%}
        \grade\fail 	Few to no contribution has been made to the wiki.
        \grade 		Some contributions have been made to the wiki.
        \grade 		A modest number of contributions have been made to the wiki.
        \par		Contributions are non-trivial.
        \par		Some attempt to engage in community discourse.
        \grade 		Many contributions have been made to the wiki.
        \par		Contributions are well-reasoned.
        \par		Active engagement in the community discourse.
        \grade 		A considerable number of contributions have been made to the wiki.
        \par		Contributions are well-reasoned and academically sound.
        \par		Steering of the community discourse.
        \grade 		A significant number of contributions have been made to the wiki.
        \par		Contributions are well-reasoned and academically sound.
        \par		Played a key role in driving the community discourse.
%
    \criterion{Wiki Quality}{10\% $\ddagger$}
        \grade\fail 	There is no structure, or the structure is unclear.
        \grade 		There is little structure and conformance to standards.
        \grade 		There is some structure and conformance to standards.
        \par 		The overall quality of the wiki is sufficient.
        \grade 		There is much structure and conformance to standards.
        \par 		The overall quality of the wiki is adequate.
        \grade 		There is considerable structure and conformance to standards.
        \par 		The overall quality of the wiki is high, giving a professional impression.
        \grade 		There is significant structure and conformance to standards.
        \par 		The overall quality of the wiki is very high, giving a professional impression.
\end{markingrubric}

\end{document}
