\chapter{Assignment Structure for COMP320}

\section*{Introduction}

You are required to deliver a major \textbf{research project} as part of your degree; either in the form of \textbf{empirical research} relating to computing for games, or \textbf{practice-based research} related to game development. Individually, you explore a field that interests you, and for which there is a clearly identified need. This module forms the first part of this project and provides the opportunity to conduct a literature review, as well as to collect and analyse data using appropriate methods and statistics.

These assignments are formed of several parts:
\begin{enumerate}[label=(\Alph*)]
	\item \textbf{Complete} an Application for Research Ethics Approval form,
		which will:
		\begin{enumerate}[label=(\roman*)]
			\item \textbf{briefly propose} the research to be undertaken in your project
			\item \textbf{outline} any potential ethical issues around your proposed work,
				particularly with respect to research involving human subjects
		\end{enumerate}
		\textbf{Note that you \textit{must} complete and submit Part A
		\textit{before} carrying out any experimental work.}
	\item \textbf{Deliver} an 8-minute presentation that will:
		\begin{enumerate}[label=(\roman*)]
			\item \textbf{explain} the context of your project
			\item \textbf{identify} and \textbf{discuss} the scientific literature
				relevant to your project
			\item \textbf{propose} one or more research questions for your project
		\end{enumerate}
	\item \label{part:review-draft} \textbf{Write} a draft research review and proposal
		that will:
		\begin{enumerate}[label=(\roman*)]
			\item \textbf{identify} and \textbf{analyse} the scientific literature
				relevant to your project
			\item \textbf{propose} one or more research questions for your project
			\item \textbf{justify} your choice of research question(s)
				and how they will contribute to the state of knowledge
			\item \textbf{outline} and \textbf{justify} the methodology to be used
				in addressing the research question(s)
			\item \textbf{present} any preliminary results you have obtained so far
		\end{enumerate}
	\item \textbf{Write} a final research review and proposal that will:
		\begin{enumerate}[label=(\roman*)]
			\item \textbf{address} any issues raised in Part~\ref{part:review-draft}
		\end{enumerate}
	\item \textbf{Produce} a prototype research artefact that will:
		\begin{enumerate}[label=(\roman*)]
			\item \textbf{facilitate} the collection of empirical data for your project
			\item \textbf{demonstrate} the technical feasibility of your proposed artefact
			\item \textbf{provide} a basis for further development and
				experimentation in the second study block
		\end{enumerate}
	\item \textbf{Deliver} a 15-minute presentation that will:
		\begin{enumerate}[label=(\roman*)]
			\item \textbf{summarise} the context and research question(s)
				of your project
			\item \textbf{outline} and \textbf{justify} the methodology to be used in
				addressing the research question(s)
			\item \textbf{present} any preliminary results you have obtained so far
		\end{enumerate}
\end{enumerate}

\subsection*{Assignment Setup}

These assignments form a \textbf{research} task, consisting of \textbf{academic reading}, \textbf{academic writing},
\textbf{software development} and \textbf{scientific experimentation}.

For the \textbf{prototype research artefact}, there is no set GitHub repository.
However you are strongly encouraged to create one.
Ensure that you set up the \texttt{.gitignore} file for your chosen development environment,
and maintain the \texttt{readme.md} file.

For the \textbf{research review and proposal}, fork the GitHub repository at the following URL:

\indent \url{https://github.com/Falmouth-Games-Academy/comp320-proposal}

Use the existing LaTeX template, which is based on the IEEE Transactions style,
to write your literature review and proposal.

\subsection*{Part A}

Part A consists of a \textbf{single formative submission}.
This work is \textbf{individual} and will be assessed on a \textbf{threshold} basis.

To complete Part A, download the Research Ethics Approval Application Form
from the COMP320 area on LearningSpace.
Complete the form in consultation with your project supervisor.
Note that you will need to write a (brief) research proposal for this form.

Print and sign the form, and hand it to your supervisor for approval.

\textbf{Important:} carrying out \textbf{any} experimental work
without first completing Part A is a breach of Falmouth University's Research Ethics Policy.
This will be treated as a case of academic misconduct, and penalised accordingly.

\subsection*{Part B}

Part B consists of a \textbf{single formative submission}.
This work is \textbf{individual} and will be assessed on a \textbf{threshold} basis.
To pass, answer the following questions:

\begin{enumerate}[label=(\roman*)]
	\item What is the context of your project? How does it fit into the
		research field of computing for games?
	\item What are the key results from the literature upon which your project will be built?
	\item What is the current state of knowledge in the field?
		What are the open questions and challenges?
	\item What is (are) the key research question(s) that you will seek to
		answer in your project?
\end{enumerate}

To complete Part B, prepare a short (8-minute) presentation 
and deliver it in the scheduled session in week~7.
Prepare your slides using your choice of presentation software
(e.g.\ Beamer, reveal.js, PowerPoint).

\subsection*{Part C}

Part C consists of a \textbf{single formative submission},
however your project supervisor may also set additional \textbf{formative submissions} at their discretion.
This work is \textbf{individual} and will be assessed on a \textbf{threshold} basis.

To complete Part C, use the provided LaTeX template to write a literature review and proposal.
Do \textbf{not} modify the formatting of the provided template.
You are advised to bear in mind the page limit specified for Part~D
and attempt to adhere to it at this stage,
however you will \textbf{not} be penalised for failing to do so.

Push your document to GitHub and open a pull request
in advance of the scheduled peer review session.

You will receive immediate \textbf{informal feedback} from tutor and peers.

\subsection*{Part D}

Part D consists of a \textbf{single summative submission},
however your project supervisor may also set additional \textbf{formative submissions} at their discretion.
This work is \textbf{individual} and will be assessed on a \textbf{holistic} basis,
according to the descriptors set out at the end of this document.

To complete Part D, revise your document from Part C
to address any issues highlighted during the peer review.
Your document must not exceed \textbf{six pages} of text,
excluding figures, tables, references and appendices.
This is subject to the usual policy on word and page limits available on LearningSpace.

Upload your final \texttt{.pdf} file to the LearningSpace.
Note that LearningSpace will only accept a single \texttt{.pdf} file.

You will receive immediate and continuous \textbf{informal feedback} through meetings with your supervisor.
You will also receive \textbf{formal feedback} three weeks after the formative deadline.

\subsection*{Part E}

Part E consists of a \textbf{single summative submission},
however your project supervisor may also set additional \textbf{formative submissions} at their discretion.
This work is \textbf{individual} and will be assessed on a \textbf{threshold} basis.

To complete Part E, design and implement a piece of research software appropriate to your chosen project.
Create a Trello board setting out the key requirements for the software,
and identifying which requirements will be tackled this study block
and which are stretch goals to be left until later.
Check the source code into a version control repository regularly.
Ensure your \texttt{readme.md} file contains any information required to build and run the artefact, if appropriate.

Upload a \texttt{.zip} file containing your source code, associated assets, and screenshots of your Trello board to the LearningSpace.
Note that LearningSpace will only accept a single \texttt{.zip} file.

You will receive immediate and continuous \textbf{informal feedback} through meetings with your supervisor.
You will also receive \textbf{formal feedback} three weeks after the formative deadline.

\subsection*{Part F}

Part F consists of a \textbf{single formative submission}.
This work is \textbf{individual} and will be assessed on a \textbf{threshold} basis.
To pass, answer the following questions:

\begin{enumerate}[label=(\roman*)]
	\item What is (are) the key research question(s) that you will seek to
		answer in your project?
	\item How will answering these questions contribute to the state of knowledge
		in the field of your project?
	\item What methodology will you use to seek answers to these questions?
		Justify your methodology.
	\item What preliminary results have you obtained?
\end{enumerate}

To complete Part F, prepare a short (15-minute) presentation 
and deliver it in the scheduled session in week~13.
Prepare your slides using your choice of presentation software
(e.g.\ Beamer, reveal.js, PowerPoint).

\section*{FAQ}

\begin{itemize}
	\item 	\textbf{What is the deadline for this assignment?} \\ 
    		Falmouth University policy states that deadlines must only be specified on the MyFalmouth system.
    		
	\item 	\textbf{What should I do to seek help?} \\ 
    		You can email your tutor for informal clarifications. For informal feedback, make a pull request on GitHub. 
    		
    	\item 	\textbf{Is this a mistake?} \\ 	
    		If you have discovered an issue with the brief itself, the source files are available at: \\
    		\url{https://github.com/Falmouth-Games-Academy/bsc-assignment-briefs}.\\
    		 Please make a pull request and comment accordingly.
\end{itemize}

%\section*{Additional Resources}

%\begin{itemize}
%    \item \url{http://www.opengl-tutorial.org}
%    \item \url{http://gamedev.stackexchange.com/questions/32876/good-resources-for-learning-modern-opengl-3-0-or-later}
%    \item \url{https://google.github.io/styleguide/cppguide.html}
%\end{itemize}
