\chapter{Assignment Structure for COMP320}

\section*{Introduction}

You are required to deliver a major \textbf{research project} as part of your degree; either in the form of \textbf{empirical research} relating to computing for games, or \textbf{practice-based research} related to game development. Individually, you explore a field that interests you, and for which there is a clearly identified need. This module forms the first part of this project and provides the opportunity to conduct a literature review, as well as to collect and analyse data using appropriate methods and statistics.

These assignments are formed of several parts:

\begin{enumerate}[label=(\Alph*)]
	\item \textbf{Identify} a project supervisor \textbf{and agree} a topic for your research, by:
		\begin{enumerate}[label=(\roman*)]
			\item \textbf{reviewing} the academic literature \textbf{in addition to} the state-of-the-art
			\item \textbf{and deriving} a key question or problem from the review to motivate your work
		\end{enumerate}
	\item \textbf{Produce} a prototype computing artefact that will:
		\begin{enumerate}[label=(\roman*)]
			\item \textbf{facilitate} the collection of empirical data for your project
			\item \textbf{demonstrate} the technical feasibility of your proposed artefact
			\item \textbf{provide} a basis for further development and
				experimentation in the second study block
		\end{enumerate}
		\textbf{Note that you \textit{must} complete and submit Part C
		\textit{before} carrying out any experimental work.}
	\item \textbf{Complete} an Application for Research Ethics Approval form,
		which will:
		\begin{enumerate}[label=(\roman*)]
			\item \textbf{briefly propose} the research to be undertaken in your project
			\item \textbf{outline} any potential ethical issues around your proposed work,
				particularly with respect to research involving human subjects
		\end{enumerate}
	\item \textbf{Incorporate} computational statistics into your prototype computing artefact, indicating how you will:
		\begin{enumerate}[label=(\roman*)]
			\item \textbf{collect, analyse, and present} data relevant to your research
		\end{enumerate}
	\item \textbf{Deliver} a 10-minute practice presentation that will:
		\begin{enumerate}[label=(\roman*)]
			\item \textbf{explain} the context of your project
			\item \textbf{identify} and \textbf{discuss} the scientific literature
				relevant to your project
			\item \textbf{propose} one or more research questions for your project
			\item \textbf{articulate} the ethical considerations you have made
			\item \textbf{illustrate} your approach to collecting, analysing, and presenting data
		\end{enumerate}
	\item \label{part:review-draft} \textbf{Write} a draft research review and proposal
		that will:
		\begin{enumerate}[label=(\roman*)]
			\item \textbf{identify} and \textbf{analyse} the scientific literature
				relevant to your project
			\item \textbf{propose} one or more research questions for your project
			\item \textbf{justify} your choice of research question(s)
				and how they will contribute to the state of knowledge
			\item \textbf{justify} the methodology to be used
				in addressing the research question(s)
			\item \textbf{outline} the key legal, social, ethics, and professional considerations	
			\item \textbf{present} your data analysis approach \textbf{and} any preliminary results
		\end{enumerate}
	\item \textbf{Write} a final research review and proposal that will:
		\begin{enumerate}[label=(\roman*)]
			\item \textbf{address} any issues raised in Part~\ref{part:review-draft}
		\end{enumerate}
	\item \textbf{Deliver} a 10-minute presentation that will:
		\begin{enumerate}[label=(\roman*)]
			\item \textbf{summarise} the context and research question(s)
				of your project
			\item \textbf{outline} and \textbf{justify} the methodology to be used in
				addressing the research question(s)
			\item \textbf{assess} legal, social, ethical, and professional issues
			\item \textbf{present} any preliminary results you have obtained so far, \textbf{emphasising} your
				approach to data collection, analysis, and presentation
		\end{enumerate}
\end{enumerate}

\subsection*{Assignment Setup}

These assignments form a \textbf{research} task, consisting of \textbf{literature review}, \textbf{academic writing},
\textbf{software development} and \textbf{scientific experimentation}. For the \textbf{research proposal} with \textbf{prototype computing artefact}, there are no set git repositories or templates for you to use. Though, you may choose to use one or more of the templates available on our GitHub Enterprise Server: \indent \url{https://github.falmouth.ac.uk}

You are strongly encouraged to use a repository for your prototype computing artefact in order to demonstrate your refactoring process. Ensure that you set up the \texttt{.gitignore} file for your chosen development environment, and maintain the \texttt{readme.md} file with design documentation appropriate to your project. Select a suitable license for your work.

For the written component, you may use IEEE Transactions style or ACM style, to write your research proposal. There are templates available on LearningSpace.

\subsection*{Part A}

Part A consists of a \textbf{single formative submission} early in academic year, by a date agreed with your supervisor.
This work is \textbf{individual} and will be assessed on a \textbf{threshold} basis.

To complete Part A, email your chosen supervisor and agree a topic for your proposal within the area of that
supervisor's expertise. Then, email confirmation of that agreement to the module leader.

You will receive immediate and continuous \textbf{informal feedback} through timetabled weekly meetings with your supervisor.

\subsection*{Part B}

Part B consists of a \textbf{multiple formative submissions},
set at the discretion of your project supervisor based on agreed milestones and the nature of your proposed project.
This work is \textbf{individual} and will be assessed on a \textbf{threshold} basis.

To complete Part B, design and implement a computing artefact appropriate to your chosen project.
Create a task board setting out the key requirements for the system,
and identifying which requirements will be tackled this study block
and which are stretch goals to be left until later.
Check the source code into a version control repository regularly.
Ensure your \texttt{readme.md} file contains any information required to build and run the artefact as well as
relevant design documentation (including relevant UML diagrams).

You will not directly submit this work. Instead, your supervisor will track your work on the BitBucket server.

You will receive immediate and continuous \textbf{informal feedback} through meetings with your supervisor.

\todo[inline]{\textbf{Important:} Carrying out \textbf{any} data collection or experimental work
without first securing ethical approval is a breach of Falmouth University's Research Ethics Policy.
This will be treated as a case of academic misconduct, and penalised accordingly.}

\subsection*{Part C}

Part C consists of a \textbf{single formative submission}.
This work is \textbf{individual} and will be assessed on a \textbf{threshold} basis.

To complete Part C, download the Research Ethics Approval Application Form
from the LearningSpace. Complete the form in consultation with your project supervisor.
Note that you will need to write a (brief) research proposal for this form. Print and sign the form, 
and hand it to your supervisor for approval.

You will receive immediate \textbf{informal feedback} from your supervisor and a \textbf{decision} 
after the ethics committee have convened, according to the academic calendar.

\subsection*{Part D}

Part D consists of a \textbf{multiple formative submissions},
set at the discretion of your project supervisor based on agreed milestones and the nature of your proposed project.
This work is \textbf{individual} and will be assessed on a \textbf{threshold} basis.

To complete Part D, incorporate computational statistics into the repository of your prototype computing artefact.
It is anticipated that you will write \texttt{R} code or \texttt{NumPy} code (or similar) to support your use of data in your project.
Ensure your \texttt{readme.md} file contains any information required to build and execute your statistical tests.

If you have not yet conducted a pilot study, it is advisable to generate sample data through simulation so that you can 
prepare your data analysis and presentation techniques.

You will receive immediate and continuous \textbf{informal feedback} through meetings with your supervisor.

\subsection*{Part E}

Part E consists of a \textbf{single formative submission}.
This work is \textbf{individual} and will be assessed on a \textbf{threshold} basis.
To pass, answer the following questions:

\begin{enumerate}[label=(\roman*)]
	\item What is the context of your project? How does it fit into the
		research field of computing for games?
	\item What are the key results from the literature upon which your project will be built?
	\item What is the current state of knowledge in the field?
		What are the open questions and challenges?
	\item What is (are) the key research question(s) that you will seek to
		answer in your project?
	\item What are the key legal, social, ethical, and/or professional issues 
		associated with your project?
\end{enumerate}

To complete Part E, prepare a short presentation 
and deliver it in the scheduled session in week~11.
Prepare your slides using your choice of presentation software
(e.g.\ Beamer, reveal.js, PowerPoint).

You will receive immediate \textbf{informal feedback} from your tutor.

\subsection*{Part F}

Part F consists of a \textbf{single formative submission},
however your project supervisor may also set additional \textbf{formative submissions} at their discretion.
This work is \textbf{individual} and will be assessed on a \textbf{threshold} basis.

To complete Part F, use the provided LaTeX template to write a literature review and proposal. Then, attend the 
relevant session in week~12.

Do \textbf{not} modify the formatting of the provided template. You are advised to bear in mind the page limit 
and attempt to adhere to it at this stage, however you will \textbf{not} be penalised for failing to do so. 
Please also ensure that you include a clear link to your git repo somewhere in the document.

You will receive immediate \textbf{informal feedback} from your peers.

\subsection*{Part G}

Part G consists of a \textbf{single summative submission},
however your project supervisor may also set additional \textbf{formative submissions} at their discretion.
This work is \textbf{individual} and will be assessed on a \textbf{holistic} basis,
according to the descriptors set out at the end of this document.

To complete Part G, revise your document from Part E
to address any issues highlighted during the peer review. Also, incorporate references to the data analysis
approaches that you intend to deploy (as per Part F).
Your document must not exceed \textbf{six pages} of text,
excluding figures, tables, references and appendices.
This is subject to the usual policy on word and page limits available on LearningSpace.

Please ensure that you include a clear link to your git repo somewhere in the document.

Upload your final \texttt{.pdf} file to the LearningSpace.
Note that LearningSpace will only accept a single \texttt{.pdf} file.

You will receive \textbf{informal feedback} through a meetings with your supervisor.
You will also receive \textbf{formal feedback} three weeks after the deadline.

\subsection*{Part H}

Part H consists of a \textbf{single formative submission}.
This work is \textbf{individual} and will be assessed on a \textbf{threshold} basis.
To pass, answer the following questions:

\begin{enumerate}[label=(\roman*)]
	\item What is (are) the key research question(s) that you will seek to
		answer in your project?
	\item How will answering these questions contribute to the state of knowledge
		in the field of your project?
	\item What methodology will you use to seek answers to these questions?
		Justify your methodology.
	\item What preliminary results have you obtained?
\end{enumerate}

To complete Part H, prepare a short presentation 
and deliver it in the scheduled session in week~13.
Prepare your slides using your choice of presentation software
(e.g.\ Beamer, reveal.js, PowerPoint).

You will receive immediate \textbf{informal feedback} from a panel of academics.
You will also receive \textbf{formal feedback} three weeks after the deadline.

\section*{FAQ}

\begin{itemize}
	\item 	\textbf{What is the deadline for this assignment?} \\ 
    		Falmouth University policy states that deadlines must only be specified on the MyFalmouth system.
    		
	\item 	\textbf{What should I do to seek help?} \\ 
    		You can email your tutor for informal clarifications. For informal feedback, make a pull request on GitHub. 
    		
    	\item 	\textbf{Is this a mistake?} \\ 	
    		If you have discovered an issue with the brief itself, the source files are available at: \\
    		\url{https://github.com/Falmouth-Games-Academy/bsc-assignment-briefs}.\\
    		 Please make a pull request and comment accordingly.
\end{itemize}

%\section*{Additional Resources}

%\begin{itemize}
%    \item \url{http://www.opengl-tutorial.org}
%    \item \url{http://gamedev.stackexchange.com/questions/32876/good-resources-for-learning-modern-opengl-3-0-or-later}
%    \item \url{https://google.github.io/styleguide/cppguide.html}
%\end{itemize}
