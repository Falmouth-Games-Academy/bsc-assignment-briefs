\chapter{Marking Descriptors}

Note that the assignments which constitute your dissertation project are \textbf{not} marked in a criterion-based fashion.
Instead, your project supervisor will assign an overall grade by considering the following descriptors in relation to your project.

You \textbf{must} demonstrate success at every learning outcome in order to pass.

\rubrictitle{COMP320 Research Proposal with Computing Prototype}

\begin{markingrubric}
    \firstcriterion{Ethics Approval}{---}
		\grade\fail An ethics approval form has not been submitted.
			\par \textbf{This is an automatic fail, regardless of other criteria.}
		\grade Ethics approval has been sought and obtained.
	%
    \criterion{Basic Competency Threshold}{---}
		\grade\fail Parts~B and~C are not completed or are unsatisfactory.
			\par Referencing of sources is unsatisfactory.
		\grade A satisfactory presentation is delivered for Part~B.
			\par The student participates in the peer review activity for Part~C, with enough work present for a meaningful review.
			\par Sources are referenced, with no obvious errors or omissions.
	%
    \criterion{Breadth of literature review}{---}
        \grade \fail 	The literature review is missing or unsatisfactory.
        	\par 		Many key sources are omitted.
        \grade 		The literature review falls far short of comprehensive.
        	\par 		There are many obvious omissions.
        \grade		The literature review falls short of comprehensive.
        	\par 		There are some obvious omissions.
        \grade 		The literature review is somewhat comprehensive.
        	\par 		There are few obvious omissions.
        \grade		The literature review is very comprehensive.
        	\par 		There are very few obvious omissions.
        \grade		The literature review is extremely comprehensive.
        	\par 		There are no obvious omissions.
	%
    \criterion{Depth of insight}{---}
        \grade\fail 	Little insight is demonstrated.
        \par		Papers are summarised in the student's own words.
        \grade		Some insight is demonstrated.
        \par		Attempts are made at discussion beyond summary.
        \grade		Much insight is demonstrated.
        \par		Discussion is inferential in nature.
        \grade		Considerable insight is demonstrated.
        \par		Discussion is analytical in nature.
        \grade		Significant insight is demonstrated.
        \par		Discussion is analytical and evaluative in nature.
        \grade		Extensive insight is demonstrated.
        \par		Discussion is analytical and evaluative in nature.
	%
    \criterion{Specificity, verifiability \& accuracy of claims}{---}
        \grade\fail 	Few claims have a clear source of evidence.
        \par 		Significant errors and/or misinterpretations.
        \grade 		Some claims have a clear source of evidence.
        \par 		Many errors and/or misinterpretations.
        \grade 		Many claims have a clear source of evidence.
        \par 		Some errors and/or misinterpretations.
        \grade 		Most claims have a clear source of evidence.
        \par 		Few errors and/or misinterpretations.
        \grade 		All claims have a clear source of evidence.
        \par 		Almost no errors and/or misinterpretations.
        \grade 		All claims have a clear source of evidence.
        \par 		No errors and/or misinterpretations.
    %
    \criterion{Quality of research question(s)}{---}
        \grade\fail 	Research questions are absent or not satisfactory.
        \grade 		Research questions show basic understanding of the field.
        \par		Research questions are unambitious or unoriginal.
        \grade 		Research questions show strong understanding of the field.
        \par		Research questions are original.
        \grade 		Research questions show some insight into the field.
        \par		Research questions are original and ambitious.
        \grade 		Research questions show much insight into the field.
        \par		Research questions are original, ambitious and timely.
        \grade 		Research questions show significant insight into the field.
        \par		Research questions are at the cutting edge of the field.
    %
    \criterion{Methodology}{---}
        \grade\fail 	Methodology is not specified, not justified, or either of these is unsatisfactory.
        \grade 		The proposed methodology is somewhat plausible.
        \par		The justification is not very convincing.
        \grade 		The proposed methodology is plausible.
        \par		The justification is somewhat convincing.
        \grade 		The proposed methodology is sound.
        \par		The justification is convincing.
        \grade 		The proposed methodology is very sound.
        \par		The justification is very convincing.
        \grade 		The proposed methodology is extremely sound.
        \par		The justification is extremely convincing.
%
\end{markingrubric}
    
\begin{markingrubric}
    \firstcriterion{Data Management Plan}{---}
        \grade\fail 	There is little to no description of how to collect, analyse, and present data.
        \grade 		There is an adequete description of how to collect, analyse, and present data to support the research process.
         \par		The proposed use of computational statistics is described with some clarity.        
        \grade 		There is a competent explanation of how to collect, analyse, and present data to support the research process.
         \par		The proposed use of computational statistics is sensible and is described clearly.         
        \par		Excerpts of the data analysis code are present and described accurately.                 
        \grade 		There is a very good analysis of how to collect, analyse, and present data to support the research process.
         \par		The proposed use of computational statistics is well devised and has been defined with precision.                  
        \par		Excerpts of the data analysis code are present and are well explained.   
        \grade 		There is an excellent evaluation of how to collect, analyse, and present data to support the research process.
         \par		The proposed use of computational statistics is very well devised and has been defined in a way that supports replication.      
        \par		A walkthrough of the data analysis code is present and is justified.   
        \grade 		There is an outstanding description of how to collect, analyse, and present data to support the research process.
         \par		The proposed use of computational statistics is likely to lead to new insights and has been defined in a way that make it easy to replicate.        
        \par		A walkthrough of the data analysis code is present and is justified very well. 
%
    \criterion{Artefact Test Plan}{---}
        \grade\fail 	There is no plan to validate, verify, and test the computing artefact, or it is inadequete.
        \grade 		There is an adequete plan to validate, verify, and test the computing artefact.
        \grade 		There is a competent plan to validate, verify, and test the computing artefact.
        \grade 		There is a very good plan to validate, verify, and test the computing artefact.
        \grade 		There is a excellent plan to validate, verify, and test the computing artefact.
        \grade 		There is a outstanding plan to validate, verify, and test the computing artefact.    
    %
     \criterion{Ethical Considerations}{---}
        \grade\fail 	Inadequete assessment of legal, social, ethical, and professional issues.
         \par		The proposed project is not ethically sound.
        \grade 		There is a somewhat accurate assessment of legal, social, ethical, and professional issues.
        \par		The proposed project is ethically sound, but there may be several concerns to address through the approval process.
        \grade 		There is an accurate assessment of legal, social, ethical, and professional issues; accompanied by a somewhat sensible analysis of key concerns.
        \par		The proposed project is ethically sound, but there may be a few concerns to address through the approval process.
        \grade 		There is an accurate assessment of legal, social, ethical, and professional issues; accompanied by a sensible analysis of key concerns.
        \par		The proposed project is ethically sound, and there a little to no concerns.
        \grade 		There is an accurate and comprehensve assessment of legal, social, ethical, and professional issues; accompanied by a sensible discussion of prominent concerns.
        \par		The proposed project is ethically sound, and there a little to no concerns.
        \grade 		There is an accurate and comprehensive assessment of legal, social, ethical, and professional issues; accompanied by a deeply intellectual discussion of prominent concerns.
        \par		The proposed project is ethically sound, and there a little to no concerns.
%
\end{markingrubric}

\rubrictitle{COMP360 Dissertation with Computing Artefact}
	
\begin{markingrubric}
%
    \firstcriterion{Ethics Approval \& Academic Conduct}{---}
		\grade\fail Ethical approval not obtained or evident lack of academic integrity.
		%\grade{}
		\grade Ethics approval has been sought and obtained. \par Academic conduct is acceptable.
%
    \criterion{Novice Competency Threshold}{---}
		\grade\fail Any criterion does not meet the `adequate' descriptor.
		%\grade{}
		\grade All criteria are at least `adequete'.
                       \par Satisfactory presentations are delivered.
		  \par Adequete participation in peer review exercises, with enough to enable meaningful review alongside critical and constructive comments provided to peers.
		  \par Link between the dissertation and the research artefact is clear and appropriate.
		  \par An appropriate number of sources are referenced.
		  \par All sources are referenced appropriately, with no obvious errors or omissions.
		  \par Claims critical to the conclusion are well-specified, verifiable, and accurate.
		  \par BCS requirements satisfied.
%
    \criterion{Functional Coherence}{---}
        \grade \fail The computing artefact is not deployable.
        \grade Some useful and relevant features have been implemented.
            \par Some obvious bugs are detected.
            \par The computing artefact is deployable.
        \grade Many useful and relevant features have been implemented.
            \par There is some evidence of feature creep.
            \par Few obvious bugs  are detected.
            \par The computing artefact is deployable.
        \grade Almost all pertinent features have been implemented.
            \par There is little evidence of feature creep.
            \par Some minor bugs  are detected.
            \par The computing artefact is deployable.
        \grade All pertinent features have been implemented.
            \par There is almost no evidence of feature creep.
            \par Some bugs, purely cosmetic and/or superficial in nature, are detected.
            \par The computing artefact is deployable.
        \grade All pertinent features have been implemented.
            \par There is no evidence of feature creep.
            \par Few to no bugs are detected.
            \par The computing artefact is deployable.
%
    \criterion{Sophistication}{---}
        \grade \fail There is little evidence of insight into programming constructs.
        \grade Some insight into the appropriate use of programming constructs is evident from the source code.
            \par The program structure is adequate.
        \grade Much insight into the appropriate use of programming constructs is evident from the source code.
            \par The program structure is appropriate.
        \grade Considerable insight into the appropriate use of programming constructs is evident from the source code.
            \par The program structure is effective.
        \grade Significant insight into the appropriate use of programming constructs is evident from the source code.
            \par The program structure is highly effective, with high cohesion and low coupling.
        \grade Extensive insight into the effective use of programming constructs is evident from the source code.
            \par The program structure is impressive, with very high cohesion and low coupling.
%
    \criterion{Maintainability}{---}
        \grade \fail There is little to no maintainability. 
        \grade The code is somewhat well commented.
            \par Some identifier names are descriptive and appropriate.
            \par Most code adheres to a sensible formatting style.
             \par There is little obvious duplication of code or of literal values.           
        \grade The code is reasonably well commented.
            \par Most identifier names are descriptive and appropriate.
            \par Most code adheres to a sensible formatting style.
             \par There is almost no obvious duplication of code or of literal values.   
        \grade The code is reasonably well commented.
            \par Almost all identifier names are descriptive and appropriate.
            \par Almost all code adheres to a sensible formatting style.
             \par There is no obvious duplication of code or of literal values.
        \grade The code is very well commented.
            \par All identifier names are descriptive and appropriate.
            \par All code adheres to a sensible formatting style.
             \par There is no obvious duplication of code or of literal values.
        \grade The code is extremely well commented.
            \par All identifier names are descriptive and appropriate.
            \par All code adheres to a sensible formatting style.
            \par There is no duplication of code or of literal values.
%
    \criterion{Refactoring}{---}
        \grade \fail There is little to no evidence of refactoring. 
        \grade There is some evidence of refactoring in version control. 
            \par There is a README.md file describing the state of the artefact.  
        \grade There is much evidence of refactoring in version control.
            \par There is supplementary documentation in the repository describing the state of the artefact and how it has been refactored with some clarity.  
        \grade There is considerable evidence of refactoring in version control.  
            \par There is supplementary documentation in the repository analysing the state of the artefact and how it has been refactored with considerable clarity.  
        \grade There is significant evidence of refactoring in version control.  
            \par There is supplementary documentation in the repository analysing the state of the artefact and the appropriateness of its refactoring.  
            \par There is evidence of bug tracking, indicating the future development of the artefact.  
        \grade There is extensive evidence of refactoring in version control. 
            \par There is supplementary documentation in the repository critically analysing the state of the artefact and the appropriateness of its refactoring.  
            \par There is evidence of bug tracking and roadmapping further development of the artefact.  
%
\end{markingrubric}

\begin{markingrubric}
%
    \firstcriterion{Abstract}{---}
        \grade \fail There is no abstract, or it does not summarise the key findings.
      \grade A general overview of the dissertation is provided.
        	\par There is some structure.		
      \grade A concise, but targeted summary of key points in the dissertation is provided.
        	\par There is much structure.	      	
       \grade A concise, but highly targeted outline of the purpose and key claims made in the dissertation is provided.
        	\par The abstract broadly incorporates and follows the OCAR elements.	  
      \grade A concise, but highly targeted outline of the relevant academic discourse and potential contributions made in the dissertation is provided.
        	\par The abstract incorporates and follows the OCAR elements.	      	
       \grade A concise, but impressively highlighted gap in the literature and outline of intellectual contributions made in the dissertation to plug that gap is provided. 
        	\par The abstract incorporates and follows the OCAR elements.
        	\par The key contribution of the dissertation is made clear in the abstract.
%
    \criterion{Introduction}{---}
        \grade \fail Introduction does not contextualise the research and development.
      \grade Introduction has much structure.
        	\par The field of research is somewhat contextualised.
      \grade Introduction has much structure.
        	\par The field of research is somewhat contextualised and motivated.	      	
       \grade Introduction has considerable structure.
        	\par The motivation for the research is somewhat justified.	  
      \grade Introduction has considerable structure.
        	\par The motivation for the research is clearly justified.	 	      	
       \grade Introduction has significant structure.
        	\par The potential impact of the research is justified effectively.
%  		
    \criterion{Literature Review}{---}
        \grade \fail Inappropriate literature review.
        \grade The literature review falls short of being comprehensive, but has some merit.
        	\par  Synthesis is adequete, hinting at a need for the work.
        \grade The literature review falls short of comprehensive, but has much merit.
        	\par  Synthesis is appropriate, highlighting a need for the work.
        \grade The literature review is somewhat comprehensive, having considerable merit.
        	\par  Synthesis is effective, criticising existing work to reveal a gap.
        \grade The literature review is comprehensive, having significant merit.
        	\par Synthesis is effective, criticising existing work to reveal a key gap.
        \grade The literature review is comprehensive, incorporating all important seminal and contemporary works.
        	\par Synthesis is highly effective, critically evaluating existing work to showcase an important gap.
%
    %
    \criterion{Research Questions}{---}
        \grade\fail 	Research questions are absent or not satisfactory.
        \grade 		Research questions show basic understanding of the field.
        \par		Research questions are unambitious or unoriginal.
        \grade 		Research questions show strong understanding of the field.
        \par		Research questions are original.
        \grade 		Research questions show some insight into the field.
        \par		Research questions are original and ambitious.
        \grade 		Research questions show much insight into the field.
        \par		Research questions are original, ambitious and timely.
        \grade 		Research questions show significant insight into the field.
        \par		Research questions are at the cutting edge of the field.
%
\end{markingrubric}

\begin{markingrubric}
%    
    \firstcriterion{Hypotheses}{---}
            \grade\fail   No hypotheses.
        \grade 		A set of hypotheses are implied.
        \grade 		Hypotheses are stated explicitly and clearly.
        \grade 		Hypotheses are stated explicitly and clearly, following the research question.
        \par		Hypotheses are reasonably well-formed.
        \grade 		Hypotheses are stated explicitly and clearly, clearly decomposing the research question.
        \par		Hypotheses are precise and well-formed.
        \grade 		Hypotheses are stated explicitly and clearly, clearly linking to the research question and the underlying literature motivating the research.
        \par		Hypotheses are precise, well-formed, and related clearly to underlying statistical tests.
%
    \criterion{Computing Artefact}{---}
        \grade \fail The computing artefact does not exist, or is otherwise poorly described.
        \grade The computing artefact is clearly described, and most features are distinguished from other work, and easily measured.
            \par Requirements somewhat correspond to the proposed research.
            \par The design and refactoring of the computing artefact is mentioned, with reference to relevant theories, practice, and discourse in the computing sector.
        \grade The computing artefact is clearly described with some technical detail, and nearly all features are distinguished from other work, and easily measured.
            \par Requirements correspond to the proposed research.
            \par The design and refactoring of the computing artefact is described well in technical detail, with reference to relevant theories, practice, and discourse in the computing sector.
        \grade The computing artefact is clearly described in much technical detail, and all features are distinguished from other work, and easily measured.
            \par Requirements clearly correspond to the proposed research.
            \par The design and refactoring of the computing artefact is described comprehensively, with critical reference to relevant theories, practice, and discourse in the computing sector.
        \grade The computing artefact is clearly described in considerable technical detail, and all features are distinguished from other work, and easily measured.
            \par Requirements clearly and comprehensively correspond to the proposed research.
            \par The design and refactoring of the computing artefact is analysed, with critical reference to relevant theories, practice, and discourse in the computing sector.
        \grade The computing artefact is clearly described in significant technical detail, and all features are distinguished from other work, and easily measured.
            \par Requirements clearly and comprehensively correspond to the proposed research.
            \par The design and refactoring of the computing artefact is analysed deeply, with critical reference to relevant theories, practice, and discourse in the computing sector.
%
    \criterion{Validation \& Verification}{---}
        \grade \fail There is no plan to validate, verify, and test the computing artefact, or it is inadequete.
            \par No evidence of source code validation and verification.
        \grade There is an adequete plan to validate, verify, and test the computing artefact.
            \par Some evidence of source code validation and verification.
            \par Basic unit testing is evident.       
        \grade There is a competent plan to validate, verify, and test the computing artefact.
            \par Much evidence of source code validation and verification.
            \par Basic unit testing is evident, with much coverage.
        \grade There is a very good plan to validate, verify, and test the computing artefact.
            \par Considerable evidence of source code validation and verification.
            \par Sophisticated unit testing is evident, with considerable coverage.
            \par Appropriate testing techniques beyond unit testing have been applied, with some effectiveness at improving integrity/quality.
        \grade There is a excellent plan to validate, verify, and test the computing artefact.
            \par Significant evidence of source code validation and verification.
            \par Sophisticated unit testing is evident, with considerable coverage.
            \par Appropriate testing techniques beyond unit testing have been applied, with much effectiveness at improving integrity/quality.
        \grade There is a outstanding plan to validate, verify, and test the computing artefact. 
            \par Extensive evidence of source code validation and verification.
            \par Sophisticated unit testing is evident, with significant coverage.
            \par Innovative or cutting-edge testing techniques have been applied, with considerable rigor.
%
    \criterion{Research Methodology}{---}
        \grade\fail 
        \grade 		The methodology applied could plausibly lead to a somewhat meaningful claim.
        \par		Methods are justified.
        \grade 		The methodology applied could plausibly lead to a meaningful claim.
        \par		The methodology and integrity of the research are justified.
        \grade 		The applied methodology is scientifically rigorous, and potentially able to lead to a sound claim.
        \par		The methodology and integrity of the research justified somewhat convincingly.
        \grade 		The applied methodology is scientifically rigorous, and quite likely to lead to a sound claim.
        \par		The methodology and integrity of the research justified convincingly.
        \grade 		The applied methodology is extremely scientifically rigorous, perhaps a gold standard, and highly likely to lead to a sound claim.
        \par		The methodology and integrity of the research are justified very convincingly.
%
\end{markingrubric}

\begin{markingrubric}
%  
    \firstcriterion{Data Management Plan}{---}
        \grade\fail 	There is little to no description of how to collect, analyse, and present data.
        \grade 		There is an adequete description of how to collect, analyse, and present data to support the research process.
         \par		The proposed use of computational statistics is described with some clarity.        
        \grade 		There is a competent explanation of how to collect, analyse, and present data to support the research process.
         \par		The proposed use of computational statistics is sensible and is described clearly.         
        \par		Excerpts of the data analysis code are present and described accurately.                 
        \grade 		There is a very good analysis of how to collect, analyse, and present data to support the research process.
         \par		The proposed use of computational statistics is well devised and has been defined with precision.                  
        \par		Excerpts of the data analysis code are present and are well explained.   
        \grade 		There is an excellent evaluation of how to collect, analyse, and present data to support the research process.
         \par		The proposed use of computational statistics is very well devised and has been defined in a way that supports replication.      
        \par		A walkthrough of the data analysis code is present and is justified.   
        \grade 		There is an outstanding description of how to collect, analyse, and present data to support the research process.
         \par		The proposed use of computational statistics is likely to lead to new insights and has been defined in a way that make it easy to replicate.        
        \par		A walkthrough of the data analysis code is present and is justified very well. 
%
     \criterion{Ethical Considerations}{---}
        \grade\fail 	Inadequete assessment of legal, social, ethical, and professional issues.
         \par		The proposed project is not ethically sound.
        \grade 		There is a somewhat accurate assessment of legal, social, ethical, and professional issues.
        \par		The proposed project is ethically sound, but there may be several concerns to address through the approval process.
        \grade 		There is an accurate assessment of legal, social, ethical, and professional issues; accompanied by a somewhat sensible analysis of key concerns.
        \par		The proposed project is ethically sound, but there may be a few concerns to address through the approval process.
        \grade 		There is an accurate assessment of legal, social, ethical, and professional issues; accompanied by a sensible analysis of key concerns.
        \par		The proposed project is ethically sound, and there a little to no concerns.
        \grade 		There is an accurate and comprehensve assessment of legal, social, ethical, and professional issues; accompanied by a sensible discussion of prominent concerns.
        \par		The proposed project is ethically sound, and there a little to no concerns.
        \grade 		There is an accurate and comprehensive assessment of legal, social, ethical, and professional issues; accompanied by a deeply intellectual discussion of prominent concerns.
        \par		The proposed project is ethically sound, and there a little to no concerns.
%  
    \criterion{Results \& Analysis}{---}
        \grade\fail 
        \grade 		The results and their analysis is basic.
        \grade 		The results and their analysis approach sophistication.
        \par 		The method of analysis is appropriate to the project.
        \grade 		The results and their analysis are somewhat sophisticated and reflect good practice.
        \par 		The method of analysis is appropriate to the project.
        \grade 		The results and their analysis are highly sophisticated and reflect good practice.
        \par 		The method of analysis is suited and tailored to the project.
       \par 		Some care has been taken to guard against potential criticisms of the analysis and its assumptions.
        \par		The analysis shows progress towards publishable quality.
        \grade 		The results and their analysis are very highly sophisticated and could be considered best practice.
        \par 		The method of analysis is not only suited and tailored to the project.
        \par 		Much care has been taken to guard against potential criticisms of the analysis and its assumptions.
        \par		The results and analysis is of publishable quality.
%
    \criterion{Discussion \& Conclusion}{---}
        \grade\fail 
        \grade		Some insight into the topic is evident.
        \par		Attempts are made at discussion beyond summary.
        \grade		Much insight into the topic is evident.
        \par		Discussion is inferential in nature.
        \par		Key take-away points can be inferred from the work.
        \grade		Considerable insight into the topic is evident.
        \par		Discussion is analytical in nature.
        \par		The limitations of the research are made clear.
        \par		Key take-away points are made clear in the conclusion.
        \grade		Significant  insight into the topic is evident.
        \par		Discussion is both analytical and evaluative in nature.
        \par		The limitations of the research are made explicitly clear.
        \par		Meaningful take-away points are made clear in the conclusion. There might be recommendations.
        \grade		Extensive insight into the topic is evident.
        \par		Discussion is both analytical and evaluative in nature.
        \par		The limitations of the research have been thought through well, with their mitigation and/or acknowledgement made explicitly clear.
        \par		Meaningful contributions to discourse in the field are made clear in the conclusion alongside key recommendations/impact.       
%
\end{markingrubric}

\begin{markingrubric}
%
    \firstcriterion{Critical Addendum}{---}
        \grade\fail  	Insufficient critical self-appraisal.
        \grade		Some relevant and critical self-appraisal.
        \grade		Much relevant and critical self-appraisal.
        \par		Outcomes and lessons learned are made clear, and loosely tied to goals to improve.
        \grade		Considerable self-appraisal whcih is both relevant and critical in nature.
        \par		Outcomes and lessons learned are analysed and tied to somewhat well-formed SMART objectives.
        \grade		Significant self-appraisal whcih is both pertinent and critical in nature.
        \par		Outcomes and lessons learned are analysed in much depth and clearly tied to well-formed SMART objectives.
        \grade		Extensive self-appraisal whcih is both pertinent and critical in nature.
        \par		Outcomes and lessons learned are analysed in considerable depth and tied explicitly to well-formed SMART objectives. 
       \par		Evaluation of the project leads to recomendations that others could leverage to improve their research. 
%
    \criterion{Artefact Testing Addendum}{---}
        \grade\fail       Insufficient evidence of testing.
        \grade		Some evidence of appropriate testing of the computing artefact.
        \grade		Much evidence of appropriate testing of the computing artefact.
        \grade		Considerable evidence of appropriate testing of the computing artefact.
        \par		Good range of different techniques used to validate, verify, and test the computing artefact.
        \grade		Significant evidence of appropriate testing of the computing artefact.
        \par		Very good range of different techniques used to validate, verify, and test the computing artefact.
        \grade		Extensive evidence of appropriate testing of the computing artefact.
        \par		Excellent range of different techniques used to validate, verify, and test the computing artefact.
\end{markingrubric}
