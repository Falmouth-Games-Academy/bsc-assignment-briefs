\chapter{Introduction}

You are required to deliver a major \textbf{research project} as part of your degree;
either in the form of \textbf{empirical research} relating to computing for games,
or \textbf{practice-based research} related to game development.
Individually, you explore a field that interests you, and for which there is a clearly identified need.

The strongest projects constitute original research, making a contribution to knowledge in their field;
the very strongest projects in the past have been submitted and accepted for publication in international
scientific conferences. This is not a requirement, but a bar to aim for.

\section*{Selecting a project}

Your \textbf{project supervisor} is responsible for giving you guidance and feedback throughout your project,
through regular group and individual supervision meetings.

Every potential supervisor has their own area of expertise and research interests.
Towards the end of your second year, a list of suggested project titles will be published;
these titles have been selected by potential supervisors on the basis of their expertise
and their assessment of feasibility and interestingness.
You are also encouraged to propose your own project title,
which your prospective supervisor will then help you refine.
Either way, it is strongly advised that you discuss your choice with your prospective supervisor over the summer vacation.

One of the few restrictions on the project topic is that it must allow you to develop some form of \textbf{software artefact},
as this is a component of the assessment.
This doesn't have to be a piece of software intended for an end user ---
for example it could be a software component for collecting and analysing statistics within a pre-existing software system.
Your supervisor can advise on what would constitute a relevant and suitable artefact for your project.

\section*{Module overview}

The project is split across \textbf{two} modules, each worth \textbf{20 credits}.
Thus in total your dissertation project makes up \textbf{one third} of the credits for your final year.

\subsubsection*{COMP320: Research Practice}

In study block 1, you have two deliverables: a written \textbf{research review and proposal},
and a \textbf{prototype software artefact}.

Your \textbf{research review and proposal} will give a comprehensive review of existing scientific literature
on your chosen topic. In writing this you will become familiar with the wider context of your project
and how it relates to state-of-the-art research results.
Within this framework, you will formulate a research question to be addressed by your project.
You will also demonstrate awareness of the ethical issues surrounding your project.

Your \textbf{prototype software artefact} will demonstrate your familiarity with the software tools and analysis techniques
that you will use for the remainder of the project.
The most time-consuming aspect of many projects is data collection and analysis,
especially where human participants are involved;
a solid prototype software artefact will allow you to ``hit the ground running'' in study block 2.

\subsubsection*{COMP360: Research Dissertation}

In study block 2, you again have two deliverables: a written \textbf{dissertation},
and a \textbf{software artefact}. Each of these builds on the corresponding deliverable from study block 1.

Your \textbf{dissertation} will present and discuss the results of your project.
It is likely that your research review and proposal from COMP320, with changes, will constitute roughly the first half
of your dissertation.
You will present your results in a scientifically rigorous way, and demonstrate critical reflection
on your findings and on the course of the project itself.
It is no coincidence that the document style we ask you to use for your dissertation is the same style
as used for scientific journals and conference proceedings published by IEEE ---
this is the tone and level of rigour for which you should strive.

Your \textbf{software artefact} will refine the prototype from study block 1.
The nature of the software artefact will vary dramatically between projects,
however your work will be assessed on the usual software engineering criteria of writing
functionally coherent and maintainable code, making sophisticated use of appropriate tools and techniques.

\subsubsection*{Assessment weighting}

As noted above, the project constitutes a total of 40 credits, or one third of the credits for your final year.
Thus the expectation is that the project will constitute approximately \textbf{400~hours} of work 
(including supervision meetings and timetabled sessions as well as independent study).

In both modules, the written work is worth 70\% of the marks, and the software artefact 30\%.
Thus it is vitally important that you allocate ample time for writing up, and do not treat it as an afterthought!
