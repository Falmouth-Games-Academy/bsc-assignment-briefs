\chapter{Assignment Structure for COMP360}

\section*{Introduction}

These assignments form the second part of your \textbf{major research project}; whether \textbf{empirical research} relating to computing for games, or \textbf{practice-based research} in game development. It is your opportunity to collect and analyse primary data and prepare a manuscript to disseminate your findings.

Together, they are formed of several parts:

\begin{enumerate}[label=(\Alph*)]
	\item \textbf{Complete} your computing artefact, ensuring that you:
		\begin{enumerate}[label=(\roman*)]
			\item \textbf{apply} a rigorous software project management approach;
			\item \textbf{follow} best practices in software engineering;
			\item \textbf{and clearly demonstrate} validation, verification, testing, and refactoring;
		\end{enumerate}
	\item \textbf{Conduct} the research required to answer your research question,
		ensuring that you:
		\begin{enumerate}[label=(\roman*)]
			\item \textbf{apply} a rigorous research method;
			\item \textbf{leverage} your research artefact appropriately;
			\item and \textbf{conform} with all ethical requirements.
		\end{enumerate}
	\item \textbf{Deliver} a 10-minute presentation that will:
		\begin{enumerate}[label=(\roman*)]
			\item \textbf{analyse and synthesise} your findings.
		\end{enumerate}
	\item \textbf{Write} a draft 12-page academic dissertation in IEEE format
		that will:
		\begin{enumerate}[label=(\roman*)]
			\item \textbf{identify} and \textbf{review} the scientific literature relevant to your project;
			\item \textbf{outline and justify} one or more research questions;
			\item \textbf{outline and justify} the methodology that was applied in addressing the research question(s);
			\item \textbf{analyse} data you have obtained;
			\item as well as \textbf{interpret and discuss} your findings.
		\end{enumerate}
	\item \textbf{Prepare} the final dissertation \textbf{and} appendicies in IEEE format that will:
		\begin{enumerate}[label=(\roman*)]
			\item \textbf{address} any issues raised through peer-review;
			\item \textbf{attach} the \texttt{R} code used to analyse the data;
			\item \textbf{evaluate} the outcome of the project through a reflective addendum;
			\item \textbf{and clearly demonstrate} the way you validated, verified, and tested your computing artefact;
		\end{enumerate}
	\item \textbf{Deliver} a 10-minute presentation that will:
		\begin{enumerate}[label=(\roman*)]
			\item \textbf{showcase} the final computing artefact;
			\item and \textbf{defend} the claims you have made in the academic dissertation.
		\end{enumerate}
\end{enumerate}

\subsection*{Assignment Setup}

These assignments form a \textbf{research} task, consisting of \textbf{academic reading}, \textbf{academic writing},
\textbf{software engineering}, and \textbf{scientific experimentation}.

Please continue to use the BitBucket repositories that you setup earlier, having forked:

\indent \url{https://gamesgit.falmouth.ac.uk/scm/fyp/dissertation.git}

\indent \url{https://gamesgit.falmouth.ac.uk/scm/fyp/computing-artefact}

\subsection*{Part A}

Part A consists of a \textbf{single summative submission}.
However, your supervisor may set additional \textbf{formative submissions} at their discretion.
This work is \textbf{individual} and will be assessed on a \textbf{holistic} basis, according to the descriptors set out at the end of this document.

To complete Part A, implement your research artefact and show it to your research supervisor in a timetabled meeting. As
the requirements for the research artefact will vary by project, consult with your supervisor to verify whether or not the
artefact is adequete for the desired purpose.

You will receive immediate \textbf{informal feedback} from your supervisor.

Then, upload a \texttt{.zip} file containing the final version of your source code and any assets/dependencies to the LearningSpace.
Note that LearningSpace will only accept a single \texttt{.zip} file.

You will receive \textbf{formal feedback} three weeks after the summative deadline.

\subsection*{Part B}

Part B consists of a \textbf{single formative submission}.
This work is \textbf{individual} and will be assessed on a \textbf{threshold} basis.

To complete Part B, collect data and codify it into a digital dataset. Show this to your research supervisor in a timetabled meeting. 
As the research questions and methods will vary, consult with your supervisor to verify whether or not the
data is adequete for the desired purpose.

You will receive immediate \textbf{informal feedback} from your supervisor.

\todo[inline]{\textbf{Important:} Carrying out \textbf{any} data collection or experimental work
without first securing ethical approval is a breach of Falmouth University's Research Ethics Policy.
This will be treated as a case of academic misconduct, and penalised accordingly.}

\subsection*{Part C}

Part C consists of a \textbf{single formative submission}.
This work is \textbf{individual} and will be assessed on a \textbf{threshold} basis.
To pass, answer the following questions:

\begin{enumerate}[label=(\roman*)]
	\item What is (are) the key research question(s) that you will seek to answer in your project?
	\item How will answering these questions contribute to the state of knowledge in the field of your project?
	\item Why have you applied particular methods in seeking answers to these questions?
	\item What results have you obtained?
	\item How have you analysed and interpreted these results?
	\item What are you likely to conclude on the basis of this research?
	\item What are the potential implications of your discoveries?
\end{enumerate}

To complete Part C, prepare a 10-minute presentation and deliver it in the scheduled session. Prepare your slide-deck using your choice of presentation software (e.g.\ Beamer, reveal.js, PowerPoint). You may extend your existing slides from COMP320, but \textbf{must} incorporate new material and new findings.

You will receive immediate \textbf{informal feedback} from your tutors.

\subsection*{Part D}

Part D consists of a \textbf{single formative submission}. 
However, your supervisor may set additional \textbf{formative submissions} at their discretion.
This work is \textbf{individual} and will be assessed on a \textbf{threshold} basis.

To complete Part D, prepare a draft of the dissertation. This should \textit{build upon} and \textit{extend} your research proposal and
literature review. As such, overlap with the COMP320 submission is expected. However, there should be new material.
Use the provided LaTeX template and do \textbf{not} modify the formatting. Bring these to meetings with your supervisor.

You will receive immediate \textbf{informal feedback} from your supervisor.

Then, bring \textbf{two physical print-outs} of your dissertation to the timetabled peer-review session. 
You are advised to consider the specified page limit and attempt to adhere to it at this stage; although, you will \textbf{not} be penalised at this point for failing to do so.  Also, ensure you also bring pens and highlighters.

You will receive immediate \textbf{informal feedback} from peers and \textit{in the following week} a \textbf{meta-review} from tutors.

\subsection*{Part E}

Part E consists of a \textbf{single summative submission}. However, your supervisor may set additional \textbf{formative submissions} at their discretion.
This work is \textbf{individual} and will be assessed on a \textbf{holistic} basis, according to the descriptors set out at the end of this document.

To complete Part E, revise your manuscript to address any issues highlighted during the peer review.
Your document must not exceed \textbf{twelve pages} of text, excluding figures, tables, references and appendices.
This is subject to the policy on word and page limits available on the LearningSpace course page.

As an addendum to the research manuscript including: (i) an appendix containing any \texttt{R} code used for statistical data analysis; (ii) an appendix containing up to \textbf{TWO} pages of critical reflection and self-evaluation should be incorporared; (iii) an appendix containing supplementary evidence of the way the computing artefact was validated, verified, and tested.

Upload your final \texttt{.pdf} file to the LearningSpace.
Note that LearningSpace will only accept a single \texttt{.pdf} file.

You will receive \textbf{formal feedback} three weeks after the formative deadline.

\subsection*{Part F}

Part F consists of a \textbf{single formative submission}.
This work is \textbf{individual} and will be assessed on a \textbf{threshold} basis.
To pass, answer the following questions:

\begin{enumerate}[label=(\roman*)]
	\item What was the purpose of your research?
	\item How did you approach the research?
	\item What did you discover?
	\item What are the implications of your findings?
\end{enumerate}

To complete Part F, prepare a 10-minute presentation in the timetabled session after the submission deadline.
Prepare your slides using your choice of presentation software (e.g.\ Beamer, reveal.js, PowerPoint).

You will receive immediate \textbf{informal feedback} from tutors.

\section*{FAQ}

\begin{itemize}
	\item 	\textbf{What is the deadline for this assignment?} \\ 
    		Falmouth University policy states that deadlines must only be specified on the MyFalmouth system.
    		
	\item 	\textbf{What should I do to seek help?} \\ 
    		You can email your tutor for informal clarifications. For informal feedback, make a pull request on GitHub. 
    		
    	\item 	\textbf{Is this a mistake?} \\ 	
    		If you have discovered an issue with the brief itself, the source files are available at: \\
    		\url{https://github.com/Falmouth-Games-Academy/bsc-assignment-briefs}.\\
    		 Please make a pull request and comment accordingly.
\end{itemize}

\section*{Additional Resources}

\begin{itemize}
   \item \url{http://www.bcs.org/category/6030}
   \item \url{http://www.bcs.org/content/ConWebDoc/39988}
   \item Turk, C. and Kirkman, J. (2001) Effective writing: improving scientific, technical and business communication. 2nd edition. London: Spon.
   \item Sides, C. (1999) How to write and present technical information. 3rd edition. Cambridge: Cambridge University Press.
   
   
\end{itemize}

