\documentclass{../../../fal_assignment}
\graphicspath{ {../../../} }

\usepackage{amsmath}
\usepackage{enumitem}
\setlist{nosep} % Make enumerate / itemize lists more closely spaced
\usepackage[T1]{fontenc} % http://tex.stackexchange.com/a/17858
\usepackage{url}
\usepackage{todonotes}
\usepackage{hyperref}

\title{Worksheet 2: Computer Graphics \& VFX}
\author{Brian McDonald}
\module{COMP280}
\version{1.0}

\begin{document}

\maketitle


\section*{Introduction}

In this worksheet, you will implement a VFX techniques which will demonstrate your understanding of the graphics pipeline.

Begin by \textbf{forking} the following git repository:

\begin{center}
	\url{https://github.com/Falmouth-Games-Academy/comp280-worksheet-3}
\end{center}

To complete this worksheet:

\begin{enumerate}
	\item \textbf{Fork} the repository linked above.
	\item \textbf{Implement} a basics VFX technique, this must include a procedural mesh element.
	\item \textbf{Iterate} on your VFX technique.
	\item \textbf{Open a pull request} by 6th of December to receive formative feedback.
\end{enumerate}

\section*{Additional guidance}

VFX in video games is a combination of art and programming practice, it is useful to have an understanding of how the graphics pipeline and how to exploit its feature to achieve an effect

The following are some ideas for VFX techniques you could implement

\begin{itemize}
	\item Fireball
	\item Deflector Shield 
	\item Liquid
	\item Flames
	\item Basic Animation (waving flags)
\end{itemize}

For more guidance, please read the over-arching assignment brief for this assignment on LearningSpace.

\begin{markingrubric}
	\firstcriterion{VFX Technique}{60\%}
        \grade\fail There is no VFX Technique implemented.
		\grade The VFX technique is very basic (e.g. some basic animation) and the procedural mesh is a basic primitive.
		\grade The VFX technique show efforts towards sophistication, but the procedural mesh is a basic primitive.
		\grade The VFX technique is moderately sophisticated, with some effort to link it into a game context (e.g. Fireball is cast by a player character).
		\grade The VFX technique is sophisticated, well thought out, and is linked to a game action.
		\grade The VFX technique is extremely sophisticated, well thought out, and would fit well into a game.

    \criterion{Sophistication and maintainability}{40\%}
        \grade \fail The project shows little or no understanding of relevant techniques and tools.
			\par Code is very poorly structured and commented, to the point of being unreadable.
        \grade The project shows basic understanding of relevant techniques and tools.
			\par Code is readable but lacking in structure and commenting.
        \grade The project shows some knowledge of relevant techniques and tools.
			\par Efforts towards structuring and commenting of code.
        \grade The project shows good knowledge of relevant techniques and tools.
			\par Code is well structured and commented.
			\par Some parameters of the VFX technique are easily tunable outside the code.
        \grade The project shows extensive knowledge of relevant techniques and tools.
			\par Code is very well structured and commented.
			\par Many parameters are easily tunable outside the code.
        \grade The project shows mastery of relevant techniques and tools.
			\par Code is extremely well structured and commented.
			\par All relevant parameters are easily tunable outside the code
\end{markingrubric}

\end{document}
