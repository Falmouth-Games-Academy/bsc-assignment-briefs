\documentclass{../../fal_assignment}
\graphicspath{ {../../} }

\usepackage{amsmath}
\usepackage{enumitem}
\setlist{nosep} % Make enumerate / itemize lists more closely spaced
\usepackage[T1]{fontenc} % http://tex.stackexchange.com/a/17858
\usepackage{url}
\usepackage{todonotes}
\usepackage{hyperref}

\title{Worksheet 2: Artificial Intelligence with Unreal and Behaviour Trees}
\author{Ed Powley}
\module{COMP280}
\version{1.0}

\begin{document}

\maketitle

\marginpicture{pacman}{
    \emph{Pac-Man} is a classic example of relatively simple AI behaviours giving rise to a compelling gameplay experience.
}

\section*{Introduction}

In this worksheet, you will implement a set of AI behaviours for the enemies in a simple game.

Begin by \textbf{forking} the following git repository:

\begin{center}
	\url{https://github.com/Falmouth-Games-Academy/comp280-worksheet-2}
\end{center}

This repository contains an Unreal project based on the classic arcade game \emph{Pac-Man}.
The player uses the WASD keys to run around a maze collecting pills.
The maze is also inhabited by four ghosts:
these will kill the player upon contact, unless a power pill is active, in which case the player kills the ghost.
In the template project, the ghosts (implemented in the \texttt{GhostCharacter} class) stay still in the centre of the maze;
your task for this worksheet is to implement suitable AI behaviours.

To complete this worksheet:

\begin{enumerate}
	\item \textbf{Fork} the repository linked above.
	\item \textbf{Implement} a basic Behaviour Tree based AI controller for the ghosts.
		A good starting tutorial is available at \url{https://wiki.unrealengine.com/Blueprint_Behavior_Tree_Tutorial}.
	\item \textbf{Iterate} on your AI to produce suitably complex and engaging behaviours.
	\item \textbf{Open a pull request} by 22nd November 2019 to receive formative feedback.
\end{enumerate}

\section*{Additional guidance}

In many genres of game, the design of AI agents is key to delivering a good player experience.
It would be easy to create virtually unbeatable ghost AI for Pac-Man,
however this would not lead to enjoyable gameplay.

The following are some ideas for how you might make your solution more interesting (all of these are optional suggestions):

\begin{itemize}
	\item Give each of the ghosts its own distinctive behaviour. (Note that if you choose to subclass \texttt{GhostCharacter} to allow for this,
	you can edit the \texttt{GhostSpawner}s in the level to cause them to spawn instances of your new class.)
	\item Have the ghosts behave differently based on whether a power pill is active.
	\item Give the ghosts limited perception so that they have to find the player.
	\item Have the ghosts take the state of the game, e.g.\ locations of pills left, into account.
\end{itemize}

For more guidance, please read the over-arching assignment brief for this assignment on LearningSpace.

\begin{markingrubric}
	\firstcriterion{Basic competency threshold}{30\%}
		\grade\fail	A reasonable attempt at the worksheet was not submitted by the formative deadline.
	\gradespan{5}{A reasonable attempt at the worksheet was submitted by the formative deadline.
		\par		There is no evidence of academic misconduct.}
		
    \criterion{Behaviour design}{40\%}
        \grade\fail AI behaviours are non-existent or are extremely basic.
		\grade AI behaviours are basic, but the gameplay experience is not well balanced.
		\grade AI behaviours show efforts towards sophistication, but the gameplay experience is not well balanced.
		\grade AI behaviours are moderately sophisticated, with efforts towards a well balanced gameplay experience.
		\grade AI behaviours are sophisticated, well thought out, and give a well balanced gameplay experience.
		\grade AI behaviours are extremely sophisticated, well thought out, and give a very well balanced gameplay experience.

    \criterion{Sophistication and maintainability}{30\%}
        \grade \fail The project shows little or no understanding of relevant techniques and tools.
			\par Code is very poorly structured and commented, to the point of being unreadable.
        \grade The project shows basic understanding of relevant techniques and tools.
			\par Code is readable but lacking in structure and commenting.
        \grade The project shows some knowledge of relevant techniques and tools.
			\par Efforts towards structuring and commenting of code.
        \grade The project shows good knowledge of relevant techniques and tools.
			\par Code is well structured and commented.
			\par Some parameters to the AI are easily tunable outside the blueprints / behaviour trees.
        \grade The project shows extensive knowledge of relevant techniques and tools.
			\par Code is very well structured and commented.
			\par Many parameters are easily tunable outside the blueprints / behaviour trees.
        \grade The project shows mastery of relevant techniques and tools.
			\par Code is extremely well structured and commented.
			\par All relevant parameters are easily tunable outside the blueprints / behaviour trees.
\end{markingrubric}

\end{document}
