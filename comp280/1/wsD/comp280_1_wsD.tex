\documentclass{../../../fal_assignment}
\graphicspath{ {../../../} }

\usepackage{enumitem}
\setlist{nosep} % Make enumerate / itemize lists more closely spaced
\usepackage[T1]{fontenc} % http://tex.stackexchange.com/a/17858
\usepackage{url}
\usepackage{todonotes}

\title{Networking Challenge}
\author{Gareth Lewis}
\module{COMP280}
\version{2.0.1}

\begin{document}

\maketitle

\section*{Introduction}

\begin{marginquote}
``...a folk definition of insanity is to do the same thing over and over again and to expect the results to be different. By this definition, we in fact require that programmers of multithreaded systems be insane. Were they sane, they could not understand their programs.''

--- Edward A. Lee

\marginquoterule

``No one can write correct programs in a language where a=a+1 is not deterministic.''

\par --- Luiz Henrique de Figueiredo

\marginquoterule

``Frameworks don't solve scalability problems, design solves scalability problems.''

\par --- Ryan Tomayko

\end{marginquote}
\marginpicture{flavour_pic}{
    Multi-threading is commonly used to improve performance in games.
}

In this assignment, you are required to design and implement algorithms that process data in a \textit{distributed} manner.

Games are resource intensive. Compounding this issue, players are sensitive to performance issues. It is critical, then, to leverage available resources to ensure adequate performance. Distributed processing is one solution. Apply the principles of coordination and agreement, and you will be successful. 

This assignment is formed of several parts:

\begin{enumerate}[label=(\Alph*)]
    \item \textbf{Design} a distributed processing architecture in UML for a MUD that will:
    	\begin{enumerate}[label=\roman*.]
    		\item \textbf{support} multiple client instances on a single computer;
    		\item \textbf{enable} players to navigate \textbf{multiple} locations in a virtual dungeon;
    		\item \textbf{allow} players to be aware of other players in the same location;
    		\item and \textbf{permit} players in the same room to communicate.
	\end{enumerate}
    \item \textbf{Implement} a MUD prototype that will:
    	\begin{enumerate}[label=\roman*.]
    		\item \textbf{support} multiple clients using socket-based networking;
    		\item \textbf{incorporate} distributed processing using threads;
     		\item and be \textbf{realised} as a client-server architecture.
	\end{enumerate}
    \item \textbf{Implement} a more refined design \textbf{and} MUD prototype that will:
    	\begin{enumerate}[label=\roman*.]
    		\item \textbf{revise} any issues raised by your tutor and/or your peers.
	\end{enumerate}
    \item \textbf{Present} a practical demonstation of the MUD prototype that will:
    	\begin{enumerate}[label=\roman*.]
    		\item \textbf{show} academic integrity \textbf{and} technical communication skills.
	\end{enumerate}
\end{enumerate}

\subsection*{Assignment Setup}

This assignment is a \textbf{programming task}. Fork the GitHub repository at:

\indent \url{https://github.com/Falmouth-Games-Academy/comp260-server}

Use the existing directory structure and, as required, extend this structure with sub-directories. Ensure that you maintain the \texttt{readme.md} file.

Modify the \texttt{.gitignore} to the defaults for \textbf{C\#} and \textbf{Visual Studio}. Please, also ensure that you add editor-specific files and folders to \texttt{.gitignore}. 

\subsection*{Part A}

Part A consists of a \textbf{single formative submission}. This work is \textbf{individual} and will be assessed on a \textbf{threshold} basis. This deliverable is not assessed and is intended to be advisory at this stage.

To complete Part A, incorporate the design, using UML, into the \texttt{readme.md} document.  Show this to your tutor in-class.  If acceptable, it will be signed-off. 

You will receive immediate \textbf{informal feedback} from your \textbf{tutor}.

\subsection*{Part B}

Part B is a \textbf{single formative submission}. This work is \textbf{individual} and will be assessed on a \textbf{threshold} basis. The following criteria are used to determine a pass or fail:

\begin{enumerate}[label=(\alph*)]
	\item Submission is timely;
	\item Enough work is available to conduct a meaningful review;
	\item A broadly appropriate review of a peer's work is submitted.
\end{enumerate}

To complete Part B, prepare a draft version of the MUD. Please ensure that the source code and related assets are pushed to GitHub and are made available prior to the scheduled peer-review workshop. Then, attend the scheduled session.

You will receive immediate \textbf{informal feedback} from your \textbf{peers}.

\subsection*{Part C}

Part C is a \textbf{single summative submission}. This work is \textbf{individual} and will be assessed on a \textbf{criterion-referenced} basis. Please refer to the marking rubric at the end of this document for further detail.

To complete Part C, revise the MUD based on the feedback you have received. Then, upload it to the LearningSpace. Ensure that you include the \texttt{readme.md} document containing the design that you developed in Part A. Please note, the LearningSpace will only accept a single \texttt{.zip} file.

You will receive \textbf{formal feedback} from your \textbf{tutor} three weeks after the final submission deadline.

\subsection*{Part D}

Part D is a \textbf{single summative submission}. This work is \textbf{individual} and will be assessed on a \textbf{threshold} basis.  The following criteria are used to determine a pass or fail:

\begin{enumerate}[label=(\alph*)]
	\item Enough work is available to hold a meaningful discussion;
	\item Clear evidence of programming knowledge \textbf{and} communication skills;
	\item No breaches of academic integrity.
\end{enumerate}

To complete Part D, prepare a practical demonstration of the computer programs. Ensure that the source code and related assets are pushed to GitHub and a pull request is made prior to the scheduled viva session. Then, attend the scheduled viva session.

You will receive immediate \textbf{informal feedback} from your \textbf{tutor}.

\section*{Additional Guidance}

A common pitfall is poor planning or time management. Many underestimate the work involved in designing and implementing multiplayer games. It simply cannot be crammed into a last minute deluge just before a deadline. There is a critical and time-consuming phase of testing! It is, therefore, very important that you begin work early and sustain a consistent pace: little and often.

The first deadline is  close to the start of the module and not much material will have been covered by this point. Please rest assured. This first formative submission is supposed to be a simple analysis of design. It is advisory---and a way to kick start the project such that you recieve early feedback to give you some direction and to encourage you to practice your programming skills.

\section*{FAQ}

\begin{itemize}
	\item 	\textbf{What is the deadline for this assignment?} \\ 
    		Falmouth University policy states that deadlines must only be specified on the MyFalmouth system.
    		
	\item 	\textbf{What should I do to seek help?} \\ 
    		You can email your tutor for informal clarifications. For informal feedback, make a pull request on GitHub. 
    		
    	\item 	\textbf{Is this a mistake?} \\ 	
    		If you have discovered an issue with the brief itself, the source files are available at: \\
    		\url{https://github.com/Falmouth-Games-Academy/bsc-assignment-briefs}.\\
    		 Please raise an issue and comment accordingly.
\end{itemize}

\section*{Additional Resources}

\begin{itemize}
    \item Additional resources have been migrated to the Talis Aspire system, which is available at: \url{http://resourcelists.falmouth.ac.uk/}.
\end{itemize}

\rubricyeartwo
%\rubrichead{Criteria marked with a $\ddagger$ are shared by the group. Criteria marked with a $\dagger$ are weighted by individual contribution to a shared deliverable and viva performance. All other criteria are individual.}
\begin{markingrubric}
%
    \firstcriterion{Threshold}{40\%}
        \gradespan{1}{\fail At least one part is missing or is unsatisfactory.}
        \gradespan{5}{Parts A---D are complete and timely.
        	\par Enough work is available to hold a meaningful discussion.
        	\par Provided a meaningful review of a peer's work.
	\par Clear evidence of programming knowledge and communication skills.
	\par Apprioriate use of GitHub for version control.
	\par No breaches of academic integrity.}
%
    \criterion{Design}{20\%}
        \grade\fail 	Little to no design work.
            \par 		Design does not incorporate concurrency and/or a databasing.
          \par 		Little to no UML.
          
        \grade 		Design has some merit.
            \par 		Design sufficiently incorporates concurrency and databasing. 
          \par 		Sufficient use of UML.
          
        \grade 		Design has modest merit.
            \par 		Design adequetely incorporates concurrency and databasing.  
          \par 		UML and other notation is leveraged adequetely.
            \par 		Scale and performance are briefly mentioned.
            
        \grade 		Design has much merit.
            \par 		Design appropriately incorporates concurrency and databasing.  
          \par 		UML and other notation is leveraged somewhat appropriately.
            \par 		Scale and performance are considered.
            
        \grade 		Design has considerable merit.
            \par 		Design, with high appropriateness, incorporates concurrency and databasing.  
          \par 		UML and other notation is leveraged appropriately.
            \par 		Issues such as scale and performance, with reference to CAP theorem, are considered.
            
        \grade 		Design has significant merit.
            \par 		Design, with high appropriateness, incorporates concurrency and databasing.  
          \par 		UML and other notation is leveraged highly appropriately.
            \par 		Issues such as scale and performance, with reference to CAP theorem, are carefully considered.

%
    \criterion{MUD Architecture}{20\%}
        \grade\fail Little to no evidence of distributed processes.

        \grade Some distributed processes.
	\par  Player chat system and/or space navigation operable.
	\par  Supports at least two client instances.
        \grade Modest number of distributed processes.
	\par  Player chat system and space navigation operable.
	\par  Basic commands implemented.
	\par  Supports serveral client instances.
        \grade Many distributed processes.
	\par  Player chat system and space navigation implemented well.
	\par  Interesting commands implemented.
	\par  Evidence to suggest scalability to support many client instances.
        \grade Considerable number of distribued processes.
	\par  Player chat system and space navigation implemented effectively.
	\par  Sophisticated commands implemented.
	\par  Evidence to suggest scalability to support a great many client instances.
        \grade Significant number of distributed processes.
	\par  Player chat system and space navigation implemented highly effectively.
	\par  Highly sophisticated commands implemented.
	\par  Evidence to suggest scalability to support hundreds of client instances.
%
    \criterion{Demo}{20\%}
        \grade\fail No demo.
            \par Little to no ability to articulate either networking or concurrency concepts.
        \grade Demo somewhat sufficient to illustrate key distributed processing concepts.
            \par Some ability to articulate either networking or concurrency concepts.
        \grade Demo sufficient to illustrate key distributed processing concepts.
            \par Modest ability to articulate either networking or concurrency concepts.
            \par Some ability to articulate design decisions.
        \grade Demo adequete to illustrate key distributed processing concepts.
            \par Much ability to articulate either networking or concurrency concepts.
            \par Modest ability to articulate design decisions.
        \grade Demo appropriate to illustrate key distributed processing concepts.
            \par Considerable ability to articulate either networking or concurrency concepts.
            \par Much ability to articulate design decisions.
        \grade Demo highly appropriate to illustrate key distributed processing concepts.
            \par Significant ability to articulate either networking or concurrency concepts.
            \par Considerable ability to articulate design decisions.
\end{markingrubric}

\end{document}