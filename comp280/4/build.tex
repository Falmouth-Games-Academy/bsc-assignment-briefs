\documentclass{../../fal_assignment}
\graphicspath{ {../../} }

\usepackage{enumitem}
\setlist{nosep} % Make enumerate / itemize lists more closely spaced
\usepackage[T1]{fontenc} % http://tex.stackexchange.com/a/17858
\usepackage{url}
\usepackage{todonotes}
\usepackage{hyperref}
\hypersetup{
    colorlinks=true,
    linkcolor=blue,
    filecolor=magenta,      
    urlcolor=cyan,
}

\title{Worksheet 4: Interfaces and Interactions}
\author{ Lucy Stent	}
\module{COMP280}
\version{1.0}

\begin{document}

\maketitle

\section*{Introduction}

\begin{marginquote}
``Design is really an act of communication, which means having a deep understanding of the person with whom the designer is communicating.''
\par --- \'Donald A. Norman
\end{marginquote}
%\marginpicture{flavour_pic}{
 %   \emph{SpaceChem} is a puzzle game in which players must apply computational thinking to build circuits which assemble
 %   chemical molecules.
%}

To complete this worksheet, you are expected to rapid prototype a simple user Interface (chosen from the game contract list available from the LearningSpace) in Unreal Engine or Unity. Once complete, the interface will be subject to a heuristic evaluation and then improved upon to accommodate the findings of the evaluation. 

To complete this worksheet: 

\begin{enumerate}[label=(\roman*)]
    \item \textbf{Design} HUDs for a Game Contract using Adobe XD or similar. The HUDs you must create are:
    \begin{itemize}
    	\item Pause Menu
	\item Main Menu
	\item Game HUD
	\item Controls window
	\item \textbf{Stretch Goal} - Game Contract Specific Menu
    \end{itemize}
    \item \textbf{implement} the HUDs in Unreal Engine or Unity
    \item \textbf{carry out} a heuristic evaluation on the HUD 
    \item \textbf{enhance} the initial HUD using the findings from the heuristic evaluation
    \item \textbf{STRETCH GOAL: } if you finish this worksheet early, ask a peer to evaluate your interface and use the new results to inform a second iteration of your interface. 
\end{enumerate}

\subsection*{Assignment Setup} 

Begin by \textbf{forking} the following BitBucket repository: 

\hyperlink{https://gamesgit.falmouth.ac.uk/projects/COMP280/repos/worksheet-4/browse}{https://gamesgit.falmouth.ac.uk/projects/COMP280/repos/worksheet-4/browse} 

Use this repo to track all files related to this worksheet. Once you have completed the worksheet, make a pull request to receive formative feedback. 

To fulfil the requirements of this worksheet you are expected to submit evidence of:

\begin{itemize}
	\item an initial interface design in the form of interactive prototype;
	\item screenshots of the first implementation in engine;
	\item the results from a heuristic evaluation;
	\item and then screenshots of the final interface with enhancements based on the results from the heuristic evaluation
\end{itemize}


\section*{FAQ}

\begin{itemize}
	\item 	\textbf{What is the deadline for this assignment?} \\ 
			Each worksheet has its own formative deadline, specified on that worksheet and also communicated in class.
    		Falmouth University policy states that summative deadlines must only be specified on the MyFalmouth system.
    		
	\item 	\textbf{What should I do to seek help?} \\ 
    		You can email your tutor for informal clarifications. For informal feedback, make a pull request on GitHub. 
    		
	\item 	\textbf{How will I receive feedback on my work?} \\ 
    		You will be given verbal feedback on your work during the session in which it is marked.
    		If you require more in-depth feedback or discussion, please book an appointment with your tutor.
    		
    	\item 	\textbf{Is this a mistake?} \\ 	
    		If you have discovered an issue with the brief itself, the source files are available at: \\
    		\url{https://github.com/Falmouth-Games-Academy/bsc-assignment-briefs}.\\
    		 Please make a pull request and comment accordingly.
\end{itemize}

\rubrichead{Each individual worksheet is marked according to this rubric.}
\rubricfoot{Each worksheet is worth 11.1\% of the total marks for this assignment.}

\begin{markingrubric}
	\firstcriterion{Functionality}{30\%}
		\grade\fail	There is no functioning user interface
		\grade{There is one user interface with limited functionality}
		\grade{There is two user interfaces with some functionality}
		\grade{There is three user interfaces with good functionality}
		\grade{There is four user interfaces with good functionality}
		\grade{There is four user interfaces with much functionality}

		
    \criterion{Heuristic Evaluation}{50\%}
    		\grade\fail	There is no evidence of an evaluation
		\grade{There is evidence of a very basic evaluation}
		\grade{There is evidence of a basic evaluation}
		\grade{There is evidence of a good evaluation}
		\grade{There is evidence of a good evaluation with academic references}
		\grade{There is  evidence of a very good evaluation with many academic references}

\criterion{Depth of insight}{20\%}
    		\grade\fail	None of the issues identified in the evaluation have been addressed
		\grade{Few of the issues identified in the evaluation have been addressed}
		\grade{Some of the issues identified in the evaluation have been addressed}
		\grade{Many of the issues identified in the evaluation have been addressed}
		\grade{All of the issues identified in the evaluation have been addressed }
		\grade{A second iteration of the evaluation process has been carried out}

\end{markingrubric}

% https://www.reddit.com/r/unrealengine/comments/a7w2vq/how_to_design_a_fps_hud/
% https://www.youtube.com/watch?v=5Gt444eN7Rg&list=PLSlkDq2rO1t5r1tj6gLWD_EesQ8IZtN_h

\end{document}