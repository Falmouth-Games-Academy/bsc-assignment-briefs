\chapter{British Computer Society Requirements}

An individual project is an expectation within undergraduate, integrated masters, and postgraduate masters programmes in computing. Students must be provided with written guidance on all aspects of the project, including selection,conduct, supervision, milestones, format of the report and the criteria for assessment. All projects should reflect the aims and learning outcomes which characterise the programme to which they contribute as set out in the programme specification.

It is expected that within an undergraduate programme, students will undertake a major computing project, normally in
their final year and normally as an individual activity, giving them the opportunity to demonstrate:

\begin{itemize}
\item their ability to apply practical and analytical skills present in the programme as a whole
\item innovation and/or creativity
\item synthesis of information, ideas and practices to provide a quality solution together with an evaluation of that solution
\item that their project meets a real need in a wider context
\item the ability to self-manage a significant piece of work
\item critical self-evaluation of the process
\end{itemize}

Projects must involve the production of a report which should include:

\begin{itemize}
\item Elucidation of the problem and the objectives of the project
\item an in-depth investigation of the context and literature, and where appropriate, other similar products (this section is
likely to be emphasised less for an IEng project)
\item where appropriate, a clear description of the stages of the life cycle undertaken
\item where appropriate, a description of how verification and validation were applied at these stages
\item where appropriate, a description of the use of tools to support the development process
\item a critical appraisal of the project, indicating the rationale for any design/implementation decisions, lessons learnt
during the course of the project, and evaluation (with hindsight) of the project outcome and the process of its
production (including a review of the plan and any deviations from it)
\item a description of any research hypothesis
\item in the event that the individual work is part of a group enterprise, a clear indication of the part played by the author in
achieving the goals of the project and its effectiveness
\item references 
\end{itemize}

In the event of this major activity being undertaken as part of a group enterprise, there is a requirement that the
assessment is such that the individual contribution of each student is measured against all the above learning
outcomes.

For accreditation for CITP, CEng or CSci, the individual project should be worth at least 30 credit points at level 6 or
above. The project must be passed without compensation.
For accreditation for IEng the individual project should be worth at least 20 credit points at level 5 or above. The project
must be passed without compensation. 
