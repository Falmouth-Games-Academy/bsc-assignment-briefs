\chapter{Guidance on Completing your Project}

\section*{Ethical Clearance}

Computing professionals are expected to follow the ethical standards that have been outlined by the British Computer Society (BCS), The Chartered Institute for IT. \textbf{All} students are expected to familiarize themselves with the BCS Code of Conduct. Although scientific work at undergraduate level is unlikely to raise serious ethical concerns, there are many topics pertinent to games where Sections 1 and 2 of the Code of Conduct and the Interpretation of the BCS Code of Conduct are likely to apply. Research artefacts with the potential for practical application or future commercialisation are one example and will normally also have wider ethical implications for you to consider. As such, heed the points in Section 2 on technological procedures and standards as well as those on complying with legislation.

\textbf{All} students must follow Falmouth University's Research Ethics Policy. In practice, this means you need to complete the Research Ethics Approval Application Form and obtain ethical clearance \textbf{before} using the research artefact you have created to collect primary data. Even if collecting such data presents minimal to no risk. Furthermore, if your project involves human participants (e.g. for testing a game system you have developed), or presents a significant ethical risk (e.g., systems that process personal data) then you will need to have completed the full Research Ethics Approval Application Form available on LearningSpace. In these cases, you should also discuss these ethical issues in your dissertation and pay particular attention to how you addressed these issues in your research methodology.

If you did not obtain ethical approval at the proposal stage, email a revised version of the application form to the module leader and \textbf{do not} start to collect data until you have been advised to do so. Failure to follow the policy will constitute academic misconduct and will result in a failing grade. If you are unsure of the status of your application, contact the module leader.

\section*{Project Management}

The final year project can be quite a daunting and intimidating prospect. Many students find it challenging because of its self-directed nature and the accompanying discipline and dedication needed to follow it through. However, do not worry! Completing the project is not only intellectually liberating and a mark of your academic independence, but will provide you with a sense of achievement and satisfaction.

It will also constitute an key indicator---a symbol---of your competence to potential employers. It gives you something you can showcase in interview and discuss in considerable depth and with enthusiasm. Successful completion of such a project demands a mastery of core employability skills including: initiative; problem solving; communication, both written and spoken; self-regulated learning; as well as planning and management. Though the stretching of your ability in these areas to limits you never thought possible will likely be rather uncomfortable, you feel better for it when you graduate.

A pitfall that many students fall into, however, is time management. Minimise your procrastination and try to chip away at your work a little every day! Although your milestones will vary depending on the nature of your project, ideally the research artefact should be near its completion towards then end of January, permitting you to collect and analyse data across February and March ahead of the interim presentation. Then, in the later half of the study block focuses on interpreting and contextualising your discoveries and writing them up as an academic dissertation. Do not underestimate this process! Your dissertation will take weeks to write!

It is, therefore, critically important that you consider \textbf{project management} again at this stage. Consider the system development life-cycle that is appropriate to your project. Ensure you use appropriate project management tools including critical path analysis, Gantt charts, and burn-down charts to keep track of your progress. Also, do not underestimate the importance of the \textbf{validation and verification} aspect of the research artefact. You must ensure sufficient time is made to enact quality assurance practices that will defend the integrity of your research by showing that your research project was appropriately managed and your research artefact was constructed through the sound application of software engineering practices. 

Explicitly outline your management approach and transparently evidence how you actually tested your software!

If you encounter any issues with respect to your time management, please consult your supervisor who can provide you with advice. They are there to support you, so take advantage of their experience. Also, ensure that you take advantage of the support services offered by the Academic Skills Team (ASK): \url{http://ask.fxplus.ac.uk}

\section*{Preparing the Manuscript}

The final manuscript should be prepared following the formal IEEE Transactions template with referencing in IEEE style. The manuscript should contain 12-pages of academic content ( excluding figures, tables, the reference section, or any additional material) and a 1-page addendum that evaluates the project. This means there should be 12-pages of content. You should discuss the structure with your supervisor.

\begin{itemize}

\setlength\itemsep{1em}

\item \textbf{Front Matter}: The first page should display your name, the title of the project, the abstract, and a copyright notice. Please do not include your supervisor as a co-author (see acknowledgements). Leverage \texttt{thank\{\} } to make a statement of originality together with any intellectual property rights agreements that you have made. The statement of originality should be worded similarly to: `This dissertation is submitted as a requirement for the degree of Bachelor of Science at Falmouth University. It presents work conducted exclusively by the author except where indicated in the text. The report may be freely copied and distributed provided the source is acknowledged.'

\item \textbf{Introduction}: This section should make clear the motivation for the project. The aims of the project should at least be stated in the first paragraph, but preferably in the first sentence. Ensure to include a road-map which explains the structure of the manuscript.

\item \textbf{Body}: The content and structure of the body of your work will vary greatly depending on the nature of the project. You should not have a section called body. Instead, this should be broken down into any number of appropriate sections. For example: literature review; requirements analysis and specification;  design of the research artefact; implementation; research methodology; method of evaluating the research artefact (including how you validated and verified the integrity of your artefact); and findings. For empirical research, ensure that you include clearly identifiable sections that highlight: research questions; hypotheses; study design; results; data analysis; and a discussion that interprets and contextualises your discoveries. Consult your supervisor to develop an appropriate structure for your work.

\item \textbf{Professional Considerations}: Do not neglect the need to acknowledge the ethical and/or professional issues raised by your work. This is an opportunity to refer to the Codes of Conduct published by the BCS and other professional bodies. Ensure that you also acknowledge any provisions that you incorporated into your methodology on ethical grounds.

\item \textbf{Conclusion}: Make clear your discoveries and the key take-away points your research. Highlight any important results. If your discoveries contribute to discourse in your field of study then ensure you have explicitly stated what these contributions are and how they do so. Also, ensure that you make clear the implications of your work and if you have derived any recommendations from your results, then explicitly state these also. 

\item \textbf{Acknowledgements}: It is important to declare those who supported your research: your supervisor(s); other students if part of a related project; and any other person or organisation that has helped you at all.

\item \textbf{References}: There is no room for error. IEEE-style references with 100\% accuracy is required. Zero tolerance. Sources should be cited in the body of your report where appropriate. This includes web pages, which historically have been cited incorrectly by many students.

\item \textbf{Appendices}: you should include your project logs and any additional relevant system or test data as appendices. You may also include any technical material which you estimate as too detailed for the main body of the report. You should make sure that any technical material is appropriately annotated and consistently presented. The source code of your project (including documentation) must be submitted electronically and should not be included in the Appendix. Also, any material that does not fit in the report's Appendix that you nevertheless deem important can be included in the electronic submission. Note, however, that it will be left to the markers' discretion whether they wish to look at such extra material in the electronic submission.

\item \textbf{Reflective Addendum}: Please do not forget to include an appendix in which you evaluate your final year project. This is your opportunity to reflect on what went well and what went wrong. No project will have been executed to perfection. So long as you identify and acknowledge mistakes, showing that you learned from them, you can mitigate any of the issues. Also you should consider, in hindsight, what you would do differently to avoid such issues in the future. It is important to be critical in these self-reflections. It is \textbf{not} a description or a diary. It must be both \textbf{analytical and evaluative}. These will be assessed on a holistic basis than your CPD reports, according to the same principles and similar criteria. 

\end{itemize}

