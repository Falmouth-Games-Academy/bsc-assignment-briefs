\documentclass{../fal_assignment}
\graphicspath{ {../} }

\usepackage{enumitem}
\usepackage[T1]{fontenc} % http://tex.stackexchange.com/a/17858
\usepackage{url}
\usepackage{todonotes}

\title{Evaluation --- Teamwork Review}
\author{Dr Michael Scott}

\begin{document}

\maketitle
\begin{marginquote}
    ``It is the long history of humankind: those who learned to collaborate and improvise most effectively have prevailed''
    
    --- Charles Darwin
\end{marginquote}
\marginpicture{MakeyMakey.jpg}{
    The \emph{MaKey~MaKey} allows a multitude of materials to be used to create videogame controllers.
}
\section*{Introduction}

In this assignment, you are required to critically reflect on your teamwork and team dynamic across the COMP150 group project, using this as a lense to review how your teamwork-related skills influenced the quality of the work that the team produced. You will also develop a plan of action to mitigate any obstacles, and overcome any challenges, that you may have encountered; thereby, improving the quality of your team output in future assignments.

Such reflection and planning is an extremely important part of learning games development, and in particular computer programming within a small team. They are key components in a technique known as deliberate practice, which research has shown to be very successful at nurturing expertise in software engineering. Everyone that properly adopts this technique eventually succeeds, despite the challenging nature of the subject. It is, therefore, very important that reflection and planning are not dismissed as an afterthought at the end of the course. For this reason, this assessment has two components, A and B, as follows:

\begin{enumerate}[label=(\alph*)]
    \item A weekly team blog that must:
    	\begin{enumerate}[label=\roman*.]
    		\item be authored \textbf{collaboratively};
    		\item \textbf{describe} the progress of the team project;
    		\item \textbf{assess} any obstacles or challenges that the team has encountered;
    		\item and then \textbf{outline} some specific actions to take to overcome them.
	\end{enumerate}
    \item A final 500-word report that must:
    	\begin{enumerate}[label=\roman*.]
    		\item \textbf{identify three} key teamwork skills that you consider weaknesses;
    		\item \textbf{assess} your application of \textbf{each} of these skills, \textbf{describing how} they affected the quality of your submissions \textbf{and suggesting why} they became challenges;
    		\item and then \textbf{identify how} to improve \textbf{each} of these skills, with reference to SMART actions.
	\end{enumerate}
\end{enumerate}

Part A consists of \textbf{multiple formative collaborative submissions} with deadlines at the end of each sprint review across the Semester. This work will be assessed on a \textbf{threshold} basis. The threshold is set at 15\%. This means that 15\% of the total marks available for the coursework overall are awarded on a pass or fail basis. In other words, satisfactory submissions will be awarded 15\%. However, unsatisfactory submissions will receive 0\%.

The following criteria are used to determine a pass or fail for each submission in Part A:

\begin{enumerate}[label=(\alph*)]
	\item Progress has been described with adequate detail;
	\item Obstacles and challenges have been clearly explained and assessed;
	\item There is balance between teamwork-related issues as well as technical issues;
	\item No witch-hunting and appropriate anonymitiy afforded for genuine absences (e.g. illness);
	\item At least one SMART action, appropriate for resolving the issue, has been outlined;
\end{enumerate}

You will receive ongoing feedback at the end of each sprint review and you will have the opportunity to revise the work prior to the final deadline. It is especially important that work that is not yet satisfactory is revised based on the feedback provided as \textbf{all reports must be satisfactory to pass}.

Part B is a \textbf{single individual summative submission} and will be assessed on a \textbf{criterion-referenced} basis. This submission is expected to take students from the threshold of 15\% (F) up to the maximum of 100\% (A*). This means that 85\% of the total marks available for the coursework overall will be awarded.

The following criteria are used to allocate marks:

\begin{enumerate}[label=(\alph*)]
	\item Appropriateness and Specificity of Selection of Key Teamwork Skills;
	\item Adequacy of Self-Appraisal in Relation to Key Teamwork Skills;
	\item Depth of Reflection on Key Teamwork Skills;
	\item Appropriateness of Plan for the Future;
	\item Quality of Academic Writing;
\end{enumerate}

\section*{Submission Instructions}

\subsection*{Part A}

Part A must be completed as a formative submission on GitHub. Fork the GitHub project at the following URL:

\indent \url{https://github.com/Falmouth-Games-Academy/comp150-evaluation}

Provide a link to your weekly blog in the \texttt{readme.md} file. The blog itself should be in Markdown format (i.e., *.md file) hosted within the main comp150-game repository. Use the comp150-game repository for any other digital assets you create (e.g.\ images), checking them in regularly as you work on your projects. For the reports, images should be embedded directly in the relevant \texttt{readme.md} file. Videos should be uploaded to a video sharing site (e.g.\ YouTube, Vimeo, Vine) and linked from the \texttt{readme.md} file.

You will need to show your most recent reports to your tutor during each of your sprint review sessions, at which point they will be signed-off.

\subsection*{Part B}

Part B must be completed as a single PDF document, prepared in LaTeX. The LaTeX source files should be hosted on GitHub in the comp150-evaluation repository. A single PDF document must be submitted to the LearningSpace by the final submission deadline shown on LearningSpace. Please note that the LearningSpace will only accept a single PDF document.

You will receive formal feedback three weeks after the submission deadline shown on LearningSpace.

\begin{marginquote}
    ``Individual commitment to a group effort--that is what makes a team work, a company work, a society work, a civilization work.''
    
    --- Vince Lombardi
    
        \marginquoterule
        
    ``Coming together is a beginning. Keeping together is progress. Working together is success.''
    
    --- Henry Ford 
\end{marginquote}
\marginpicture{guitarhero}{
    Rhythm games such as \emph{Guitar Hero} and \emph{Rock Band} are excellent examples of games
    which make use of unique input devices to enhance gameplay.
}
\section*{Additional Guidance}

In previous evaluative reports, you have been expected to write in a reflective style. Please continue to use first-person and follow a similar style. However, take great care to assume an as objective position as possible. There must be sufficient analysis and reference to evidence in order to support your claims. More so than would be expected in a general reflective report.

On this note, the teamwork analysis is \textbf{not} the place where you indulge in witch-hunting. That is, a blame game. While it is inevitable that incidents occur and these may block progress (e.g. due to illness, due to different levels of productivity as a consequence of differing levels of skill or differing levels of extra-curricular committments, due to technical mishaps or unforseeable circumstances), this is not an excuse to blame other students for shortcomings in a project. This is not only poor form, but is indicative of the report's author having poor teamworking skills. Incidents should be taken as an opportunity for reflection. Why wasn't a contingency in place? Why had nobody shared responsibility for a critical user story? Why wasn't there a backup? These are opportunities for reflection and growth.

A common type of concern is lack of engagement by other members of the team. Such concerns must be sufficiently anonymised, such that no other student is identified in either the report or the blog. Moreover, the concern must be sufficiently evidenced and analysed (e.g. by the scrum-master keeping an attendence record). For example, why was the design of the game not re-scoped if it became clear that one of the specialists on the team was unlikely to deliver key components upon which a proposed design relied?

As you author your reports, this contributes to the continuous cycle of reflection and planning that forms the cornerstone of deliberate practice. Just as you should avoid treating this reflection as an afterthought, you should avoid delegating the blog to one member of the team to produce. The blog should be the output from a team discussion and review that takes place at the end of each week. 

Furthermore, do not pay too much attention on the end-product (i.e., the game itself) in the blog. A short description alongside some screenshorts or pseudocode illustrating work-in-progress is more than sufficient. It is the reflection upon and analysis of your working practice that is important. Namely, \textit{how} and \textit{why} influence quality in more abstract terms. As such, presenting your work in terms of `lessons learned' is appropriate. This will likely lead to higher quality submissions in the future and such lessons could form some of the components of your final report. Critically, however, lessons learned form a strong foundation for improving your teamwork.

As with previous evaluations, the most common mistake when planning future actions to take is being too general. It is, therefore, important to consider SMART goals: specific; measurable; achievable; relevant; and time-bound. Teamwork is critically important in the games industry, whether this be through an out-sourcing, remote collaboration, or indeed pair-programming. So, do ensure that you engage with this process of reflection and planning to improve your teamworking skills.

\begin{marginquote}
    ``Remember, teamwork begins by building trust. And the only way to do that is to overcome our need for invulnerability.''
    
    --- Patrick Lencioni
    
    \marginquoterule

        ``Talent wins games, but teamwork and intelligence win championships.''
    
    --- Michael Jordan
\end{marginquote}
\marginpicture{fishing}{
    The \emph{Dreamcast Fishing Controller}, released as a peripheral for the game \emph{Sega Bass Fishing}.
    Even peripherals which appeal to only a small audience can enjoy moderate commercial success.
}
\section*{Additional Resources}

\begin{itemize}
    \item Belbin, R.M. (2012) Team roles at work. Routledge.
    \item Baker, D.P. and Salas, E. (1992) Principles for Measuring Teamwork Skills. Human Factors, 34(4), pp.469-475.
    \item Williams, L. and Kessler, R. (2002) Pair programming Illuminated. Addison-Wesley.
\end{itemize}

\begin{markingrubric}
%
    \firstcriterion{Satisfactory Preparation of Weekly Reports}{15\%}
        \gradespan{5}{\fail At least one weekly blog post has not been submitted, is incomplete, or is unsatisfactory.}
        \grade 		All weekly blog posts have been signed-off by your tutor by the deadline.
%
    \criterion{Appropriateness, Specificity, and Relevance of Selection of Key Teamwork Skills}{10\%}
        \grade\fail 	Less than two appropriate key skills are mentioned.
        \grade 		At least two appropriate key skills are mentioned.
        \grade 		At least three appropriate key skills are mentioned.
        \grade 		At least three appropriate key skills are mentioned.
        \par 		At least two of the key skills are both specific and relevant.
        \grade 		At least three appropriate key skills are mentioned.
        \par 		At least three of the key skills are both specific and relevant.
        \grade 		At least three appropriate key skills are mentioned.
        \par 		At least two of the key skills are both specific and a priority.
%
    \criterion{Adequacy of Self-Criticism in Relation to Key Teamwork Skills}{20\%}
        \grade\fail 	No self-criticism is made.
        \grade 		Little self-criticism is made.
        \grade 		Some self-criticism is made.
        \grade 		Much self-criticism is made.
        \grade 		A significant level of self-criticism is made.
            \par 		Some of the self-criticism is accurate and pertinent.
        \grade 		An exception level of self-criticism is made.
            \par 		Much of the self-criticism is accurate and pertinent.
%
    \criterion{Depth of the Reflection on the Application of Key Teamwork Skills}{20\%}
        \grade\fail 	No reflection is evident.
        \grade 		Little reflection is evident.
        \grade 		Some reflection is evident.
        \grade 		Much reflection is evident.
        \par 		Some depth of insight is demonstrated.
        \grade 		Significant reflection is evident.
        \par 		Much depth of insight is demonstrated.
        \grade 		Exemplary reflection is evident.
        \par 		Significant depth of insight is demonstrated.
%
    \criterion{Appropriateness of Plan for Future Development}{20\%}
        \grade\fail 	No appropriate plans are proposed.
        \grade 		At least one generally appropriate plan is proposed.
        \grade 		At least two specific and achievable plans are proposed. 
        \grade 		At least three specific and achievable plans are proposed. 
        \par 		At least two of the plans are also relevant.
        \grade 		At least three specific, relevant, and achievable plans are proposed. 
        \par 		At least two of the plans are also measurable and time-bound.
        \grade 		At least three specific, measurable, achievable, relevant, and time-bound plans are proposed. 
%
    \criterion{Appropriateness of Reflective Writing Style}{5\%}
        \grade\fail 	Demonstrates no evidence of ability in reflective writing.
        \grade 		Demonstrates evidence of little ability in reflective writing.
        \grade 		Demonstrates evidence of some ability in reflective writing.  
        \grade 		Demonstrates evidence of partial mastery of reflective writing.
        \grade 		Demonstrates evidence of mastery in reflective writing.
        \grade 		Demonstrates significant evidence of mastery in reflective writing.
%
    \criterion{Appropriateness of Spelling and Grammar}{5\%}
        \grade\fail 	Substantial spelling and/or grammar errors.
        \grade 		Many spelling and/or grammar errors.
        \grade 		Some spelling and/or grammar errors.  
        \grade 		Few spelling and/or grammar errors.
        \grade 		Nearly no spelling and/or grammar errors.
        \grade 		No spelling and/or grammar errors.
%
    \criterion{Appropriateness of Essay Structure}{5\%}
        \grade\fail 	There is no structure, or the structure is unclear.
        \grade 		There is little structure.
        \grade 		There is some structure.
        \par 		A few sentences and paragraphs are well constructed.
        \grade 		There is much structure.
        \par 		Some sentences and paragraphs are well constructed.
        \par 		There is a clear introduction and conclusion.
        \grade 		There is much structure, highlighting the key skills.
        \par 		Most sentences and paragraphs are well constructed.
        \par 		There is a clear and well-constructed introduction and conclusion.
        \grade 		There is much structure, highlighting the key skills.
        \par 		All sentences and paragraphs are well constructed.
        \par 		There is a clear and well-constructed introduction and conclusion.
\end{markingrubric}

\end{document}