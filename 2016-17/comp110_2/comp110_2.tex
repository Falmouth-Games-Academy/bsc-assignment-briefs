\documentclass{../fal_assignment}
\graphicspath{ {../} }

\usepackage{enumitem}
\setlist{nosep} % Make enumerate / itemize lists more closely spaced
\usepackage[T1]{fontenc} % http://tex.stackexchange.com/a/17858
\usepackage{url}
\usepackage{todonotes}

\title{Research Journal --- Computing}
\author{Dr Ed Powley}

\begin{document}

\maketitle

\section*{Introduction}

\begin{marginquote}
  ``Individuals and Interactions over Processes and Tools''
  
  ``Working Software over Comprehensive Documentation''
  
  ``Customer Collaboration over Contract Negotiation''
  
  ``Responding to Change over Following a Plan''
    
    --- Agile Manifesto
\end{marginquote}
\marginpicture{MakeyMakey.jpg}{
    The \emph{MaKey~MaKey} allows a multitude of materials to be used to create videogame controllers.
}

In this assignment, you will produce a journal detailing your research on the principles of computing.
In addition to your individual journal, as a group you will create a community wiki to collect and discuss your findings.

Familiarity with the scientific literature is extremely helpful for the computing professional,
both to understand the seminal works that lay the foundations of the field,
and to keep abreast of recent developments at the cutting edge.
Scientific papers are written in a way that is sometimes daunting to newcomers;
it is essential to practice the skill of reading and comprehending such papers.
Keeping a research journal is a useful way to record your thoughts
(questions, hypotheses, connections, ideas, ...) as you explore the literature.

This assignment is formed of several parts:

\begin{enumerate}[label=\Alph*.]
    \item \textbf{Read} a number of papers from the scientific literature on computing.
    	One paper will be suggested in class every two weeks; this is the bare minimum that you should read.
    	You should also read around the subject areas covered by these papers.
    \item \textbf{Write} a journal entry for each paper you read. The journal entry should record your thoughts on the paper,
    	which may include but not necessarily be limited to:
    	\begin{enumerate}[label=\roman*.]
    		\item A \textbf{summary}, in your own words, of the content of your paper;
    		\item Why you think the paper is \textbf{significant} and/or \textbf{influential};
    		\item Anything in the paper that you \textbf{do not understand};
    		\item Anything in the paper that is \textbf{counterintuitive}, \textbf{surprising} or that you \textbf{disagree with};
    		\item Any \textbf{research questions} that the paper suggests to you;
    		\item Ways in which this paper \textbf{supports} or \textbf{contradicts} other papers you have read.
    	\end{enumerate}
    \item \textbf{Collate} your individual journal entries and \textbf{synthesise} them into a cohesive report.
    \item \textbf{Edit} the community wiki to share and debate your findings with your peers.
    \item \textbf{Discuss} your research journal with the tutor in the viva session in class.
\end{enumerate}

\subsection*{Assignment Setup}

This assignment is an \textbf{academic research and writing task}. Fork the GitHub repository at the following URL:

\indent \url{https://github.com/Falmouth-Games-Academy/comp110-journal}

Use the existing directory structure and, as required, extend this structure with sub-directories.

Modify the \texttt{.gitignore} to the defaults for \textbf{TeX}. Please, also ensure that you add editor-specific files and folders to \texttt{.gitignore}. 

\subsection*{Part A}

No separate submission is required for Part~A, however it is a prerequisite for completing the other parts.

\subsection*{Part B}

Part B consists of \textbf{multiple formative submissions}. This work is \textbf{individual} and will be assessed on a \textbf{threshold} basis.

To complete Part B, edit your \texttt{readme.md} file to contain a journal entry for each article you read,
as well as any other material you feel will be relevant or useful in preparing your final report.
Bring this to your personal tutor meetings to be signed off, at least once every three weeks during semester~1.

You will receive \textbf{immediate informal feedback}.

\subsection*{Part C}

Part C is a \textbf{single summative submission}. This work is \textbf{individual} and will be assessed on a \textbf{criterion-referenced} basis
using the criteria listed in the marking rubric at the end of this document.

To complete Part C, edit your \texttt{readme.md} file to contain your final report.
You should already have written much of the material for your report in Part~B;
at this stage you should concentrate on revising that material in response to feedback
and on ensuring that the material is structured in a coherent way.
How you structure your report is up to you;
however for higher grades the report should be formed around a clear thread of arguments framed by a suitable introduction and conclusion.

Compress your \texttt{readme.md} file, along with any required images or other external files,
into a \texttt{.zip} file and upload it to LearningSpace.

You will receive \textbf{formal feedback} three weeks after the final deadline.

\subsection*{Part D}

Part D consists of \textbf{multiple summative submissions}. This work is \textbf{group-based} and will be assessed on a \textbf{criterion-referenced} basis.
The criterion used to assess this part relates to the level and nature of your participation.

To complete Part D, contribute to the wiki at the following URL:

\url{https://github.com/Falmouth-Games-Academy/comp110-journal/wiki}

Please ensure that you are editing the wiki for the \texttt{Falmouth-Games-Academy} repository,
and \textbf{not} the wiki for your fork of the repository!
The wiki is intended to be a group counterpart to your individual research journal. \todo{bla}

During your personal tutor meetings, make your tutor aware of your contributions to the wiki.

You will receive \textbf{immediate informal feedback} as well as \textbf{ongoing peer feedback}.

\subsection*{Part E}

Part E is a viva, after the summative deadline. \todo{bla}

\section*{Additional Guidance}

Use your experience from the previous essays. Identify weaknesses and feed-\textit{forward}. University is an opportunity for improvement and an effective way to do this is to compare past and current performance.

Developing the research question is the most challenging aspect of this assignment. It is very unlikely that you will settle on the first research question that you propose. This is because the question will often arise out of your individual research and reading efforts. Furthermore, the question should relate to working practices for game developers. An example might be: ``how can game developers make effective use of the daily scrum?''. You will need to discuss your question with your tutor and your peers to help focus it.

Areas where students tend to lose marks are: depth of insight; analytical skill; and evaluative skill. Depth of insight implies rigorous research, addressing one key challenge in much detail, rather than several challenges with weaker research and/or in less detail. Adequete analysis implies going beyond mere descrption, perhaps through: performing calculatons, comparing sources, or even deploying reasoning to generate new insights. Adequete evaluation implies making appropriate reference to evidence and ensuring that evidence is of appropriate quality. Further to this, sound and valid arguments are constructed, criticising the claims made by other authors.

Focus on answering your research question. You have but 1000-words! Depth over breadth. Quality over quantitiy. Write concisely. Your ability to recall facts is not under assessment, your ability to construct an argument through critical analysis and making it relevant to practice is.

\section*{FAQ}

\begin{itemize}
	\item 	\textbf{What is the deadline for this assignment?} \\ 
    		Falmouth University policy states that deadlines must only be specified on LearningSpace. Please examine the assignment area where you located this document.
    		
	\item 	\textbf{What should I do to seek help?} \\ 
    		You can email your tutor for informal clarifications. For informal feedback, make a pull request on GitHub. 
    		
    	\item 	\textbf{Is this a mistake?} \\ 	
    		If you have discovered an issue with the brief itself, the source files are available at: \\
    		\url{https://github.com/Falmouth-Games-Academy/bsc-assignment-briefs}.\\
    		 Please make a pull request and comment accordingly.
\end{itemize}

\section*{Additional Resources}

\begin{itemize}
    \item Keith, C. (2010) Agile Game Development with Scrum. Pearson Education.
    \item http://agilemanifesto.org/
\end{itemize}

\begin{markingrubric}
%
    \firstcriterion{Participation in viva}{Threshold 10\%}
        \gradespan{5}{\fail The student did not participate in the viva, or did not submit sufficient work to discuss in the viva.}
        \grade 		The student participated in the viva.
%
    \criterion{Breadth of reading}{15\%}
        \grade\fail	No articles are referenced.
        \grade		Some but not all of the suggested articles are referenced.
        \grade		All of the suggested articles are referenced.
        \par		No other relevant scholarly articles are referenced.
        \grade		All of the suggested articles are referenced.
        \par		Some other relevant scholarly articles are referenced.
        \grade		All of the suggested articles are referenced.
        \par		Many other relevant scholarly articles are referenced.
        \grade		All of the suggested articles are referenced.
        \par		A wide variety of highly relevant scholarly articles are also referenced.
%
    \criterion{Depth of insight}{20\%}
        \grade\fail F
        \grade		D
        \grade		C
        \grade		B
        \grade		A
        \grade		A$^*$
%
    \criterion{Specificity, verifiability \& accuracy of claims}{10\%}
        \grade\fail 	No citations to evidence to claims.
        \par 		Substantial errors and/or misinterpretations.
        \grade 		Few claims have a clear source of evidence.
        \par 		Significant errors and/or misinterpretations.
        \grade 		Some claims have a clear source of evidence.
        \par 		Many errors and/or misinterpretations.
        \grade 		Many claims have a clear source of evidence.
        \par 		Some errors and/or misinterpretations.
        \grade 		Most claims have a clear source of evidence.
        \par 		Few errors and/or misinterpretations.
        \grade 		All claims have a clear source of evidence.
        \par 		Almost no errors and/or misinterpretations.
%
    \criterion{Synthesis}{20\%}
        \grade\fail F
        \grade		D
        \grade		C
        \grade		B
        \grade		A
        \grade		A$^*$
%
    \criterion{Community engagement}{15\%}
        \grade\fail 	No contribution has been made to the wiki.
        \grade 		A few minor contributions have been made to the wiki.
        \par		Contributions have not added to the community discussion.
        \grade 		Some contributions have been made to the wiki.
        \par		The student has made some attempt to engage in community discussion.
        \grade 		Many contributions have been made to the wiki.
        \par		The student has actively engaged in the community discussion.
        \grade 		Significant contributions have been made to the wiki.
        \par		The student has participated in steering the community discussion.
        \grade 		Extensive contributions have been made to the wiki.
        \par		The student has played a key role in driving the community discussion.
%
    \criterion{Appropriateness of spelling \& grammar}{5\%}
        \grade\fail 	Substantial spelling and/or grammar errors.
        \grade 		Many spelling and/or grammar errors.
        \grade 		Some spelling and/or grammar errors.  
        \grade 		Few spelling and/or grammar errors.
        \grade 		Almost no spelling and/or grammar errors.
        \grade 		No spelling or grammar errors.
%
    \criterion{Appropriateness of journal structure}{5\%}
        \grade\fail 	There is no structure, or the structure is unclear.
        \grade 		There is little structure.
        \grade 		There is some structure.
        \par 		A few sentences and paragraphs are well constructed.
        \grade 		There is much structure.
        \par 		Some sentences and paragraphs are well constructed.
        \par 		There is a clear introduction and conclusion.
        \grade 		There is much structure, highlighting the key themes.
        \par 		Most sentences and paragraphs are well constructed.
        \par 		There is a clear and well-constructed introduction and conclusion.
        \grade 		There is much structure, highlighting the key themes.
        \par 		All sentences and paragraphs are well constructed.
        \par 		There is a clear and well-constructed introduction and conclusion.
\end{markingrubric}

\end{document}