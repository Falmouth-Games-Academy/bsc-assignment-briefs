\documentclass{../../fal_assignment}
\graphicspath{ {../../} }

\usepackage{amsmath}
\usepackage{enumitem}
\setlist{nosep} % Make enumerate / itemize lists more closely spaced
\usepackage[T1]{fontenc} % http://tex.stackexchange.com/a/17858
\usepackage{url}
\usepackage{todonotes}
\usepackage{algorithm}
\usepackage{algpseudocode}

\usepackage{listings}
\lstset{
	basicstyle=\ttfamily,
	frame=single,
	showstringspaces=false,
	breaklines=true,
	prebreak={\space\hbox{\textcolor{gray}{$\hookleftarrow$}}}
}
\lstset{
	commentstyle=\ttfamily\textit,
	keywordstyle=\ttfamily\textbf,
	stringstyle=\ttfamily,
	rulecolor=\color{black}
}
\lstset{language=C++}

\title{Worksheet C: Introduction to Unreal}
\module{COMP130}
\author{Dr Ed Powley}

\begin{document}

\maketitle

\section*{Introduction}

\begin{marginquote}
``The hero is a hand holding a gun.''

--- San Jose Mercury News, reviewing Doom (1994)
\end{marginquote}
\marginpicture{flavour_pic}{
    Although the Unreal Engine can now be used to develop games of almost any genre,
    it originated with the 1998 first-person shooter Unreal.
}

In this worksheet you will create a simple first-person shooter (FPS) game using Unreal Engine~4.

To complete this worksheet:
\begin{enumerate}[label=(\alph*)]
	\item \textbf{Create} a new Unreal project based on the \textbf{C++ $\to$ First Person} template.
	\item \textbf{Modify} the project to include the following features:
		\begin{enumerate}[label=(\roman*)]
			\item One or more enemy AIs which seek out and attack the player, and respawn on death;
			\item A `rifle' weapon, which uses \textbf{line tracing} (also known as \textbf{raycasting})
				to instantaneously apply damage to any enemy in its line of fire;
			\item A `grenade launcher' weapon, which fires a projectile that causes area-of-effect damage to enemies
				and applies force to all physics bodies in the blast radius;
			\item A way of switching between these two weapons;
			\item As a stretch goal, one or more additional features of your choosing.
		\end{enumerate}
\end{enumerate}

\section*{Submission instructions}

Begin by \textbf{forking} the GitHub repository at the following URL:

\url{https://github.com/Falmouth-Games-Academy/comp130-worksheets}

Create your Unreal project inside the \url{worksheet_C} directory.
Modify the C++ First Person template to implement the requirements listed above.

Attend the timetabled COMP130 session in \textbf{week commencing 20th February 2017},
ensuring that you have uploaded all material to GitHub before this time.

\section*{Marking criteria}

Remember that \textbf{it is better to submit incomplete work than to submit nothing at all}.
Any attempt, even unfinished, will receive a passing grade.

Note that this is a \textbf{programming assignment},
therefore you will \textbf{not} be marked on the aesthetic qualities of your game.

You should aim to implement much of the functionality using C++ code,
however you may also use Blueprint scripting if appropriate.
Note that your Blueprint scripts will be assessed according to the same criteria as your code,
so in particular you should ensure that your Blueprint scripts are maintainable.

Your work will be marked according to the following criteria:
\begin{itemize}
	\item \textbf{Functional coherence}. Is your implementation correct and free of obvious bugs?
		Does it adhere to the specification outlined in this worksheet?
	\item \textbf{Sophistication}. Have you made use of appropriate code structures and data structures?
		Note the emphasis is on \textbf{appropriate}; extra credit will \textbf{not} be given for unnecessarily complex solutions.
	\item \textbf{Maintainability}. Is your code well commented? Are your identifier names appropriate and descriptive?
		Have you adhered to appropriate coding standards?
	\item \textbf{Stretch goal}. Have any additional features been implemented? How well do they work?
\end{itemize}

\end{document}
