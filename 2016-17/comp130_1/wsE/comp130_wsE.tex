\documentclass{../../fal_assignment}
\graphicspath{ {../../} }

\usepackage{amsmath}
\usepackage{enumitem}
\setlist{nosep} % Make enumerate / itemize lists more closely spaced
\usepackage[T1]{fontenc} % http://tex.stackexchange.com/a/17858
\usepackage{url}
\usepackage{todonotes}
\usepackage{algorithm}
\usepackage{algpseudocode}

\usepackage{listings}
\lstset{
	basicstyle=\ttfamily,
	frame=single,
	showstringspaces=false,
	breaklines=true,
	prebreak={\space\hbox{\textcolor{gray}{$\hookleftarrow$}}}
}
\lstset{
	commentstyle=\ttfamily\textit,
	keywordstyle=\ttfamily\textbf,
	stringstyle=\ttfamily,
	rulecolor=\color{black}
}
\lstset{language=C++}

\title{Worksheet E: Individual Specialist Contribution}
\module{COMP130}
\author{Dr Ed Powley}

\begin{document}

\maketitle

\section*{Introduction}

\begin{marginquote}
``The hero is a hand holding a gun.''

--- San Jose Mercury News, reviewing Doom (1994)
\end{marginquote}
\marginpicture{flavour_pic}{
    Although the Unreal Engine can now be used to develop games of almost any genre,
    it originated with the 1998 first-person shooter Unreal.
}

This worksheet assesses your individual specialist contribution to your team game
development project.

To complete this worksheet:
\begin{enumerate}[label=(\alph*)]
	\item \textbf{Write} a brief summary, no more than \textbf{one side of A4 paper},
		stating what tasks you have completed in each sprint of development.
\end{enumerate}

This document is not assessed \textit{per se}, but serves as a reference for your Product Owner
to recall your main contributions to the project.

\section*{Submission instructions}

If you have not already, begin by \textbf{forking} the GitHub repository at the following URL:

\url{https://github.com/Falmouth-Games-Academy/comp130-worksheets}

Use LaTeX or your choice of word processing software to prepare your one-page summary.
Your document should contain \textbf{one heading per sprint}, beginning with Sprint~0 and ending with your final sprint.
Under each heading, write \textbf{2--3 bullet points} outlining the main tasks which you undertook in that sprint,
and identifying with whom you collaborated on these tasks where appropriate.

Save your document as \textbf{PDF}, and upload it to the \url{worksheet_E} directory.

Upload all material to GitHub and open a pull request
by \textbf{TODO date}.

\section*{Marking criteria}

Your work will be marked according to the following criteria:
\begin{itemize}
	\item \textbf{Contribution and value to concept}. Is the contribution adding value in line with the game's concept?
		\begin{itemize}
			\item Does the student's work add appropriate value to the game?
			\item Given the available time and resources, are the student's efforts suitable and relevant?
			\item Has the work been crafted in line with the game's high concept and USPs?
		\end{itemize}
	\item \textbf{Effective use of specialist skills}. Has the student used appropriate techniques from their specialist route to create work of a suitably professional standard?
		\begin{itemize}
			\item This is considered from the perspective of a player. For example, to a programmer it might mean the game runs smoothly and is free of obvious bugs.
			\item Note that in this instance you are \textbf{not} being assessed on factors that are not visible
to the player, for example the maintainability of your code.
		\end{itemize}
	\item \textbf{Use of Agile and Scrum principles}. Has the student effectively utilised Agile \& Scrum principles?
		To what extent has the student...
		\begin{itemize}
			\item  Participated in scrum meetings (sprint planning, stand-ups, sprint reviews)?
			\item  Engaged with the iterative sprint cycle (plan, build, play-test, evaluate)?
			\item  Shown willingness to adapt and to accommodate change?
		\end{itemize}
	\item \textbf{Teamworking skills}. Has the student shown good teamworking skills?
		\begin{itemize}
			\item  Has the student communicated professionally and considerately with their peers?
			\item  Has the student shown an ability to both give and receive constructive criticism well?
			\item  Has the student demonstrated good teamworking skills such as: listening, questioning,
			persuading, respecting, helping, sharing, and participating?
		\end{itemize}
\end{itemize}

These criteria will be assessed by your Product Owner, based on what they have seen of your work
and of your professional practice throughout the semester,
particularly in sprint reviews and retrospectives.

\end{document}
