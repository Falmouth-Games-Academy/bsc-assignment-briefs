\documentclass{../../fal_assignment}
\graphicspath{ {../../} }

\usepackage{amsmath}
\usepackage{enumitem}
\setlist{nosep} % Make enumerate / itemize lists more closely spaced
\usepackage[T1]{fontenc} % http://tex.stackexchange.com/a/17858
\usepackage{url}
\usepackage{todonotes}
\usepackage{algorithm}
\usepackage{algpseudocode}

\usepackage{listings}
\lstset{
	basicstyle=\ttfamily,
	frame=single,
	showstringspaces=false,
	breaklines=true,
	prebreak={\space\hbox{\textcolor{gray}{$\hookleftarrow$}}}
}
\lstset{
	commentstyle=\ttfamily\textit,
	keywordstyle=\ttfamily\textbf,
	stringstyle=\ttfamily,
	rulecolor=\color{black}
}
\lstset{language=C++}

\title{Worksheet D: Game Component}
\module{COMP130}
\author{Dr Ed Powley}

\begin{document}

\maketitle

\section*{Introduction}

\begin{marginquote}
``Todo margin quote.''

--- Whoever said it
\end{marginquote}
\marginpicture{flavour_pic}{
    Although the Unreal Engine can now be used to develop games of almost any genre,
    it originated with the 1998 first-person shooter Unreal.
}

In this worksheet you will describe a component (e.g.\ a particular algorithm or subsystem)
which you have created for your team game development project.
Note that you are \textbf{not} expected to develop a component specifically for this worksheet;
you are being assessed on your ability to describe a piece of technical work which is already
in progress or completed.

To complete this worksheet:
\begin{enumerate}[label=(\alph*)]
	\item \textbf{Select} a component which you have implemented, or are in the process of implementing,
		for your team development project.
		This component should be primarily your own work, although it is fine if others have also contributed.
	\item \textbf{Write} at most \textbf{1000 words} describing:
		\begin{enumerate}[label=(\roman*)]
			\item The name and high concept of your game;
			\item The purpose of the component, and how it fits into the game concept;
			\item Technical details of how the component works;
			\item Challenges encountered in developing the component;
			\item Any contributions made by others to your component.
		\end{enumerate}
		This should be written in the style of development blog post, for an audience of fellow game developers.
	\item \textbf{Include} in your blog post:
		\begin{enumerate}[label=(\roman*)]
			\item \textbf{At least two} relevant \textbf{static images, GIFs or videos} (e.g.\ screenshots or diagrams);
			\item \textbf{At least one} sample of \textbf{C++ code} or \textbf{Blueprint scripting};
			\item Links or bibliographic details for any \textbf{sources}
				(e.g.\ tutorials, blog posts, forum discussions, academic papers, textbooks)
				used in developing your component.
		\end{enumerate}
		This material does \textbf{not} contribute to the 1000 word limit, so you are encouraged to use
		as much as you feel is appropriate.
\end{enumerate}

\section*{Submission instructions}

If you have not already, begin by \textbf{forking} the GitHub repository at the following URL:

\url{https://github.com/Falmouth-Games-Academy/comp130-worksheets}

Within the \url{worksheet_D} directory, create a \texttt{readme.md} document
and write the content of your blog post.
Upload any images to this directory also, and use appropriate Markdown syntax to embed them in your document.

Upload all material to GitHub and open a pull request
by \textbf{TODO date}.

\section*{Marking criteria}

Remember that \textbf{it is better to submit incomplete work than to submit nothing at all}.
Any attempt, even unfinished, will receive a passing grade.

Your work will be marked according to the following criteria:
\begin{itemize}
	\item \textbf{Written communication}. Is your description of technical concepts clear and accurate?
		Is there an appropriate level of depth?
		Is spelling and grammar of a high standard?
	\item \textbf{Visual aids}. Are images relevant and high quality?
		Do they contribute to the user's understanding?
	\item \textbf{Sophistication}. Have you made use of appropriate technical solutions?
		Note the emphasis is on \textbf{appropriate}; extra credit will \textbf{not} be given for unnecessarily complex solutions.
	\item \textbf{Maintainability}. Are C++ code samples well commented, with identifier names appropriate and descriptive?
		Are Blueprint samples tidy, well commented and easy to follow?
	\item \textbf{Stretch goal}. Is your solution particularly elegant or ingenious?
		Does it demonstrate in-depth insight into complex technical concepts?
\end{itemize}

\end{document}
