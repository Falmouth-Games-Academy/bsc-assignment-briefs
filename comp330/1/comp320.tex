\documentclass{../../fal_assignment}
\graphicspath{ {../../} }

\usepackage{enumitem}
\setlist{nosep} % Make enumerate / itemize lists more closely spaced
\usepackage[T1]{fontenc} % http://tex.stackexchange.com/a/17858
\usepackage[hyphens]{url}
\usepackage{todonotes}

\title{CV \& Portfolio}
\author{Alcwyn Parker}
\module{COMP330}

\begin{document}

\maketitle

\section*{Introduction}

\begin{marginquote}
``.''

--- P

    \marginquoterule

``.''

--- A

\end{marginquote}
\marginpicture{flavour_pic}{}


A degree alone does not guarantee success in the future but the skills and experiences acquired throughout your studies will certainly make you very employable in the future. The key to success is being proactive in how you communicate your skills, experience and passion for the industry in a way the appeals to future employers/collaborators. Making yourself visible is crucial. This means creating a profile both digitally and IRL that reflects your 
professionalism, passion and evidences your skill sets and drive to learn. A professional profile usually consists of several components: a portfolio website, personal statement, CV and more often than not, some engagement with professional networks such as LinkedIn, Twitter, StackOverflow and so on. 

For this assignment, you are expected to carry out industry research based on your career aspirations and use this insight to inform the design and development of your own professional profile. The output should be tailored to your specific aspirations and clearly communicates your passion, experiences and skill sets. The required components are, a personal portfolio and CV but you are encouraged to go beyond the MVP for the sake of your own career success. 

This assignment is formed of several parts:
\begin{enumerate}[label=(\Alph*)]
	\item carry out \textbf{research} based on your career aspirations that:
		\begin{enumerate}[label=(\roman*)]
			\item \textbf{identifies} existing and future employment opportunities  
			\item helps to \textbf{distinguish} industry experts as potential mentors
			\item \textbf{informs} your own personal branding
		\end{enumerate}
	
	\item  \textbf{Write} a draft curriculum vitae that will:
		\begin{enumerate}[label=(\roman*)]
			\item \textbf{identify} and \textbf{analyse} the relevant job roles that 
			\item \textbf{propose} one or more research questions for your project
			\item \textbf{communicate} your passion for the industry
			\item \textbf{outline} and \textbf{justify} the methodology to be used
			\item \textbf{present} any preliminary results you have obtained so far
		\end{enumerate}
		
	\item \textbf{Design \& Develop} a portfolio website that will:
		\begin{enumerate}[label=(\roman*)]
			\item \textbf{showcase} your best work
			\item \textbf{demonstrate} your technical skills
			\item \textbf{communicate} your passion for the subject
			\item \textbf{signpost} various components of your online profile
			\item \textbf{host} aN up-to-date version of your CV
		\end{enumerate}
	\end{enumerate}

\subsection*{Assignment Setup}

Fork the GitHub repository at:

\indent \url{https://github.com/Falmouth-Games-Academy/comp330-cv-portfolio }

Use the \texttt{readme.md} file to document your research and sign post the components of your professional profile. 

Please ensure that you add editor-specific files and folders to \texttt{.gitignore}. 


\subsection*{Part A}

Part A consists of a \textbf{single formative submission}.
This work is \textbf{individual} and will be assessed on a \textbf{threshold} basis.

To complete Part A, complete a career focused competitor analysis of industry experts already working in your chosen field. Identify three or more career opportunities advertised that align with your aspirations. Find examples of individuals who brand themselves well. Seek out advice and examples of best practice for personal portfolios and CV writing. A good place to start is the careers hub here at Falmouth University: \url{http://careerhub.falmouth.ac.uk} 

Document all research carried out by adding or linking to it within the repo's readme.md

\subsection*{Part B}
Part B consists of a \textbf{single summative submission},
however your project supervisor may also set additional \textbf{formative submissions} at their discretion.
This work is \textbf{individual} and will be assessed on a \textbf{threshold} basis.

To complete Part B, Design, implement and populate a CV. The CV should express your individuality and your passion for the industry. 
You must include information about: 

\begin{itemize}
	\item profile (short summary about yourself)
	\item personal details
	\item employment \& experience
	\item skills
	\item qualifications
	\item interests \& activities
\end{itemize}

Add a \texttt{.pdf} file containing your CV to the assignment repo and link to it from both your readme file and the portfolio website. 

You will receive immediate and continuous \textbf{informal feedback} through meetings with your supervisor.
You will also receive \textbf{formal feedback} three weeks after the formative deadline.

\subsection*{Part C}

Part C consists of a \textbf{single summative submission},
however your project supervisor may also set additional \textbf{formative submissions} at their discretion.
This work is \textbf{individual} and will be assessed on a \textbf{threshold} basis.

To complete Part C, design and implement a portfolio website that showcases your best work, signposts potential employers/collaborators to your contact details and hosts an up-to-date version of your CV. 

You can choose to host the portfolio using GitHub Pages or any hosting platform you like but you must link to it from the readme.md file in your repo. 

You will receive immediate and continuous \textbf{informal feedback} through meetings with your supervisor.
You will also receive \textbf{formal feedback} three weeks after the formative deadline.



Finally, upload a zipped version of your repo to the LearningSpace. Please note, the LearningSpace will only accept a single \texttt{.zip} file.


\section*{Additional Guidance}

Additional guidance can be found on the University's Career Hub which, can be found at:

\url{http://careershub.falmouth.ac.uk}

\section*{FAQ}

\begin{itemize}
	\item 	\textbf{What is the deadline for this assignment?} \\ 
    		Falmouth University policy states that deadlines must only be specified on the MyFalmouth system.
    		
	\item 	\textbf{What should I do to seek help?} \\ 
    		You can email your tutor for informal clarifications. For informal feedback, make a pull request on GitHub. 
    		
    	\item 	\textbf{Is this a mistake?} \\ 	
    		If you have discovered an issue with the brief itself, the source files are available at: \\
    		\url{https://github.com/Falmouth-Games-Academy/bsc-assignment-briefs}.\\
    		 Please make a pull request and comment accordingly.
\end{itemize}

%\section*{Additional Resources}

%\begin{itemize}
%    \item \url{http://www.opengl-tutorial.org}
%    \item \url{http://gamedev.stackexchange.com/questions/32876/good-resources-for-learning-modern-opengl-3-0-or-later}
%    \item \url{https://google.github.io/styleguide/cppguide.html}
%\end{itemize}


\rubrictitle{Marking Descriptors: Research Review and Proposal}
\begin{markingrubric}
	%
    \firstcriterion{Basic Competency Threshold}{---}
		\grade\fail Parts~B and~C are not completed or are unsatisfactory.
			\par Referencing of sources is unsatisfactory.
		\gradespan{5}{A satisfactory presentation is delivered for Part~B.
			\par The student participates in the peer review activity for Part~C, with enough work present for a meaningful review.
			\par Sources are referenced, with no obvious errors or omissions.}
	%
    \criterion{Breadth of Industry Research}{---}
        \grade \fail The research is missing or unsatisfactory.
        	\par 
        \grade The research falls far short of comprehensive.
        	\par 
        \grade The research falls short of comprehensive.
        	\par 
        \grade The research is somewhat comprehensive.
        	\par 
	\grade The research is very comprehensive.
        	\par 
	\grade The research is extremely comprehensive.
        	\par 
	%
	  \criterion{Synthesis}{15\%}
        \grade\fail No connections are evident between the research and 
        \grade		Superficial connections are made between different sources.
        \grade		Basic connections are made between different sources.
        \grade		Reasonable connections are made between different sources.
        \par		Connections go beyond mere description.
        \grade		Strong connections are made between different sources.
        \par		Connections are analytical in nature.
        \grade		Strong connections are made between different sources.
        \par		Connections are analytical and evaluative in nature.
	%
      \criterion{Appropriateness, Specificity, and Relevance of Information Presennted}{5\%}
%        \grade\fail 	Fewer than four appropriate key skills are mentioned.
        \grade \fail  	Fewer than five appropriate key skills are mentioned.
         \par 		Few skill domains have been considered.
        \grade 		Five appropriate key skills are mentioned.
        \par 		Most skill domains have been considered.
        \grade 		Five appropriate key skills are mentioned.
        \par 		All skill domains have been considered.
        \par 		At least two of the key skills are both specific and relevant.
        \grade 		Five appropriate key skills are mentioned.
        \par 		All skill domains have been considered.
        \par 		At least three of the key skills are both specific and relevant.
        \grade 		Five appropriate key skills are mentioned.
        \par 		All skill domains have been considered. There is little to no overlap.
        \par 		At least four of the key skills are both specific and a relevant.
        \par 		At least two of the key skills are a priority.
        \grade 		Five appropriate key skills are mentioned.
        \par 		All skill domains have been considered. There is little to no overlap.
        \par 		At least five of the key skills are both specific and a relevant.
        \par 		At least four of the key skills are a priority.
%
	 \criterion{Appropriateness of Structure, Layout, and Style}{2.5\%}
        \grade\fail 	There is little to no structure.
        \grade 		There is some structure.
        \par 		A few sentences and paragraphs are well constructed.
        \grade 		There is much structure.
        \par 		Some sentences and paragraphs are well constructed.
        \par 		Website layout and page navigation provides some support for the structure of the analysis.
        \grade 		There is considerable structure.
        \par 		Most sentences and paragraphs are well constructed.
        \par 		Website layout and page navigation provides much support for the structure of the analysis.
        \grade 		There is significant structure.
        \par 		Nearly all sentences and paragraphs are well constructed.
        \par 		Website layout and page navigation provides considerable support for the structure of the analysis.
        \grade 		There is extensive structure.
        \par 		Website layout and page navigation provides significant support for the structure of the analysis.

    %
    \criterion{Creative Flair}{10\%}
        \grade\fail Little or no creativity.
            \par The design is uncreative, showing no divergent and/or subversive thinking.
            \par The work delivers little or no fun and/or engagement.
        \grade Some creativity.
            \par The design demonstrates very little divergent and/or subversive thinking.
            \par The work delivers some fun and/or engagement.
        \grade Much creativity.
            \par The design demonstrates little divergent and/or subversive thinking.
            \par The work delivers much fun and/or engagement.
        \grade Considerable creativity.
            \par The design demonstrates some divergent and/or subversive thinking.
            \par The work delivers considerable fun and/or engagement.
        \grade Significant creativity.
            \par The design demonstrates significant divergent and/or subversive thinking.
            \par The work delivers significant fun and/or engagement.
        \grade Extensive creativity.
            \par The design demonstrates extensive divergent and/or subversive thinking.
            \par The work delivers extensive fun and/or engagement.
    %
        \criterion{Spelling \& grammar}{5\%}
        \grade\fail 	Substantial spelling and/or grammatical errors.
        \grade 		Many spelling and/or grammatical errors.
        \grade 		Some spelling and/or grammatical errors.  
        \grade 		Few spelling and/or grammatical errors.
        \grade 		Almost no spelling and/or grammatical errors.
        \grade 		No spelling or grammatical errors.
\end{markingrubric}



\end{document}