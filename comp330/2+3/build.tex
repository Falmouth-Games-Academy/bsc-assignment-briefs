\documentclass{../../fal_assignment}
\graphicspath{ {../../} }

\usepackage{enumitem}
\usepackage[T1]{fontenc} % http://tex.stackexchange.com/a/17858
\usepackage{url}
\usepackage{todonotes}

\usepackage{listings}
\lstset{
    basicstyle=\ttfamily,
	frame=single,
	tabsize=4,
	showstringspaces=false,
	breaklines=true,
    language=C++
}

\title{Pre-Production Tasks \& Posters}
\author{Dr Michael Scott}
\module{COMP230}

\begin{document}

\maketitle

\begin{marginquote}
    ``You don't make money by working on things;
    
    You make money by shipping things.''
    
    --- Dave Lang
    
    \marginquoterule
    
    ``To evoke emotions through art,
in games or anything else
you have to break down your medium.
What's the minimal amount of information 
we can use to tell you
what you need to know about an object?

Use that, and no more.''
    
    --- Luis Antonio
\end{marginquote}
\marginpicture{flavour_pic}{
    The team behind \textit{Portal} were once students and were hired by Valve when they graduated on the basis of their final year game project \textit{Narbacular Drop}.
}

\section*{Introduction}

This is your final year game development project. It is a major undertaking, affording the opportunity to present a portfolio piece that showcases what you have learned across the degree. There are many configurations for this work. You can: (i) join an existing game development team as a `technical specialist' to work on one or more game components; (ii) act individually as a `consultant' for multiple BA teams to work with one particular emerging technology---delivering a tech demo; (iii) form your own team, extending your research projects and integrating them into an original game as cutting-edge game components; or (iv) work as an individual to develop a game-making tool or game middleware. Use any programming language or technology the Games Academy has access to.

Gaming is a diverse ecology. Experiencing different roles and cross-disciplinary challenges is important. Computing professionals frequently encounter them.

This assignment is formed of several parts:

\begin{enumerate}[label=(\Alph*)]
    \item \textbf{Implement}, as an \textbf{individual or group}, an initial pre-production prototype of the product which will:
    	\begin{enumerate}[label=\roman*.]
    		\item \textbf{illustrate} the project's core objectives, features, and unique selling points.
	\end{enumerate}
    \item \textbf{Implement}, as an \textbf{individual or group}, a final pre-production prototype of the product which will:
    	\begin{enumerate}[label=\roman*.]
    		\item \textbf{revise} any issues raised by your product owner and/or your peers.
	\end{enumerate}
    \item \textbf{Prepare}, as an \textbf{individual}, an A3 poster and double-sided handout which:
    	\begin{enumerate}[label=\roman*.]
    		\item \textbf{outlines} the core features and unique selling points of your project;
		\item \textbf{illustrates} your individual contribution.
    		\item and \textbf{justifies}, where relevant, the market need for the project---whether research \textbf{or} enterprise.
	\end{enumerate}
    \item \textbf{Present}, as an \textbf{individual}, a `poster demo` that will:
    	\begin{enumerate}[label=\roman*.]
    		\item \textbf{clarify} the product and its design, as shown in the poster and handout;
    		\item and \textbf{demonstrate} the final pre-production prototype.
	\end{enumerate}
\end{enumerate}

\subsection*{Assignment Setup}

This assignment is a \textbf{product development task}.

There is no defined method of version control or repository.

\todo[inline]{If you are not working with other students, you are a group of one. As such, all references to group, collaborative, and/or teamwork activities refer to yourself as an individual. Involving BA students in such activites is optional.}

\subsection*{Part A}

Part A is formed of \textbf{multiple formative submissions}. This work is either \textbf{individual} pr \textbf{collaborative}depending on your mode of work. You will be assessed on a \textbf{threshold} basis. The following criteria are used to determine a pass or fail:

\begin{enumerate}[label=(\alph*)]
	\item Playable builds are made available in a timely manner;
	\item Enough work is available to conduct a meaningful review.
\end{enumerate}

To complete Part A, prepare draft versions of the pre-production prototype ahead of the \textbf{sprint review meetings} with your product owner. Ensure that the source code and related assets are made available prior to \textbf{EACH} scheduled meeting. Then, attend the scheduled meetings.

You will receive immediate \textbf{informal feedback} from your \textbf{product owner}.

\subsection*{Part B}

Part B is a \textbf{single summative submission}.This work is either \textbf{individual} pr \textbf{collaborative}depending on your mode of work. It will be assessed on a \textbf{criterion-referenced} basis. Please refer to the marking rubric at the end of the brief for details on the criteria.

To complete Part B, revise the pre-production prototype based on the feedback you have received and finish any incomplete features. Then, upload the source code to the LearningSpace. If you are working with an existing game development team, they will submit the final executable build physically. Otherwise, you must include a link to your build in your submission to the LearningSpace. Please note, the LearningSpace will only accept a single \texttt{.zip} file. 

You will receive \textbf{formal feedback} three weeks after the final deadline.

\subsection*{Part C}

Part C is a \textbf{single summative submission}. This work is \textbf{individual} and will be assessed on a \textbf{threshold} basis. Ensure the presentation addresses the following:

\begin{itemize}
	\item The title and high concept of the product
	\item The target audience and market
	\item The concept's unique selling points and how they distinguish it from competitors
	\item The design's technical and production feasibility
	\item The project's commercial feasibility
\end{itemize}

To complete Part C, prepare the poster and double-sided handout using any word processing and/or presentation tool. Then, upload the relevant files to the LearningSpace. Please note, the LearningSpace will only accept a single \texttt{.zip} file.

You will receive \textbf{informal feedback} from your tutor during Part D.

\subsection*{Part D}

Part D is a \textbf{single summative submission}. This work is \textbf{individual} and will be assessed on a \textbf{criterion-referenced} basis. Please refer to the marking rubric at the end of the brief for details on the criteria.

To complete Part D, practice the delivery of the poster demo. Ensure that you are comfortable with the presentation medium and discuss any concerns with your tutor. Then, attend the scheduled poster demo session. Please ensure that you print out and bring both your poster and a sufficient number of copies of your handout to the session. Please, also ensure that you setup a workstation with the pre-production demo ahead of time.

You will receive \textbf{formal feedback} three weeks after the poster-demo session.

\section*{Additional Guidance}

Avoid poor planning and time management. By now this will be a familiar phrase, but it is no less true.
In particular, avoid underestimating the effort required to implement even simple software; always consider scope.
From the pitch stage, you should consider very carefully what is feasible.

For the most part, your work will be marked as a group effort.
However we want to avoid the situation where students try to ``coast'' through the assignment
on their fellow group members' work,
and equally the situation where one member of the group takes the lion's share of the work
and prevents the others from contributing effectively.
Marks will be weighted by a multiplier for \textbf{individual contribution},
which aims to penalise both of these behaviours.
We assess this by several means, including but not limited to: sprint reviews; individual vivas; feedback from your peers;
attribution in the source code; and GitHub commit logs.
Any student who has contributed their \textit{fair share} of effort to the project will receive a fair \% for their effort,
so any student who is putting in the appropriate level of effort has no need to worry.
Note that effort is not the same as productivity.

The first step in planning your implementation should be to break your concept down into \textbf{user stories}. 
Your user stories should be \textbf{distinguishable} (i.e.\ there should be little overlap between them)
and \textbf{easily measured} (i.e.\ it should be easy to tell when each user story has been implemented).
They should also be \textbf{comprehensive}, i.e.\ the user stories should completely capture the
desired functionality of the game, with no gaps.
Imagine giving your user stories to a developer who has never seen a product of this type before.
Would they be able to implement the software correctly, or would they miss key features?

%Wherever possible, you should clearly \textbf{attribute} the author(s) of segments of your code.
%The easiest way to do this is by inserting appropriate comments, for example:
%\begin{lstlisting}
%/* This function was written by student 1511111 */
%\end{lstlisting}
%\begin{lstlisting}
%/* This function was pair-programmed by students 1522222 and 1533333 */
%\end{lstlisting}
%\begin{lstlisting}
%/* This function was adapted by student 1544444 from an example at http://stackoverflow.com/a/1657490 */
%\end{lstlisting}

Your code will be assessed on \textbf{functional coherence}:
how well the finished product corresponds to the user stories,
and whether it has any obvious bugs.
Correspondence to user stories runs both ways:
implementing features that were not present in the design (``feature creep'')
is just as bad as neglecting to implement features.

Your code will also be assessed on \textbf{sophistication}.
To succeed on a project of this size and complexity,
you will need to make use of appropriate algorithms, data structures, libraries, and object oriented programming concepts.
Appropriateness to the task at hand is key:
you will \textbf{not} receive credit for complexity  
where something simpler would have sufficed.

\textbf{Maintainability} is important in all programming projects,
but doubly so when working in a team.
Use \textbf{comments} liberally to improve code comprehension,
and carefully choose the \textbf{names} for your files, classes, functions and variables.
Use a well-established commenting convention
for \textbf{high-level documentation}.
The open-source tool Doxygen supports several such conventions.
Also ensure that all code corresponds to a sensible and consistent \textbf{formatting style}:
indentation, whitespace, placement of curly braces, etc.
Hard-coded \textbf{literals} (numbers and strings) within the source should be avoided,
with values instead defined as constants together in a single place.
Consider allowing some literal values, where appropriate, to be ``tinkered'' without changing the source code,
e.g.\ by defining them in an external file read at startup.

As with all assignments on this course, you are expected to display a level of
\textbf{innovation and creative flair} befitting Falmouth University's reputation as a world-leading
arts institution.
We are looking for creativity; but, 
you will \textbf{not} be judged on the quality of your art assets.
One approach to promoting creativity is
\textbf{divergent thinking}: generating ideas by exploring many possible solutions.
Often the most interesting ideas are \textbf{subversive}: they deliberately go against
convention or obvious solutions.

\section*{FAQ}

\begin{itemize}
	\item 	\textbf{What is the deadline for this assignment?} \\ 
    		Falmouth University policy states that deadlines must only be specified on the MyFalmouth system.
    		
	\item 	\textbf{What should I do to seek help?} \\ 
    		You can email the module leader for informal clarifications. 
    		
    	\item 	\textbf{Is this a mistake?} \\ 	
    		If you have discovered an issue with the brief itself, the source files are available at: \\
    		\url{https://github.com/Falmouth-Games-Academy/bsc-assignment-briefs}.\\
    		 Please create an `issue' and comment accordingly.
\end{itemize}

\section*{Additional Resources}

\begin{itemize}
    \item Stroustrup, B. (2014) Programming: Principles and Practice using C++. Second Edition. Addison Wesley.
    \item DeLoura, M. (2000) Game Programming Gems. Charles River Media.
    \item Keith, C. (2010) Agile Game Development with Scrum. Pearson.
    \item Gregory, J. (2010) Game Engine Architecture. CRC Press.
    \item \url{https://www.mountaingoatsoftware.com/}
    \item \url{https://literateprogramming.com}
    \item \url{http://gameprogrammingpatterns.com/}
    \item \url{https://blog.codinghorror.com/}
    \item \url{https://git-scm.com/book/en/v2}
    \item \url{http://martinfowler.com/articles/continuousIntegration.html}
    \item \url{https://travis-ci.org}
    \item \url{https://doxygen.org}
    \item \url{http://dopresskit.com/}
    \item \url{http://www.binpress.com/blog/2015/04/06/}\\ \url{guide-launching-indie-games-part-three-getting-press/}
    \item \url{https://c9.io}
    \item \url{http://www.gamasutra.com/blogs/RogerPaffrath/20131115/204871/What_NOT_to_do_when_starting_as_an_indie_game_developer.php}
\end{itemize}

\rubricyearthree

\rubrictitle{Marking Rubric (Pre-Production Prototype)}
\rubrichead{Criteria marked with a $\ddagger$ are shared by the group. Criteria marked with a $\dagger$ are weighted by individual contribution to a shared deliverable. All other criteria are individual.}
\begin{markingrubric}
    \firstcriterion{Sprint Reviews}{40\%}
        \gradespan{1}{\fail Inadequete participation in sprint reviews.}
        \gradespan{5}{The student participates adequetely in all sprint reviews (excepting mitigating circumstances). A meaningfully playable build is available in most sprint reviews.}
%
    \criterion{Progress}{15\% $\dagger$}
        \grade \fail There is no working build or the build has little to no functional coherence.
	\par Only some user stories have been implemented.
            \par Set sprint goals set were tenuous or otherwise inappropriate.
        \grade Many user stories have been implemented.
            \par There is some evidence of feature creep.
            \par Somewhat appropriate sprint goals were set.
        \grade Almost all user stories have been implemented.
            \par There is little evidence of feature creep.
            \par Somwhat appropriate sprint goals were set.
        \grade All user stories have been implemented.
            \par There is almost no evidence of feature creep.
            \par Appropriate sprint goals were set.
        \grade All user stories have been implemented.
            \par There is no evidence of feature creep.
            \par Appropriate sprint goals were set.
        \grade All user stories have been implemented.
            \par There is no evidence of feature creep.
            \par Well-scoped sprint goals were set, providing considerable focus to the project and aiding productivity.
%
    \criterion{Value}{15\% $\dagger$}
        \grade \fail Contributions are a poor fit to the overall project goals.
        \grade Contributions are a somewhat adequate fit to the overall project goals.       
        \grade Contributions are an adequete fit to the overall project goals.
        \grade Contributions are a good fit to the overall project goals and enhance the outcome.
        \grade Contributions are a good fit to the overall project goals and considerably enhance the outcome.
        \grade Contributions are a good fit to the overall project goals and significantly enhance the outcome.
%
    \criterion{Professionalism}{10\% $\ddagger$}
        \grade\fail The group has demonstrated no, little, or only some professionalism.
            \par Agile pratices have not been used, or were implemented in such a way they provided only some support for the project. 
        \grade The group has demonstrated much professionalism, functioning somewhat effectively in a team and/or customer engagment context.
            \par Agile practices have provided much support for the project.
        \grade The group has demonstrated considerable professionalism, functioning somewhat effectively in a team and/or customer engagment context.
            \par Agile practices have provided considerable support for the project.
        \grade The group has demonstrated significant professionalism, functioning somewhat effectively in a team and/or customer engagment context.
            \par Agile practices have provided significant support for the project.
        \grade The group has demonstrated extensive professionalism, functioning somewhat effectively in a team and/or customer engagment context.
            \par Agile practices have provided extensive support for the project.
        \grade The group has demonstrated extensive professionalism, functioning somewhat effectively in a team and/or customer engagment context.
            \par Agile practices have provided extensive support for the project.
            \par There is evidence that the team are engaged in research-led practice or research-driven practice.
%
    \criterion{Application of Specialist Skills \& Polish}{20\% $\dagger$}
        \grade\fail Product will not execute at all on another machine, for reasons related to code portability, even if they are trivially resolvable.
            \par There is only some insight into code sophisticiation evident in the source code.
            \par There is only some insight into code maintainability evident in the source code.
        \grade Some portability issues are present.
            \par Much insight into code sophisticiation is evident in the source code.
            \par Much insight into code maintainability is evident in the source code.
        \grade Few portability issues are present.
            \par Considerable insight into code sophisticiation is evident in the source code.
            \par Considerable insight into code maintainability is evident in the source code.
        \grade Almost no portability issues are present.
            \par Significant insight into code sophisticiation is evident in the source code.
            \par Significant insight into code maintainability is evident in the source code.
        \grade No portability issues are present.
            \par Extensive insight into code sophisticiation is evident in the source code.
            \par Extensive insight into code maintainability is evident in the source code.
        \grade No portability issues are present. 
	 \par Much evidence that portability has been considered in the design of the software architecture.
            \par Extensive insight into code sophisticiation is evident in the source code.
            \par Extensive insight into code maintainability is evident in the source code.
            \par Some evidence of novelty and originality in the software architecture, commesurate with cutting-edge practice.
%
    \criterion{Individual Contribution}{Multiplier for criteria marked $\dagger$}
        \grade \fail Little to no evidence of individual contributions to source code.
        \grade Some evidence of individual contributions to source code.
        \grade Much evidence of individual contributions to source code.
        \grade Considerable evidence of individual contributions to source code.
        \grade Significant evidence of individual contributions to source code.
        \grade Extensive evidence of individual contributions to source code.
\end{markingrubric}

\rubrictitle{Marking Rubric (Project Pitch --- Poster-Demo)}
\rubrichead{Criteria marked with a $\ddagger$ are shared by the group. All other criteria are individual.}
\begin{markingrubric}
    \firstcriterion{Basic Competency Threshold}{40\%}
        \gradespan{1}{\fail No poster-demo is delivered, or either is inappropriate. \par There is no executable and working build of the project. \par There is evidence of academic misconduct.}
        \gradespan{5}{A broadly appropriate individual poster and demo is on display during the demo day. \par A working build is available.}
%
    \criterion{Pre-Production Demo}{25\%}
        \grade\fail There is no handout, or it describes the product concept with little adequacy.
        \grade The product concept is described with some adequacy.
        \grade The product concept is concisely described with much adequacy.
            \par The use of figures and tables is somewhat effective.
        \grade The product concept is concisely described with considerable adequacy.
            \par The use of figures and tables is quite effective.
        \grade The product concept is concisely described with significant adequacy.
            \par The use of figures and tables is very effective.
        \grade The product concept is concisely described with extensive adequacy.
            \par The use of figures and tables is extremely effective.
%
    \criterion{Poster}{20\%}
        \grade \fail Delivered with little enthusiasm. 
            \par Little connection with the audience.
            \par The product concept has been articulated with little clarity.
        \grade Delivered with some enthusiasm, conveying a basic argument. 
            \par Some connection with the audience.
            \par The product concept has been articulated with some clarity.
        \grade Delivered with much enthusiasm, conveying a persuasive argument. 
            \par Much connection with the audience.
            \par The product concept has been articulated with much clarity.
            \par The aesthetics have impact.
        \grade Delivered with considerable enthusiasm, conveying a persuasive argument. 
            \par Considerable connection with the audience.
            \par The product concept has been articulated with considerable clarity.
            \par The aesthetics have considerable impact.
        \grade Delivered with significant enthusiasm, conveying a very persuasive argument and passion for the project.
            \par Significant connection with the audience.
            \par The product concept has been articulated with significant clarity.
            \par The aesthetics have significant impact.
        \grade Delivered with extensive enthusiasm, conveying a very persuasive argument and passion for the project.
            \par Extensive connection with the audience.
            \par The product concept has been articulated with extensive clarity.
            \par The aesthetics have extensive impact.
%
    \criterion{Handout}{5\%}
        \grade\fail There is no handout, or it describes the product concept with little adequacy.
        \grade The product concept is described with some adequacy.
        \grade The product concept is concisely described with much adequacy.
            \par The use of figures and tables is somewhat effective.
        \grade The product concept is concisely described with considerable adequacy.
            \par The use of figures and tables is quite effective.
        \grade The product concept is concisely described with significant adequacy.
            \par The use of figures and tables is very effective.
        \grade The product concept is concisely described with extensive adequacy.
            \par The use of figures and tables is extremely effective.
%
    \criterion{Market Need}{5\%}
        \grade \fail The target audience is mentioned briefly.
            \par There is little evidence of market research.
            \par The proposed product design has few, if any, unique selling points.
        \grade The target audience is explicitly defined.
            \par There is some evidence of market research.
            \par The proposed product design has some unique selling points.
        \grade The target audience is well-defined and somewhat appropriate for the design.
            \par There is much evidence of market research.
            \par The proposed product design has several unique selling points.
        \grade The target audience is well-defined and appropriate to the design.
            \par There is considerable evidence of rigorous market research.
            \par The proposed product design has several unique selling points.
        \grade The target audience is well-defined and justifiably mapped to the design.
            \par There is significant evidence of rigorous market research.
            \par The proposed product design has several unique selling points.
        \grade The target audience is very well-defined and justifiably mapped to the design.
            \par There is extensive evidence of rigorous market research.
            \par The proposed product design has many unique selling points.
%
    \criterion{Innovation and Creative Flair}{5\%}
        \grade\fail Little to no innovation and/or creativity.
            \par The product concept is a clone of existing works with only cosmetic alterations, or is derivative of existing works, with only minor alterations.
        \grade Some innovation and/or creativity.
            \par The product concept is derivative of existing works, but shows emerging divergent and/or subversive thinking.
        \grade Much innovation and/or creativity.
            \par The product concept is somewhat original, with an attempt at divergent and/or subversive thinking.
            \par The product shows promise of fun, engagement, and/or utility.
        \grade Considerable innovation and/or creativity.
            \par The product concept is original, with evidence of divergent and/or subversive thinking.
            \par The product is somewhat fun, engaging, and/or useful.
        \grade Significant innovation and/or creativity.
            \par The product concept is highly original, with strong evidence of divergent and/or subversive thinking.
            \par The product is fun and engaging, or is useful.
        \grade Extensive innovation and/or creativity.
            \par The product concept is highly original, with very strong evidence of divergent and/or subversive thinking.
            \par The product is highly fun and engaging, or is very useful.
\end{markingrubric}

\end{document}
