\documentclass{../fal_assignment}
\graphicspath{ {../} }

\usepackage{enumitem}
\usepackage[T1]{fontenc} % http://tex.stackexchange.com/a/17858
\usepackage{url}
\usepackage{todonotes}

\title{Coding Task II: Game Component}
\author{Dr Michael Scott}

\begin{document}

\maketitle
\begin{marginquote}
    ``Students come into programming classes with a broad range of backgrounds;
    some have experience in several programming languages, others have never programmed before in their life.
    
    Being able to engage with the community and support each other is important.
    Upload your code to GitHub and receive feedback from experienced peers.
    Review your peers' work yourself and really consider what `quality' actually means
    and what `good' source code looks like.
    Debate, argue, and question others about it ---
    an open and sustained discourse is an excellent way to learn ---
    for both beginners and adepts!''
\end{marginquote}
\marginpicture{MakeyMakey.jpg}{
    The \emph{MaKey~MaKey} allows a multitude of materials to be used to create videogame controllers.
}
\section*{Introduction}

In this assignment, you are required to construct a game component and integrate it into an existing game. You will then present a demonstration of your game component.

Games are often comprised of a rich architecture incorporating many components. As such, coding tasks in the games industry may require you to develop bespoke code on a particular aspect of a game. Many components could be modelled. For example: level design; in-game items; characters; art or audio assets; in-game events; an AI director; and so on. Through this project, you will become acquainted in a practical way with the different techniques and methods that help you to work effectively to build computing solutions. You will gain a better knowledge of coding and its various aspects.

This assignment is formed of four parts: A, B, C, and D.

\begin{enumerate}[label=(\alph*)]
    \item Present a brief proposal that will:
    	\begin{enumerate}[label=\roman*.]
    		\item \textbf{identify} a component of game architecture that needs creating;
    		\item \textbf{describe} what will be created, using appropriate annotations and user stories;
    		\item and then \textbf{outline how} it will be integrated into an existing game project.
	\end{enumerate}
    \item Prepare one draft computer program that implements the game component according to the requirements highlighted in the proposal.
    \item Prepare one final computer program that will:
    	\begin{enumerate}[label=\roman*.]
    		\item \textbf{revise} any issues raised by your tutor or your peers;
	\end{enumerate}
    \item Prepare a practical demonstration of your work
\end{enumerate}

\todo[inline]{\textbf{Note:} All submissions must be clearly distinctive. Members of the same COMP150 team must \textbf{not} target the same problem.}

Part A consists of a \textbf{single formative submission} with a deadline set in LearningSpace. This work will be assessed on a \textbf{threshold} basis. The threshold is set at 5\%. This means that 5\% of the total marks available for the coursework overall are awarded on a pass or fail basis. In other words, satisfactory submissions will be awarded 5\%. However, unsatisfactory submissions will receive 0\%.

The following criteria are used to determine a pass or fail for each submission in Part A:

\begin{enumerate}[label=(\alph*)]
	\item Submission is timely;
	\item Proposed game component is relevant and feasible to implement;
	\item Coding task is non-trivial.
\end{enumerate}

Part B is a \textbf{single formative submission} with a deadline set in LearningSpace. This work will be assessed on a \textbf{threshold} basis. The threshold is set at 5\%. This means that 5\% of the total marks available for the coursework overall are awarded on a pass or fail basis. In other words, satisfactory submissions will be awarded 5\%. However, unsatisfactory submissions will receive 0\%.

Part C is a \textbf{single summative submission} and will be assessed on a criterion-referenced basis. This submission is expected to take students from the threshold of 10\% (F) up to the maximum of 100\% (A*). This means that 80\% of the total marks available for the coursework overall will be awarded.

The following criteria are used to allocate marks:

\begin{enumerate}[label=(\alph*)]
	\item Appropriateness of Requirements;
	\item Appropriateness of Design;
	\item Functional Coherence of Executable Solution;
	\item Maintainability of Source Code;
	\item Sophistication of Source Code;
	\item Professional Practice;
\end{enumerate}

Part D is not grade-bearing; however, attendance is mandatory. Absence will result in a grade capped at 40\% (D-).


\begin{marginquote}
    ``Remember, learning to program can take a surprising amount of time \& effort --- students may get there at different rates, but all students who put in the time \& effort get there eventually. Making good use of [reflection and deliberate practice] are an essential part of this process.''
    
    --- Professor Quintin Cutts
\end{marginquote}
\marginpicture{guitarhero}{
    Rhythm games such as \emph{Guitar Hero} and \emph{Rock Band} are excellent examples of games
    which make use of unique input devices to enhance gameplay.
}
\section*{Submission Instructions}

\subsection*{Part A}

Part A must be completed as a formative submission on GitHub. Fork the GitHub project at the following URL:

\indent \url{https://github.com/Falmouth-Games-Academy/comp110-coding-task-2}

Write your proposal in the \texttt{readme.md} file. Also use this repository for any other digital assets you create (e.g. diagrams and pseudocode), checking them in regularly as you work on your projects. For the proposal, images should be embedded directly in the \texttt{readme.md} file.

You will need to show your proposal to your tutor to be signed-off before the deadline. This will need to be arranged in one of your fortnightly tutorials prior to the deadline.

\subsection*{Part B}

Part B will be completed as an in-session code review during the appropriate workshop, as scheduled on the LearningSpace. The source code and related assets should be hosted on GitHub before the code review takes place. The single PDF document must be submitted to the LearningSpace by the final submission deadline as shown on LearningSpace. Please note that the LearningSpace will only accept a single PDF document.

You will receive formal feedback three weeks after the submission deadline shown on LearningSpace.

\subsection*{Part C}

Part C must be submitted as a single zip file, containing the source code and assets from the relevant Git repository, on the LearningSpace. This must be submitted by the final submission deadline as shown on LearningSpace. Please note that the LearningSpace will only accept a single zip file (.rar format must not be used).

\subsection*{Part D}

No formal submission is required. Please turn up to the demo session at the end of the Semester, with an executable demo of your work and be prepared to discuss it with others.

You will receive formal feedback three weeks after the final submission deadline shown on LearningSpace.

\section*{Additional Guidance}

A common mistake in this project is poor planning and poor time management. In such circumstances, it is often the case that students greatly underestimate how much work is involved in first learning programming concepts and then actually applying them in order to write a computer program. Programming is quite unlike other subjects in that it cannot be "crammed" into a last minute deluge just before a deadline, and so relying on knowledge-centric learning strategies (for which such an approach may be possible in other subject areas) will likely result in disaster in the programming context. It is, therefore, very important that students begin their work early and sustain a good pace across the duration of the course. Do a little programming every day, if you can!

It is also important to consider that 50\% of the work is pushing for that last 5\% of the polish. That is, significant effort is needed to push the quality of the work up from a B/C grade up to an A or A* grade. It is, therefore, very important to anticipate this and accommodate enough slack-time for this long phase of polishing in your plan --- especially with competing demands from your other assignments.

You may worry about the first deadline as it is quite soon after the assignment is set and not much material has been covered. Please rest assured. The formative submission is supposed to be an early proof-of-concept. It is expected to contain errors, be riddled with mistakes, fail to fulfil the requirements, and generally be of poor quality. However, it is very important to make a start on this project so you are actively using your programming skills across the entire duration of the course. Ideally, you should be programming every day. Considering this, producing implementing the first 3 algorithms within 3 weeks is feasible.

The peer-review component of this work does sometimes raise alarm. However, the only way to learn how to review code is by reviewing code. Your tutor will guide you through the process and provide advice. With practice, it will become clear what is not satisfactory by discussing the quality of work with your peers and your tutor during the review sessions. Ongoing reviewing and discussion on GitHub is advised.

\begin{marginquote}
    ``The first 90 percent of the code accounts for the first 90 percent of the development time.
    
    ``The remaining 10 percent of the code accounts for the other 90 percent of the development time.''
    
    --- Tom Cargill
    
    \marginquoterule
    
    ``Hofstadter's Law:
    
    ``It always takes longer than you expect, even when you take into account Hofstadter's Law.''
    
    --- Douglas Hofstadter
\end{marginquote}
\marginpicture{fishing}{
    The \emph{Dreamcast Fishing Controller}, released as a peripheral for the game \emph{Sega Bass Fishing}.
    Even peripherals which appeal to only a small audience can enjoy moderate commercial success.
}
\section*{Additional Resources}

\begin{itemize}
    \item As listed in the module guide.
\end{itemize}

\begin{markingrubric}
%
    \firstcriterion{Satisfactory Preparation of Proposal}{5\%}
        \gradespan{5}{\fail The proposal is inappropriate and/or is late.}
        \grade 		The proposal has been signed-off by your tutor by the deadline.
%
    \criterion{Satisfactory Completion of Peer-Review Tasks}{5\%}
        \gradespan{5}{\fail No work was submitted for peer-review and/or no peer-review has been submitted and/or either is late.}
        \grade 		Work submitted for peer-review on time and reviews of peers' work submitted on time.
%
    \criterion{Appropriateness of Requirements}{5\%}
        \grade\fail 	No user stories are provided.
        \grade 		Few user stories are appropriately formatted, distinguishable, and easily measured.
        \grade 		Some user stories are appropriately formatted, distinguishable, and easily measured.
        \grade 		Most user stories are appropriately formatted, distinguishable, and easily measured.
        \par 		The scope and relevance of all requirements is appropriate.
        \grade 		Nearly all user stories are appropriately formatted, distinguishable, and easily measured.
        \par 		The scope and relevance of all requirements is appropriate.
        \grade 		All user stories are appropriately formatted, distinguishable, and easily measured.
        \par 		The scope and relevance of all requirements is appropriate.
%
    \criterion{Appropriateness of Design}{10\%}
        \grade\fail 	No design is presented.
        \grade 		The design is very flawed and/or very poorly described.
        \grade 		The design is flawed and/or poorly described.
        \grade 		The design is acceptable and adequately described.
        \grade 		The design is sound and well described.
        \grade 		The design is exceptional and very well described.
%
    \criterion{Functional Coherence}{15\%}
        \grade\fail 	The component is non-functional.
        \grade 		Few requirements have been met.
        \par 		There are many obvious bugs.
        \grade 		Some requirements have been met.
        \par 		There are some obvious bugs.
        \grade 		Many requirements have been met.
        \par 		There are few obvious bugs.
        \grade 		The game component is fit-for-purpose.
        \par 		There are almost no obvious bugs.
        \grade 		The game component is fit-for-purpose.
        \par 		There are no obvious bugs.
%
    \criterion{Sophistication}{25\%}
        \grade\fail 	No insight into the appropriate use of programming constructs is evident from the source code.
        \grade 		Little insight into the appropriate use of programming constructs is evident from the source code.
        \grade 		Some insight into the appropriate use of programming constructs is evident from the source code.
        \grade 		Much insight into the appropriate use of programming constructs is evident from the source code.
        \par 		The program is structured appropriately.
        \grade 		Significant insight into the appropriate use of programming constructs is evident from the source code.
        \par 		The program is structured effectively, such that there is high cohesion and low coupling.
        \grade 		Exemplary insight into the appropriate use of programming constructs is evident from the source code.
        \par 		The program is structured very effectively, such that there is very high cohesion and very low coupling.
%
    \criterion{Maintainability}{25\%}
        \grade\fail 	The source code cannot be maintained.
        \grade 		There are many problems which affect the maintainability of the source code.
        \grade 		There are some problems which affect the maintainability of the source code. 
        \par 		Some clear and appropriate comments are present.
        \grade 		There are few problems which affect the maintainability of the source code. 
        \par 		Many clear and appropriate comments are present.
        \grade 		There are almost no problems which affect the maintainability of the source code. 
        \par 		Source code is well commented.
        \par 		Doc strings (or equivalent) are provided.       
        \grade 		There are no problems which affect the maintainability of the source code.
        \par 		Source code is exceptionally well commented.
        \par 		Appropriate doc strings (or equivalent) are provided. 
%
    \criterion{Professional Practice}{10\%}
        \grade\fail 	GitHub has not been used.
        \grade 		Source code and assets have been checked into the repository only just before a deadline.
        \grade 		Source code and assets have seldom been checked into the repository.
        \grade 		Source code and assets have regularly been checked into the repository.
        \par 		An attempt has been made to document the project using \texttt{readme.md} and \texttt{changelog.md}.
        \grade 		Source code and assets have regularly been checked into the repository.
        \par  		The first check-in to the repository is in the first half of the semester.
        \par 		The project is appropriately documented using \texttt{readme.md} and \texttt{changelog.md}.
        \par 		There is evidence of some engagement with the Falmouth Games Academy community (e.g. reviewing peers' pull requests).
        \grade 		Source code and assets have regularly been checked into the repository.
        \par  		The first check-in to the repository is in the first quarter of the semester.
        \par 		The project is exemplary documented using \texttt{readme.md} and \texttt{changelog.md}.
        \par 		There is evidence of much engagement with the Falmouth Games Academy community (e.g. reviewing peers' pull requests).
\end{markingrubric}

\end{document}