\documentclass{../../fal_assignment}
\graphicspath{ {../../} }

\usepackage{enumitem}
\setlist{nosep} % Make enumerate / itemize lists more closely spaced
\usepackage[T1]{fontenc} % http://tex.stackexchange.com/a/17858
\usepackage{url}
\usepackage{todonotes}

\title{Market Evaluation \& Business Case}
\author{Gareth Lewis \& Dr Michael Scott}
\module{COMP240}
\version{2.0}

\newcommand{\essayWordCount}{14}
\newcommand{\presentationMinutes}{5}
\newcommand{\minReferenceCount}{4 }

\definecolor{announce}{HTML}{6495ed}

\begin{document}

\maketitle

\section*{Introduction}

\begin{marginquote}
  ``Research is formalized curiosity, it is poking and prying with a purpose.''
  
   --- Zora Neale Hurston
   
    \marginquoterule
    
    ``Listening is hearing the needs of the customer, understanding those needs and making sure the company recognizes the opportunity they present.''
    
    --- Frank Eliason
    
    \marginquoterule
    
    ``Innovation needs to be part of your culture. Customers are transforming faster than we are, and if we don't catch up, we're in trouble.''
    
    --- Ian Schafer
      
\end{marginquote}
\marginpicture{flavour_pic}{
    Team Niche extensively marketed their game at events such as \textit{PAX} prior to and during its KickStarter campaign. It was funded.
}

In this assignment, you will research business models for your COMP240 project and present a business case. Specifically, to: 

\begin{enumerate}[label=(\roman*)]
    \item explore the market and possible routes to market for your game;
    \item and to compare two business propositions, recommending which you think will be the most successful commercialisation strategy.
\end{enumerate}

Making games is a business. For the most part, game studios exists to make money. There are many business models, each having its own constraints and viability---even for those soley interested in games for art. All game developers must eat and sleep! To maximise your chance of success in the games industry and to confer some of the entrepreneurial agency that Falmouth University is famed for, this assignment will help you develop an appreciation for the commercial forces behind game development. An evaluation of a market and the construction of a business case will expose some of these forces.

\todo[inline, color=announce]{\textbf{Role-Play Scenario:} Your current group is taking the game to market using \texttt{Itch.IO} as the sole distributor. You estimate the game will sell 10,000 units and agree a minimum net profit of \textsterling 5 per unit, once pricing and other costs have been assessed. You anticipate production will require six more months to make the game polished and functionally complete. After which, you have several game ideas you would like to develop. So far, the team enjoys working together and you all feel that you are getting good results.

\vspace{1em}

Until this point, the team has run `under the radar' by working remotely and drawing benefits during development. However, an opportunity has presented itself which would enable everyone to: (i) get paid; (ii) have a real company with an office to work in; and (iii) be able to tell people that they are real game developers. A publisher likes the look of your game and is proposing a deal (see the additional guidence section). It is your job to conduct a strategic evaluation of the two business options (to stick with self-publishing or to go with the publisher).}

This assignment is formed of several parts:

\begin{enumerate}[label=(\Alph*)]
    \item \textbf{Present} a \presentationMinutes-minute market evaluation and business case that will:
    	\begin{enumerate}[label=\roman*.]
    		\item \textbf{outline how} lucrative the target market is, with reference to relevant business concepts and elements of the business canvas;
    		\item \textbf{contrast} self-publishing to the deal offered by the publisher;
    		\item and \textbf{recommend} which business proposition to pursue.
	\end{enumerate}
    \item \textbf{Write} a draft \essayWordCount-page report which will:
    	\begin{enumerate}[label=\roman*.]
    		\item  \textbf{evaluate} the market \textbf{and} the two routes to maket;
    		\item  \textbf{investigate and contrast} the two business models;
    		\item  as well as \textbf{justify} the \textbf{most} appropriate business case.
	\end{enumerate}
    \item \textbf{Write} a final \essayWordCount-page report that will:
    	\begin{enumerate}[label=\roman*.]
    		\item \textbf{revise} any issues raised by your tutor and/or your peers.
	\end{enumerate}
\end{enumerate}

\subsection*{Assignment Setup}

This assignment is a \textbf{business task}. There is no GitHub repository. Use any word-processing, graphics, and/or desktop publishing tools as well as the platform(s) of your choice to prepare your presentation slide-deck and report.

\subsection*{Part A}

Part A is a \textbf{single formative submission}. This work is \textbf{individual} and will be assessed on a \textbf{threshold} basis. The following criteria are used to determine a pass or fail:

\begin{enumerate}[label=(\alph*)]
	\item Business modelling is adequately addressed using CANVAS;
	\item The self-publishing and publisher-deal models are suitably compared;
	\item A recommendation is made, with at least some sensible justification.
\end{enumerate}

To complete Part A, prepare a brief 5-minute presentation. Create about 6 slides using a tool of your choice (i.e., one cover slide; two slides, one for each business proposition; two slides, one for financial insight into each option; and one final slide outlining your decision). Leverage details specific to the game you are working on to formulate a sensible a business model and recommend one of the two propositions. You are presenting to mentors as if making a business decision. So, tell us your decision and justify why. Analyse your options and practice your argument ahead of your presentation slot in Week 7. Then, attend the scheduled presentation session.

You will receive \textbf{informal feedback} from your tutor within 3 working days.

\subsection*{Part B}

Part B is a \textbf{single formative submission}. This work is \textbf{individual} and will be assessed on a \textbf{threshold} basis. The following criteria are used to determine a pass or fail:

\begin{enumerate}[label=(\alph*)]
	\item Submission is timely;
	\item Enough work is available to conduct a meaningful review;
	\item A broadly appropriate review of a peer's work is submitted.
\end{enumerate}

To complete Part B, prepare a draft version of the business report. This is quite an extensive piece, so ensure that sufficient time has been allocated to, both: (i) explore and refine the underlying business models that will underpin your recommendation; and (ii) prepare a well-structured and visually interesting business report. Ensure this is made available as a \texttt{.pdf} document and uploaded to LearningSpace ahead of the peer-review workshop in Week 8. Then, attend the scheduled session.

You will receive \textbf{peer feedback} within 3 working days after the session.

\subsection*{Part C}

Part C is a \textbf{single summative submission}. This work is \textbf{individual} and will be assessed on a \textbf{criterion-referenced} basis. Please refer to the marking rubric at the end of this document for further detail.

To complete Part C, revise the market evaluation and business case report based on the feedback you have received. Take care to ensure the recommendation is clear and sufficiently justified by the body of the report. Then, upload the report to the LearningSpace. Please note, the LearningSpace will only accept a single \texttt{.pdf} file.

You will receive \textbf{formal feedback} three weeks after the final deadline.

\section*{Additional Guidance}

It is critically important that you carefully review the details of the deal offered to you by the publisher in making your decision:

\todo[inline, color=announce]{\textbf{Publisher Deal:} The publisher will fund up to 12 months (about \textsterling 100k of development) using the current team through an advance that will be recouped against sales. There is a local business park that can accommodate the whole team and provide suitable development machines and associated resources for \textsterling 1200 per calendar month.

\vspace{1em}

The game will be initially release on Steam at a gross price of \textsterling 15. The publisher will commit a further \textsterling 50k marketing advance that will be also be recouped against sales. The publisher estimates that the game should sell between 30-50,000 units and once the initial advances have been recouped, there will be a 50:50 royalty split. Once the game meets the upper end of the sales targets (40,000 units), console ports will be triggered. This will be a go / no go decision after 3 months of sales data and the PC version will be launched as soon as it is complete.

\vspace{1em}

For console ports, the publisher will provide a \textsterling 50k advance for development of PS4, XboxOne \& Switch skus. It is expected that all 3 versions can be turned over within 6 months and a skeleton team of 2 programmers and a producer / designer will be required working across all platform skus. The publisher will manage all sku testing.
 
\vspace{1em}

On completion, the publisher will commit a further \textsterling 150k marketing advance to ensure that the game will get great coverage on release.
Console targets suggest an average of 20,000 sales per platform with average net revenue of \textsterling 5 per sale and this would be split 50:50 once advances had been recovered.
The developer would retain ownership of the current IP, though the publisher would retain the right of first refusal on publishing future development work with the IP. Further work on novel IP would be discussed on completion of the console versions.}

Use the \textit{Business CANVAS} as an analytic framework to frame your recommendation. It will be presented as a professional management report. The page count may seem intimidating but, in terms of prep, is broadly equivilient to the essays you have written until this point. Indicately, the report should include:

\begin{enumerate}[label=(\alph*)]
\item Front Cover
    	\begin{enumerate}[label=\roman*.]
    		\item This should be a single page with your company and game name.
	\end{enumerate}
\item Excutive Summary
    	\begin{enumerate}[label=\roman*.]
    		\item Report Highlights (1 page, but only about half-a-page of content)
	\end{enumerate}
\item Market Evaluation
    	\begin{enumerate}[label=\roman*.]
    		\item Game Value Proposition (1 page);
    		\item Marketing Personas (at least 4 personas across 2 pages).
    	\end{enumerate}		
\item Route to Market --- Self-Publishing for PC Release	
    	\begin{enumerate}[label=\roman*.]	
    		\item Market Position and Route to Market using \texttt{Itch.IO} (1 page);
    		\item Revenue Spreadsheet (1 page).
	\end{enumerate}
\item Route to Market --- Publisher-Deal for Console Release  
    	\begin{enumerate}[label=\roman*.]	
    		\item Market Position and Route to Market using Publisher (1 page);
    		\item Revenue Spreadsheet (1 page).
	\end{enumerate}	
\item Development Plan
     	\begin{enumerate}[label=\roman*.]	
    		\item PC Development (1 page);
    		\item Console development with a reduced headcount (1 page);
    		\item Development cash-flow (1 page).
	\end{enumerate}		
\item Conclusion	
     	\begin{enumerate}[label=\roman*.]	
    		\item Assumptions and risks (half a page);
    		\item Accuracy and/or relevance of publisher information based on your game (half a page);
    		\item Recommendation and justification as to whether to accept the publisher offer (1 page).
	\end{enumerate}		
\end{enumerate}

\section*{FAQ}

\begin{itemize}    		
    	\item 	\textbf{Is this a mistake?} \\ 	
    		If you have discovered an issue with the brief itself, the source files are available at: \\
    		\url{https://github.com/Falmouth-Games-Academy/bsc-assignment-briefs}.\\
    		 Please make a pull request and comment accordingly.
\end{itemize}

\section*{Additional Resources}

\begin{itemize}
    \item \url{http://voyager.falmouth.ac.uk/vwebv/holdingsInfo?bibId=687701}
    \item \url{http://voyager.falmouth.ac.uk/vwebv/holdingsInfo?bibId=1079269}
    \item \url{https://www.youtube.com/watch?v=mt3povGsIGw}
    \item \url{https://unilearning.uow.edu.au/report/4bi1.html}
    \item \url{https://hbr.org/2014/07/the-right-way-to-present-your-business-case}
    \item \url{http://voyager.falmouth.ac.uk/vwebv/holdingsInfo?bibId=293746}
\end{itemize}

\rubricyeartwo

\begin{markingrubric}
%
    \firstcriterion{Basic Proficiency Threshold}{40\% (Threshold)}
        \gradespan{1}{\fail Parts A---C have not been submitted, are incomplete, or are unsatisfactory.
        \par		No recommendation is provided.}
        \gradespan{5}{Parts A---C are complete. 
          \par		There is at least some allusion to a recommendation.
          \par		Evidence to suggest that business ethics were considered as a factor in the decision.
        \par		All criteria are met at the basic proficiency level.}
%
    \criterion{Depth of Insight into the Market and the Business Propositions}{20\%}
%        \grade\fail 	No depth of insight into software engineering principles.
        \grade  \fail	Little to no depth of insight and/or analysis into commercial feasibility.
        \par 		No personas included.
        \grade 		Some depth of insight into one or both of the business propositions and its/their commercial feasibility.
        \par 		At least two broadly sufficient marketing personas are described.
        \grade 		Modest depth of insight into the market and commercial viability.
        \par 		At least two adequete marketing personas are described.
        \par 		Adequete insight from the underpinning market research used to construct a sound business case.
        \grade 		Much depth of insight into the market and commercial viability.
        \par 		At least four adequete marketing personas are described.
        \par 		Much insight from the underpinning market research used to construct a sound business case.
    	\par 		Insights suggest some level of commercial awareness.
        \grade 		Considerable depth of insight into the market and commercial viability.
        \par 		At least four appropriate and detailed marketing personas are described.
        \par 		Considerable insight from the underpinning market research used to construct a critical business case.
    	\par 		Insights suggest an appropriate level of commercial awareness.
        \grade 		Significant depth of insight into the market and commercial viability.
        \par 		At least four appropriate and detailed marketing personas are described.
        \par 		Significant insight from the underpinning market research used to construct a critical business case.
    	\par 		Insights suggest a highly appropriate level of commercial awareness.
%
%
    \criterion{Adequacy of Analysis of Propositions}{20\%}
%        \grade\fail 	No analysis has been presented.
        \grade  \fail	Little to no analysis has been presented.
         \par		No evidence to suggest knowledge of the business canvas.
        \grade 		Some analysis has been presented. 
         \par		The business canvas is referred to.
        \grade 		Moderate analysis has been presented.
         \par		Elements of the business canvas have been leveraged sufficiently.
         \par		A recommendation is made reasonably clear.
        \grade 		Much analysis has been presented.
         \par		Elements of the business canvas have been leveraged sufficiently.
         \par		Recommendations are clear and are somewhat justified.
        \grade 		Considerable analysis has been presented.
         \par		The business canvas has been leveraged adequetely.
         \par		Recommendations are lucid and have been justified appropriately.
         \par		Key limitations of the analysis are explicitly acknowledged.
         \grade 		Significant analysis has been presented.
         \par		The business canvas has been leveraged adequetely.
         \par		Recommendations are lucid and have been justified with much detail.
	\par		Key limitations of the analysis are explicitly acknowledged.
%
    \criterion{Appropriateness of Business Writing}{20\%}
        \grade  \fail	Little to no evidence for mastery of written communication skills.
        \par 		Considerable spelling and/or grammar problems.
        \par 		Little to no structure.
     
        \grade 		Some evidence of mastery of written communication skills.
        \par 		Many obvious spelling and/or grammar errors are present in places.
        \par 		Some structure.   
        \par 		Some sentences and paragraphs are well constructed.
        
        \grade 		Modest evidence of mastery of written communication skills.
        \par 		Some obvious spelling and/or grammar errors are present in places.
        \par 		Modest structure.   
        \par 		A modest proportion of the sentences and paragraphs are well constructed.
        \par 		There is a clear executive summary and closing statement.
        
        \grade 		Much evidence of mastery of written communication skills.

        \par 		Few obvious spelling and/or grammar errors.
        \par 		Much structure.   
        \par 		Most sentences and paragraphs are well constructed.
        \par 		There is a clear and well-constructed introduction and conclusion.
        
        \grade 		Considerable evidence of mastery of written communication skills.
        
        \par 		Close to no obvious spelling and/or grammar errors.
        \par 		Considerable structure, leveraged to effectively highlight the argument and key takeaway points.  
        \par 		Nearly all sentences and paragraphs are well constructed.
        \par 		There is a clear and well-constructed introduction and conclusion.
        \par 		All in-text citations and quotations are correct. 
        
        \grade 		Significant evidence of mastery of written communication skills.
        
        \par 		No obvious spelling and/or grammar errors. 
        \par 		Significant structure, leveraged to effectively highlight the argument and key takeaway points.  
        \par 		All sentences and paragraphs are well constructed.
        \par 		There is a clear and well-constructed introduction and conclusion.
        \par 		All in-text citations and quotations are correct.
       \par 		Active voice is prevalent.
        \par 		Grammar is leveraged deliberately to draw attention to salient points.    
        
\end{markingrubric}

\end{document}