\documentclass{../../fal_assignment}
\graphicspath{ {../../} }

\usepackage{enumitem}
\usepackage[T1]{fontenc} % http://tex.stackexchange.com/a/17858
\usepackage{url}
\usepackage{todonotes}

\usepackage{listings}
\lstset{
    basicstyle=\ttfamily,
	frame=single,
	tabsize=4,
	showstringspaces=false,
	breaklines=true,
    language=C++
}

\title{Production Tasks \& Game Demo}
\author{Dr Michael Scott}
\module{COMP130}
\version{1.2-alpha}

\begin{document}

\maketitle

\begin{marginquote}
    ``It seems that perfection is attained not when there is nothing more to add,
        but when there is nothing more to remove.''
    
    --- Antoine de Saint-Exup\'ery
    
    \marginquoterule
    
    ``Good judgment comes from experience and experience comes from bad judgment!''
    
    --- Fred Brooks Jr
    
    \marginquoterule
    
    ``Debugging is twice as hard as writing the code in the first place.
     Therefore, if you write the code as cleverly as possible, you are, by definition, not smart enough to debug it.'' 
 
         --- Brian Kernighan
     
\end{marginquote}
\marginpicture{flavour_pic}{
    A poster and technical demo is a common format for presenting novel computing techniques at technology conferences such as \textit{ACM Multimedia}.
}

\section*{Introduction}

In this assignment, you will prepare a playable early-access game. Your game will be written in \textit{C++} and \textit{Blueprints} using \textit{Unreal Engine 4}. You will work collaboratively in a multi-disciplinary team with other BA and BSc students.

The success of a digital game is not just determined by the fun of its design, but also by the soundness of its architecture. Put simply, games with serious bugs do not sell. As such, the value of software engineering principles cannot be emphasised enough! Reflecting upon these principles, using the lens of a development process that mirrors industry practice is therefore critically important. Further to this, there are key quality assurance techniques and anlysis methods to appy. This will be an ongoing process, so engagement will need to persist throughout the production cycle.

This assignment is formed of several parts:

\begin{enumerate}[label=(\Alph*)]
    \item \textbf{Launch}, as a \textbf{group}, a GitHub Pages website \textbf{and} promotional video, which will:
    	\begin{enumerate}[label=\roman*.]
    		\item \textbf{describes} the game concept \textbf{and} instructions on how to play;
    		\item as well as \textbf{showcase} the game's unique selling points and aesthetic.
	\end{enumerate}
    \item \textbf{Prepare}, as an \textbf{individual}, a draft A3 research-style poster which will:
    	\begin{enumerate}[label=\roman*.]
    	    	\item \textbf{identify ONE} component for which \textbf{you} are individually accountable;
    		\item and \textbf{describe} its specification \textbf{and} design using UML.
	\end{enumerate}
    \item \textbf{Implement}, as a \textbf{group}, a playable game prototype which:
    	\begin{enumerate}[label=\roman*.]
    		\item \textbf{acts} as a stand-alone executable suitable for play-testing;
    		\item and \textbf{illustrates} the core game mechanic.
	\end{enumerate}
    \item \textbf{Write}, as an \textbf{individual}, a blog post for the GitHub Pages site which will:
    	\begin{enumerate}[label=\roman*.]
    		\item \textbf{introduce} yourself \textbf{and} the area of the game you are working on;
		\item \textbf{describe} the architecture and engineering of the game;
    		\item \textbf{analyse} the engineering quality of the game \textbf{and your} component.
	\end{enumerate}	
    \item \textbf{Implement}, as a \textbf{group}, an early-access release canditate which will:
    	\begin{enumerate}[label=\roman*.]
    		\item \textbf{revise} any issues raised by your tutor and/or your peers.
	\end{enumerate}
    \item \textbf{Prepare}, as an \textbf{individual}, a final A3 research-style poster that:
    	\begin{enumerate}[label=\roman*.]
    		\item \textbf{analyses} the architectural and engineering qualities of the early-access release canditate;
    		\item and \textbf{describes} the component \textbf{you} implemented using pseudocode, code snippets, UML, \textbf{and} software quality metrics. 
	\end{enumerate}
    \item \textbf{Present}, a \textbf{group} `tech demo' with, \textbf{individually}, your poster, which:
    	\begin{enumerate}[label=\roman*.]
    		\item \textbf{clarifies} the technical content of \textbf{EACH} poster;
    		\item and \textbf{showcases} the early-access release canditate of the game.
	\end{enumerate}
\end{enumerate}
  
\subsection*{Assignment Setup}

This assignment is a \textbf{product development task}. For this project you \textbf{MUST} use the repository allocated to your team by GA technicians for version control.

Use a sensible directory structure and, as required, extend this structure with sub-directories and trunks/branches. Add and maintain a \texttt{readme.md} file. 

\subsection*{Part A}

Part A consists of a \textbf{single formative submission}. This work is \textbf{collaborative} and will be assessed on a \textbf{threshold} basis. The following criteria are used to determine a pass or fail:

\begin{enumerate}[label=(\alph*)]
	\item The GitHub Pages site is live;
	\item Instructions are adequete to enable an external assessor to play;
	\item The unique selling points are clearly listed;
	\item The aesthetic is illustrated with at least four screenshots;
	\item The core game mechanics are showcased through a video.
\end{enumerate}

To complete Part A, setup GitHub Pages and populate a single \texttt{HTML} page with promotional material about the game. Show it to your product owner prior to the March `Show and Tell' event. Leverage the material the BA students are prearing for the event and mirror these, within reason, on the website.

You will receive immediate \textbf{informal feedback} from your \textbf{product owner}.

\subsection*{Part B}

Part B consists of a \textbf{single formative submissions}. This work is \textbf{individual} and will be assessed on a \textbf{threshold} basis. The following criteria are used to determine a pass or fail:

\begin{enumerate}[label=(\alph*)]
	\item An appropriate game component is adequetely described;
	\item and a UML class diagram is used to describe the design.
\end{enumerate}

To complete Part B, prepare a draft version of the poster. Submit the poster, in \texttt{.pdf} format, to the review activity in LearningSpace ahead of the workshop session in Week 5.

You will receive immediate \textbf{informal feedback} from your \textbf{module tutor}.

\subsection*{Part C}

Part C is formed of \textbf{multiple formative submissions}. This work is \textbf{collaborative} and will be assessed on a \textbf{threshold} basis. The following criteria are used to determine a pass or fail:

\begin{enumerate}[label=(\alph*)]
	\item A working build is demonstrated at \textbf{EACH} sprint review;
	\item Enough work is available to conduct meaningful reviews.
\end{enumerate}

To complete Part C, build a stand-alone executable from the master branch ahead of \textbf{EACH} sprint review meeting. It is standard practice to maintain an archive of working builds in case of emergency. Ensure assets are merged in regularly. Showcase the executable at \textbf{EACH} sprint review meeting.

You will receive immediate \textbf{informal feedback} from your \textbf{product owner}.

\subsection*{Part D}

Part D consists of a \textbf{single formative} submission. This work is \textbf{individual} and will be assessed on a \textbf{threshold} basis.  The following criteria are used to determine a pass or fail:

\begin{enumerate}[label=(\alph*)]
	\item The constribution is communicated in a clear and concise manner;
	\item The contribution is meaningful and will, or will potentially, add value.
\end{enumerate}

To complete Part D, reflect on how your invididual contributions are adding value. Focus on \textbf{one} key contribution. Gather your thoughts into a short blog post to your GitHub pages site (from Part A). Illustrate the engineering aspects using appropriate diagrams and data analyses. Upload a link to your blog post to the review activity on LearningSpace ahead of the session in Week 9. 

You will receive \textbf{informal feedback} in the session from the \textbf{module tutor}.

\subsection*{Part E}

Part E is a \textbf{single summative submission}. This work is \textbf{collaborative} and will be assessed on a \textbf{criterion-referenced} basis. Please refer to the marking rubric at the end of the brief for details on the criteria.

To complete Part E, revise the prototype based on the feedback that you have received. Build the project and copy it to a USB storage device. Please note, you \textbf{MUST} include \textbf{both} the source code \textbf{and} a stand-alone playable build. These should be in two separate and clearly labelled folders. Instructions on how to play the game \textbf{MUST} also be included. Then, submit it physically alongside your colleagues on the other courses.

You will receive \textbf{formal feedback} three weeks after the final deadline.

\subsection*{Part F}

Part F is a \textbf{single summative submission}. This work is \textbf{individual} and will be assessed on a \textbf{criterion-referenced} basis. Please refer to the marking rubric at the end of the brief for details on the criteria.

To complete Part F, finish the poster (from Part B) using any software of your choice. Upload the relevant files to the LearningSpace. Please note, the LearningSpace will only accept a single \texttt{.zip} file.

You will receive \textbf{formal feedback} three weeks after the final deadline.

\subsection*{Part G}

Part G is a \textbf{single summative submission}. This work is \textbf{collaborative} and will be assessed on a \textbf{threshold} basis. The following criteria are used to determine a pass or fail:

\begin{enumerate}[label=(\alph*)]
	\item A working \textbf{and} playable early-access release canditate is showcased;
	\item Analyses of engineering quality are sufficient \textbf{and} somewhat accurate.
\end{enumerate}

To complete Part G, prepare for the `tech demo'. Ensure you are comfortable discussing both the game and the technical content of your poster. Then, attend the scheduled session. Print your poster ahead of time and bring it with you. This is your own responsibility! It should be A3 paper and portrait. Please, also ensure that someone is available to setup ahead of the session.

You will receive immediate \textbf{informal feedback} from \textbf{attending staff}.

\section*{Additional Guidance}

This is a complex and challenging project. Accordingly, it is double-weighted in terms of academic credit. This project alone is worth the same as an entire module of study. Invest a substantial proportion of your time into it: about one third! There are \textbf{FOUR} main deliverables. These are:

\begin{enumerate}[label=(\roman*)]
	\item A website about your game project, which you prepare collectively as a group on GitHub and show to your product owner;
	\item A blog post, in which review you---as an individual---review your key contribution(s) to the group project and show to your module tutor;
	\item A working and playable version of the early-acces release canditate, submitted physically as a group on a USB device, by the production task deadline;
	\item and, a poster highlighting \textbf{ONE} game component that you implemented, submitted online as an invidual to the LearningSpace, by the production task deadline.
\end{enumerate}

Carefully select the algorithms and/or components that you take ownership of and implement. These will not only need to interface with other game components, but also other kinds of contributions made by artistic colleagues. Aim for high cohesion and low coupling! Firmly specify the requirements and ensure they are not too broad. This will help you to avoid overlap with the work of your peers and overburdening yourself with too much work. 

Commit and to push your source code and related assets to the repository frequently. This will make it easier for you to maintain a backup of your work. It will also help you to assess your productivity. Version control \textbf{MUST} become a core and essential part of your workflow. 

Poor planning and poor time management can have a substantial impact on this assignment. Programming is quite unlike many of the other subjects you may have studied. It cannot be ``crammed'' into a last minute deluge. Sustain a steady pace across the duration of the course. Do a little programming every day---from the start!

For the most part, this is a group effort and will be marked accordingly.
However, we want to avoid the situation where students try to ``coast'' through the assignment,
and equally the situation where one member of the group takes the lion's share of the work
and prevents the others from contributing effectively.
Marks will be weighted by a multipliers for \textbf{individual contribution}, which penalises both of these behaviours.
Any student who has contributed their \textit{fair share} of effort to the project will receive a fair \% for their effort,
so any student who is putting in the appropriate level of effort has no need to worry. Note that effort is not the same as productivity.

\todo[inline]{\textbf{Note:} In addition to sprint retrospectives, peer-review activities, and your repository, your product owners will track your individual contributions---alongside other deliverables, as described above. Engage with your PO if you are encountering difficulty! They are your mentors and lifeline. }

You are expected to act professionally. This is a university environment, not school. Among many notable differences, it is left to you to arrange daily stand-up meetings and it is mandatory for every member of the team to engage with agile practices. Agile should be maintained every weekday throughout the duration of the project, including the studio period and days when you have sessions. 

It is also \textbf{mandatory} to attend every single meeting with your product owner. Your attendence to these meetings will be monitored. If you cannot attend a meeting, you \textbf{MUST} send a email in advance to your product owner. This should detail your apologies just as if you were emailing a senior colleague or manager in a real business context. You will be penalised on your individual contribution multiplier for any unauthorised absences. 

Now that you are working in larger, multi-disciplinary teams, you will encounter collaborative complexity. There will be many more lines of communication, and therefore greater opportunity for disorganisation and conflict. Likewise, some members of the team may become agitated if others appear to be less productive than they should be. Whatever forms that these may take, it is important that you recognise such challenges and approach them in a collegiate and constructive manner. It is \textit{critically important} that you maintain your professionalism. Be helpful and understanding. Your criticisms should be a gift to help your peers grow, improve, and mature. This means that concerns should relate to directly to working practice (or lack thereof) and be reasonably actionable. When raised, the discussion should be fair and tolerate the possibility of a compromise, as well as the your potential involvement in providing direct support for your peer.    

Do not permit conflicts to fester and stagnate. Tackle these head-on at the earliest opportunity. However, if there is a breach of professional conduct, contact your product owner. 

\section*{FAQ}

\begin{itemize}
	\item 	\textbf{What is the deadline for this assignment?} \\ 
    		Falmouth University policy states that deadlines must only be specified on the MyFalmouth system.
    		
	\item 	\textbf{What should I do to seek help?} \\ 
    		You can email your tutor for informal clarifications. For informal feedback, make a pull request on GitHub. 
    		
    	\item 	\textbf{Is this a mistake?} \\ 	
    		If you have discovered an issue with the brief itself, the source files are available at: \\
    		\url{https://github.com/Falmouth-Games-Academy/bsc-assignment-briefs/issues}.\\
    		 Please create an issue and comment accordingly.
\end{itemize}

\section*{Additional Resources}

\begin{itemize}
    \item Stroustrup, B. (2014) Programming: Principles and Practice using C++. Second Edition. Addison Wesley.
    \item Keith, C. (2010) Agile Game Development with Scrum. Pearson.
    \item \url{https://www.mountaingoatsoftware.com/agile/user-stories}
    \item \url{https://literateprogramming.com}
    \item \url{http://gameprogrammingpatterns.com/}
    \item \url{https://blog.codinghorror.com/}
    \item \url{http://martinfowler.com/articles/continuousIntegration.html}
    \item \url{https://doxygen.org}
    \item \url{http://www.binpress.com/blog/2015/04/06/}\\ \url{guide-launching-indie-games-part-three-getting-press/}
    \item \url{http://www.gamasutra.com/blogs/RogerPaffrath/20131115/204871/What_NOT_to_do_when_starting_as_an_indie_game_developer.php}
\end{itemize}

\rubrictitle{Marking Rubric (Production Tasks --- [i] Invidual Specialist Contribution) - 40\%}
\rubrichead{Criteria marked with a $\ddagger$ are shared by the group. All other criteria are individual.}
\begin{markingrubric}
    \firstcriterion{Threshold}{40\%}
        \gradespan{1}{\fail Unsatisfactory participation, academic misconduct, and/or no working build delivered.}
        \gradespan{5}{An individual blog post to the GitHub Pages site clearly articulates the individual contribution to the game project.
        \par Adequete attendence to crit and PO meetings.
             \par A playable build is delivered in each sprint review.
             \par There is no evidence of academic misconduct.}
%
    \criterion{Value of Individual Specialist Contribution}{10 \%}
        \grade\fail Few to no individual contributions that are relevant to a computing specialism.
        \grade Contributions add some value to the project.
        \grade Contributions add much value to the project.
            \par An attempt has been made to align contributions with the high concept and USPs.
        \grade Contributions add considerable value to the project.
            \par Contributions are somewhat aligned with the high concept and USPs.
        \grade Contributions add considerable value to the project.
            \par Contributions are adequetely aligned with the high concept and USPs.
        \grade Contributions add significant value to the project.
            \par Contributions are clearly and directly aligned with the high concept and USPs.
            \par Contributions have provided some support for peers.
%
    \criterion{Quality of Individual Specialist Contribution}{20 \%}
        \grade\fail Few to no individual contributions that are relevant to a computing specialism.
        \grade Little engineering quality.
        \grade Some engineering quality.
            \par Blog posts elicit appropriate technical details.
            \par UML diagrams and pseudocode are used.
        \grade Much engineering quality.
            \par Blog posts elicit appropriate technical details.
            \par UML diagrams and pseudocode are clear and somewhat effective.
        \grade Considerable engineering quality.
            \par Blog posts leverage technical details effectively.
            \par UML diagrams and pseudocode are clear and effective.
        \grade Significant engineering quality.
            \par Blog posts provide clear and effective technical detail.
            \par UML diagrams and pseudocode are clear and very effective.        
%
    \criterion{Agile Working Practice}{15\% $\ddagger$}
        \grade\fail Working practices are unacceptable and/or agile principles have not been applied.
        \grade Little engagement with agile working practice.
            \par Clear evidence of attendance at regular sprint planning meetings.
            \par Clear evidence of attendance regular sprint retrospective meetings.
            \par Little to no evidence of attendance at daily stand-ups.
        \grade Some engagement with agile working practice.
            \par Clear evidence of attendance at regular sprint planning meetings.
            \par Clear evidence of attendance regular sprint retrospective meetings.
            \par Some evidence of attendance at daily stand-ups.
        \grade Much engagement with agile working practice.
            \par Clear evidence of attendance at regular sprint planning meetings.
            \par Clear evidence of attendance regular sprint retrospective meetings.
            \par Clear evidence of attendance at daily stand-ups.
            \par There is evidence of a little use of CI, TDD, and/or bug tracking to support the project.
        \grade Considerable engagement with agile working practice.
            \par Clear evidence of attendance at regular sprint planning meetings.
            \par Clear evidence of attendance regular sprint retrospective meetings.
            \par Clear evidence of attendance at daily stand-ups.
            \par There is evidence of some use of CI, TDD, and/or bug tracking to support the project.
        \grade Significant engagement with agile working practice.
            \par Attendence to all daily stand-ups and regular meetings.
            \par CI, TDD, and bug tracking have been used to support the project.
%
    \criterion{Collaborative Practice}{15\% $\ddagger$}
        \grade\fail Version control has not been used.
        \grade Material has been checked into version control at least once per sprint.
            \par Communication has been sufficient.
        \grade Material has been checked into version control at least once per week.
            \par Commit messages are clear, concise and relevant.
            \par Communication has been somewhat appropriate.
            \par Conduct and teamwork in meetings has been sufficient.
        \grade Material has been checked into version control several times per week.
            \par Commit messages are clear, concise and relevant.
            \par Branches are used sensibly.
            \par There is a little evidence of engagement with peers (e.g.\ code review).
            \par Communication has been effective.
            \par Conduct and teamwork in meetings has been somewhat appropriate.
        \grade Material has been checked into version control frequently each week.
            \par Commit messages are clear, concise and relevant.
            \par Branches are used somewhat effectively.
            \par There is some evidence of engagement with peers (e.g.\ code review).
            \par Communication has been effective.
            \par Conduct and teamwork in meetings has been appropriate.
        \grade Material has been checked into version control very frequently each week.
            \par Commit messages are clear, concise and relevant.
            \par Branches are used effectively.
            \par There is much evidence of engagement with peers (e.g.\ code review).
            \par Communication has been very effective.
            \par Conduct and teamwork in meetings has been effective.
\end{markingrubric}


\rubrictitle{Marking Rubric (Production Tasks --- [ii] Game with Instructions) - 50\%}
\rubrichead{Criteria marked with a $\dagger$ are weighted by individual contribution to a shared deliverable. All other criteria are individual.}
\begin{markingrubric}
    \firstcriterion{Threshold}{40\%}
        \gradespan{1}{\fail Unsatisfactory participation, academic misconduct, and/or no working build delivered.}
        \gradespan{5}{A working build is produced and submitted on the USB stick.
        \par A GitHub Pages site is available to describe the game and provide adequete instructions on how to play the game
        \par There is no evidence of academic misconduct.}
%
    \criterion{Conceptual Coherence}{20\% $\dagger$}
        \grade\fail No user stories and/or sprint plans are provided.
        \grade Few user stories are distinguishable and easily measured.
            \par Sprint plans provide little support for the project.
        \grade Some user stories are distinguishable and easily measured.
            \par Sprint plans provide some support for the project.
        \grade Most user stories are distinguishable and easily measured.
            \par User stories correspond to the game design.
            \par Sprint plans provide much support for the project.
        \grade Nearly all user stories are distinguishable and easily measured.
            \par User stories clearly correspond to the game design.
            \par Sprint plans provide considerable support for the project.
        \grade All user stories are distinguishable and easily measured.
            \par User stories clearly and comprehensively correspond to the game design.
            \par Sprint plans provide significant support for the project.
%
    \criterion{Functional Coherence}{20\% $\dagger$}
        \grade\fail No gameplay elements have been implemented and/or the code fails to compile or run.
        \grade Few gameplay elements have been implemented.
            \par There are many obvious and serious bugs.
        \grade Some gameplay elements have been implemented.
            \par There are some obvious bugs.
        \grade Many gameplay elements have been implemented.
            \par There is some evidence of feature creep.
            \par There are few obvious bugs.
        \grade Almost all gameplay elements have been implemented.
            \par There is little evidence of feature creep.
            \par There are some minor bugs.
        \grade All gameplay elements have been implemented.
            \par There is no evidence of feature creep.
            \par Bugs, if any, are purely cosmetic and/or superficial.
%            
    \criterion{Creative Innovation}{10\% $\dagger$}
        \grade\fail There is no website or press kit, or the game does not resemble either.
        \grade Few promised gameplay elements are in-game.
            \par The website and press kit have little clarity.
        \grade Some promised gameplay elements are in-game.
            \par The website and press kit have some clarity.
        \grade Many promised gameplay elements are in-game.
            \par The website and press kit have much clarity.
            \par Promotional material evokes some excitement.
        \grade Almost all promised gameplay elements are in-game.
            \par The website and press kit have considerable clarity.
            \par Promotional material evokes much excitement.
        \grade All promised gameplay elements are in-gamed.
            \par The website and press kit have significant clarity.
            \par Promotional material evokes considerable excitement.
%
    \criterion{Engagement}{10\% $\dagger$}
        \grade\fail No insight into the appropriate use of programming constructs is evident from the source code.
            \par No attempt to structure the program is evident (e.g. one monolithic source file).
        \grade Little insight into the appropriate use of programming constructs is evident from the source code.
            \par The program structure is poor.
        \grade Some insight into the appropriate use of programming constructs is evident from the source code.
            \par The program structure is adequate.
        \grade Much insight into the appropriate use of programming constructs is evident from the source code.
            \par The program structure is appropriate.
        \grade Considerable insight into the appropriate use of programming constructs is evident from the source code.
            \par The program structure is effective. There is high cohesion and low coupling.
        \grade Significant insight into the appropriate use of programming constructs is evident from the source code.
            \par The program structure is very effective. There is high cohesion and low coupling.            
\end{markingrubric}

\rubrictitle{Marking Rubric (Production Tasks --- [iii] Technical Demo) - 10\%}
\rubrichead{Criteria marked with a $\ddagger$ are shared by the group. All other criteria are individual.}
\begin{markingrubric}
    \firstcriterion{Basic Competency Threshold}{40\%}
        \gradespan{1}{\fail No individual poster and/or demo is delivered and/or no evidence of C++ having been used for the demonstrated component, or either are inappropriate.}
        \gradespan{5}{Present at the demo. 
        \par A broadly appropriate poster and tech demo are delivered in a timely fashion. 
        \par Evidence of understanding of game and software engineering practices used to construct it. 
        \par There is no evidence of academic misconduct.}
%
    \criterion{Poster Quality}{20\%}
        \grade\fail There is no poster or it does not describe the engineering of a non-trivial game component. 
        \grade The engineering of the component (e.g., class designs) is described with little adequacy.
        \grade The engineering of the component (e.g., class designs) is described with some adequacy.
            \par UML diagrams and source code excerpts are present.
        \grade The engineering of the component (e.g., class designs) is concisely described with much adequacy.
            \par The use of UML diagrams and source code excerpts is somewhat effective.
        \grade The engineering of the component (e.g., class designs) is concisely described with considerable adequacy.
            \par The use of UML diagrams and source code excerpts is quite effective.
        \grade The engineering of the software (e.g., class designs) is concisely described with significant adequacy.
            \par The use of UML diagrams and source code excerpts is very effective.
%
    \criterion{Technical Insight}{20\%}
        \grade\fail No or inappropriate and/or irrelevant technical insight.
        \grade Little insight into the technical qualities of the individual algorithm.
            \par Little ability to explain how the algorithm fits into the game's components and architecture.
        \grade Some insight into the technical qualities of the individual algorithm.
            \par Some ability to explain how the algorithm fits into the game's components and architecture.
        \grade Much insight into the technical qualities of the individual algorithm.
            \par Much ability to explain how the algorithm fits into the game's components and architecture.
            \par The relevance of the contribution is justified.
        \grade Considerable insight into the technical qualities of the individual algorithm.
            \par Considerable ability to explain how the algorithm fits into the game's components and architecture.
            \par The relevance and value of the individual algorithm are justified.
        \grade Significant insight into the technical qualities of the individual algorithm.
            \par Significant ability to explain how the algorithm fits into the game's components and architecture.
            \par The relevance and value of the  individual algorithm are justified.
            \par The  individual algorithm is somewhat important to the design of the game.
%
    \criterion{Demo Quality}{20\% $\ddagger$}
        \grade\fail There is no demo, or it is non-functional.
        \grade The demo demonstrates few key mechanics and interfaces.
        \grade The demo demonstrates some key mechanics and interfaces.
        \grade The demo demonstrates most core game mechanics.
            \par Although there may be a backup video, at least some aspect of the demo is live using the production prototype.
        \grade The demo demonstrates all core game mechanics.
            \par Although there may be a backup video, much of the demo is live using the production prototype.
        \grade The demo demonstrates all core game mechanics.
            \par Although there may be a backup video, a considerable part of the demo is live using the production prototype.
            \par There is some innovation in terms of technologies incorporated into the demo.
\end{markingrubric}

\end{document}
