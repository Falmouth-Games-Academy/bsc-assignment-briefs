\documentclass{../../fal_assignment}
\graphicspath{ {../../} }

\usepackage{enumitem}
\setlist{nosep} % Make enumerate / itemize lists more closely spaced
\usepackage[T1]{fontenc} % http://tex.stackexchange.com/a/17858
\usepackage[hyphens]{url}
\usepackage{todonotes}

\title{Research Artefact \& Dissertation}
\author{Dr Michael Scott}
\module{COMP360}
\version{1.1}

\begin{document}

\maketitle

\section*{Introduction}

\begin{marginquote}
``Most papers in computer science describe how their author learned what someone else already knew.''

--- Peter Landin

    \marginquoterule

``We can only see a short distance ahead, but we can see plenty there that needs to be done.''

--- Alan Turing

\end{marginquote}
\marginpicture{flavour_pic}{}

 These assignments form the second part of your \textbf{major research project}; whether \textbf{empirical research} relating to computing for games, or \textbf{practice-based research} in game development. It is your opportunity to collect and analyse primary data and prepare a manuscript to disseminate your findings.

Together, they are formed of several parts:

\begin{enumerate}[label=(\Alph*)]
	\item \textbf{Complete} your research artefact, ensuring that you:
		\begin{enumerate}[label=(\roman*)]
			\item \textbf{apply} a rigorous software project management approach;
			\item and \textbf{follow} best practices in software engineering.
		\end{enumerate}
	\item \textbf{Collect} data relating to your research question,
		ensuring that you:
		\begin{enumerate}[label=(\roman*)]
			\item \textbf{apply} a rigorous research method;
			\item \textbf{leverage} your research artefact appropriately;
			\item and \textbf{conform} with all ethical requirements.
		\end{enumerate}
	\item \textbf{Deliver} a 20-minute presentation that will:
		\begin{enumerate}[label=(\roman*)]
			\item \textbf{analyse and synthesise} your findings.
		\end{enumerate}
	\item \label{part:review-draft} \textbf{Write} a 12-page academic dissertation in IEEE format
		that will:
		\begin{enumerate}[label=(\roman*)]
			\item \textbf{identify} and \textbf{review} the scientific literature relevant to your project;
			\item \textbf{outline and justify} one or more research questions;
			\item \textbf{outline and justify} the methodology that was applied in addressing the research question(s);
			\item \textbf{analyse} data you have obtained;
			\item as well as \textbf{interpret and discuss} your findings.
		\end{enumerate}
	\item \textbf{Prepare} the final dissertation \textbf{and} addendum in IEEE format that will:
		\begin{enumerate}[label=(\roman*)]
			\item \textbf{address} any issues raised through peer-review;
			\item \textbf{and evaluate} the outcome of the project.
		\end{enumerate}
	\item \textbf{Deliver} a 20-minute presentation that will:
		\begin{enumerate}[label=(\roman*)]
			\item \textbf{showcase} the final research artefact;
			\item and \textbf{defend} the claims made in the academic dissertation.
		\end{enumerate}
\end{enumerate}

\subsection*{Assignment Setup}

These assignments form a \textbf{research} task, consisting of \textbf{academic reading}, \textbf{academic writing},
\textbf{software development}, and \textbf{scientific experimentation}.

For the \textbf{prototype research artefact}, there is no set GitHub repository.
However you are strongly encouraged to create one.
Ensure that you set up the \texttt{.gitignore} file for your chosen development environment,
and maintain the \texttt{readme.md} file.

For the \textbf{research dissertation}, continue using the GitHub repository at the following URL:

\indent \url{https://github.com/Falmouth-Games-Academy/comp320-comp360-dissertation}

Use the existing LaTeX template, which is based on the IEEE Transactions style,
to write your dissertation.

\subsection*{Part A}

Part A consists of a \textbf{single summative submission}.
However, your supervisor may set additional \textbf{formative submissions} at their discretion.
This work is \textbf{individual} and will be assessed on a \textbf{holistic} basis, according to the descriptors set out at the end of this document.

To complete Part A, implement your research artefact and show it to your research supervisor in a timetabled meeting. As
the requirements for the research artefact will vary by project, consult with your supervisor to verify whether or not the
artefact is adequete for the desired purpose.

You will receive immediate \textbf{informal feedback} from your supervisor.

Then, upload a \texttt{.zip} file containing the final version of your source code and any assets/dependencies to the LearningSpace.
Note that LearningSpace will only accept a single \texttt{.zip} file.

You will receive \textbf{formal feedback} three weeks after the summative deadline.

\subsection*{Part B}

Part B consists of a \textbf{single formative submission}.
This work is \textbf{individual} and will be assessed on a \textbf{threshold} basis.

To complete Part B, collect data and codify it into a digital dataset. Show this to your research supervisor in a timetabled meeting. 
As the research questions and methods will vary, consult with your supervisor to verify whether or not the
data is adequete for the desired purpose.

You will receive immediate \textbf{informal feedback} from your supervisor.

\todo[inline]{\textbf{Important:} Carrying out \textbf{any} data collection or experimental work
without first securing ethical approval is a breach of Falmouth University's Research Ethics Policy.
This will be treated as a case of academic misconduct, and penalised accordingly.}

\subsection*{Part C}

Part C consists of a \textbf{single formative submission}.
This work is \textbf{individual} and will be assessed on a \textbf{threshold} basis.
To pass, answer the following questions:

\begin{enumerate}[label=(\roman*)]
	\item What is (are) the key research question(s) that you will seek to answer in your project?
	\item How will answering these questions contribute to the state of knowledge in the field of your project?
	\item Why have you applied particular methods in seeking answers to these questions?
	\item What results have you obtained?
	\item How have you analysed and interpreted these results?
	\item What are you likely to conclude on the basis of this research?
	\item What are the potential implications of your discoveries?
\end{enumerate}

To complete Part C, prepare a 20-minute presentation and deliver it in the scheduled session. Prepare your slide-deck using your choice of presentation software (e.g.\ Beamer, reveal.js, PowerPoint). You may extend your existing slides from COMP320, but must incorporate new material and new findings.

You will receive immediate \textbf{informal feedback} from your tutors.

\subsection*{Part D}

Part D consists of a \textbf{single formative submission}. 
However, your supervisor may set additional \textbf{formative submissions} at their discretion.
This work is \textbf{individual} and will be assessed on a \textbf{threshold} basis.

To complete Part D, prepare a draft of the dissertation. This should \textit{build upon} and \textit{extend} your research proposal and
literature review. As such, overlap with the COMP320 submission is expected. However, there should be new material.
Use the provided LaTeX template and do \textbf{not} modify the formatting. Bring these to meetings with your supervisor.

You will receive immediate \textbf{informal feedback} from your supervisor.

Then, bring \textbf{two physical print-outs} of your dissertation to the timetabled peer-review session. 
You are advised to consider the specified page limit and attempt to adhere to it at this stage; although, you will \textbf{not} be penalised at this point for failing to do so.  Also, ensure you also bring pens and highlighters.

You will receive immediate \textbf{informal feedback} from peers and \textit{in the following week} a \textbf{meta-review} from tutors.

\subsection*{Part E}

Part E consists of a \textbf{single summative submission}. However, your supervisor may set additional \textbf{formative submissions} at their discretion.
This work is \textbf{individual} and will be assessed on a \textbf{holistic} basis, according to the descriptors set out at the end of this document.

To complete Part E, revise your manuscript to address any issues highlighted during the peer review.
Your document must not exceed \textbf{twelve pages} of text, excluding figures, tables, references and appendices.
This is subject to the policy on word and page limits available on the LearningSpace course page.

As an addendum to the research manuscript, an appendix containing at most one-page of critical reflection and self-evaluation should be incorporared. 
This should be a critical appraisal of the project, indicating the rationale for any key decisions, lessons learnt, and evaluation (with hindsight) of the process
and outcome of the project. 

Upload your final \texttt{.pdf} file to the LearningSpace.
Note that LearningSpace will only accept a single \texttt{.pdf} file.

You will receive \textbf{formal feedback} three weeks after the formative deadline.

\subsection*{Part F}

Part F consists of a \textbf{single formative submission}.
This work is \textbf{individual} and will be assessed on a \textbf{threshold} basis.
To pass, answer the following questions:

\begin{enumerate}[label=(\roman*)]
	\item What was the purpose of your research?
	\item How did you approach the research?
	\item What did you discover?
	\item What are the implications of your findings?
\end{enumerate}

To complete Part F, prepare a 20-minute presentation in the timetabled session after the submission deadline.
Prepare your slides using your choice of presentation software (e.g.\ Beamer, reveal.js, PowerPoint).

You will receive immediate \textbf{informal feedback} from tutors.

\section*{Additional Guidance}

\subsubsection*{Ethical Clearance}

Computing professionals are expected to follow the ethical standards that have been outlined by the British Computer Society (BCS), The Chartered Institute for IT. \textbf{All} students are expected to familiarize themselves with the BCS Code of Conduct. Although scientific work at undergraduate level is unlikely to raise serious ethical concerns, there are many topics pertinent to games where Sections 1 and 2 of the Code of Conduct and the Interpretation of the BCS Code of Conduct are likely to apply. Research artefacts with the potential for practical application or future commercialisation are one example and will normally also have wider ethical implications for you to consider. As such, heed the points in Section 2 on technological procedures and standards as well as those on complying with legislation.

\textbf{All} students must follow Falmouth University's Research Ethics Policy. In practice, this means you need to complete the Research Ethics Approval Application Form and obtain ethical clearance \textbf{before} using the research artefact you have created to collect primary data. Even if collecting such data presents minimal to no risk. Furthermore, if your project involves human participants (e.g. for testing a game system you have developed), or presents a significant ethical risk (e.g., systems that process personal data) then you will need to have completed the full Research Ethics Approval Application Form available on LearningSpace. In these cases, you should also discuss these ethical issues in your dissertation and pay particular attention to how you addressed these issues in your research methodology.

If you did not obtain ethical approval at the proposal stage, email a revised version of the application form to the module leader and \textbf{do not} start to collect data until you have been advised to do so. Failure to follow the policy will constitute academic misconduct and will result in a failing grade. If you are unsure of the status of your application, contact the module leader.

\subsubsection*{Project Management}

The final year project can be quite a daunting and intimidating prospect. Many students find it challenging because of its self-directed nature and the accompanying discipline and dedication needed to follow it through. However, do not worry! Completing the project is not only intellectually liberating and a mark of your academic independence, but will provide you with a sense of achievement and satisfaction.

It will also constitute an key indicator---a symbol---of your competence to potential employers. It gives you something you can showcase in interview and discuss in considerable depth and with enthusiasm. Successful completion of such a project demands a mastery of core employability skills including: initiative; problem solving; communication, both written and spoken; self-regulated learning; as well as planning and management. Though the stretching of your ability in these areas to limits you never thought possible will likely be rather uncomfortable, you feel better for it when you graduate.

A pitfall that many students fall into, however, is time management. Minimise your procrastination and try to chip away at your work a little every day! Although your milestones will vary depending on the nature of your project, ideally the research artefact should be near its completion towards then end of January, permitting you to collect and analyse data across February and March ahead of the interim presentation. Then, in the later half of the study block focuses on interpreting and contextualising your discoveries and writing them up as an academic dissertation. Do not underestimate this process! Your dissertation will take weeks to write!

It is, therefore, critically important that you consider \textbf{project management} again at this stage. Consider the system development life-cycle that is appropriate to your project. Ensure you use appropriate project management tools including critical path analysis, Gantt charts, and burn-down charts to keep track of your progress. Also, do not underestimate the importance of the \textbf{validation and verification} aspect of the research artefact. You must ensure sufficient time is made to enact quality assurance practices that will defend the integrity of your research by showing that your research project was appropriately managed and your research artefact was constructed through the sound application of software engineering practices. 

Explicitly outline your management approach and transparently evidence how you actually tested your software!

If you encounter any issues with respect to your time management, please consult your supervisor who can provide you with advice. They are there to support you, so take advantage of their experience. Also, ensure that you take advantage of the support services offered by the Academic Skills Team (ASK): \url{http://ask.fxplus.ac.uk}

\subsubsection*{Preparing the Manuscript}

The final manuscript should be prepared following the formal IEEE Transactions template with referencing in IEEE style. The manuscript should contain 12-pages of academic content ( excluding figures, tables, the reference section, or any additional material) and a 1-page addendum that evaluates the project. This means there should be 12-pages of content. You should discuss the structure with your supervisor.

\begin{itemize}

\setlength\itemsep{1em}

\item \textbf{Front Matter}: The first page should display your name, the title of the project, the abstract, and a copyright notice. Please do not include your supervisor as a co-author (see acknowledgements). Leverage \texttt{\\thank\{\} } to make a statement of originality together with any intellectual property rights agreements that you have made. The statement of originality should be worded similarly to: `This dissertation is submitted as a requirement for the degree of Bachelor of Science at Falmouth University. It presents work conducted exclusively by the author except where indicated in the text. The report may be freely copied and distributed provided the source is acknowledged.'

\item \textbf{Introduction}: This section should make clear the motivation for the project. The aims of the project should at least be stated in the first paragraph, but preferably in the first sentence. Ensure to include a road-map which explains the structure of the manuscript.

\item \textbf{Body}: The content and structure of the body of your work will vary greatly depending on the nature of the project. You should not have a section called body. Instead, this should be broken down into any number of appropriate sections. For example: literature review; requirements analysis and specification;  design of the research artefact; implementation; research methodology; method of evaluating the research artefact (including how you validated and verified the integrity of your artefact); and findings. For empirical research, ensure that you include clearly identifiable sections that highlight: research questions; hypotheses; study design; results; data analysis; and a discussion that interprets and contextualises your discoveries. Consult your supervisor to develop an appropriate structure for your work.

\item \textbf{Professional Considerations}: Do not neglect the need to acknowledge the ethical and/or professional issues raised by your work. This is an opportunity to refer to the Codes of Conduct published by the BCS and other professional bodies. Ensure that you also acknowledge any provisions that you incorporated into your methodology on ethical grounds.

\item \textbf{Conclusion}: Make clear your discoveries and the key take-away points your research. Highlight any important results. If your discoveries contribute to discourse in your field of study then ensure you have explicitly stated what these contributions are and how they do so. Also, ensure that you make clear the implications of your work and if you have derived any recommendations from your results, then explicitly state these also. 

\item \textbf{Acknowledgements}: It is important to declare those who supported your research: your supervisor(s); other students if part of a related project; and any other person or organisation that has helped you at all.

\item \textbf{References}: There is no room for error. IEEE-style references with 100\% accuracy is required. Zero tolerance. Sources should be cited in the body of your report where appropriate. This includes web pages, which historically have been cited incorrectly by many students.

\item \textbf{Appendices}: you should include your project logs and any additional relevant system or test data as appendices. You may also include any technical material which you estimate as too detailed for the main body of the report. You should make sure that any technical material is appropriately annotated and consistently presented. The source code of your project (including documentation) must be submitted electronically and should not be included in the Appendix. Also, any material that does not fit in the report's Appendix that you nevertheless deem important can be included in the electronic submission. Note, however, that it will be left to the markers' discretion whether they wish to look at such extra material in the electronic submission.

\item \textbf{Reflective Addendum}: Please do not forget to include an appendix in which you evaluate your final year project. This is your opportunity to reflect on what went well and what went wrong. No project will have been executed to perfection. So long as you identify and acknowledge mistakes, showing that you learned from them, you can mitigate any of the issues. Also you should consider, in hindsight, what you would do differently to avoid such issues in the future. It is important to be critical in these self-reflections. It is \textbf{not} a description or a diary. It must be both \textbf{analytical and evaluative}. These will be assessed on a holistic basis than your CPD reports, according to the same principles and similar criteria. 

\end{itemize}

\section*{FAQ}

\begin{itemize}
	\item 	\textbf{What is the deadline for this assignment?} \\ 
    		Falmouth University policy states that deadlines must only be specified on the MyFalmouth system.
    		
	\item 	\textbf{What should I do to seek help?} \\ 
    		You can email your tutor for informal clarifications. For informal feedback, make a pull request on GitHub. 
    		
    	\item 	\textbf{Is this a mistake?} \\ 	
    		If you have discovered an issue with the brief itself, the source files are available at: \\
    		\url{https://github.com/Falmouth-Games-Academy/bsc-assignment-briefs}.\\
    		 Please make a pull request and comment accordingly.
\end{itemize}

\section*{Additional Resources}

\begin{itemize}
   \item \url{http://www.bcs.org/category/6030}
   \item \url{http://www.bcs.org/content/ConWebDoc/39988}
   \item Turk, C. and Kirkman, J. (2001) Effective writing: improving scientific, technical and business communication. 2nd edition. London: Spon.
   \item Sides, C. (1999) How to write and present technical information. 3rd edition. Cambridge: Cambridge University Press.
   
   
\end{itemize}

\rubricyearthree
\rubrictitle{Marking Descriptors: Research Artefact (30\%)}
\rubrichead{Note that this assignment is \textbf{not} marked in a criterion-based fashion.
	Instead, your project supervisor will assign an overall grade by considering the following descriptors in relation to your project.}
	
\begin{markingrubric}
	%
    \firstcriterion{Ethics Approval \& Academic Conduct}{---}
		\grade\fail Ethical approval not obtained or evident lack of academic integrity.
		%\grade{}
		\gradespan{5}{Ethics approval has been sought and obtained. \par Academic conduct is acceptable.}
	%
    \criterion{Novice Competency Threshold}{---}
		\grade\fail Any criterion not achieved at the novice competency.
		%\grade{}
		\gradespan{5}{Research artefact is relevant and satisfactory. \par All novice competency crtieria are met.}
	%
    \criterion{Requirements}{---}
        \grade \fail 
        \grade Some requirements are distinguishable and easily measured.
            \par Requirements somewhat correspond to the proposed research.
        \grade Most requirements are distinguishable and easily measured.
            \par Requirements correspond to the proposed research.
        \grade Nearly all requirements are distinguishable and easily measured.
            \par Requirements clearly correspond to the proposed research.
        \grade All requirements are distinguishable and easily measured.
            \par Requirements clearly and comprehensively correspond to the proposed research.
        \grade All requirements are distinguishable and easily measured.
            \par Requirements clearly and comprehensively correspond to the proposed research.
%
    \criterion{Functional Coherence}{---}
        \grade \fail 
        \grade Some useful and relevant features have been implemented.
            \par Some obvious bugs are detected.
        \grade Many useful and relevant features have been implemented.
            \par There is some evidence of feature creep.
            \par Few obvious bugs  are detected.
        \grade Almost all pertinent features have been implemented.
            \par There is little evidence of feature creep.
            \par Some minor bugs  are detected.
        \grade All pertinent features have been implemented.
            \par There is almost no evidence of feature creep.
            \par Some bugs, purely cosmetic and/or superficial in nature, are detected.
        \grade All pertinent features have been implemented.
            \par There is no evidence of feature creep.
            \par Few to no bugs are detected.
%
    \criterion{Sophistication}{---}
        \grade \fail 
        \grade Some insight into the appropriate use of programming constructs is evident from the source code.
            \par The program structure is adequate.
        \grade Much insight into the appropriate use of programming constructs is evident from the source code.
            \par The program structure is appropriate.
        \grade Considerable insight into the appropriate use of programming constructs is evident from the source code.
            \par The program structure is effective.
        \grade Significant insight into the appropriate use of programming constructs is evident from the source code.
            \par The program structure is highly effective, with high cohesion and low coupling.
        \grade Extensive insight into the effective use of programming constructs is evident from the source code.
            \par The program structure is impressive, with very high cohesion and low coupling.
%
    \criterion{Maintainability}{---}
        \grade \fail 
        \grade The code is somewhat well commented.
            \par Some identifier names are descriptive and appropriate.
            \par Most code adheres to a sensible formatting style.
             \par There is little obvious duplication of code or of literal values.           
        \grade The code is reasonably well commented.
            \par Most identifier names are descriptive and appropriate.
            \par Most code adheres to a sensible formatting style.
             \par There is almost no obvious duplication of code or of literal values.   
        \grade The code is reasonably well commented.
            \par Almost all identifier names are descriptive and appropriate.
            \par Almost all code adheres to a sensible formatting style.
             \par There is no obvious duplication of code or of literal values.
        \grade The code is very well commented.
            \par All identifier names are descriptive and appropriate.
            \par All code adheres to a sensible formatting style.
             \par There is no obvious duplication of code or of literal values.
        \grade The code is extremely well commented.
            \par All identifier names are descriptive and appropriate.
            \par All code adheres to a sensible formatting style.
            \par There is no duplication of code or of literal values.
%
    \criterion{Validation \& Verification}{---}
        \grade \fail 
        \grade Some evidence of source code validation and verification.
            \par Basic unit testing is evident.       
        \grade Much evidence of source code validation and verification.
            \par Basic unit testing is evident, with much coverage.
        \grade Considerable evidence of source code validation and verification.
            \par Sophisticated unit testing is evident, with considerable coverage.
            \par Appropriate testing techniques beyond unit testing have been applied, with some effectiveness at improving integrity/quality.
        \grade Significant evidence of source code validation and verification.
            \par Sophisticated unit testing is evident, with considerable coverage.
            \par Appropriate testing techniques beyond unit testing have been applied, with much effectiveness at improving integrity/quality.
        \grade Extensive evidence of source code validation and verification.
            \par Sophisticated unit testing is evident, with significant coverage.
            \par Innovative or cutting-edge testing techniques have been applied, with considerable rigor.
            
\end{markingrubric}

\rubrictitle{Marking Descriptors: Dissertation (70\%)}
\begin{markingrubric}
	%
    \firstcriterion{Ethics Approval \& Academic Conduct}{---}
		\grade\fail Ethical approval not obtained or evident lack of academic integrity.
		%\grade{}
		\gradespan{5}{Ethics approval has been sought and obtained. \par Academic conduct is acceptable.}
	%
    \criterion{Novice Competency Threshold}{---}
		\grade\fail Any criterion not achieved at the novice competency.
		\gradespan{5}{Satisfactory presentations are delivered.
			\par Adequete participation in peer review exercises, with enough to enable meaningful review alongside critical and constructive comments provided to peers.
			\par All novice competency crtieria are met.
			\par An appropriate number of academic sources are referenced.
			\par All academic sources are referenced appropriately, with no obvious errors or omissions.
			\par All important claims are well-specified, verifiable, and accurate.}
		%\gradespan{4}{}
	%
    \criterion{Abstract}{---}
        \grade \fail 
      \grade A general overview of the dissertation is provided.
        	\par There is some structure.		
      \grade A concise, but targeted summary of key points in the dissertation is provided.
        	\par There is much structure.	      	
       \grade A concise, but highly targeted outline of the purpose and key claims made in the dissertation is provided.
        	\par The abstract broadly incorporates and follows the OCAR elements.	  
      \grade A concise, but highly targeted outline of the relevant academic discourse and potential contributions made in the dissertation is provided.
        	\par The abstract incorporates and follows the OCAR elements.	      	
       \grade A concise, but impressively highlighted gap in the literature and outline of intellectual contributions made in the dissertation to plug that gap is provided. 
        	\par The abstract incorporates and follows the OCAR elements.
        	\par The key contribution of the dissertation is made clear in the abstract.

    \criterion{Introduction}{---}
        \grade \fail 
      \grade Introduction has much structure.
        	\par The field of research is somewhat contextualised.
      \grade Introduction has much structure.
        	\par The field of research is somewhat contextualised and motivated.	      	
       \grade Introduction has considerable structure.
        	\par The motivation for the research is somewhat justified.	  
      \grade Introduction has considerable structure.
        	\par The motivation for the research is clearly justified.	 	      	
       \grade Introduction has significant structure.
        	\par The potential impact of the research is justified effectively.
        		
    \criterion{Literature Review}{---}
        \grade \fail 
        	\par 
        \grade The literature review falls short of being comprehensive, but has some merit.
        	\par  Synthesis is adequete, hinting at a need for the work.
        \grade The literature review falls short of comprehensive, but has much merit.
        	\par  Synthesis is appropriate, highlighting a need for the work.
        \grade The literature review is somewhat comprehensive, having considerable merit.
        	\par  Synthesis is effective, criticising existing work to reveal a gap.
        \grade The literature review is comprehensive, having significant merit.
        	\par Synthesis is effective, criticising existing work to reveal a key gap.
        \grade The literature review is comprehensive, incorporating all important seminal and contemporary works.
        	\par Synthesis is highly effective, critically evaluating existing work to showcase an important gap.
	%
    \criterion{Research Questions \& Hypotheses}{---}
            \grade\fail 
        \grade 		Research questions suggest a basic understanding of concepts and principles in the field.
        \grade 		Research questions suggest a broad understanding of concepts and principles in the field.
        \par		Research questions are original.
        \par		Where relevant, hypotheses are made clear.
        \grade 		Research questions suggest a thorough understanding of the field as well as some insight into its discourse.
        \par		Research questions are original and ambitious.
      \par		Where relevant, hypotheses are made explicit, clear, and well-formed.
        \grade 		Research questions suggest a thorough understanding of the field as well as much insight into its discourse.
        \par		Research questions are original, ambitious, and timely.
      \par		Where relevant, hypotheses are made explicit, clear, and well-formed.
        \grade 		Research questions suggest an comprehensive understanding of the field as well as significant insight into its discourse.
        \par		Research questions are at the cutting edge of the field.
      \par		Where relevant, hypotheses are stated explicitly with precision.
        %
    %
    \criterion{Methodology}{---}
        \grade\fail 
        \grade 		The methodology applied could plausibly lead to a somewhat meaningful claim.
        \par		Methods are justified.
        \grade 		The methodology applied could plausibly lead to a meaningful claim.
        \par		The methodology and integrity of the research are justified.
        \grade 		The applied methodology is scientifically rigorous, and potentially able to lead to a sound claim.
        \par		The methodology and integrity of the research justified somewhat convincingly.
        \grade 		The applied methodology is scientifically rigorous, and quite likely to lead to a sound claim.
        \par		The methodology and integrity of the research justified convincingly.
        \grade 		The applied methodology is extremely scientifically rigorous, perhaps a gold standard, and highly likely to lead to a sound claim.
        \par		The methodology and integrity of the research are justified very convincingly.
    %
    \criterion{Results \& Analysis}{---}
        \grade\fail 
        \grade 		The results and their analysis is basic.
        \grade 		The results and their analysis approach sophistication.
        \par 		The method of analysis is appropriate to the project.
        \grade 		The results and their analysis are somewhat sophisticated and reflect good practice.
        \par 		The method of analysis is appropriate to the project.
        \grade 		The results and their analysis are highly sophisticated and reflect good practice.
        \par 		The method of analysis is suited and tailored to the project.
       \par 		Some care has been taken to guard against potential criticisms of the analysis and its assumptions.
        \par		The analysis shows progress towards publishable quality.
        \grade 		The results and their analysis are very highly sophisticated and could be considered best practice.
        \par 		The method of analysis is not only suited and tailored to the project.
        \par 		Much care has been taken to guard against potential criticisms of the analysis and its assumptions.
        \par		The results and analysis is of publishable quality.
    %
    \criterion{Discussion \& Conclusion}{---}
        \grade\fail 
        \grade		Some insight into the topic is evident.
        \par		Attempts are made at discussion beyond summary.
        \grade		Much insight into the topic is evident.
        \par		Discussion is inferential in nature.
        \par		Key take-away points can be inferred from the work.
        \grade		Considerable insight into the topic is evident.
        \par		Discussion is analytical in nature.
        \par		The limitations of the research are made clear.
        \par		Key take-away points are made clear in the conclusion.
        \grade		Significant  insight into the topic is evident.
        \par		Discussion is both analytical and evaluative in nature.
        \par		The limitations of the research are made explicitly clear.
        \par		Meaningful take-away points are made clear in the conclusion. There might be recommendations.
        \grade		Extensive insight into the topic is evident.
        \par		Discussion is both analytical and evaluative in nature.
        \par		The limitations of the research have been thought through well, with their mitigation and/or acknowledgement made explicitly clear.
        \par		Meaningful contributions to discourse in the field are made clear in the conclusion alongside key recommendations/impact.
       %
    \criterion{Critical Evaluation}{---}
        \grade\fail 
        \grade		Some relevant and critical self-appraisal.
        \grade		Much relevant and critical self-appraisal.
        \par		Outcomes and lessons learned are made clear, and loosely tied to goals to improve.
        \grade		Considerable self-appraisal whcih is both relevant and critical in nature.
        \par		Outcomes and lessons learned are analysed and tied to somewhat well-formed SMART objectives.
        \grade		Significant self-appraisal whcih is both pertinent and critical in nature.
        \par		Outcomes and lessons learned are analysed in much depth and clearly tied to well-formed SMART objectives.
        \grade		Extensive self-appraisal whcih is both pertinent and critical in nature.
        \par		Outcomes and lessons learned are analysed in considerable depth and tied explicitly to well-formed SMART objectives. 
       \par		Evaluation of the project leads to recomendations that others could leverage to improve their research. 
\end{markingrubric}

\section*{Appendix: British Computer Society Requirements}

An individual project is an expectation within undergraduate, integrated masters, and postgraduate masters programmes in computing. Students must be provided with written guidance on all aspects of the project, including selection,conduct, supervision, milestones, format of the report and the criteria for assessment. All projects should reflect the aims and learning outcomes which characterise the programme to which they contribute as set out in the programme specification.

It is expected that within an undergraduate programme, students will undertake a major computing project, normally in
their final year and normally as an individual activity, giving them the opportunity to demonstrate:

\begin{itemize}
\item their ability to apply practical and analytical skills present in the programme as a whole
\item innovation and/or creativity
\item synthesis of information, ideas and practices to provide a quality solution together with an evaluation of that solution
\item that their project meets a real need in a wider context
\item the ability to self-manage a significant piece of work
\item critical self-evaluation of the process
\end{itemize}

Projects must involve the production of a report which should include:

\begin{itemize}
\item Elucidation of the problem and the objectives of the project
\item an in-depth investigation of the context and literature, and where appropriate, other similar products (this section is
likely to be emphasised less for an IEng project)
\item where appropriate, a clear description of the stages of the life cycle undertaken
\item where appropriate, a description of how verification and validation were applied at these stages
\item where appropriate, a description of the use of tools to support the development process
\item a critical appraisal of the project, indicating the rationale for any design/implementation decisions, lessons learnt
during the course of the project, and evaluation (with hindsight) of the project outcome and the process of its
production (including a review of the plan and any deviations from it)
\item a description of any research hypothesis
\item in the event that the individual work is part of a group enterprise, a clear indication of the part played by the author in
achieving the goals of the project and its effectiveness
\item references 
\end{itemize}

In the event of this major activity being undertaken as part of a group enterprise, there is a requirement that the
assessment is such that the individual contribution of each student is measured against all the above learning
outcomes.

For accreditation for CITP, CEng or CSci, the individual project should be worth at least 30 credit points at level 6 or
above. The project must be passed without compensation.
For accreditation for IEng the individual project should be worth at least 20 credit points at level 5 or above. The project
must be passed without compensation. 

\end{document}