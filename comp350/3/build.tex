\documentclass{../../fal_assignment}
\graphicspath{ {../../} }

\usepackage{enumitem}
\setlist{nosep} % Make enumerate / itemize lists more closely spaced
\usepackage[T1]{fontenc} % http://tex.stackexchange.com/a/17858
\usepackage{url}
\usepackage{todonotes}

\title{Research Journal --- Algorithms \& Optimisations}
\author{Brian}
\module{COMP350}

\begin{document}

\maketitle

\section*{Introduction}

\begin{marginquote}
  ``There are two ways of constructing a software design: One way is to make it so simple that there are obviously no deficiencies and the other way is to make it so complicated that there are no obvious deficiencies.''
    
    --- C.A.R. Hoare
    
    \marginquoterule
    
    ``The computing scientist's main challenge is not to get confused by the complexities of [their] own making.''
    
    --- E. W. Dijkstra
    
    \marginquoterule

        ``Controlling complexity is the essence of computer programming.''
    
    --- Brian Kernighan
            
\end{marginquote}
\marginpicture{flavour_pic}{
    \textit{ACM SIGGRAPH} is the premier conference on computer graphics, with contributions to it highly respected by the games industry.
}

In this assignment, you will produce a journal detailing your research on computer graphics. In addition to your individual journal, as a group you will create a community wiki to collect and discuss your findings.

Familiarity with the academic literature is extremely helpful for the computing professional,
both to understand the seminal works that lay the foundations of the field,
and to keep abreast of recent developments at the cutting edge.
Games technology is a fast-moving field, and keeping up is important.
However scientific papers are written in a way that is sometimes daunting to newcomers,
so it is essential to practice the skill of reading and comprehending such papers.
Keeping a research journal is a useful way to record your thoughts
(questions, hypotheses, connections, ideas, ...) as you explore the literature.

This assignment is formed of several parts:

	\begin{enumerate}[label=(\Alph*)]
		\item \textbf{Write} a draft outline in the wiki, that will:
		\begin{enumerate}[label=(\roman*)]
			\item \textbf{Identify} the key topics to be covered by the wiki
			\item \textbf{Structure} these topics in a sensible way
		\end{enumerate}
		\item \textbf{Populate} the wiki with content that will:
    	\begin{enumerate}[label=(\roman*)]
			\item \textbf{share} your findings on what you have read;
			\item \textbf{debate} your findings with your peers.
		\end{enumerate}
	\end{enumerate}

\subsection*{Assignment Setup}

This assignment is an \textbf{academic writing task} and \textbf{wiki task}. Fork the GitHub repository at the following URL:

\indent \url{https://github.com/Falmouth-Games-Academy/comp350-research-journal}

Use the existing directory structure and, as required, extend this structure with sub-directories.

Modify the \texttt{.gitignore} to the defaults for \textbf{TeX}. Please, also ensure that you add editor-specific files and folders to \texttt{.gitignore}. 

\subsection*{Part A}

Part A consists of a \textbf{single formative submission}. This work is \textbf{collaborative}.

To complete Part A, as a class, begin populating the wiki with an outline of the topics to be covered.
Discuss and iterate upon this as a class.
It is expected that the edit log for the wiki will show that everyone has contributed to this process.

You will receive \textbf{immediate informal feedback} in class in week 3.

\subsection*{Part B}

Part B consists of a \textbf{single summative submission}. This work is \textbf{collaborative}.

To complete Part B, populate the wiki with content.
Again, this is expected to be an iterative process to which everybody contributes.

Please ensure that all changes have been made to the wiki by the deadline.

To submit, clone the wiki to your local machine using Git.
Compress the wiki as a \texttt{.zip} file and upload it to the submission area on LearningSpace.
Note that although this is a collaborative assignment,
\textbf{everyone must} submit individually via LearningSpace.

You will receive \textbf{formal feedback} three weeks after the deadline.

\section*{Additional Guidance}

To make the most efficient use of your time,
focus your reading on papers that are relevant to the techniques you have chosen to implement
in your portfolio task for this module.
That said, it is also beneficial to read more widely around the subject area
to better understand the context within which works are situated.
As much as possible you should focus your reading on peer-reviewed scholarly sources reporting primary research:
articles in scientific workshops, conferences, journals, and some books or book chapters.
Other sources tend to be less rigorous, and should be used only for background information
or in cases where their use can be convincingly justified.

A common pitfall is to focus too much on summarising the content of the papers you have read.
For higher marks you need to demonstrate \textbf{insight} into what you have read:
forming inferences and analyses beyond what is written in the paper.
Some questions you might ask yourself are:
Why is the paper significant and/or influential?
Why did the researchers choose the approach that they did?
Is there anything counterintuitive or surprising in the paper?
Do you disagree with any of the assumptions or claims it makes?
Does the paper suggest any further research questions?

A related pitfall is to structure the journal as a sequence of disconnected entries.
Instead aim to \textbf{synthesise} multiple papers into a cohesive whole,
drawing connections between works by different authors.
Forming a holistic picture of a field is much more valuable than simply understanding individual works.

The wiki is primarily intended to become a useful shared resource for the cohort.
As such, students are expected to direct themselves and their peers in populating, structuring and editing the wiki.
If appropriate, you may copy and paste material from your individual journal into the wiki.
However note that a wiki is not a piece of academic writing
and thus will tend to have a less formal tone than you should be aiming for in your journal.
Edits may be required to ensure a consistent tone for the wiki.

You can, and indeed should, take inspiration from others' wiki contributions when working on your own journal.
However you \textbf{must not} copy verbatim material written by others;
doing so will be considered academic misconduct.

Most researchers write scientific papers for the intended audience of their fellow researchers.
Thus some papers can seem impenetrable to the novice reader.
Don't lose heart!
Discuss the paper with your peers.
Follow up the papers it cites to find alternative explanations.
If all else fails, continue reading the paper --- often a difficult paragraph is clarified by something which appears later.

\section*{FAQ}

\begin{itemize}
	\item 	\textbf{What is the deadline for this assignment?} \\ 
    		Falmouth University policy states that deadlines must only be specified on the MyFalmouth system.
    		
	\item 	\textbf{What should I do to seek help?} \\ 
    		You can email your tutor for informal clarifications. For informal feedback, make a pull request on GitHub. 
    		
    	\item 	\textbf{Is this a mistake?} \\ 	
    		If you have discovered an issue with the brief itself, the source files are available at: \\
    		\url{https://github.com/Falmouth-Games-Academy/bsc-assignment-briefs}.\\
    		 Please make a pull request and comment accordingly.
\end{itemize}

	\rubrichead{Criteria marked with a $\ddagger$ are shared by the group. All other criteria are individual.}
	\rubricyearthree
	\begin{markingrubric}
		%
		\firstcriterion{Basic Competency Threshold}{40\%}
		\gradespan{1}{\fail At least one part is missing or is unsatisfactory.}
		\gradespan{5}{Submission of all parts is timely.
			\par The student participated in the viva, with enough work available to hold a meaningful discussion.
			\par Sources have been cited in an appropriate manner, without any obvious errors.
			\par There are no breaches of academic integrity.
		}
		%
		\criterion{Wiki quality}{10\% $\ddagger$}
		\grade\fail There is no structure to the wiki.
		\par 	Substantial spelling and/or grammatical errors.
		\par	Style and tone are not appropriate.
		\grade		There is little structure to the wiki.
		\par 	Many spelling and/or grammatical errors.
		\par	Style and tone are adequate.
		\grade 		There is some structure to the wiki.
		\par 	Some spelling and/or grammatical errors.  
		\par	Style and tone are somewhat appropriate.
		\grade 		The wiki is structured somewhat sensibly.
		\par 	Few spelling and/or grammatical errors.
		\par	Style and tone are mostly appropriate.
		\grade 		The wiki is structured somewhat effectively.
		\par 	Almost no spelling and/or grammatical errors.
		\par	Style and tone are appropriate.
		\grade 		The wiki is structured effectively.
		\par 	No spelling or grammatical errors.
		\par	Style and tone are highly appropriate.
		%
		\criterion{Coverage}{10\% $\ddagger$}
		\grade\fail		The wiki gives almost no coverage of the topic.
		\par			There is a majority of superfluous material.
		\grade			The wiki gives a superficial coverage of the topic.
		\par			There is much superfluous material.
		\grade 		The wiki gives an incomplete coverage of the topic.
		\par			There is some superfluous material.
		\grade 		The wiki gives an adequate coverage of the topic.
		\par			There is some superfluous material.
		\grade 		The wiki gives a somewhat comprehensive coverage of the topic.
		\par			There is little superfluous material.
		\grade 		The wiki gives a comprehensive coverage of the topic.
		\par			There is very little superfluous material.
		%
				\criterion{Synthesis}{10\% $\ddagger$}
		\grade\fail No connections are made between different sources.
		\grade		Basic connections are made between different sources.
		\grade		Reasonable connections are made between different sources.
		\par		Connections go beyond mere description.
		\grade		Strong connections are made between different sources.
		\par		Connections are analytical in nature.
		\grade		Sources are synthesised into a coherent narrative.
		\par		Connections are analytical and evaluative in nature.
		\grade		Sources are synthesised into a highly coherent narrative.
		\par		Connections are analytical and evaluative in nature.
		%
		\criterion{Specificity, verifiability \& accuracy of claims}{10\%}
		\grade\fail 	No citations to evidence to claims.
		\par 		Substantial errors and/or misinterpretations.
		\grade 		Few claims have a clear source of evidence.
		\par 		Significant errors and/or misinterpretations.
		\grade 		Some claims have a clear source of evidence.
		\par 		Many errors and/or misinterpretations.
		\grade 		Many claims have a clear source of evidence.
		\par 		Some errors and/or misinterpretations.
		\grade 		Most claims have a clear source of evidence.
		\par 		Few errors and/or misinterpretations.
		\grade 		All claims have a clear source of evidence.
		\par 		Almost no errors and/or misinterpretations.
		%
		\criterion{Depth of insight}{10\%}
		\grade\fail No insight is demonstrated.
		\par		A poor attempt at summary in the student's own words.
		\grade		Little insight is demonstrated.
		\par		Material is summarised in the student's own words.
		\grade		Some insight is demonstrated.
		\par		Attempts are made at discussion beyond summary.
		\grade		Much insight is demonstrated.
		\par		Discussion is inferential in nature.
		\grade		Considerable insight is demonstrated.
		\par		Discussion is analytical in nature.
		\grade		Significant insight is demonstrated.
		\par		Discussion is analytical and evaluative in nature.
		%
		\criterion{Value of individual contribution}{10\%}
		\grade\fail 	No contribution has been made to the wiki.
		\grade 		A few contributions have been made to the wiki..
		\grade 		Some contributions have been made to the wiki.
		\par		The student's contributions are somewhat integrated with the contributions of others.
		\grade 		Many contributions have been made to the wiki.
		\par		The student's contributions are reasonably integrated with the contributions of others.
		\grade 		A significant number of contributions have been made to the wiki.
		\par		The student's contributions are well integrated with the contributions of others.
		\grade 		An extensive number of contributions have been made to the wiki.
		\par		The student's contributions are tightly integrated with the contributions of others.
	\end{markingrubric}

\end{document}
