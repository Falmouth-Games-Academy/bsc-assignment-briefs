\documentclass{../../fal_assignment}
\graphicspath{ {../../} }

\usepackage{enumitem}
\setlist{nosep} % Make enumerate / itemize lists more closely spaced
\usepackage[T1]{fontenc} % http://tex.stackexchange.com/a/17858
\usepackage[hyphens]{url}
\usepackage{todonotes}

\title{Assignment 1 - Optimisation Task}
\author{Brian McDonaldS}
\module{COMP350}

\begin{document}

\maketitle

%

\section*{Introduction}

\begin{marginquote}
``We should forget about small efficiencies, say about 97\% of the time: premature optimization is the root of all evil. ''

--- Donald Knuth

    \marginquoterule

``There is a famous rule in performance optimization called the 90/10 rule: 90\% of a program's execution time is spent in only 10\% of its code. The standard inference from this rule is that programmers should find that 10\% of the code and optimize it, because that's the only code where improvements make a difference in the overall system performance. ''

--- Richard E. Pattis
\end{marginquote}

\marginpicture{flavour_pic}{
	Unreal Engine CPU Profiler
}

In this assignment, you are required to \textbf{optimise} a PS4 project. You will document the process and the tools you have used in a blog.

Optimisation is the process of modifying a system to make it more efficient. This will often take the form of identifying via profiling and then making a change to improve performance and then re-profiling. 

This assignment is formed of several parts:
\begin{enumerate}[label=(\Alph*)]
	\item \textbf{Write} an initial blog which contains the following
		\begin{enumerate}[label=(\roman*)]
			\item \textbf{describe} the project;
			\item \textbf{outline} the required tools;
		\end{enumerate}
	\item \textbf{Review} the work of your peers
	\item \textbf{Carry out} and \textbf{document}, a series of optimisation steps
	\item \textbf{Present}, as an \textbf{individual}, a practical demo of the computer program to your tutor that will:
	\begin{enumerate}[label=\roman*.]
		\item \textbf{demonstrate} your academic integrity;
		\item as well as \textbf{demonstrate} your \textbf{individual} programming knowledge \textbf{and} communication skills.
	\end{enumerate}
\end{enumerate}

\subsection*{Assignment Setup}

This assignment is a \textbf{programming} task. Fork the GitHub repository at the following URL:

\indent \url{https://github.com/Falmouth-Games-Academy/comp350-optimisation}

Use the existing directory structure and, as required, extend this structure with sub-directories.
Ensure that you maintain the \texttt{readme.md} file.

Modify the \texttt{.gitignore} to the defaults for platform or engine of your project.
Please, also ensure that you add editor-specific files and folders to \texttt{.gitignore}. 

For the blog, you need to log onto the following

\indent \url{https://journal.falmouth.ac.uk}

You should be already attached to a 'class' and Blog page has already been created for you.

\subsection*{Part A}

Part A consists of a \textbf{single formative submission}. This work is \textbf{individual} and will be assessed on a \textbf{threshold} basis. Answer the following questions to pass:

\begin{itemize}
	\item What is the details of the project you are optimising?
	\item What tools are you going to use?
\end{itemize}

To complete Part A, prepare a blog post before \textbf{5pm on Friday of Week 2}.

\subsection*{Part B}

Part B consists of a \textbf{single formative submission}. This work is \textbf{individual} and will be assessed on a \textbf{threshold} basis. You will upload your coursework repository for peer review.

You will be assessed on the following criteria 

\begin{enumerate}[label=(\alph*)]
	\item Submission is timely;
	\item Enough work is available to conduct a meaningful review;
	\item A broadly appropriate review of a peer's work is submitted.
\end{enumerate}

To complete Part B, prepare draft versions of the computer programs. Ensure that the source code and related assets are pushed to GitHub and a pull request is made prior to the scheduled peer-review session. Then, attend the scheduled peer-review session.

You will receive immediate \textbf{informal feedback} from your \textbf{peers}.

\subsection*{Part C}

Part C is a \textbf{single formative submission}. This work is \textbf{individual} and will be assessed on a \textbf{threshold} basis. The following criteria are used to determine a pass or fail:

\begin{enumerate}[label=(\alph*)]
	\item Submission is timely;
	\item Enough work is available to conduct a meaningful review;
	\item A broadly appropriate review of a peer's work is submitted.
\end{enumerate}

To complete Part C, prepare draft versions of the computer programs. Ensure that the source code and related assets are pushed to GitHub and a pull request is made prior to the scheduled sprint review sessions. Then, attend the scheduled sprint review sessions.

\subsection*{Part D}

Part D is a \textbf{single summative submission}. This work is \textbf{individual} and will be assessed on a \textbf{criterion-referenced} basis. Please refer to the marking rubric at the end of this document for further detail.

To complete Part D, revise your program based on the feedback you have received. Then, upload a \texttt{.zip} file to LearningSpace containing the following:

\begin{enumerate}[label=(\alph*)]
	\item The source code for your project;
	\item A \textbf{readme.md} which has URL to your Dev Blog
\end{enumerate}

Please note, the LearningSpace will only accept a single \texttt{.zip} file. You will receive \textbf{formal feedback} three weeks after the final submission deadline.

\section*{Additional Guidance}

As always, avoid underestimating the effort required to implement even simple software; always consider scope.
From the proposal stage, you should consider very carefully what is feasible. The important aspect about this coursework is the profiling process, you should approach this like an experiment and document each step and iteration in the process.



\section*{FAQ}

\begin{itemize}
	\item 	\textbf{What is the deadline for this assignment?} \\ 
    		Falmouth University policy states that deadlines must only be specified on the MyFalmouth system.
    		
	\item 	\textbf{What should I do to seek help?} \\ 
    		You can email your tutor for informal clarifications. For informal feedback, make a pull request on GitHub. 
    		
    	\item 	\textbf{Is this a mistake?} \\ 	
    		If you have discovered an issue with the brief itself, the source files are available at: \\
    		\url{https://github.com/Falmouth-Games-Academy/bsc-assignment-briefs}.\\
    		 Please make a pull request and comment accordingly.
\end{itemize}

\section*{Additional Resources}

\begin{itemize}
    \item \url{http://gameprogrammingpatterns.com/optimization-patterns.html}
    \item \url{https://docs.unrealengine.com/latest/INT/Engine/Performance/}
    \item \url{https://docs.unity3d.com/Manual/Profiler.html}
    \item \url{https://msdn.microsoft.com/en-us/library/ms182372.aspx}
\end{itemize}

\rubricyeartwo
\begin{markingrubric}
	\firstcriterion{Basic Competency Threshold}{40\%}
		\grade\fail At least one part, is missing or is unsatisfactory.
		\gradespan{5}{Submission is timely.
			\par Considerable engagement with version control, commesurate with at least three or more commits per week.
			\par Enough work is available to hold a meaningful discussion.
			\par Clear evidence of programming knowledge and communication skills.
			\par Adequete participation in-class peer-review activities at least at the level of basic competency.
			\par No breaches of academic integrity.}
%
    \criterion{Appropriateness of Project Description}{10\%}
        \grade \fail The description lack detail or is unclear
        \grade The description is poor and requires more detail
            \par There is no list of tools to be used during the optimisation process
        \grade The description is of a good level of detail but it is over-scoped 
            \par Some tools were listed but no detail was given on these
        \grade The description is of a good level of detail and is of appropriate scope
            \par Some tools were listed and there is a brief description of the tool
        \grade The description is of a very good level of detail and is of appropriate scope
            \par Some tools were listed and there is a description of good level of detail
        \grade The description is of an excellent level of detail and is of appropriate scope
            \par  Some tools were listed and there is a description of an excellent level detail
%
    \criterion{Sophistication}{10\%}
        \grade \fail Little insight into the appropriate use of programming constructs is evident from the source code.
            \par The program structure is poor or non-existant.
        \grade Some insight into the appropriate use of programming constructs is evident from the source code.
            \par The program structure is adequate.
        \grade Much insight into the appropriate use of programming constructs is evident from the source code.
            \par The program structure is appropriate.
        \grade Considerable insight into the appropriate use of programming constructs is evident from the source code.
            \par The program structure is effective. There is high cohesion and low coupling.
        \grade Significant insight into the appropriate use of programming constructs is evident from the source code.
            \par The program structure is very effective. There is high cohesion and low coupling.
        \grade Extensive insight into the appropriate use of programming constructs is evident from the source code.
            \par The program structure is extremely effective. There is very high cohesion and very low coupling.
%
    \criterion{Maintainability}{15\%}
        \grade \fail The code is only sporadically commented, if at all, or comments are unclear.
            \par Few identifier names are clear or inappropriate.
            \par Code formatting hinders readability.
        \grade The code is well commented.
            \par Some identifier names are descriptive and appropriate.
            \par An attempt has been made to adhere to a consistent formatting style.
             \par There is little obvious duplication of code or of literal values.           
        \grade The code is reasonably well commented.
            \par Most identifier names are descriptive and appropriate.
            \par Most code adheres to a sensible formatting style.
             \par There is almost no obvious duplication of code or of literal values.   
        \grade The code is reasonably well commented, with appropriate Doxygen-compatiable documentation.
            \par Almost all identifier names are descriptive and appropriate.
            \par Almost all code adheres to a sensible formatting style.
             \par There is no obvious duplication of code or of literal values. Some literal values can be easily ``tinkered''. 
        \grade The code is very well commented, with comprehensive appropriate Doxygen-compatiable documentation.
            \par All identifier names are descriptive and appropriate.
            \par All code adheres to a sensible formatting style.
             \par There is no obvious duplication of code or of literal values. Most literal values are, where appropriate, easily ``tinkered'' outside of the source.  
        \grade The code is commented extremely well, with comprehensive appropriate Doxygen-compatiable documentation.
            \par All identifier names are descriptive and appropriate.
            \par All code adheres to a sensible formatting style.
            \par There is no duplication of code or of literal values. Nearly all literal values are, where appropriate, easily ``tinkered'' outside of the source.  
%
	\criterion{Peer Review}{10\%}
		\grade\fail No peer-review submitted.
		\grade There is evidence of some engagement with peers (e.g.\ code review).
		\grade There is evidence of some engagement with peers (e.g.\ code review).
		\par Comments to peers are somewhat constructive and provide some insight.
		\grade There is evidence of much engagement with peers (e.g.\ code review).
		\par Comments to peers are reasonably constructive and provide much insight.
		\grade There is evidence of much engagement with peers (e.g.\ code review).
		\par Comments to peers are reasonably constructive and provide considerable insight.
		\grade There is evidence of considerable engagement with peers (e.g.\ code review).
		\par Comments to peers are highly constructive and provide significant insight.
%
	\criterion{Optimisation Process}{15\%}
		\grade\fail Optimisation has been carried out but there is no supporting documentation
		\grade Optimisation has been carried out with some basic supporting documentation
		\par The documentation lacks detail and clarity, there is also a lack of figures, charts and tables
		\grade Optimisation has been carried out with more detailed supporting documentation
		\par The documentation lacks detail and clarity, but there is evidence of more analytical approach (including diagrams)
		\grade Optimisation has been carried out and the supporting documentation is of a good level
		\par The documentation is of a good level of detail and there is evidence of more analytical approach (including diagrams)
		\grade Optimisation has been carried out and the supporting documentation is of a very good level
		\par The documentation is of a good level of detail and the commentary on the optimisation process is of a good level of detail
		\grade Optimisation has been carried out and the supporting documentation is of a excellent level
		\par The documentation is of a good level of detail and the commentary on the optimisation process is of a excellent level of detail
\end{markingrubric}

\end{document}