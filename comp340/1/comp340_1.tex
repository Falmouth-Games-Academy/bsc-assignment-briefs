\documentclass{../../fal_assignment}
\graphicspath{ {../../} }

\usepackage{enumitem}
\setlist{nosep} % Make enumerate / itemize lists more closely spaced
\usepackage[T1]{fontenc} % http://tex.stackexchange.com/a/17858
\usepackage{url}
\usepackage{todonotes}

\title{Job Market Evaluation}
\author{Dr Michael Scott \& Alcwyn Parker}
\module{COMP340}

\newcommand{\proposalWordCount}{100}
\newcommand{\essayWordCount}{1000}
\newcommand{\presentationMinutes}{8}
\newcommand{\minReferenceCount}{8}

\begin{document}

\maketitle

\section*{Introduction}

\begin{marginquote}
  ``Research is formalized curiosity, it is poking and prying with a purpose.''
  
   --- Zora Neale Hurston
   
    \marginquoterule
    
    ``Listening is hearing the needs of the customer, understanding those needs and making sure the company recognizes the opportunity they present.''
    
    --- Frank Eliason
    
    \marginquoterule
    
    ``Innovation needs to be part of your culture. Customers are transforming faster than we are, and if we don't catch up, we're in trouble.''
    
    --- Ian Schafer
      
\end{marginquote}
\marginpicture{flavour_pic}{
    Team Niche extensively marketed their game at events such as \textit{PAX} prior to and during its KickStarter campaign. It was funded.
}

In this assignment, you review potential career trajectories, exploring: 

\begin{enumerate}[label=(\roman*)]
    \item Which skills are required for entry-level roles in your area of interest;
    \item and what further professional practice you will need to engage in.
\end{enumerate}

Employers don't care about your piece of paper. At best, it will get you invited to interview. Employers care about the value that you will bring to their business. Typically, evidenced through your experience doing the things that they want you to do. It is, therefore, critica to exercise your commercial awareness and identify those skills that are relevant to employers. This will maximise your chances of securing employment in your industry of choice. Adopt a frank, dispassionate, and commercial perspective in evaluating your marketable skills and then reflect on how your skillset brings value to employers alongwhile identifying gaps which will demand further practice.

This assignment is formed of several parts:

\begin{enumerate}[label=(\Alph*)]
    \item \textbf{Write} a \proposalWordCount-word overview \textbf{with} sources which must:
    	\begin{enumerate}[label=\roman*.]
    		\item \textbf{illustrate} the job market in your field of interest;
    		\item and then \textbf{analyse} at least \textbf{\minReferenceCount} job adverts.
	\end{enumerate}
    \item \textbf{Present} a \presentationMinutes-minute summary of your research that will:
    	\begin{enumerate}[label=\roman*.]
    		\item \textbf{outline} your career aspirations;
    		\item \textbf{analyse} the key skills required;
    		\item \textbf{suggest how} your projects help showcase these skills;
    		\item and \textbf{evaluate} gaps which require further professional practice.
	\end{enumerate}
    \item \textbf{Write} a draft \essayWordCount-word report which will:
    	\begin{enumerate}[label=\roman*.]
    		\item  \textbf{evaluate} the job market and skills required in your field of interest;
    		\item  \textbf{conduct} a sound competitor analysis;
    		\item  and \textbf{evaluate how} your current CV and portfolio aids you.
	\end{enumerate}
    \item \textbf{Write} a final \essayWordCount-word report which will:
    	\begin{enumerate}[label=\roman*.]
    		\item \textbf{revise} any issues raised by your tutor and/or your peers.
	\end{enumerate}
\end{enumerate}

\todo[inline]{\textbf{Note:} This work is individual and so all submissions must be distinctive, even though your projects are collaborative.}

\subsection*{Assignment Setup}

This assignment is an \textbf{business writing task}. Fork the GitHub repository at the following URL:

\indent \url{https://github.com/Falmouth-Games-Academy/comp340-market}

Use the existing directory structure and, as required, extend this structure with sub-directories. Ensure that you maintain the \texttt{readme.md} file.

Modify the \texttt{.gitignore} to the defaults for \textbf{TeX}. Please, also ensure that you add editor-specific files and folders to \texttt{.gitignore}. 

\subsection*{Part A}

Part A consists of a \textbf{single formative submission}. This work is \textbf{individual} and will be assessed on a \textbf{threshold} basis. The following criteria are used to determine a pass or fail:

\begin{enumerate}[label=(\alph*)]
	\item Submission is timely;
	\item Field of employment is clear and appropriate;
	\item At least \minReferenceCount ~relevant job adverts are sourced.
\end{enumerate}

To complete Part A, write your proposal in the \texttt{readme.md} document, ensuring that you include copies of the job adverts and provide links to the original source.  Show these to your personal tutor.  If acceptable, this will be signed-off. 

You will receive immediate \textbf{informal feedback}.

\subsection*{Part B}

Part B is a \textbf{single formative submission}. This work is \textbf{individual} and will be assessed on a \textbf{threshold} basis. The following criteria are used to determine a pass or fail:

\begin{enumerate}[label=(\alph*)]
	\item Research questions are adequately addressed;
	\item Some evidence of analytical and evaluative skills;
	\item Some insight into the skills required for an identifiable field.
\end{enumerate}

To complete Part B, prepare a presentation, and practice your debate and discussion. Prepare a slide deck. Use the job averts and other professional sources that you have identified to support your analysis. Ensure that the slides and any notes are prepared ahead of the scheduled session. Then, attend the scheduled session.

You will receive immediate \textbf{informal feedback}.

\subsection*{Part C}

Part C is a \textbf{single formative submission}. This work is \textbf{individual} and will be assessed on a \textbf{threshold} basis. The following criteria are used to determine a pass or fail:

\begin{enumerate}[label=(\alph*)]
	\item Submission is timely;
	\item Enough work is available to conduct a meaningful review;
	\item A broadly appropriate review of a peer's work is submitted.
\end{enumerate}

To complete Part C, prepare a draft version of the report. Ensure that you submit a \texttt{.pdf} to the peer-review activity on the LearningSpace prior to the scheduled peer-review session. Then, attend the scheduled session.

You will receive \textbf{peer feedback} within 1 working day after the session.

\subsection*{Part D}

Part D is a \textbf{single summative submission}. This work is \textbf{individual} and will be assessed on a \textbf{criterion-referenced} basis. Please refer to the marking rubric at the end of this document for further detail.

To complete Part D, revise the report based on the feedback you have received. Then, upload the essay to the LearningSpace. Please note, the LearningSpace will only accept a single \texttt{.pdf} file.

You will receive \textbf{formal feedback} three weeks after the final deadline.

\section*{Additional Guidance}

As you progress into your final year, you will have discovered that a much greater level of intellectual independence is expected of you. Sessions now focus on student-driven dialogues where important issues are explored instead of merely presentedl. Your tutor is there to highlight opportunities for learning and to facilitate the dialogue. It is up to you to analyse the material and consider its value in the contexct of your own aspirations. It is, therefore, critically important that you engage in research across the study block and review relevant materials ahead of attending each session.

Identifying the most appropriate skills to address and propose to develop through further professional practice is the primary goal of this assignment. It is unlikely that you will settle on the first type of role or set of job averts that you find, but stick with it. You can always repeat the exercise to prepare yourself for a change of role in the future. 

Areas where students tend to lose marks are: depth of insight; analytical skill; and evaluative skill. Depth of insight implies rigorous research, addressing a defined and focused field and associated skills in significant detail, rather than broad field or a general area like `computing'. Adequate analysis implies going beyond mere description, perhaps through: performing calculations, comparing sources, or even deploying reasoning to generate new insights. Adequate evaluation implies making appropriate reference to evidence and ensuring that evidence is of appropriate quality. Further to this, sound and valid arguments are constructed, criticising the claims made by other authors. In your final year, your submissions \textbf{must clearly} illustrate your ability to analyse and evaluate as a minimum requirement to pass!

You have but 1000-words! Depth over breadth. Quality over quantity. Write concisely. Your research skills, such as your ability to lookup facts and figures, is not under assessment so focus on construct a sound competitor analysis and arguing critically about the skills that you need to develop and the approaches to professional practice that you are or will engage in to get there.

\section*{FAQ}

\begin{itemize}
	\item 	\textbf{What is the deadline for this assignment?} \\ 
    		Falmouth University policy states that deadlines must only be specified on the MyFalmouth system.
    		
	\item 	\textbf{What should I do to seek help?} \\ 
    		You can email your tutor for informal clarifications. For informal feedback, make a pull request on GitHub. 
    		
    	\item 	\textbf{Is this a mistake?} \\ 	
    		If you have discovered an issue with the brief itself, the source files are available at: \\
    		\url{https://github.com/Falmouth-Games-Academy/bsc-assignment-briefs}.\\
    		 Please make a pull request and comment accordingly.
\end{itemize}

\section*{Additional Resources}

\begin{itemize}
    \item \url{https://www.kent.ac.uk/careers/sk/top-ten-skills.htm}
    \item \url{http://creativeskillset.org/creative_industries/games/job_roles}
    \item \url{https://www.gamasutra.com/view/feature/131064/so_you_want_to_be_a_game_.php}
    \item \url{https://blog.codinghorror.com/recommended-reading-for-developers/}
    \item \url{http://tinysubversions.com/2005/10/effective-networking-in-the-games-industry-introduction/index.html}
    \item \url{http://www.sloperama.com/advice.html}
\end{itemize}

\rubricyearthree

\begin{markingrubric}
%
    \firstcriterion{Basic Proficiency Threshold}{40\% (Threshold)}
        \gradespan{1}{\fail Parts A---C have not been submitted, are incomplete, or are unsatisfactory.
                \par 		The reference section is incomplete and/or malformed.
                \par 		Some in-text citations and quotations are incorrect.}
        \gradespan{5}{Parts A---C are complete. 
        \par At least ten suitable soures have been referenced.
        	\par 		The reference section is complete and well-formed in either ACM or IEEE format.
                \par 		All in-text citations and quotations are correct.
                \par 		All novice competency criteria met in full.}
%
    \criterion{Relevance \& Focus}{10\%}
%        \grade\fail 	No focus on the research question.
        \grade  \fail	Some to no focus on relevant roles, skills, competition and/or professional practice.
        \grade 		Much focus on relevant roles, skills, competition and/or professional practice.
            \par 		Important skills for the field are explicitly referred to.
        \grade 		Considerable focus on relevant roles, skills, competition and/or professional practice.
            \par 		Important skills for the field are related to the CV and portfolio  pieces.
        \grade 		Significant focus on relevant roles, skills, competition and/or professional practice.
            \par 		Skills critical to the field are highlighted across the CV, portfolio, and future professional practice.
        \grade 		Extensive focus on relevant roles, skills, competition and/or professional practice.
            \par 		Skills critical to the field are analysed in depth across the CV, portfolio, and future professional practice.
        \grade 		Extensive focus on relevant roles, skills, competition and/or professional practice.
            \par 		Skills critical to the field are critically analysed in depth across the CV, portfolio, and future professional practice.
            \par 		Focus on roadmap to success.
%
    \criterion{Depth of Insight}{20\%}
%        \grade\fail 	No depth of insight into software engineering principles.
        \grade  \fail	Some to no depth of insight into roles, skills, competition and/or professional practice.
        \grade 		Much depth of insight into any of roles, skills, competition and/or professional practice.
        \grade 		Considerable depth of insight into any of roles, skills, competition and/or professional practice.
        \par 		Insights highlight the viability of current and future professional practice.
        \grade 		Significant depth of insight into each of roles, skills, competition and professional practice.
        \par 		Insights highlight the trajectory of current and future professional practice.
        \grade 		Extensive depth of insight into each of roles, skills, competition and professional practice.
        \par 		Critical insights highlight the trajectory of current and future professional practice.
        \grade 		Extensive depth of insight into each of roles, skills, competition and professional practice.
        \par 		Critical insights highlight, in detail, a sound pathway to employment with sound recommendations.
%
    \criterion{Analysis \& Evaluation}{10\%}
%        \grade\fail 	No analysis has been presented.
        \grade  \fail	Some to no analysis has been presented.
        \grade 		Much analysis.
        \grade 		Considerable analysis and evaluation.
        \grade 		Significant analysis and evaluation.
        \grade 		Extensive analysis and evaluation.
        \grade 		Extensive analysis and evaluation.
        \par 		Presented analyses are empirically sound.
%
    \criterion{Report Mastery}{10\%}
    
        \grade  \fail	Little to no evidence for mastery of written communication skills.
        \par 		Many obvious spelling and/or grammar errors.
        \par 		Some to no structure.
        
        \grade 		Some evidence of mastery of written communication skills.
        \par 		Most in-text citations and quotations are correct.
        \par 		Only some obvious spelling and/or grammar errors.
        \par 		Much structure.   
        \par 		Many sentences and paragraphs are well constructed.
        \par 		There is a clear introduction and conclusion.
        
        \grade 		Much evidence of mastery of written communication skills.

        \par 		Only few obvious spelling and/or grammar errors.
        \par 		Considerable structure.   
        \par 		Most sentences and paragraphs are well constructed.
        \par 		There is a clear and well-constructed introduction and conclusion.
        
        \grade 		Considerable evidence of mastery of written communication skills.
        \par 		All in-text citations and quotations are correct.
        \par 		No obvious spelling and/or grammar errors.
        \par 		Significant structure, , leveraged to effectively highlight the argument and key takeaway points.  
        \par 		Nearly all sentences and paragraphs are well constructed.
        \par 		There is a clear and well-constructed introduction and conclusion.
        
        \grade 		Significant evidence of mastery of written communication skills.
        \par 		All in-text citations and quotations are correct.
        \par 		No obvious spelling and/or grammar errors.
       \par 		Active voice is prevalent.
        \par 		Grammar is leveraged deliberately to draw attention to salient points.     
        \par 		Extensive structure, , leveraged to effectively highlight the argument and key takeaway points. 
        \par 		All sentences and paragraphs are well constructed.
        \par 		There is a clear and well-constructed introduction and conclusion.
        
        \grade 		Extensive evidence of mastery of written communication skills, commesurate with possible publication.
        \par 		All in-text citations and quotations are correct.
        \par 		No obvious spelling and/or grammar errors.
        \par 		Active voice is prevalent.
        \par 		Grammar is leveraged deliberately to emphasise salient points.     
        \par 		Extensive structure, leveraged to effectively highlight the argument and key takeaway points.
        \par 		All sentences and paragraphs are well constructed.
        \par 		There is a clear and well-constructed introduction and conclusion.
\end{markingrubric}

\end{document}