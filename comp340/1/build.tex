\documentclass{../../fal_assignment}
\graphicspath{ {../../} }

\usepackage{enumitem}
\setlist{nosep} % Make enumerate / itemize lists more closely spaced
\usepackage[T1]{fontenc} % http://tex.stackexchange.com/a/17858
\usepackage{url}
\usepackage{todonotes}

\title{Job Market Evaluation}
\author{Dr Michael Scott \& Gareth Lewis}
\module{COMP340}

\newcommand{\proposalWordCount}{100}
\newcommand{\essayWordCount}{1500}
\newcommand{\presentationMinutes}{8}
\newcommand{\minReferenceCount}{3}

\begin{document}

\maketitle

\section*{Introduction}

\begin{marginquote}
  ``Research is formalized curiosity, it is poking and prying with a purpose.''
  
   --- Zora Neale Hurston
   
    \marginquoterule
    
    ``Listening is hearing the needs of the customer, understanding those needs and making sure the company recognizes the opportunity they present.''
    
    --- Frank Eliason
    
    \marginquoterule
    
    ``Innovation needs to be part of your culture. Customers are transforming faster than we are, and if we don't catch up, we're in trouble.''
    
    --- Ian Schafer
      
\end{marginquote}
\marginpicture{flavour_pic}{
    Team Niche extensively marketed their game at events such as \textit{PAX} prior to and during its KickStarter campaign. It was funded.
}

In this assignment, you review potential career trajectories, exploring: 

\begin{enumerate}[label=(\roman*)]
    \item Which skills are required for entry-level roles in your area of interest;
    \item and what further professional practice you will need to engage in.
\end{enumerate}

Employers don't care about your piece of paper. At best, it will get you invited to interview. Employers care about the value that you will bring to their business. Typically, evidenced through your experience doing the things that they want you to do. It is, therefore, critical to exercise your commercial awareness and identify those skills desired by employers in particular areas. This will help you to secure employment in your industry of choice. Adopt a frank, dispassionate, and commercial perspective in evaluating your marketable skills and then reflect on how your skillset brings value to employers alongwhile identifying gaps which will demand further practice.

This assignment is formed of several parts:

\begin{enumerate}[label=(\Alph*)]
    \item \textbf{Write} a draft \essayWordCount-word report which will:
    	\begin{enumerate}[label=\roman*.]
    		\item \textbf{illustrate} the job market in your field of interest;
    		\item \textbf{analyse} at least \textbf{\minReferenceCount} related job adverts;
       		\item and then \textbf{compile} insights gathered by interviewing relevant experts and mentors. 		
	\end{enumerate}
    \item \textbf{Prepare} your short-form CV \textbf{and enhance} your portfolio site such that it:
    	\begin{enumerate}[label=\roman*.]
    		\item \textbf{incorporates} yourshort-form CV;
    		\item \textbf{highlights}your key skills and abilities in your desired field;
    		\item \textbf{leverages} professional insights to \textbf{enhance} your personal brand;
    		\item \textbf{projects} your professionalism;
    		\item and \textbf{emphasises} your professional practice.
	\end{enumerate}
    \item \textbf{Write} a final \essayWordCount-word report which will:
    	\begin{enumerate}[label=\roman*.]
    		\item \textbf{revise} any issues raised by your tutor and/or your peers.
	\end{enumerate}
    \item \textbf{Present} a final short-form CV and portfolio website which will:
    	\begin{enumerate}[label=\roman*.]
    		\item \textbf{align} with an explicitly declared entry-level role in your report;
    		\item \textbf{revise} any issues raised by your tutor and/or your peers.
	\end{enumerate}
\end{enumerate}

\subsection*{Assignment Setup}

This assignment is a \textbf{personal branding task}. 

\todo[inline]{Please follow the conventions of written reports in Parts A and C, whilst being personable and creative in Parts B and D.}

Use any tools and the platform of your choice for preparing the report, CV, and portfolio website.

\subsection*{Part A}

Part A consists of a \textbf{single formative submission}. This work is \textbf{individual} and will be assessed on a \textbf{threshold} basis. The following criteria are used to determine a pass or fail:

\begin{enumerate}[label=(\alph*)]
	\item Field of employment is clear and appropriate;
	\item At least \minReferenceCount ~relevant job adverts are sourced.
\end{enumerate}

To complete Part A, prepare a draft of your professional practice report. Ensure that, where relevant, you include copies of the job adverts and provide links to the original source.  Interview appropriate members of staff and external mentors to gather insights into these roles and how they align with the skills that you have developed. Show these to your module tutor.  If acceptable, the draft will be signed-off. 

You will receive immediate \textbf{informal feedback} from the module tutor.

\subsection*{Part B}

Part B is a \textbf{single formative submission}. This work is \textbf{individual} and will be assessed on a \textbf{threshold} basis. The following criteria are used to determine a pass or fail:

\begin{enumerate}[label=(\alph*)]
	\item Promotional materials are complete and appropriate;
	\item Relevant skills and experience are highlighted;
	\item Spelling and grammar are immaculate.
\end{enumerate}

To complete Part B, prepare your promotional materials, to include at least a short-form CV and a portfolio website. Build upon the work you delivered in the previous study block. Leverage the job averts and professional insights in the design of these materials. Ensure that these are prepared ahead of the session in Week 7. Then, attend the scheduled session.

You will receive immediate \textbf{informal feedback} from peers.

\subsection*{Part C}

Part C is a \textbf{single summative submission}. This work is \textbf{individual} and will be assessed on a \textbf{criterion-referenced} basis. Please refer to the marking rubric at the end of this document for further detail.

To complete Part C, revise the promotional materials---including CV and online portfolio---based on the feedback you have received. Then, upload the materials to the LearningSpace. Please note, the LearningSpace will only accept a single \texttt{.zip} file.

You will receive \textbf{formal feedback} three weeks after the final deadline.

\subsection*{Part D}

Part D is a \textbf{single summative submission}. This work is \textbf{individual} and will be assessed on a \textbf{criterion-referenced} basis. Please refer to the marking rubric at the end of this document for further detail.

To complete Part D, revise the report based on the feedback you have received. Then, upload the report (together with the material from Part C) to the LearningSpace. Please note, the LearningSpace will only accept a single \texttt{.zip} file.

You will receive \textbf{formal feedback} three weeks after the final deadline.

\section*{Additional Guidance}

As you progress into your final year, you will have discovered that a much greater level of intellectual independence is expected of you. Sessions now focus on student-driven dialogues where important issues are explored instead of merely presented. Your tutor is there to highlight opportunities for learning and to facilitate the dialogue. It is up to you to analyse the material and consider its value in the contexct of your own aspirations. It is, therefore, critically important that you engage in research across the study block and review relevant materials ahead of attending each session.

Identifying the most appropriate skills to address and propose to develop through further professional practice is the primary goal of this assignment. It is unlikely that you will settle on the first type of role or set of job averts that you find, but stick with it. You can always repeat the exercise to prepare yourself for a change of role in the future. 

In order to help you prepare for life in professional practice, you are required to write a professional practice report based on discussions with your product owner/mentor, with members of the team based in your area of game development and on discussions with others who can give you an insight into professional life and development in your area of practice. There are many games and business/enterprise professionals in the university, as well as art, audio, writing and animation professionals; now is the time to actively seek them out and talk to them about professional and business practice.

It is really important that you ensure that you have questions prepared before speaking to someone. Do your homework on what they do and so you can get the best out of them. After speaking (or emailing them) record the material gathered and your thoughts on the topic, all of which can then be analysed and used in your report. The report invites you to critically analyse the information you've gathered and use it to help create a personalised strategy for managing your future personal and professional development.

Areas where students tend to lose marks are: depth of insight; analytical skill; and evaluative skill. Depth of insight implies rigorous research, addressing a defined and focused field and associated skills in significant detail, rather than broad field or a general area like `computing'. Adequate analysis implies going beyond mere description, perhaps through: performing calculations, comparing sources, or even deploying reasoning to generate new insights. Adequate evaluation implies making appropriate reference to evidence and ensuring that evidence is of appropriate quality. Further to this, sound and valid arguments are constructed, criticising the claims made by other authors. In your final year, your submissions \textbf{must clearly} illustrate your ability to analyse and evaluate as a minimum requirement to pass!

You have but \essayWordCount-words! Depth over breadth. Quality over quantity. Write concisely. Your research skills, such as your ability to lookup facts and figures, is not under assessment so focus on construct a sound competitor analysis and arguing critically about the skills that you need to develop and the approaches to professional practice that you are or will engage in to get there.

You can use lists, charts, visual information and bullet points in the report, it doesn't have to simply be an `essay'. What we are looking for however, is effort made in gathering and evaluating/analysing knowledge that will inform and support your own professional development.

You will then make some promotional materials for yourself for personal branding and self-marketing; most likely, a website as well as a professional CV. In some case, other things like business cards might be appropriate. The direction you take with this assessment should reflect your discipline and the path you wish to follow after university---as highlighted in your report. These obviously need to be designed to promote you and, where relevant, your game. Think creatively about your promotional and marketing materials and gear them around making you and your product stand out in a crowded marketplace while remaining professional. This is a chance to be creative or simply very professional. Presentation needs careful thought as that's all part of getting your message across.
 
\section*{FAQ}

\begin{itemize}  		
    	\item 	\textbf{Is this a mistake?} \\ 	
    		If you have discovered an issue with the brief itself, the source files are available at: \\
    		\url{https://github.com/Falmouth-Games-Academy/bsc-assignment-briefs}.\\
    		 Please make a pull request and comment accordingly.
\end{itemize}

\section*{Additional Resources}

\begin{itemize}
    \item \url{https://www.kent.ac.uk/careers/sk/top-ten-skills.htm}
    \item \url{http://creativeskillset.org/creative_industries/games/job_roles}
    \item \url{https://www.gamasutra.com/view/feature/131064/so_you_want_to_be_a_game_.php}
    \item \url{https://blog.codinghorror.com/recommended-reading-for-developers/}
    \item \url{http://tinysubversions.com/2005/10/effective-networking-in-the-games-industry-introduction/index.html}
    \item \url{http://www.sloperama.com/advice.html}
\end{itemize}

\rubricyearthree

\begin{markingrubric}
%
    \firstcriterion{Basic Proficiency Threshold}{40\% (Threshold)}
        \gradespan{1}{\fail Parts A---B have not been submitted, are incomplete, or are unsatisfactory.
                \par 		The reference section is incomplete and/or malformed.
                \par 		Some in-text citations and quotations are incorrect.}
        \gradespan{5}{Parts A---B are complete. 
        	\par 		The reference section is complete and well-formed in either ACM or IEEE format.
        	\par 		No evidence of academic misconduct.
           \par 		All novice competency criteria met in full.}
%
    \criterion{Depth of Insight into Industry}{10\%}
        \grade  \fail     Little to no, industry awareness.
        \par                 Only some, or no depth of insight into roles, skills, competition and/or professional practice.
        \grade 		Some indsutry awareness.
        \par 		Much depth of insight into any of roles, skills, competition and/or professional practice.
        \par 		Evidence that mentors were asked somewhat appropriate questions.
        \grade 		Much industry awareness.
        \par 		Considerable depth of insight into any of roles, skills, competition and/or professional practice.
      \par 		Evidence that mentors were asked somewhat appropriate questions.
        \grade 		Considerable industry awareness.
        \par 		Significant depth of insight into each of roles, skills, competition and professional practice.
        \par 		Evidence that mentors were asked appropriate questions.
        \grade 		Significant industry awareness.
        \par 		Extensive depth of insight into each of roles, skills, competition and professional practice.
        \par 		Evidence that mentors were asked appropriate questions.
        \grade 		Extensive industry awareness.
        \par 		Extensive depth of insight into each of roles, skills, competition and professional practice.
        \par 		Evidence that mentors were asked highly appropriate questions.
%
    \criterion{Sense of Strategy}{10\%}
        \grade  \fail	Little to no strategy derived from interviews.
        \grade 		Much sense of strategy.
            \par 		Skills relevant to the field are explicitly referred to in the professional practice report.
        \par 		Insights highlight the viability of current and future professional practice.
            \par 		Focus on roadmap to securing job.
        \grade 		Considerable sense of strategy.
            \par 		Skills relevant to the field are explicitly referred to in the professional practice report.
        \par 		Critical insights highlight the viability of current and future professional practice.
            \par 		Focus on roadmap to securing job.
        \grade 		Significant sense of strategy.
            \par 		Skills important to the field are highlighted across the professional practice report.
        \par 		Critical insights highlight the trajectory of current and future professional practice.
            \par 		Focus on roadmap to achieving aspirations.
        \grade 		Extensive sense of strategy.
            \par 		Skills important to the field are highlighted across the professional practice report.
        \par 		Critical insights highlight the trajectory of current and future professional practice.            
            \par 		Focus on roadmap to achieving aspirations.
        \grade 		Extensive sense of strategy.
            \par 		Skills critical to the field are highlighted across the professional practice report.
        \par 		Critical insights highlight, in detail, a sound pathway to employment via current and future professional practice.
            \par 		Focus on roadmap to achieving eminence.
%
    \criterion{Report Mastery}{10\%}
    
        \grade  \fail	Little to no evidence for mastery of written communication skills.
        \par 		Many obvious spelling and/or grammar errors.
        \par 		Some to no structure.
        
        \grade 		Some evidence of mastery of written communication skills.
        \par 		Most in-text citations and quotations are correct.
        \par 		Only some obvious spelling and/or grammar errors.
        \par 		Much structure.   
        \par 		Many sentences and paragraphs are well constructed.
        \par 		There is a clear introduction and conclusion.
        
        \grade 		Much evidence of mastery of written communication skills.

        \par 		Only few obvious spelling and/or grammar errors.
        \par 		Considerable structure.   
        \par 		Most sentences and paragraphs are well constructed.
        \par 		There is a clear and well-constructed introduction and conclusion.
        
        \grade 		Considerable evidence of mastery of written communication skills.
        \par 		All in-text citations and quotations are correct.
        \par 		No obvious spelling and/or grammar errors.
        \par 		Significant structure, , leveraged to effectively highlight the argument and key takeaway points.  
        \par 		Nearly all sentences and paragraphs are well constructed.
        \par 		There is a clear and well-constructed introduction and conclusion.
        
        \grade 		Significant evidence of mastery of written communication skills.
        \par 		All in-text citations and quotations are correct.
        \par 		No obvious spelling and/or grammar errors.
       \par 		Active voice is prevalent.
        \par 		Grammar is leveraged deliberately to draw attention to salient points.     
        \par 		Extensive structure, , leveraged to effectively highlight the argument and key takeaway points. 
        \par 		All sentences and paragraphs are well constructed.
        \par 		There is a clear and well-constructed introduction and conclusion.
        
        \grade 		Extensive evidence of mastery of written communication skills, commesurate with possible publication.
        \par 		All in-text citations and quotations are correct.
        \par 		No obvious spelling and/or grammar errors.
        \par 		Active voice is prevalent.
        \par 		Grammar is leveraged deliberately to emphasise salient points.     
        \par 		Extensive structure, leveraged to effectively highlight the argument and key takeaway points.
        \par 		All sentences and paragraphs are well constructed.
        \par 		There is a clear and well-constructed introduction and conclusion.
%        
    \criterion{Appropriateness of the Curriculum Vitae}{10\%}
        \grade \fail  			No stated job.
    		    \par 		Information presented in the CV has little to no relevance to stated job.
        		\par 			CV layout and/or style is not appropriate for a professional context.			
        		\par 			Insufficient experience, skills, and/or qualifications listed.
        \grade 				Information presented in the CV has some relevance to stated job.
        		\par 			CV layout and/or style is somewhat appropriate in a professional context.
        		\par 			Few key skills, experiences, and/or qualifications are adequetely highlighted. 
	\grade 			Information presented in the CV has much relevance to stated job.
		\par			CV layout and/or style is somewhat appropriate in a professional context.
        		\par 			Many key skills, experiences, and/or qualifications are adequetely highlighted. 
        \grade 				Information presented in the CV has considerable relevance to stated job.
		\par			CV layout and/or style is appropriate in a professional context.
		\par			No obvious spelling and/or grammatical errors.
        		\par 			A considerable number of key skills, experiences, and/or qualifications are adequetely highlighted. 
		\par			A passion for the industry is evident.
        \grade 				Information presented in the CV has significant relevance to stated job.
		\par			CV layout and/or style is appropriate in a professional context.
		\par			No obvious spelling and/or grammatical errors.
        		\par 			A comprehensive set of key skills, experiences, and/or qualifications are adequetely highlighted. 
		\par			A passion for the industry is evident.
        \grade 				Information presented in the CV has extensive relevance to stated job.
		\par			CV layout and/or style is highly appropriate in a professional context.   
		\par			No obvious spelling and/or grammatical errors.
        		\par 			A comprehensive set of key skills, experiences, and/or qualifications are adequetely highlighted. 
		\par 			A passion for the industry is apparent in all aspects of the CV.
%        
    \criterion{Appropriateness of Online Portfolio \& Promotional Material}{10\%}
        \grade \fail  			No stated field of interest or career aspiration.
        		\par 			Portfolio and/or promotional materials have little to no relevance to stated field of interest.
        		\par 			Few or no projects and/or practice are appropriate for a professional context.	
	%		
        \grade 				Portfolio and promotional materials have some relevance to stated career aspirations.
        		\par 			Some projects and/or practice are appropriate for a professional context.
	%
	\grade 			Portfolio and promotional materials have much relevance to stated career aspirations.
		\par			Most projects and/or practice are appropriate for a professional context.
        		\par 			Portfolio layout and page navigation provides some support for analysis by potential employers.
		\par			Portfolio and promotional materials evidence a skillset and experience that are somewhat relevant to industry.
	%
        \grade 				Portfolio and promotional materials have considerable relevance to career aspirations.
        		\par 			Most projects and/or practice are appropriate for a professional context.
        		\par 			Portfolio layout and page navigation provides much support for analysis by potential employers.
		\par			Portfolio and promotional materials evidence a skillset and experience relevant to industry.
		\par			A passion for the industry is evident.
	%
        \grade 				Portfolio and promotional materials have significant relevance to career aspirations.
        		\par			Most projects and/or practice are appropriate for a professional context.
        		\par 			Portfolio layout and page navigation provides considerable support for analysis by potential employers.
		\par			Portfolio and promotional materials evidence a skillset and experience relevant to industry.
		\par			A passion for the industry is evident.
	%
        \grade 				Portfolio and promotional materials have extensive relevance to career aspirations.	
      		\par 			All projects and/or practice are appropriate for a professional context.
        		\par 			Portfolio layout and page navigation provides significant support for analysis by potential employers.
		\par			Portfolio and promotional materials evidence a skillset and experience critical to industry.
		\par 			A passion for the industry is apparent in all aspects of the portfolio and promotional materials.
    
    
%
    \criterion{Creative Flair \& Originality}{10\%}
        \grade\fail Little or no creativity.
            \par Promotional materials are uncreative, show very little to no divergent and/or subversive thinking.
        \grade Some creativity.
            \par Promotional materials demonstrate little divergent and/or subversive thinking.
        \grade Much creativity.
            \par Promotional materials demonstrate some divergent and/or subversive thinking.
        \grade Considerable creativity.
            \par Promotional materials demonstrate much divergent and/or subversive thinking.
        \grade Significant creativity.
            \par Promotional materials demonstrate consierable divergent and/or subversive thinking.
        \grade Extensive creativity.
            \par Promotional materials demonstrate significant divergent and/or subversive thinking.
        
\end{markingrubric}

\end{document}