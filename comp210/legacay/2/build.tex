\documentclass{../../fal_assignment}
\graphicspath{ {../../} }

\usepackage{enumitem}
\setlist{nosep} % Make enumerate / itemize lists more closely spaced
\usepackage[T1]{fontenc} % http://tex.stackexchange.com/a/17858
\usepackage{url}
\usepackage{todonotes}

\title{Interaction \& Interfaces}
\author{Alcwyn Parker}
\module{COMP210}

\begin{document}

\maketitle

\section*{Introduction}

\begin{marginquote}
Hacker definition: ``A person who enjoys exploring the details of programmable systems and stretching their capabilities, as opposed to most users, who prefer to learn only the minimum necessary.''

--- Jargon File

\end{marginquote}
\marginpicture{flavour_pic}{
   Microsoft Hololens is a mixed reality system that will revolutionise the way we experience and interact with content. 
}

In this assignment, you are required to design and develop an interface that incorporates either augmented reality (AR) or virtual reality (VR). The interface should function as a means of interacting with the game your team are developing for the COMP230 - Game Development I: Pre-Production module. You may decide that you want to develop a new concept, separate from the work being carried out for COMP230. In this case, you will need to negotiate the potential outcome with the module coordinator. You will have access to a variety of AR/VR peripherals that can be incorporated into your project or you can develop your own. If you already have access to related hardware that you would like to work with this is also acceptable. 

In the games industry, emerging technology have a tendency to disrupt our practice on a regular basis. As a developer it is vital that you are able to adapt and move with the times. What is trending now will soon become industry norms and make way for new innovations. It is easy to become overwhelmed by the pace of technological change that our industry faces. This module aims to encourage you to embrace this tide of technology advancement, using experimentation, tinkering and play to make sense of new technologies and channel your inner creativity rather than frustration. 

This assignment is formed of several parts:

\begin{enumerate}[label=(\Alph*)]
    \item \textbf{Write}, a proposal for a VR/AR interface that will:
    	\begin{enumerate}[label=\roman*.]
    		\item \textbf{state and justify} the game that will be the basis for your interface;
		\item \textbf{assess} VR/AR market;
		\item \textbf{outline} an initial design in detail;
		\item and \textbf{list} the key requirements the prototype must fulfil.
	\end{enumerate}
    \item \textbf{design} and \textbf{Implement} your interface:
    	\begin{enumerate}[label=\roman*.]
		\item \textbf{improve} upon the design iteratively over the course of six weeks;
	\end{enumerate}
    \item \textbf{Implement} a final interface which will:
    	\begin{enumerate}[label=\roman*.]
		\item \textbf{revise} the design based on the results of the peer evaluation session.
	\end{enumerate}
    \item \textbf{Present} a practical demo of the game controller to your tutor that will:
    	\begin{enumerate}[label=\roman*.]
    		\item \textbf{demonstrate} your academic integrity;
    		\item as well as \textbf{show} your programming knowledge \textbf{and} communication skills.
	\end{enumerate}
\end{enumerate}

\subsection*{Assignment Setup}

Fork the GitHub repository at:

\indent \url{https://github.com/Falmouth-Games-Academy/comp210-vr-ar-interface }

Use the existing directory structure and, as required, extend this structure with sub-directories. Ensure that you maintain the \texttt{readme.md} file.

Modify the \texttt{.gitignore} to the defaults for \textbf{whichever} language you have chosen to implement your interface in. Please, also ensure that you add editor-specific files and folders to \texttt{.gitignore}. 

\subsection*{Part A}

Part A consists of a \textbf{single formative submission}. This work will be assessed on a \textbf{threshold} basis. The following criteria are used to determine a pass or fail:

\begin{enumerate}[label=(\alph*)]
	\item Submission is timely;
	\item Choice of interface is feasible;
	\item research is comprehensive;
	\item Design is distinctive and has creative merit.
\end{enumerate}

To complete part A, write your proposal in the \texttt{readme.md} document. Show this to your tutor in-class. If acceptable, this will be signed-off.

\subsection*{Part B}

Part B is formed of \textbf{multiple formative submissions}. This is \textbf{individual} work will be assessed on a \textbf{threshold} basis. The following criteria are used to determine a pass or fail:

\begin{enumerate}[label=(\alph*)]
	\item Submission is timely;
	\item Enough progress is made to conduct a meaningful review each week;
	\item A broadly appropriate evaluation of a peer's work is submitted.
\end{enumerate}

To complete Part B, implement your interface utilising a fast, iterative development process. Demonstrate your progress to your tutor each week in class. Ensure that any digital artefacts (including but not limited to design sketches, photographs, art assets, source code, electronic circuit designs) are pushed to GitHub prior to each weekly session. Also ensure that you bring your prototype with you each week. 

You will receive immediate \textbf{informal feedback} from your \textbf{tutor} and \textbf{peers}.

\subsection*{Part C}

Part C is a \textbf{single summative submission}. This work is \textbf{individual} and will be assessed on a \textbf{criterion-referenced} basis. Please refer to the marking rubric at the end of this document for further detail.

To complete Part C, revise the interface design based on the feedback you have received. Then, upload photos and source code to the LearningSpace. Please note, the LearningSpace will only accept a single \texttt{.zip} file.

You will receive \textbf{formal feedback} from your \textbf{tutor} three weeks after the final submission deadline.

\subsection*{Part D}

Part D is a \textbf{single summative submission}. This is \textbf{individual} work will be assessed on a \textbf{threshold} basis. The following criteria are used to determine a pass or fail: 

\begin{enumerate}[label=(\alph*)]
	\item Enough work is available to hold a meaningful discussion; 
	\item Clear evidence of programming knowledge and communication skills; 
	\item No breaches of academic integrity. 
\end{enumerate}

To complete Part D, prepare a practical demonstration of the interface. Ensure that the source code and related assets are pushed to GitHub and a pull request is made prior to the scheduled viva session. Then, attend the scheduled viva session. 

You will receive \textbf{immediate informal} feedback from your \textbf{tutor}.

\section*{Additional Guidance}
Falmouth University is internationally renowned for the arts. Despite the fact that you are studying for a BSc degree in a technical discipline, you are still expected to strive for the same level of innovation and creative flair as your peers. This assignment is more heavily weighted towards the creative than the assignments you have completed thus far. On this assignment, a competent execution of an unimaginative idea is unlikely to achieve a high grade, as opposed to an imperfect execution of a unique and ambitious concept, which will be seen favourably by examiners. Consider this when working on your design. One approach to promote creativity is divergent thinking: generate ideas by exploring many possible solutions. Often the most interesting ideas are subversive: they deliberately go against the accepted or most obvious solution.

Whichever platform you choose, you may implement your component in C++, C\#, Python, or a combination of these. Other languages may be permitted at the discretion of your tutor, if you can argue convincingly that it is appropriate for the proposed project. The design of your interface must be informed by the usability and user-experience findings from assignment one. 

The development process for your interface should be fast and iterative. In the first five weeks you will journey through the stages of discovery, interpretation and ideation. This will be followed by six weeks of prototyping through fast paced experimentation and evolution. At the end of each ?prototyping? week you should aim to have a fully functional prototype that has been improved considerably from the previous week. 

Most VR/AR headset and peripheral manufacturers release their products with SDKs, plugins and game engine integration. This helps developers utilise these products in their own projects with ease. For this assignment you must not rely on the basic examples and demoes given away for free with the developer kits.  You are expected to produce new code alongside using the appropriate libraries or plugins to create an interface that is unique, novel and innovative. 

Stand on the shoulders of giants. Although you may be new to developing for AR/VR these types of systems have been around since the 1960s. In fact, stereoscopic photo viewers have been around since 1838 (see Charles Wheatstone). Large organisations such as Google, Facebook and many others are all investing massive sums of money in research, production and documentation for VR and AR. There is a wealth of information already out there that you will need to draw upon to create a successful response to the assignment. 

You should aim to demonstrate a high level of sophistication in the technical execution of your prototype. An important part of sophistication is having the insight to choose the right tool for the job: if a simpler technique fulfils all the requirements, use it. The use of unnecessarily complicated techniques, serving only to showcase one's own cleverness, is a dangerous habit. 

The sole purpose of the recorded demonstration is to aid the external moderators and examiners. Furthermore, any photos and/or videos submitted do not need to be entertaining or highly polished.

\section*{FAQ}

\begin{itemize}
	\item 	\textbf{What is the deadline for this assignment?} \\ 
    		Falmouth University policy states that deadlines must only be specified on the MyFalmouth system.
    		
	\item 	\textbf{What should I do to seek help?} \\ 
    		You can email your tutor for informal clarifications. For informal feedback, make a pull request on GitHub. 
    		
    	\item 	\textbf{Is this a mistake?} \\ 	
    		If you have discovered an issue with the brief itself, the source files are available at: \\
    		\url{https://github.com/Falmouth-Games-Academy/bsc-assignment-briefs}.\\
    		 Please raise an issue and comment accordingly.
\end{itemize}

\section*{Additional Resources}

\begin{itemize}
    \item Jerald, Jason. (2015) Human-Centered Design for Virtual Reality. ACM Books.
    \item http://www.gdcvault.com/play/1023649/Human-Centered-Design-of-Immersive 
    \item http://www.gdcvault.com/play/1022810/Interaction-Design-in-VR-The
    \item https://docs.unity3d.com/Manual/VROverview.html
    \item https://play.google.com/store/apps/details?id=com.google.vr.cardboard.apps.designlab
    
\end{itemize}

\rubricyeartwo

% All criterion need moving down a column
\begin{markingrubric}
    \firstcriterion{Basic Competency Threshold}{40\%}
        \gradespan{1}{\fail At least one part is missing or is unsatisfactory. 
        
        There is little or no evidence of an iterative development process and no improvement over time in regards to the quality of the design and build of the prototype.}
        \gradespan{5}{Submission is timely.
        	\par Enough work is available to hold a meaningful discussion.
	\par Clear evidence of a `reasonable' iterative development process
	\par Clear evidence of programming knowledge and communication skills.
	\par Clear evidence of reflection on own performance and contribution.
	\par No breaches of academic integrity.}
%
    \criterion{Design of the solution}{15\%}
        \grade\fail No user stories are provided, or the design does not correspond to the user stories.
        \grade Some user stories are distinguishable and easily measured.
            \par The correspondence between design and user stories is tenuous.
        \grade Little user stories are distinguishable and easily measured.
            \par The design somewhat corresponds to the user stories.
        \grade Most user stories are distinguishable and easily measured.
            \par The design corresponds to the user stories.
        \grade Nearly all user stories are distinguishable and easily measured.
            \par The design clearly corresponds to the user stories.
        \grade All user stories are distinguishable and easily measured.
            \par The design clearly and comprehensively corresponds to the user stories.
%
    \criterion{Commercial awareness}{10\%}
        \grade\fail No commercial awareness is demonstrated.
        \grade Some commercial awareness is demonstrated.
            \par There is no evidence of market research.
        \grade Little commercial awareness is demonstrated.
            \par Market research is present, but with significant gaps.
        \grade Much commercial awareness is demonstrated.
            \par Market research is extensive, but with some gaps.
        \grade Considerable commercial awareness is demonstrated.
            \par Market research is comprehensive.
        \grade Significant commercial awareness is demonstrated.
            \par Market research is comprehensive and insightful.
%
    \criterion{Innovation and creative flair}{5\%}
        \grade\fail No evidence of innovation and/or creativity.
        \grade Some evidence of emerging innovation and/or creativity.
            \par The solution is purely derivative of existing products.
            \par There is no evidence of divergent thinking.
        \grade Little evidence of emerging innovation and/or creativity.
            \par The solution is mostly derivative, with some attempts at innovation.
            \par There is evidence of an attempt at divergent thinking.
        \grade Much evidence of emerging innovation and/or creativity.
            \par The solution is an interesting and somewhat innovative product.
            \par There is some evidence of divergent thinking.
        \grade Considerable evidence of mastery of innovative and creative practice.
            \par The solution is a novel and innovative product.
            \par There is much evidence of divergent thinking.
        \grade Significant evidence of mastery of innovative and creative practice.
            \par The solution is a unique and innovative product.
            \par There is significant evidence of divergent thinking.
%            
    \criterion{Functionality of the Interface}{15\%}
        \grade\fail No interface is produced, or the interface is completely non-functional.
        \grade The interface has no functionality.
            \par There are serious technical and/or constructional flaws.
        \grade The interface has some functionality.
            \par There are obvious technical and/or constructional flaws.
        \grade The interface has much functionality.
            \par There are minor technical and/or constructional flaws.
        \grade The interface has considerable functionality.
            \par There are superficial technical and/or constructional flaws.
        \grade The interface has significant functionality.
            \par The technical execution is flawless.
%
    \criterion{Sophistication: \par Software \par Electronics \par Physical construction}{10\%}
        \grade\fail The solution lacks even a basic level of sophistication in any of the three areas.
        \grade The solution evidences some sophistication in one or more of the three areas.
            \par Some insight has been demonstrated in any area.
        \grade The solution evidences little sophistication in one or more of the three areas.
            \par Little insight has been demonstrated in at least one of the areas.
        \grade The solution evidences much sophistication in one or more of the three areas.
            \par Much insight has been demonstrated in at least one of the areas.
        \grade The solution evidences considerable sophistication in one or more of the three areas.
            \par Considerable insight has been demonstrated in at least one of these areas.
        \grade The solution evidences significant sophistication in one or more of the three areas.
            \par Significant insight has been demonstrated in at least one of these areas.
%            
    \criterion{Use of Version Control}{5\%}
        \grade\fail GitHub has not been used.
        \grade Source code has rarely been checked into GitHub.
        \grade Source code  has been checked into GitHub at least once per week.
            \par Commit messages are present.
            \par There is evidence of engagement with peers (e.g.\ code review).
        \grade Source code  has been checked into GitHub several times per week.
            \par Commit messages are clear, concise and relevant.
            \par There is evidence of somewhat meaningful engagement with peers (e.g.\ code review).
        \grade Source code has been checked into GitHub several times per week.
            \par Commit messages are clear, concise and relevant.
            \par There is evidence of meaningful engagement with peers (e.g.\ code review).
        \grade Source code has been checked into GitHub several times per week.
            \par Commit messages are clear, concise and relevant.
            \par There is evidence of effective engagement with peers (e.g.\ code review).
%
\end{markingrubric}

\end{document}