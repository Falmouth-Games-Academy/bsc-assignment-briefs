\documentclass{../../fal_assignment}
\graphicspath{ {../../} }

\usepackage{enumitem}
\usepackage[T1]{fontenc} % http://tex.stackexchange.com/a/17858
\usepackage{url}
\usepackage{todonotes}

\title{Interface Task}
\author{Alcwyn Parker}
\module{COMP210}

\begin{document}
\maketitle

\begin{marginquote}
    ``I think there is a tendency in science to measure what is measurable and to decide that what you cannot measure must be uninteresting.''
    
    --- Donald Norman
    
    \marginquoterule
    
    ``To design an easy-to-use interface, pay attention to what users do, not what they say. Self-reported claims are unreliable, as are user speculations about future behavior.''
    
    --- Jakob Nielsen
\end{marginquote}

\marginpicture{flavour_pic}{
    One approach to the analysis of player interfaces is gaze tracking, which is used extensively in design of e-sports interfaces.}
\section*{Introduction}

In this assignment, you will evaluate an interface from one of your existing projects, with a focus on usability and user-experience. You will carry out multiple tests to generate both qualitative and quantitive data. Once the data is collected, you will need to analyse, compare and visualise your results then present your findings in a formal report.

The user-experience (UX) of your games heavily depends on the design of the user-interface (UI). According to Nielsen, the aim of user research is to generate, ``actionable insights and drive the company's development activities at both the tactical level (better design) and the strategic level (discovering customer needs and building products to meet and exceed these needs).'' This assignment encourages you interrogate the games you develop on a deeper level, testing assumptions and challenging preconceived ideas through involving the user at all stages of the design and development process. 

This assignment is formed of several parts:

\begin{enumerate}[label=(\alph*)]
    \item \textbf{Set up} a GitHub Pages website \textbf{and modify} the HTML/CSS code that defines the website, such that you: 
    	\begin{enumerate}[label=\roman*.]
    		\item \textbf{manipulate} the default layout to personalise the site;
    		\item and \textbf{structure} the site appropriately for an academic report.
	\end{enumerate}
    \item \textbf{Evaluate} a screen-based game interface, ensuring that you:
    	\begin{enumerate}[label=\roman*.]
    		\item \textbf{critically reflect} on your choice of evaluation methods;
		\item \textbf{note} your findings, in detail, on your GitHub Pages site.
	\end{enumerate}
    \item \textbf{Report} your findings in the form of a well-structured GitHub Pages site that documents your:
    	\begin{enumerate}[label=\roman*.]
    		\item \textbf{methodology} for each piece of user-research carried out;
    		\item \textbf{results} using diagrams and visualisations where appropriate;
    		\item \textbf{interpretation} of the results in the form of a disucussion;
		\item \textbf{synthesis} of the results into a formal conclusion;
    		\item and \textbf{recommendations} for improving the user interface in question.
	\end{enumerate}
	
\end{enumerate}

\subsection*{Assignment Setup}

This assignment is an \textbf{usability evaluation task}. Fork the GitHub repository at:

\indent \url{https://github.com/Falmouth-Games-Academy/comp210-evaluation}

Use the existing directory structure and, as required, extend this structure with sub-directories. Ensure that you maintain the \texttt{readme.md} file.

Modify the \texttt{.gitignore} to the defaults for \textbf{HTML}. Please, also ensure that you add editor-specific files and folders to \texttt{.gitignore}. 

\subsection*{Part A}

Part A is formed of \textbf{single formative submission}. This work is \textbf{individual} and will be assessed on a \textbf{threshold} basis. The following criteria are used to determine a pass or fail:

\begin{enumerate}[label=(\alph*)]
	\item Submission is timely;
	\item The website is live and can be accessed through a web browser;
\end{enumerate}

To complete Part A, follow the GitHub Pages tutorial and setup a site on your GitHub account. Create separate pages for each section of the report. Modify the content and default template to incorporate your own HTML and CSS code. Demonstrate this to your tutor.

You will receive immediate \textbf{informal feedback} from your \textbf{tutor}.

\subsection*{Part B}

Part B is formed of \textbf{multiple formative submissions}. This work is \textbf{individual} and will be assessed on a \textbf{threshold} basis. The following criteria are used to determine a pass or fail:

\begin{enumerate}[label=(\alph*)]
	\item Submission is timely;
	\item Enough progress is made to conduct a meaningful review each week;
	\item At least one qualitative and one quantitative method is applied.
\end{enumerate}

To complete Part B, carry out a thorough evaluation of your chosen interface. Then, document your process and findings using GitHub pages. Ensure that any digital artefacts (including but not limited to sketches, photographs, diagrams, raw data, and any other documentation) are included in your GitHub page. Although some written discussion will be necessary, do not overly rely on this mode of communication. Instead, experiment with embedding dynamic content such as visualisations, images and videos. Demonstrate your progress to your tutor in each session.

You will receive immediate \textbf{informal feedback} from your \textbf{tutor}.

\subsection*{Part C}

Part C is a \textbf{single summative submission}. This work is \textbf{individual} and will be assessed on a \textbf{threshold} basis. The following criteria are used to determine a pass or fail: 

\begin{enumerate}[label=(\alph*)]
	\item Enough work is available to assess; 
	\item Clear evidence of usability testing knowledge and communication skills; 
	\item No breaches of academic integrity. 
\end{enumerate}

Part C is a \textbf{single summative submission}. This work is \textbf{individual} and will be assessed on a \textbf{criterion-referenced} basis. Please refer to the marking rubric at the end of this document for further detail. Complete a report of your research and present it using a GitHub Pages site. Then, upload the GitHub Pages site to the LearningSpace. Please note, the LearningSpace will only accept a single \texttt{.zip} file. 

You will receive \textbf{formal feedback} from your \textbf{tutor} three weeks after the final submission deadline.

\section*{Additional Guidance}

Your choice of game interface should not only be complex and interesting enough to warrant interrogation but also be relevant to your interests and your aspirations as a game developer. The selection process might involve choosing multiple games and using rapid and heavily discounted evaluation methods to identify the game interface that will produce the most insightful results. Before you begin the task you are encouraged to research existing case studies and evaluations to inform your approach. 

Your evaluation must find a balance between expert reviews and usability testing. Human-centred design (HCD) puts the user at the centre of the design process, and thus relying solely on expert reviews will not produce results conducive to a HCD process. The purpose of usability testing is to evaluate the user's behaviour when interacting with an interface and identify the aspects of the interface that are most regularly a source of frustration and confusion. Tests should be designed around tasks and scenarios that represent typical end-user goals. Participants in your studies must span a range of skills and experiences for your results to be meaningful. It is important that you go beyond your course cohort to find participants. 

You should use a range of qualitative and quantitative methods of evaluation. Some suggested methods are: cognitive walkthrough; task analysis; user-story mapping; analytic tools. Many other methods can be found in the academic literature.

GitHub Pages are an invaluable tool for showcasing your work to future employers and collaborators. You will use them a lot more in the third year so it is important that you familiarise yourself with them now. GitHub Pages are created just like any other website, using HTML to layout content, CSS to control the style and JavaScript to create dynamic behaviour such as animations and interactive components. Although you have not been taught these specific languages, your existing knowledge of document markup languages such as LaTeX and programming languages such as Python and C++, combined with the wealth of tutorials available online, should make it relatively easy to pick these languages up.

Poor planning and poor time management can have a significant impact on this assignment. A comprehensive evaluation cannot be `crammed' into a last minute deluge. Sustain a steady pace across the four weeks. Aim to implement one method of evaluation per week.

Areas where students tend to lose marks are: depth of insight; analytical skill; and evaluative skill. Depth of insight implies rigorous testing of each task in detail. Adequate analysis implies going beyond mere description, perhaps through: researching UI/UX, comparing interfaces, and deploying reasoning to generate new insights. Adequate evaluation implies making appropriate reference to evidence and ensuring that evidence is of appropriate quality. Further to this, sound and valid arguments should be constructed based on common usability principles. 

\section*{FAQ}

\begin{itemize}
	\item 	\textbf{What is the deadline for this assignment?} \\ 
    		Falmouth University policy states that deadlines must only be specified on the MyFalmouth system.
    		
	\item 	\textbf{What should I do to seek help?} \\ 
    		You can email your tutor for informal clarifications. For informal feedback, make a pull request on GitHub. 
    		
    	\item 	\textbf{Is this a mistake?} \\ 	
    		If you have discovered an issue with the brief itself, the source files are available at: \\
    		\url{https://github.com/Falmouth-Games-Academy/bsc-assignment-briefs}.\\
    		 Please make a pull request and comment accordingly.
\end{itemize}

\section*{Additional Resources}
\begin{itemize}
    \item Guild, John D., and Clayton Lewis. Designing for Usability: Key Principles and What Designers Think. Communications of the ACM, 1985.
    \item Krug, Steve. Don't Make Me Think. Berkeley, 2000. 
    \item Reiss, Eric, Usable Usability : Simple Steps for Making Stuff Better. Wiley, 2012.
\end{itemize}

\rubricyeartwo

\begin{markingrubric}
%
    \firstcriterion{Basic Competency Threshold}{40\%}
        \gradespan{1}{\fail At least one part is missing or is unsatisfactory. 
        \par There is little or no evidence a usability evaluation of an interface was conducted.
        \par The qualitative and/or quantitative analyses have been omitted.}
        \gradespan{5}{Submission is timely.
        	\par Enough work is available to hold a meaningful discussion.
	\par Clear evidence of a `reasonable' evaluation process.
	\par Clear evidence of usability testing knowledge and communication skills.
	\par No breaches of academic integrity.}
%
    \criterion{Adequacy of Justification of Methods and Practice}{10\%}
        \grade\fail 	There is no justification, or little of the justification is sound.
        \grade 		Some of the methodological justification is sound.
        \par	           A few methodological limitations are acknowledged. 
        \grade 		Most of the methodological justification is sound.
        \par		Appropriate literature has been referenced to support justifications.
        \par 		A few key methodological limitations are acknowledged. 
        \grade 		A considerable amount of the methodological justification is sound.
        \par		Appropriate literature has been referenced to support justifications.
        \par 		Some key methodological limitations are acknowledged. 
        \grade 		Nearly all  of the methodological justification is sound.
        \par		Key literature is appropriately referenced to support justifications.
        \par 		Nearly all methodological limitations are acknowledged and a few are explicitly addressed.
        \grade 		Nearly all  of the methodological justification is sound.
        \par		Seminal literature is appropriately referenced while key literature offers rigorous support for justifications.
        \par		Nearly all methodological limitations are acknowledged and some are explicitly addressed.
%
    \criterion{Specificity, Verifiability, and Validity of Interpretations of Quantitative Data}{5\%}
        \grade\fail 	Quantitative data analyses have been omitted, or are poorly described, or few have a clear source of evidence.
        \grade		Some claims based on quantitative data have a clear source of evidence.
	\par 		There are considerable challenges to the validity of inferences drawn from the data.
        \grade		Many claims based on quantitative data have a clear source of evidence.
        \par 		There are several challenges to the validity of inferences drawn from the data.
	\par 		An attempt to use statistical notation and follow appropriate conventions is evident.
        \grade		A considerable number of claims based on quantitative data have a clear source of evidence.
        \par		There are some challenges to the validity of inferences drawn from the data.
        \par 		Appropriate statistical notation is used and conventions followed.
        \grade		Nearly all claims based on quantitative data have a clear source of evidence.
        \par		There are few, if any, challenges to the validity of inferences drawn from the data.
        \par 		APA format for reporting statistics has somewhat been adhered to.
        \grade		Nearly all claims based on quantitative data have a clear source of evidence.
        \par		There are few, if any, challenges to the validity of inferences drawn from the data.
        \par 		APA format for reporting statistics has been mostly adhered to.
%
    \criterion{Specificity, Verifiability, and Validity of Interpretations of Qualitative Data}{5\%}
        \grade\fail 	Qualitative data analyses have been omitted, or are poorly described, or few have a clear source of evidence.
        \grade		Some claims based on qualitative data have a clear source of evidence.
	\par 		There are considerable challenges to the validity of inferences drawn from the data.
        \grade		Many claims based on qualitative data have a clear source of evidence.
        \par 		There are several challenges to the validity of inferences drawn from the data.
	\par 		An attempt to use quotations and transparently analyse excerpts is evident.
        \grade		A considerable number of claims based on qualitative data have a clear source of evidence.
        \par		There are some challenges to the validity of inferences drawn from the data.
        \par 		An attempt to apply appropriate qualitative data reporting conventions is evident.
        \grade		Nearly all claims based on qualitative data have a clear source of evidence.
        \par		There are few, if any, challenges to the validity of inferences drawn from the data.
        \par 		The SRQR is somewhat adhered to.
        \grade		Nearly all claims based on qualitative data have a clear source of evidence.
        \par		There are few, if any, challenges to the validity of inferences drawn from the data.
        \par		The SRQR is mostly adhered to.
%
    \criterion{Ability to Discern Key Usability and User Experience Issues}{5\%}
        \grade\fail 	Many key usability and user experience issues have been missed.
        \grade		Several key usability and user experience issues have been missed.
        \grade		Only some key usability and user experience issues have been missed.
        \grade		Few, if any, key usability and user experience issues have been missed.
        \grade		Few, if any, usability and user experience issues have been missed.
        \par		Key issues highlighted are pertinent to the play experience.
        \grade		Few, if any, usability and user experience issues have been missed.
        \par		Key issues highlighted identified are priorities for improving the play experience.
%
    \criterion{Depth of Discussion about Key Usability and User Experience Issues}{15\%}
        \grade \fail	Few meaningful connections are made between the findings.
        \grade 		Some meaningful connections are made between the findings.
        \par     		Connections are largely descriptive in nature.
        \grade		Many meaningful connections are made between the findings.
        \par     		Connections are largely analytic in nature.
        \par		An attempt to tie together findings from the different evaluation methods is evident.
        \grade		Considerable connections are made between the findings.
        \par     		Connections are largely analytic in nature.
        \par		An attempt to synthesise findings from the different evaluation methods is evident.
        \grade		Significant connections are made between the findings.
        \par		Connections are largely both analytical and evaluative in nature.
        \par		Triangulation, using different evaluation methods, has been used to support findings.
        \grade		Extensive connections are made between the findings.
        \par		Connections are largely both analytical and evaluative in nature.
        \par		Triangulation, using different evaluation methods, has been used to effectively support and reinforce findings.

%
    \criterion{Adequacy of Justification of Design Recommendations}{10\%}
        \grade\fail 	No design recommendation, or one that while may be generally appropriate, lacks specificity or is unachievable.
        \grade		At least one specific and achievable design change is proposed.
        \grade		At least one specific, relevant, and achievable design change is proposed.
        \par		The design change is justified somewhat appropriately.
        \grade		At least one specific, relevant, and achievable design change is proposed.
        \par		The design change is justified appropriately.
        \grade		At least one specific, important, and achievable design change is proposed.
        \par		The design change is justified rigorously.
        \grade		At least one specific, important, and achievable design change is proposed.
        \par		The design change is justified very rigorously.
%
    \criterion{Effectiveness of Use of Media and Web Technologies}{5\%}
        \grade\fail Website is inoperable, GitHub Pages has not been used, or the content is unreadable. 
        \grade A generally appropriate GitHub Pages website is live and accessible.
        \grade The GitHub Pages website is live, accessible, and has a distinctive layout.
            \par The content has some clarity.
            \par Many images and diagrams have been included.
        \grade The GitHub Pages website is live, accessible, and has a distinctive layout.
            \par The content has much clarity.
            \par Many images and diagrams have been included.
        \grade The GitHub Pages website is live, accessible, and has a distinctive and well-designed layout.
            \par The content has considerable clarity.
            \par Many effective images and diagrams have been included.
            \par Dynamic media has been used to articulate the process and findings. 
        \grade The GitHub Pages website is live, accessible, and has a distinctive and well-designed layout.
            \par The content has significant clarity.
            \par Many effective images and diagrams have been included.
            \par Dynamic media has been used to effectively articulate the process and findings. 
%
    \criterion{Appropriateness of Structure, Layout, and Style}{2.5\%}
        \grade\fail 	There is little to no structure.
        \grade 		There is some structure.
        \par 		A few sentences and paragraphs are well constructed.
        \grade 		There is much structure.
        \par 		Some sentences and paragraphs are well constructed.
        \par 		Website layout and page navigation provides some support for the structure of the analysis.
        \grade 		There is considerable structure.
        \par 		Most sentences and paragraphs are well constructed.
        \par 		Website layout and page navigation provides much support for the structure of the analysis.
        \grade 		There is significant structure.
        \par 		Nearly all sentences and paragraphs are well constructed.
        \par 		Website layout and page navigation provides considerable support for the structure of the analysis.
        \grade 		There is extensive structure.
        \par 		Website layout and page navigation provides significant support for the structure of the analysis.
%
   \criterion{Appropriateness of Spelling and Grammar}{2.5\%}
        \grade\fail 	Many spelling and/or grammar errors.
        \grade		Some spelling and/or grammar errors.
        \grade		Few insight is demonstrated.
        \grade		Almost no spelling and/or grammar errors.
        \grade		No spelling and/or grammar errors.
        \par 		Active voice is prevalent.
        \grade		No spelling and/or grammar errors.
        \par 		Active voice is prevalent.
        \par 		Grammar is leveraged deliberately to draw attention to salient content.
\end{markingrubric}

\end{document}
