\documentclass{../../../fal_assignment}
\graphicspath{ {../../../} }

\usepackage{enumitem}
\setlist{nosep} % Make enumerate / itemize lists more closely spaced
\usepackage[T1]{fontenc} % http://tex.stackexchange.com/a/17858
\usepackage{url}
\usepackage{todonotes}

\title{COMP210 Worksheet: Google Sheets}
\module{COMP210}
\author{Alcwyn Parker}
\version{1.0}

\begin{document}

\maketitle

\section*{Introduction}

\begin{marginquote}
``R provides a wide variety of statistical (linear and nonlinear modelling, classical statistical tests, time-series analysis, classification, clustering) and graphical techniques, and is highly extensible. The S language is often the vehicle of choice for research in statistical methodology, and R provides an Open Source route to participation in that activity.''

--- https://www.r-project.org/
\end{marginquote}
\marginpicture{flavour_pic}{
    Once you have exhausted the functionality of Google Sheets, you should consider learning R. It will allow you to achieve more complex statistical analysis. 
}

\textbf{Quantitative} research can generate a lot of data. Often, this data is meaningless until it has been processed, organised, analysed and then visualised. The aim is to convert the vast sea of meaningless data into coherent and accessible information. There are a wide range of tools and techniques that can aid you in this process. In this worksheet we will interrogate the potential of Google Sheets as a statistical analysis tool. Traditionally data science is the realm of R and Python or software packages such SPSS. Using an example dataset, you will work through several basic statistical analysis examples that prove Google Sheets is a reasonable tool for basic statistics. 

To complete this worksheet:
\begin{enumerate}[label=(\alph*)]
	\item \textbf{Download} the Example data from the LearningSpace, and then upload it to Google Sheets. 
	\item \textbf{Complete} all of the empty tables and generate graphs where there are place holders. 
\end{enumerate}

\section*{Submission instructions}

Once you have completed the Google Sheet activities, email the public link to the module coordinator. 

\section*{Marking criteria}

Remember that \textbf{it is better to submit incomplete work than to submit nothing at all}. If you do not manage to finish all the activities by the deadline, submit the work you have done so far with comprehensive notes next to the activities you were struggling with. 

To demonstrate \textbf{basic competency}, complete \textbf{at least} the descriptive statistic section.

To demonstrate \textbf{basic proficiency}, complete \textbf{at least} the descriptive statistics \& correlation sheets.

To demonstrate \textbf{novice competency}, achieve basic proficiency \& test if there is a correlation between two of the other questions.
	
To demonstrate \textbf{novice proficiency}, achieve basic proficiency plus \textbf{install} the XL Miner Tool Pak add-in and use it to generate the descriptive statistics. 

To demonstrate \textbf{professional competency}, achieve basic proficiency plus \textbf{experiment} with the other functionality provided by XL Miner Tool Pak. Some useful suggestions are ANOVA or T-Tests. 

\end{document}
