\documentclass{../../fal_assignment}
\graphicspath{ {../../} }

\usepackage{enumitem}
\setlist{nosep} % Make enumerate / itemize lists more closely spaced
\usepackage[T1]{fontenc} % http://tex.stackexchange.com/a/17858
\usepackage[hyphens]{url}
\usepackage{todonotes}

\title{Computing Artefact --- AI}
\author{Dr Ed Powley}
\module{COMP250}
\version{5.0}

\begin{document}

\maketitle

\section*{Introduction}

\begin{marginquote}
``To be enjoyable, an AI must put up a good fight but lose more often than win. It must make the player feel clever, sly, cunning, and powerful. It must make the player jump from his seat shouting, `Take that, you little s**t!'\thinspace''

--- Mat Buckland
\end{marginquote}
\marginpicture{flavour_pic_1}{
    Not all video games portray AI in a positive light.
    However, many games rely on AI to create a satisfying gameplay experience.
}

In this assignment, you are required to \textbf{design} and \textbf{implement} a game component
implementing one or more artificial intelligence (AI) techniques.
The definition of ``component'' here is intentionally loose --- it could be a subsystem within a game,
or a standalone tech demo,
or even a tool (such as an editor plugin) to help game designers to author AI behaviours or systems.

In the interests of managing workload, you may consider integrating your component into your GAM240 group game development project.
However if this is not appropriate you may integrate it into another of your current or previous development projects, or into a standalone demo application.
You may use any programming language and game engine you deem appropriate,
including visual scripting languages such as Blueprints;
this choice will probably be dictated by your choice of game.

Almost all types of games use AI in some capacity, and many genres rely on advanced AI techniques.
Control of non-player characters is an important application of AI,
and a wide-ranging one: enemy AI in a realistic stealth game is very different from in an arcade shooter,
which in turn differs from a racing game.
Other applications of AI can include adversaries in board, card or strategy games,
procedural content generators,
procedural narrative engines,
assistive technologies,
AI ``directors'',
and many others.
Your final product will be a portfolio piece, which you can use in future to demonstrate your mastery of these techniques.

The ``Contracts'' section at the end of this document gives some ideas for the type of component you might consider making for this assignment.
This is not an exhaustive list, but is indicative of the scope and level of sophistication you should be aiming for.

This assignment is formed of several parts:
\begin{enumerate}[label=(\Alph*)]
	\item \textbf{Write} a 2-page handout that will:
		\begin{enumerate}[label=(\roman*)]
			\item \textbf{outline} the concept of your component;
			\item \textbf{identify} the game into which your component will be integrated, and how it will fit into
				the overall game concept;
			\item \textbf{describe} the key requirements of your component;
			\item \textbf{state} the AI technique(s) that the component will demonstrate;
			\item \textbf{justify} that the proposed component is feasible in scope.
		\end{enumerate}
	\item \textbf{Implement} your component.
	\item \textbf{Record} a 2--5 minute commentated video demonstration of your component that will:
		\begin{enumerate}[label=(\roman*)]
			\item \textbf{demonstrate} the component in action;
			\item \textbf{describe} the architecture of the component, with reference to source code as appropriate;
			\item \textbf{justify} the choices made in architecting the component.
		\end{enumerate}
\end{enumerate}

\subsection*{Assignment Setup}

This assignment is a \textbf{programming} task.
There is no template version control repository for this assignment;
you should work within the existing repository for your game,
or create a new repository on the Games Academy Git server as appropriate.
Either way, remember to modify the \texttt{.gitignore} file (or equivalent on other version control systems) to exclude temporary build files
from the repository.

\subsection*{Part A}

Part A consists of a \textbf{single formative submission}. This work is \textbf{individual} and will be assessed on a \textbf{threshold} basis. Answer the following questions to pass:

\begin{itemize}
	\item What is the title and high concept of the game or demo into which your component will be integrated?
	\item What functionality will your component include?
	\item How does your component fit into the overall concept of the game or demo?
	\item What are the key requirements?
	\item Is the scope appropriate for the product development time-frame?
\end{itemize}

To complete Part A, prepare the handout using any word processing tool.
Your handout may include images and/or links to online videos.

Show the handout to the \textbf{tutor} in the timetabled proposal review session in week 2 for immediate \textbf{informal feedback}.

\subsection*{Part B}

Part B is a \textbf{single formative submission}. This work is \textbf{individual} and will be assessed on a \textbf{threshold} basis. The following criteria are used to determine a pass or fail:

\begin{enumerate}[label=(\alph*)]
	\item Submission is timely;
	\item Enough work is available to conduct a meaningful review;
	\item A broadly appropriate review of a peer's work is submitted.
\end{enumerate}

To complete Part B, record a 2--5 minute draft video demonstrating your component --- a draft of your submission for Part C below.
Upload your video to Microsoft Stream, and submit a link to the peer review section on LearningSpace prior to the timetabled peer review session.

You will receive immediate \textbf{informal feedback} from your \textbf{peers}.

\subsection*{Part C}

Part C is a \textbf{single summative submission}. This work is \textbf{individual} and will be assessed on a \textbf{criterion-referenced} basis. Please refer to the marking rubric at the end of this document for further detail.

To complete Part C, record a 2--5 minute video demonstrating your component.
The video should show the component in action, as well as some illustration of the underlying architecture.
Optionally, the video may include a voiceover explaining the concept of the component,
commentating the demonstration, and describing the architecture.
Advanced video editing is not required --- a screen-captured video recorded using OBS or similar software is fine.

Upload your video to Microsoft Stream, and submit a link to the submission area on LearningSpace.
You will receive \textbf{formal feedback} three weeks after the final submission deadline.

\section*{Additional Guidance}

As always, avoid underestimating the effort required to implement even simple software; always consider scope.
From the proposal stage, you should consider very carefully what is feasible.

Your code will be assessed on \textbf{functional coherence}:
how well the finished product corresponds to the user stories,
and whether it has any obvious bugs.
Correspondence to user stories runs both ways:
implementing features that were not present in the design (``feature creep'')
is just as bad as neglecting to implement features.

Your code will also be assessed on \textbf{sophistication}.
To succeed on a project of this size and complexity,
you will need to make use of appropriate algorithms, data structures, libraries, and object oriented programming concepts.
Appropriateness to the task at hand is key:
you will \textbf{not} receive credit for complexity  
where something simpler would have sufficed.
Likewise, if you are using an engine such as Unity or Unreal,
you should make use of (and build upon) the AI functionality included therein;
you will \textbf{not} receive extra credit for ``rolling your own'' without good reason to do so.

Use of \textbf{third-party code}, e.g.\ from online tutorials or asset packages,
is permitted as long as you clearly identify which code is not your own, and clearly identify where it is from.
Failure to give proper credit, whether deliberate or not, will result in investigation under the university's \textbf{Academic Misconduct} procedure.

Also bear in mind that you will \textbf{not} receive credit for any code that is not your own ---
it is important to make sure you change or build upon any third-party code in a substantial way,
and are able to articulate how you did this.
For example, simply following a tutorial to implement a particular AI system will not result in a high mark;
however, using a tutorial as a starting point and then making significant contributions of your own will potentially yield much better results.

\textbf{Maintainability} is important in all programming projects,
but doubly so when working in a team.
Use \textbf{comments} liberally to improve code comprehension,
and carefully choose the \textbf{names} for your files, classes, functions and variables.
Use a well-established commenting convention
for \textbf{high-level documentation}.
The open-source tool Doxygen supports several such conventions.
Also ensure that all code corresponds to a sensible and consistent \textbf{formatting style}:
indentation, whitespace, placement of curly braces, etc.
Hard-coded \textbf{literals} (numbers and strings) within the source should be avoided,
with values instead defined as constants together in a single place.
Where appropriate, values should be exposed as properties or variables in the Unity or Unreal editor
so that they can easily be ``\textbf{tinkered}'' without changing the source code.

Maintainability is also important when using \textbf{visual scripting} systems such as \textbf{Blueprints}.
Pay special attention to the \textbf{layout} of your Blueprints,
which should be tidy and should make clear the flow of control and data.
Use \textbf{grouping}, \textbf{macros}, \textbf{functions}, \textbf{routing nodes} etc.\ to achieve this.
Blueprints which resemble bowls of spaghetti will not achieve high marks!

% As with all assignments on this course, you are expected to display a level of
% \textbf{innovation and creative flair} befitting Falmouth University's reputation as a world-leading
% arts institution.
% One approach to promoting creativity is
% \textbf{divergent thinking}: generating ideas by exploring many possible solutions.
% Often the most interesting ideas are \textbf{subversive}: they deliberately go against
% convention or obvious solutions.

%You will \textbf{not} be judged on the quality of your art assets.
%It is fine to use meshes and textures found online,
%as long as they are available under an appropriate license and are properly attributed.

\section*{FAQ}

\begin{itemize}
	\item 	\textbf{What is the deadline for this assignment?} \\ 
    		Falmouth University policy states that deadlines must only be specified on the MyFalmouth system.
    		
	\item 	\textbf{What should I do to seek help?} \\ 
            Contact the module leader, via email or Microsoft Teams message. Contact details can be found on LearningSpace.
    
    	\item 	\textbf{Is this a mistake?} \\ 	
    		If you have discovered an issue with the brief itself, the source files are available at: \\
    		\url{https://github.com/Falmouth-Games-Academy/bsc-assignment-briefs}.\\
    		 Please make a pull request and comment accordingly.
\end{itemize}

% \section*{Additional Resources}

% \begin{itemize}
% 	\item \url{http://aigamedev.com}
%     \item \url{https://docs.unity3d.com/Manual/Navigation.html}
%     \item \url{https://docs.unrealengine.com/latest/INT/Gameplay/AI/}
%     \item \url{https://google.github.io/styleguide/cppguide.html}
% \end{itemize}

\rubrichead{\ }
%\rubricyeartwo
\begin{markingrubric}
	\firstcriterion{Basic Competency Threshold}{30\%}
		\grade\fail At least one part is missing or is unsatisfactory.
		\gradespan{5}{Submission of all parts is timely and satisfactory.
			\par Enough work is available to hold a meaningful discussion.
			\par Clear evidence of programming knowledge and communication skills.
			\par No breaches of academic integrity.}
%
    \criterion{Functional coherence}{20\%}
        \grade \fail Few requirements have been implemented  and/or the code fails to compile or run.
            \par Many obvious and serious bugs are detected.
        \grade Some requirements have been implemented.
            \par Some obvious bugs are detected.
        \grade Many requirements have been implemented.
            \par There is some evidence of feature creep.
            \par Few obvious bugs  are detected.
        \grade Almost all requirements have been implemented.
            \par There is little evidence of feature creep.
            \par Some minor bugs  are detected.
        \grade All requirements have been implemented.
            \par There is almost no evidence of feature creep.
            \par Some bugs, purely cosmetic and/or superficial in nature, are detected.
        \grade All requirements have been implemented.
            \par There is no evidence of feature creep.
            \par Few to no bugs are detected.
%
    \criterion{Sophistication of architecture}{20\%}
        \grade \fail Little insight into the appropriate use of programming constructs and data structures is evident.
            \par The program structure is poor or non-existant.
        \grade Some insight into the appropriate use of programming constructs and data structures is evident.
            \par The program structure is adequate.
        \grade Much insight into the appropriate use of programming constructs and data structures is evident.
            \par The program structure is appropriate.
        \grade Considerable insight into the appropriate use of programming constructs and data structures is evident.
            \par The program structure is effective. There is high cohesion and low coupling.
        \grade Significant insight into the appropriate use of programming constructs and data structures is evident.
            \par The program structure is very effective. There is high cohesion and low coupling.
        \grade Extensive insight into the appropriate use of programming constructs and data structures is evident.
            \par The program structure is extremely effective. There is very high cohesion and very low coupling.
%
    \criterion{Sophistication of AI techniques}{30\%}
        \grade\fail Lack of basic understanding of AI techniques is evident.
            \par AI techniques are unsuited to the task, and/or justification is absent or incoherent.
            \par Application of AI is trivial.
        \grade Understanding of basic AI techniques is demonstrated.
            \par AI techniques are suited to the task, with an attempt at justification.
            \par Application of AI is basic but effective.
        \grade Working knowledge of basic AI techniques is demonstrated.
            \par AI techniques are suited to the task, with adequate justification.
            \par Application of AI is somewhat sophisticated.
        \grade Working knowledge of standard AI techniques is demonstrated.
            \par AI techniques are suited to the task, with good justification.
            \par Application of AI is reasonably sophisticated.
        \grade Extensive knowledge of advanced AI techniques is demonstrated.
            \par AI techniques are highly suited to the task, with strong justification.
            \par Application of AI is sophisticated and complex.
        \grade Mastery of cutting-edge AI techniques is demonstrated.
            \par AI techniques are highly suited to the task, with very strong justification.
            \par Application of AI is highly sophisticated, complex, and state-of-the-art.
\end{markingrubric}

\section*{Appendix: Contracts}

Choose \textbf{one} of the following contracts to implement in your component:

\begin{itemize}
    \item Non-player character behaviour --- note that this must include something more than the basic pathfinding and perception behaviours included in your chosen game engine
    \item Pathfinding with advanced movement, e.g. physics-based vehicle steering
    \item An authored AI behaviour system based on something other than finite state machines or behaviour trees, e.g. planning or utility-based AI
    \item A solver for a puzzle game
    \item An AI director or adaptive difficulty system
    \item A procedural content generation (PCG) system --- either as an in-game system or as an editor tool
    \item A tool to allow designers to specify AI behaviours in editor
    \item Another AI component of your choice,
        subject to discussion with your tutor
        and demonstration that you have researched the feasibility of your chosen techniques
\end{itemize}

\end{document}
