\title{Portfolio of game engine components --- AI (Task 2)}

\maketitle

\label{p:part2-start}

\section*{Introduction}

\begin{marginquote}
``Real stupidity beats artificial intelligence every time.''

--- Terry Pratchett
\end{marginquote}
\marginpicture{flavour_pic_2}{
    MicroRTS, originally developed by Santiago Onta\~n\'on at Drexel University,
    is an open-source software environment designed for benchmarking AI techniques for real-time strategy games.
}

In this assignment, you are required to design and implement
an \textbf{AI bot} for the \textbf{MicroRTS} game.

In most games, the role of AI agents is to provide an enjoyable experience for the player.
However, creating agents that instead try to win the game is an important area of research:
in some genres of games there is demand for such AI,
and it allows games to be used as a testbed for AI techniques that can be applied to real-world decision problems.
Real-time strategy (RTS) games are particularly challenging for AI to play well,
involving complex tactical and strategic decisions.
MicroRTS is a simplified RTS which was developed for benchmarking AI techniques against each other.
In this assignment you will develop a ``bot'' (an AI agent) that plays MicroRTS.
Your bot will compete in a tournament against your classmates' bots.

This assignment is formed of several parts:
\begin{enumerate}[label=(\Alph*)]
	\item \textbf{Implement}, over a course of several sprints, draft versions of your bot:
		\begin{enumerate}[label=(\roman*)]
			\item \textbf{research} and \textbf{devise} appropriate AI methods;
			\item \textbf{implement} these methods in your bot;
			\item \textbf{upload} your bot to the tournament server;
			\item \textbf{revise} your bot based on results from the tournament server;
			\item \textbf{repeat} iteratively.
		\end{enumerate}
	\item \textbf{Implement} a final version of your bot that will:
		\begin{enumerate}[label=(\roman*)]
			\item \textbf{revise} any issues raised by your tutor and/or your peers.
		\end{enumerate}
	\item \textbf{Present} a practical demo of your bot that will:
		\begin{enumerate}[label=(\roman*)]
			\item \textbf{demonstrate} your academic integrity;
			\item \textbf{demonstrate} your individual programming knowledge and communication skills.
		\end{enumerate}
\end{enumerate}

\subsection*{Assignment Setup}

This assignment is a \textbf{programming} task.
Begin by forking the following GitHub repository:

\indent \url{https://github.com/Falmouth-Games-Academy/comp250-bot}

Modify the provided template code to implement your bot.
The base repository's \texttt{readme.md} file contains important instructions for submitting your bot to the tournament server;
be sure to read and follow these instructions carefully.

\subsection*{Part A}

Part A is a \textbf{single formative submission}. This work is \textbf{individual} and will be assessed on a \textbf{threshold} basis. The following criteria are used to determine a pass or fail:

\begin{enumerate}[label=(\alph*)]
	\item Submission is timely;
	\item Enough work is available to conduct a meaningful review;
	\item A broadly appropriate review of a peer's work is submitted.
\end{enumerate}

To complete Part A, develop your bot in an iterative fashion.
Push your code to GitHub regularly; if you have forked the base repository correctly,
every push to the \texttt{master} branch will automatically submit your bot to the tournament server.
Use the results from the tournament server to refine your bot design.

Ensure that a sufficiently advanced draft is uploaded to GitHub, and attend the scheduled peer review session.

\subsection*{Part B}

Part B is a \textbf{single summative submission}. This work is \textbf{individual} and will be assessed on a \textbf{criterion-referenced} basis. Please refer to the marking rubric at the end of this document for further detail.

To complete Part B, revise your bot based on the feedback you have received. Then, upload it to the LearningSpace.
Please note, the LearningSpace will only accept a single \texttt{.zip} file, which must include your submissions for
\textbf{both} Task 1 (Game Component) \textbf{and} Task 2 (MicroRTS Bot).

You will receive \textbf{formal feedback} from your \textbf{tutor} three weeks after the final submission deadline.

\subsection*{Part C}

Part C is a \textbf{single summative submission}. This work is \textbf{individual} and will be assessed on a \textbf{threshold} basis.  The following criteria are used to determine a pass or fail:

\begin{enumerate}[label=(\alph*)]
	\item Enough work is available to hold a meaningful discussion;
	\item Clear evidence of programming knowledge \textbf{and} communication skills;
	\item No breaches of academic integrity.
\end{enumerate}

To complete Part C, prepare a practical demonstration of your bot. Ensure that the source code and related assets are pushed to GitHub and a pull request is made prior to the scheduled viva session. Then, attend the scheduled viva session.

\section*{Additional Guidance}

As always, avoid underestimating the effort required to implement even simple software; always consider scope.
From the proposal stage, you should consider very carefully what is feasible.

Your code will be assessed on \textbf{functional coherence}:
how well the finished product corresponds to the user stories,
and whether it has any obvious bugs.
Correspondence to user stories runs both ways:
implementing features that were not present in the design (``feature creep'')
is just as bad as neglecting to implement features.

%Unlike your previous assignments,
%you will be assessed on the \textbf{performance} of your solution.
%Real-time graphics and simulation are not just about creating aesthetically pleasing effects,
%but doing so whilst maintaining a smooth and consistent framerate free of any lag or glitches that might frustrate the player.
%It may be necessary to trade-off the complexity or fidelity of an effect
%in order to achieve acceptable performance.

Your code will also be assessed on \textbf{sophistication}.
To succeed on a project of this size and complexity,
you will need to make use of appropriate algorithms, data structures, libraries, and object oriented programming concepts.
Appropriateness to the task at hand is key:
you will \textbf{not} receive credit for complexity  
where something simpler would have sufficed.
Likewise, if you are using an engine such as Unity or Unreal,
you should make use of (and build upon) the AI functionality included therein;
you will \textbf{not} receive extra credit for ``rolling your own'' without good reason to do so.

\textbf{Maintainability} is important in all programming projects,
but doubly so when working in a team.
Use \textbf{comments} liberally to improve code comprehension,
and carefully choose the \textbf{names} for your files, classes, functions and variables.
Use a well-established commenting convention
for \textbf{high-level documentation}.
The open-source tool Doxygen supports several such conventions.
Also ensure that all code corresponds to a sensible and consistent \textbf{formatting style}:
indentation, whitespace, placement of curly braces, etc.
Hard-coded \textbf{literals} (numbers and strings) within the source should be avoided,
with values instead defined as constants together in a single place.
Consider allowing some literal values, where appropriate, to be ``tinkered'' without changing the source code,
e.g.\ by defining them in an external file read at startup.

As with all assignments on this course, you are expected to display a level of
\textbf{innovation and creative flair} befitting Falmouth University's reputation as a world-leading
arts institution.
One approach to promoting creativity is
\textbf{divergent thinking}: generating ideas by exploring many possible solutions.
Often the most interesting ideas are \textbf{subversive}: they deliberately go against
convention or obvious solutions.

%You will \textbf{not} be judged on the quality of your art assets.
%It is fine to use meshes and textures found online,
%as long as they are available under an appropriate license and are properly attributed.

\section*{FAQ}

\begin{itemize}
	\item 	\textbf{What is the deadline for this assignment?} \\ 
    		Falmouth University policy states that deadlines must only be specified on the MyFalmouth system.
    		
	\item 	\textbf{What should I do to seek help?} \\ 
    		You can email your tutor for informal clarifications. For informal feedback, make a pull request on GitHub. 
    		
    	\item 	\textbf{Is this a mistake?} \\ 	
    		If you have discovered an issue with the brief itself, the source files are available at: \\
    		\url{https://github.com/Falmouth-Games-Academy/bsc-assignment-briefs}.\\
    		 Please make a pull request and comment accordingly.
\end{itemize}

\section*{Additional Resources}

\begin{itemize}
	\item \url{https://github.com/santiontanon/microrts/wiki}
	\item \url{https://sites.google.com/site/micrortsaicompetition/home}
    \item \url{https://docs.oracle.com/javase/tutorial/}
    \item \url{https://google.github.io/styleguide/javaguide.html}
\end{itemize}

\rubricyeartwo
\begin{markingrubric}
	\firstcriterion{Basic Competency Threshold}{40\%}
		\grade\fail At least one part is missing or is unsatisfactory.
			\par The student is disqualified from the tournament.
		\gradespan{5}{Submission is timely.
			\par Enough work is available to hold a meaningful discussion.
			\par Clear evidence of programming knowledge and communication skills.
			\par No breaches of academic integrity.
			\par Tournament rules have been adhered to.
			\par A bot is pushed to GitHub and submitted to the tournament server at least once per sprint.}
%
    \criterion{Functional Coherence}{10\%}
        \grade \fail The code fails to compile or run.
            \par Many obvious and serious bugs are detected.
        \grade Some obvious bugs are detected.
        \grade Few obvious bugs  are detected.
        \grade Some minor bugs  are detected.
        \grade Some superficial bugs are detected.
        \grade Almost no bugs are detected.
%
    \criterion{Algorithmic Sophistication}{10\%}
        \grade \fail Little insight into the appropriate use of AI techniques is evident from the design of the bot.
        \grade Some insight into the appropriate use of AI techniques is evident from the design of the bot.
        \grade Much insight into the appropriate use of AI techniques is evident from the design of the bot.
        \grade Considerable insight into the appropriate use of AI techniques is evident from the design of the bot.
        \grade Significant insight into the appropriate use of AI techniques is evident from the design of the bot.
        \grade Extensive insight into the appropriate use of AI techniques is evident from the design of the bot.
%
    \criterion{Sophistication of Implementation}{10\%}
        \grade \fail Little insight into the appropriate use of programming constructs is evident from the source code.
            \par The program structure is poor or non-existant.
        \grade Some insight into the appropriate use of programming constructs is evident from the source code.
            \par The program structure is adequate.
        \grade Much insight into the appropriate use of programming constructs is evident from the source code.
            \par The program structure is appropriate.
        \grade Considerable insight into the appropriate use of programming constructs is evident from the source code.
            \par The program structure is effective. There is high cohesion and low coupling.
        \grade Significant insight into the appropriate use of programming constructs is evident from the source code.
            \par The program structure is very effective. There is high cohesion and low coupling.
        \grade Extensive insight into the appropriate use of programming constructs is evident from the source code.
            \par The program structure is extremely effective. There is very high cohesion and very low coupling.
%
    \criterion{Maintainability}{15\%}
        \grade \fail The code is only sporadically commented, if at all, or comments are unclear.
            \par Few identifier names are clear or inappropriate.
            \par Code formatting hinders readability.
        \grade The code is well commented.
            \par Some identifier names are descriptive and appropriate.
            \par An attempt has been made to adhere to a consistent formatting style.
             \par There is little obvious duplication of code or of literal values.           
        \grade The code is reasonably well commented.
            \par Most identifier names are descriptive and appropriate.
            \par Most code adheres to a sensible formatting style.
             \par There is almost no obvious duplication of code or of literal values.   
        \grade The code is reasonably well commented, with appropriate Doxygen-compatiable documentation.
            \par Almost all identifier names are descriptive and appropriate.
            \par Almost all code adheres to a sensible formatting style.
             \par There is no obvious duplication of code or of literal values. Some literal values can be easily ``tinkered''. 
        \grade The code is very well commented, with comprehensive appropriate Doxygen-compatiable documentation.
            \par All identifier names are descriptive and appropriate.
            \par All code adheres to a sensible formatting style.
             \par There is no obvious duplication of code or of literal values. Most literal values are, where appropriate, easily ``tinkered'' outside of the source.  
        \grade The code is commented extremely well, with comprehensive appropriate Doxygen-compatiable documentation.
            \par All identifier names are descriptive and appropriate.
            \par All code adheres to a sensible formatting style.
            \par There is no duplication of code or of literal values. Nearly all literal values are, where appropriate, easily ``tinkered'' outside of the source.  
%
    \criterion{Performance}{15\%}
        \grade\fail No bot is entered into the tournament, or the entered bot is disqualified.
        \gradespan{5}{Marks to be allocated according to the bot's Elo score, as measured in a round-robin tournament.}
\end{markingrubric}

\section*{Appendix: Tournament Rules}

\begin{enumerate}
	\item Bots will play in a \textbf{round-robin} (all-versus-all) tournament.
	\item Bots will be ranked by \textbf{Elo score}, based on wins and losses.
	\item A \textbf{rolling tournament} server is available, to which you may upload your bot for testing.
	\item However, \textbf{final leaderboard position} will be based on the version of your bot that is submitted to LearningSpace.
	\item You \textbf{may} upload multiple bots to the rolling tournament server,
		however \textbf{only one} submission will be accepted for the final tournament.
	\item \textbf{All Java source code} for your bot must be provided ---
		pre-compiled files (\texttt{.class}, \texttt{.jar} etc) are \textbf{not} permitted.
		This includes any third-party libraries you use.
	\item The bot will \textbf{forfeit} the current match if it:
		\begin{enumerate}
			\item Takes longer than 100 milliseconds to choose an action
			\item Allocates more than 1GB of memory
			\item Crashes by raising an uncaught exception
		\end{enumerate}
		Forfeiting a match is recorded as a loss.
	\item The bot will be \textbf{disqualified} from the tournament if it:
		\begin{enumerate}
			\item Fails to compile on the tournament server
			\item Creates additional threads or processes
			\item Accesses GPU resources
			\item Accesses the network
			\item Accesses files outside the working directory
		\end{enumerate}
		Disqualification removes the bot from the tournament, and results in the lowest possible Elo score.
		If a bot is disqualified from the rolling tournament, you may upload a new version to fix the issue.
	\item The bot will be \textbf{permanently disqualified} from the tournament if it:
		\begin{enumerate}
			\item \textbf{Deliberately} attempts to compromise the security of the tournament server
			\item \textbf{Deliberately} attempts to interfere with the smooth running of the tournament or the fairness of the results
		\end{enumerate}
		Permanent disqualification removes the bot from the tournament, and results in a ban on uploading further bots to the server.
		Your bot will also not be included in the final tournament.
		Permanent disqualification will be considered an automatic failure of the Basic Competency Threshold for this assignment.
	\item All students are expected to engage with the competition in an \textbf{ethical} and \textbf{professional} manner.
		Any perceived breaches of this, as judged by the Module Leader, will be penalised accordingly.
\end{enumerate}
\label{p:part2-end}

