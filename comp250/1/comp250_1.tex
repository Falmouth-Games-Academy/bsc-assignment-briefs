\documentclass{../fal_assignment}
\graphicspath{ {../} }

\usepackage{enumitem}
\setlist{nosep} % Make enumerate / itemize lists more closely spaced
\usepackage[T1]{fontenc} % http://tex.stackexchange.com/a/17858
\usepackage[hyphens]{url}
\usepackage{todonotes}

\title{Portfolio of game engine components --- Artificial Intelligence}
\author{Dr Ed Powley}
\module{COMP250}

\begin{document}

\maketitle

\section*{Introduction}

\begin{marginquote}
``To be enjoyable, an AI must put up a good fight but lose more often than win. It must make the player feel clever, sly, cunning, and powerful. It must make the player jump from his seat shouting, `Take that, you little s**t!'\thinspace''

--- Mat Buckland

    \marginquoterule

``Real stupidity beats artificial intelligence every time.''

--- Terry Pratchett
\end{marginquote}
\marginpicture{flavour_pic}{
    Not all video games portray AI in a positive light.
    However, many games rely on AI to create a satisfying gameplay experience.
}

In this assignment, you are required to \textbf{design} and \textbf{implement} a game component
implementing one or more artificial intelligence (AI) techniques.
Note that ``component'' here may also include tools to help game designers to author AI behaviours or systems.
Your component should be integrated into either:
\begin{enumerate}[label=(\Alph*)]
	\item A game developed by BA Digital Games students;
	\item Your COMP230/COMP240 project;
	\item A game or technical demo you developed for any previous module; or
	\item In the case of an authoring component, either a standalone application
		or an editor plugin for e.g.\ Unity or Unreal.
\end{enumerate}
You may use any programming language and game engine you deem appropriate;
this choice will probably be dictated by your choice of game.

Almost all types of games use AI in some capacity, and many genres rely on advanced AI techniques.
Control of non-player characters is an important application of AI,
and a wide-ranging one: enemy AI in a realistic stealth game is very different from in an arcade shooter,
which in turn differs from a racing game.
Other applications of AI can include adversaries in board, card or strategy games,
procedural content generators,
procedural narrative engines,
assistive technologies,
AI ``directors'',
and many others.
Your final product will be a portfolio piece, which you can use in future to demonstrate your mastery of these techniques.

%Graphics technology is one of the most obvious areas in which innovation has driven gaming technology in recent years.
%Modern gaming PCs and consoles contain powerful graphics processing units (GPUs),
%and gamers expect modern games to push this hardware to its full potential.
%In this assignment you will practice the use of advanced graphical effects.
%Your final product will be a portfolio piece, which you can use in future to demonstrate your mastery of these techniques.

This assignment is formed of several parts:
\begin{enumerate}[label=(\Alph*)]
	\item \textbf{Write} a 2-page handout that will:
		\begin{enumerate}[label=(\roman*)]
			\item \textbf{outline} the concept of your component;
			\item \textbf{identify} the game into which your component will be integrated, and how it will fit into
				the overall game concept;
			\item \textbf{describe} the key requirements of your component.
		\end{enumerate}
	\item \textbf{Populate} a Trello board with the user stories for your component.
	\item \textbf{Implement}, over a series of \textbf{three} sprints, draft versions of your component that will:
		\begin{enumerate}[label=(\roman*)]
			\item \textbf{implement} the user stories from your Trello board.
		\end{enumerate}
	\item \textbf{Implement} a final version of your program, which will:
		\begin{enumerate}[label=(\roman*)]
			\item \textbf{revise} any issues raised by your tutor and/or your peers.
		\end{enumerate}
	\item \textbf{Present} a practical demo of the computer program to your tutor that will:
		\begin{enumerate}[label=(\roman*)]
			\item \textbf{demonstrate} your academic integrity;
			\item \textbf{demonstrate} your individual programming knowledge and communication skills.
		\end{enumerate}
\end{enumerate}

\subsection*{Assignment Setup}

This assignment is a \textbf{programming} task.
There is no template GitHub repository for this assignment;
you should work within the existing source control repository for your game
or create a new GitHub repository as appropriate.
Either way, remember to modify the \texttt{.gitignore} file to exclude temporary build files
from the repository.

\subsection*{Part A}

Part A consists of a \textbf{single formative submission}. This work is \textbf{individual} and will be assessed on a \textbf{threshold} basis. Answer the following questions to pass:

\begin{itemize}
	\item What is the title and high concept of the game or demo into which your component will be integrated?
	\item What functionality will your component include?
	\item How does your component fit into the overall concept of the game or demo?
	\item What are the key requirements?
	\item Is the scope appropriate for the product development time-frame?
\end{itemize}

To complete Part A, prepare the handout using any word processing tool.
Your handout may include images and/or links to online videos.
In addition, create a Trello board and populate it with your user stories.

Show the handout to the \textbf{tutor} for immediate \textbf{informal feedback}.

\subsection*{Part B}

Part B consists of a \textbf{single formative submission}. This work is \textbf{individual} and will be assessed on a \textbf{threshold} basis. Answer the following questions to pass:

\begin{itemize}
	\item What are the user stories for your component?
	\item Which user stories are required for a minimum viable product, and which are stretch goals?
	\item Is the scope appropriate for the product development time-frame?
\end{itemize}

To complete Part B, populate a Trello board with your user stories.

Show the Trello board to the \textbf{tutor} for immediate \textbf{informal feedback}.

\subsection*{Part C}

Part C is a \textbf{single formative submission}. This work is \textbf{individual} and will be assessed on a \textbf{threshold} basis. The following criteria are used to determine a pass or fail:

\begin{enumerate}[label=(\alph*)]
	\item Submission is timely;
	\item Enough work is available to conduct a meaningful review;
	\item A broadly appropriate review of a peer's work is submitted.
\end{enumerate}

To complete Part C, prepare draft versions of the computer programs. Ensure that the source code and related assets are pushed to GitHub and a pull request is made prior to the scheduled sprint review sessions. Then, attend the scheduled sprint review sessions.

\subsection*{Part D}

Part D is a \textbf{single summative submission}. This work is \textbf{individual} and will be assessed on a \textbf{criterion-referenced} basis. Please refer to the marking rubric at the end of this document for further detail.

To complete Part D, revise the computer program based on the feedback you have received. Then, upload it to the LearningSpace. Please note, the LearningSpace will only accept a single \texttt{.zip} file.

You will receive \textbf{formal feedback} from your \textbf{tutor} three weeks after the final submission deadline.

\subsection*{Part E}

Part E is a \textbf{single summative submission}. This work is \textbf{individual} and will be assessed on a \textbf{threshold} basis.  The following criteria are used to determine a pass or fail:

\begin{enumerate}[label=(\alph*)]
	\item Enough work is available to hold a meaningful discussion;
	\item Clear evidence of programming knowledge \textbf{and} communication skills;
	\item No breaches of academic integrity.
\end{enumerate}

To complete Part E, prepare a practical demonstration of the computer program. Ensure that the source code and related assets are pushed to GitHub and a pull request is made prior to the scheduled viva session. Then, attend the scheduled viva session.

\section*{Additional Guidance}

As always, avoid underestimating the effort required to implement even simple software; always consider scope.
From the proposal stage, you should consider very carefully what is feasible.

Your code will be assessed on \textbf{functional coherence}:
how well the finished product corresponds to the user stories,
and whether it has any obvious bugs.
Correspondence to user stories runs both ways:
implementing features that were not present in the design (``feature creep'')
is just as bad as neglecting to implement features.

%Unlike your previous assignments,
%you will be assessed on the \textbf{performance} of your solution.
%Real-time graphics and simulation are not just about creating aesthetically pleasing effects,
%but doing so whilst maintaining a smooth and consistent framerate free of any lag or glitches that might frustrate the player.
%It may be necessary to trade-off the complexity or fidelity of an effect
%in order to achieve acceptable performance.

Your code will also be assessed on \textbf{sophistication}.
To succeed on a project of this size and complexity,
you will need to make use of appropriate algorithms, data structures, libraries, and object oriented programming concepts.
Appropriateness to the task at hand is key:
you will \textbf{not} receive credit for complexity  
where something simpler would have sufficed.
Likewise, if you are using an engine such as Unity or Unreal,
you should make use of (and build upon) the AI functionality included therein;
you will \textbf{not} receive extra credit for ``rolling your own'' without good reason to do so.

\textbf{Maintainability} is important in all programming projects,
but doubly so when working in a team.
Use \textbf{comments} liberally to improve code comprehension,
and carefully choose the \textbf{names} for your files, classes, functions and variables.
Use a well-established commenting convention
for \textbf{high-level documentation}.
The open-source tool Doxygen supports several such conventions.
Also ensure that all code corresponds to a sensible and consistent \textbf{formatting style}:
indentation, whitespace, placement of curly braces, etc.
Hard-coded \textbf{literals} (numbers and strings) within the source should be avoided,
with values instead defined as constants together in a single place.
Consider allowing some literal values, where appropriate, to be ``tinkered'' without changing the source code,
e.g.\ by defining them in an external file read at startup.

As with all assignments on this course, you are expected to display a level of
\textbf{innovation and creative flair} befitting Falmouth University's reputation as a world-leading
arts institution.
One approach to promoting creativity is
\textbf{divergent thinking}: generating ideas by exploring many possible solutions.
Often the most interesting ideas are \textbf{subversive}: they deliberately go against
convention or obvious solutions.

%You will \textbf{not} be judged on the quality of your art assets.
%It is fine to use meshes and textures found online,
%as long as they are available under an appropriate license and are properly attributed.

\section*{FAQ}

\begin{itemize}
	\item 	\textbf{What is the deadline for this assignment?} \\ 
    		Falmouth University policy states that deadlines must only be specified on the MyFalmouth system.
    		
	\item 	\textbf{What should I do to seek help?} \\ 
    		You can email your tutor for informal clarifications. For informal feedback, make a pull request on GitHub. 
    		
    	\item 	\textbf{Is this a mistake?} \\ 	
    		If you have discovered an issue with the brief itself, the source files are available at: \\
    		\url{https://github.com/Falmouth-Games-Academy/bsc-assignment-briefs}.\\
    		 Please make a pull request and comment accordingly.
\end{itemize}

\section*{Additional Resources}

\begin{itemize}
	\item \url{http://aigamedev.com}
    \item \url{https://docs.unity3d.com/Manual/Navigation.html}
    \item \url{https://docs.unrealengine.com/latest/INT/Gameplay/AI/}
    \item \url{https://google.github.io/styleguide/cppguide.html}
\end{itemize}

\rubricyeartwo
\begin{markingrubric}
	\firstcriterion{Basic Competency Threshold}{40\%}
		\grade\fail At least one part, or at least one sprint review, is missing or is unsatisfactory.
		\gradespan{5}{Submission is timely.
			\par Enough work is available to hold a meaningful discussion.
			\par Clear evidence of programming knowledge and communication skills.
			\par No breaches of academic integrity.
			\par The student participates in all sprint reviews}
%
    \criterion{Appropriateness of User Stories and Sprint Plans}{5\%}
        \grade \fail Few user stories are distinguishable and easily measured.
            \par Sprint plans provide little support for the project.
        \grade Some user stories are distinguishable and easily measured.
            \par Sprint plans provide some support for the project.
        \grade Most user stories are distinguishable and easily measured.
            \par User stories correspond to the product design.
            \par Sprint plans provide much support for the project.
        \grade Nearly all user stories are distinguishable and easily measured.
            \par User stories clearly correspond to the product design.
            \par Sprint plans provide considerable support for the project.
        \grade All user stories are distinguishable and easily measured.
            \par User stories clearly and comprehensively correspond to the product design.
            \par Sprint plans provide significant support for the project.
        \grade All user stories are distinguishable and easily measured.
            \par User stories clearly and comprehensively correspond to the product design.
            \par Sprint plans provide extensive support for the project.
%
    \criterion{Functional Coherence}{5\%}
        \grade \fail Few user stories have been implemented  and/or the code fails to compile or run.
            \par Many obvious and serious bugs are detected.
        \grade Some user stories have been implemented.
            \par Some obvious bugs are detected.
        \grade Many user stories have been implemented.
            \par There is some evidence of feature creep.
            \par Few obvious bugs  are detected.
        \grade Almost all user stories have been implemented.
            \par There is little evidence of feature creep.
            \par Some minor bugs  are detected.
        \grade All user stories have been implemented.
            \par There is almost no evidence of feature creep.
            \par Some bugs, purely cosmetic and/or superficial in nature, are detected.
        \grade All user stories have been implemented.
            \par There is no evidence of feature creep.
            \par Few to no bugs are detected.
%
    \criterion{Sophistication}{15\%}
        \grade \fail Little insight into the appropriate use of programming constructs is evident from the source code.
            \par The program structure is poor or non-existant.
        \grade Some insight into the appropriate use of programming constructs is evident from the source code.
            \par The program structure is adequate.
        \grade Much insight into the appropriate use of programming constructs is evident from the source code.
            \par The program structure is appropriate.
        \grade Considerable insight into the appropriate use of programming constructs is evident from the source code.
            \par The program structure is effective. There is high cohesion and low coupling.
        \grade Significant insight into the appropriate use of programming constructs is evident from the source code.
            \par The program structure is very effective. There is high cohesion and low coupling.
        \grade Extensive insight into the appropriate use of programming constructs is evident from the source code.
            \par The program structure is extremely effective. There is very high cohesion and very low coupling.
%
    \criterion{Maintainability}{15\%}
        \grade \fail The code is only sporadically commented, if at all, or comments are unclear.
            \par Few identifier names are clear or inappropriate.
            \par Code formatting hinders readability.
        \grade The code is well commented.
            \par Some identifier names are descriptive and appropriate.
            \par An attempt has been made to adhere to a consistent formatting style.
             \par There is little obvious duplication of code or of literal values.           
        \grade The code is reasonably well commented.
            \par Most identifier names are descriptive and appropriate.
            \par Most code adheres to a sensible formatting style.
             \par There is almost no obvious duplication of code or of literal values.   
        \grade The code is reasonably well commented, with appropriate Doxygen-compatiable documentation.
            \par Almost all identifier names are descriptive and appropriate.
            \par Almost all code adheres to a sensible formatting style.
             \par There is no obvious duplication of code or of literal values. Some literal values can be easily ``tinkered''. 
        \grade The code is very well commented, with comprehensive appropriate Doxygen-compatiable documentation.
            \par All identifier names are descriptive and appropriate.
            \par All code adheres to a sensible formatting style.
             \par There is no obvious duplication of code or of literal values. Most literal values are, where appropriate, easily ``tinkered'' outside of the source.  
        \grade The code is commented extremely well, with comprehensive appropriate Doxygen-compatiable documentation.
            \par All identifier names are descriptive and appropriate.
            \par All code adheres to a sensible formatting style.
            \par There is no duplication of code or of literal values. Nearly all literal values are, where appropriate, easily ``tinkered'' outside of the source.  
%
    \criterion{Creative Flair}{10\%}
        \grade\fail Little or no creativity.
            \par The work is a clone of an existing work with mere cosmetic alterations.
            \par The work delivers little or no fun and/or engagement.
        \grade Some creativity.
            \par The work is derivative of existing works, with only minor alterations.
            \par The work delivers some fun and/or engagement.
        \grade Much creativity.
            \par The work is derivative of existing works, demonstrating little divergent and/or subversive thinking.
            \par The work delivers much fun and/or engagement.
        \grade Considerable creativity.
            \par The work is somewhat novel, demonstrating some divergent and/or subversive thinking.
            \par The work delivers considerable fun and/or engagement.
        \grade Significant creativity.
            \par The work is novel, demonstrating significant divergent and/or subversive thinking.
            \par The work delivers significant fun and/or engagement.
        \grade Extensive creativity.
            \par The work is highly original, with strong evidence of divergent and/or subversive thinking.
            \par The work delivers extensive fun and/or engagement.
%
    \criterion{Portability and Navigability}{5\%}
        \grade\fail Product will not execute at all on another machine, for reasons related to code portability, even if they are trivially resolvable.
            \par The directory structure inside the submitted zip file is unclear.
            \par Provided template has not been followed well, if at all.
        \grade Several portability issues are present.
            \par The directory structure inside the submitted zip file is somewhat confusing.
            \par The provided template has mostly been followed.
        \grade Some portability issues are present.
            \par The directory structure inside the submitted zip file is adequate.
            \par The provided template has been followed.
        \grade Few portability issues are present.
            \par The directory structure inside the submitted zip file is mostly sensible.
            \par The provided template has been followed.
        \grade Almost no portability issues are present.
            \par The directory structure inside the submitted zip file is sensible.
            \par The provided template has been followed.
        \grade No portability issues are present.
            \par There is cross-platform compatibility.
            \par The directory structure inside the submitted zip file is sensible.
            \par The provided template has been followed.
%
    \criterion{Use of Version Control}{5\%}
        \grade \fail Material has been checked into GitHub less frequently than once per sprint.
        \grade Code has been checked into GitHub at least once per sprint.
        \grade Code has been checked into GitHub several times per sprint.
            \par Commit messages are clear, concise and relevant.
            \par There is some evidence of engagement with peers (e.g.\ code review).
        \grade Code has been checked into GitHub several times per sprint.
            \par Commit messages are clear, concise and relevant.
            \par There is much evidence of engagement with peers (e.g.\ code review).
        \grade Code has been checked into GitHub several times per sprint.
            \par Commit messages are clear, concise and relevant.
            \par There is significant evidence of engagement with peers (e.g.\ code review).
        \grade Code has been checked into GitHub several times per week.
            \par Commit messages are clear, concise and relevant.
            \par There is extensive evidence of engagement with peers (e.g.\ code review).
\end{markingrubric}

\end{document}