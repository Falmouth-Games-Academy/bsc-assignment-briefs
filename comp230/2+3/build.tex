\documentclass{../../fal_assignment}
\graphicspath{ {../../} }

\usepackage{enumitem}
\usepackage[T1]{fontenc} % http://tex.stackexchange.com/a/17858
\usepackage{url}
\usepackage{todonotes}

\usepackage{listings}
\lstset{
    basicstyle=\ttfamily,
	frame=single,
	tabsize=4,
	showstringspaces=false,
	breaklines=true,
    language=C++
}

\title{Pre-Production Tasks \& Pitches}
\author{Dr Michael Scott}
\module{COMP230}
\version{2.1}

\begin{document}

\maketitle

\begin{marginquote}
    ``The first 90 percent of the code accounts for the first 90 percent of the development time.
    
    ``The remaining 10 percent of the code accounts for the other 90 percent of the development time.''
    
    --- Tom Cargill
    
    \marginquoterule
    
    ``Hofstadter's Law:
    
    ``It always takes longer than you expect, even when you take into account Hofstadter's Law.''
    
    --- Douglas Hofstadter
\end{marginquote}
\marginpicture{flavour_pic}{
    Don't make \textit{Hammer}, but don't under-estimate the value of tool programming.
}

\section*{Introduction}

You will prepare a pre-production prototype of a gamet. You will work  in a tea with peers drawn from other courses in the Games Academy. You will be involved in ideation, but will primarily adopt a role as a `technical specialist' and `data abalyst' to develop game components and facilitate play-testing.

Game development is a diverse ecology. Tackling the challenges of multidisciplinarity is important---notably, appreciating the roles of computing experts.

This assignment is formed of several parts:

\begin{enumerate}[label=(\Alph*)]
    \item \textbf{Present}, as an \textbf{individual}, a 10-minute `world pitch' that will:
    	\begin{enumerate}[label=\roman*.]
    		\item \textbf{describe} an original world;
    		\item and \textbf{identify affordances} suitable for the context of a game.
	\end{enumerate}
    \item \textbf{Implement}, as a \textbf{group}, a pre-production prototype that will:
    	\begin{enumerate}[label=\roman*.]
    		\item \textbf{illustrate} the core gameplay and unique selling points;
    		\item and \textbf{facilitate} play testing.
	\end{enumerate}
    \item \textbf{Prepare}, as an \textbf{individual}, a video of the pre-production prototype that:
    	\begin{enumerate}[label=\roman*.]
    		\item \textbf{highlights} the key features of the game;
    		\item and \textbf{shows} issues \textbf{you} identified during play-testing.
	\end{enumerate}
    \item \textbf{Demonstrate}, as a \textbf{group}, the pre-production prototype:
    	\begin{enumerate}[label=\roman*.]
    		\item \textbf{demonstrate} your academic integrity;
    		\item as well as \textbf{showcase EACH} of your individual contributions.
	\end{enumerate}
\end{enumerate}

\subsection*{Assignment Setup}

This assignment is a \textbf{product development task}. 

You will need to devise an approach to communication, task management, and version control with your peers. Strive to be as adaptable as possible and heed the advise of your product owner.

You may use any languages, tools, and development environment available within the Games Academy.

\subsection*{Part A}

Part A consists of a \textbf{single summative submission}. This work is \textbf{individual} and will be assessed on a \textbf{criterion-referenced} basis. Please refer to the marking rubric at the end of the brief for details on the criteria.

To complete Part A, devise an original world. Then, prepare a slide-deck and handout that describe that world. You will either: (i) present these as a live 10-minute presentation to tutors; or (ii) record a 10-minute video. Attend the relevant sessions and then upload your materials to the LearningSpace.

You will receive \textbf{formal feedback} from your \textbf{tutor} 3-weeks after the deadline.

\subsection*{Part B}

Part B is a \textbf{multiple formative submissions}. This work is \textbf{collaborative} and will be assessed on a \textbf{threshold} basis. The following criteria are used to determine a pass or fail:

\begin{enumerate}[label=(\alph*)]
	\item Playable builds are made available in a timely manner;
	\item Enough work is available to conduct a meaningful review.
\end{enumerate}

To complete Part B, prepare draft versions of the pre-production prototype ahead of the \textbf{sprint review meetings} with your product owner. Ensure that the source code and related assets are made available prior to \textbf{EACH} scheduled meeting. Then, attend the scheduled meetings.

You will receive immediate \textbf{informal feedback} from your \textbf{product owner}.

\subsection*{Part C}

Part C is a \textbf{single summative submission}. This work has both \textbf{individual} and \textbf{collaborative} elements, and will be assessed on a \textbf{criterion-referenced} basis. Please refer to the marking rubric at the end of the brief for details on the criteria.

To complete Part C, revise the pre-production prototype \textbf{with your team} based on the feedback you have received through play-testing, crits, and your product owner. Then, \textbf{as an individual}, record a a 5-minute video of the gameplay. The video should showcase the core elements of the game and draw attention to the evolution of the design; particularly your own contributions and your own insights from play-testing. Fading through a few different versions of the build is a possibility. Upload the video to the LearningSpace.

You will receive \textbf{formal feedback} three weeks after the final deadline.

\subsection*{Part D}

Part D is a \textbf{single formative submission}. This work has both \textbf{individual} and \textbf{collaborative} elements, and will be assessed on a \textbf{threshold} basis. The following criteria are used to determine a pass or fail:

\begin{enumerate}[label=(\alph*)]
	\item A working and playable build is presented;
	\item Enough work is available to conduct a meaningful review;
	\item Individual contributions are both clearly articulated and explained.
\end{enumerate}

To complete Part D, prepare a live demonstration of the pre-production prototype. You are permitted to demo any version of the pre-production prototype, but it must be a playable and working version. Then, attend the scheduled demo-day.

You will receive immediate \textbf{informal feedback} from \textbf{tutors}.

\section*{Additional Guidance}

Avoid poor planning and time management. By now this will be a familiar phrase, but it is no less true.
In particular, avoid underestimating the effort required to implement even simple software; always consider scope.
From the pitch stage, you should consider very carefully what is feasible.

For the most part, your work will be marked as a group effort.
However we want to avoid the situation where students try to ``coast'' through the assignment
on their fellow group members' work,
and equally the situation where one member of the group takes the lion's share of the work
and prevents the others from contributing effectively.
Marks will be weighted by a multiplier for \textbf{individual contribution},
which aims to penalise both of these behaviours.
We assess this by several means, including but not limited to: sprint reviews; individual vivas; feedback from your peers;
attribution in the source code; and commit logs.
Any student who has contributed their \textit{fair share} of effort to the project will receive a fair \% for their effort,
so any student who is putting in the appropriate level of effort has no need to worry.
Note that effort is not the same as productivity.

The first step in planning your implementation should be to break your concept down into \textbf{user stories}. 
Your user stories should be \textbf{distinguishable} (i.e.\ there should be little overlap between them)
and \textbf{easily measured} (i.e.\ it should be easy to tell when each user story has been implemented).
They should also be \textbf{comprehensive}, i.e.\ the user stories should completely capture the
desired functionality of the game, with no gaps.
Imagine giving your user stories to a developer who has never seen a product of this type before.
Would they be able to implement the software correctly, or would they miss key features?

Your game will be assessed on \textbf{conceptual coherence} and \textbf{player engagement}. This loosely corresponds to the \textbf{functional coherence} criterion you have previously been familiar with:
how well the finished product corresponds to the user stories,
and whether it has any obvious bugs.
Correspondence to user stories runs both ways:
implementing features that were not present in the design (``feature creep'')
is just as bad as neglecting to implement features.
However, the strength of the concept and its ability to engage players are integrated into the definition. It mustn't just work---it should also be fun!

Your source code will also be assessed as part of your \textbf{specialist contribution}. This is a single criterion consisting of \textbf{both sophistication and maintainability}.
Your code will also be assessed on \textbf{sophistication}.
To succeed on a project of this size and complexity,
you will need to make use of appropriate algorithms, data structures, libraries, and object oriented programming concepts.
Appropriateness to the task at hand is key:
you will \textbf{not} receive credit for complexity  
where something simpler would have sufficed.
\textbf{Maintainability} is important in all programming projects,
but doubly so when working in a team.
Use \textbf{comments} liberally to improve code comprehension,
and carefully choose the \textbf{names} for your files, classes, functions and variables.
Use a well-established commenting convention
for \textbf{high-level documentation}.
The open-source tool Doxygen supports several such conventions.
Also ensure that all code corresponds to a sensible and consistent \textbf{formatting style}:
indentation, whitespace, placement of curly braces, etc.
Hard-coded \textbf{literals} (numbers and strings) within the source should be avoided,
with values instead defined as constants together in a single place.
Consider allowing some literal values, where appropriate, to be ``tinkered'' without changing the source code,
e.g.\ by defining them in an external file read at startup.

\section*{FAQ}

\begin{itemize}
	\item 	\textbf{What is the deadline for this assignment?} \\ 
    		Falmouth University policy states that deadlines must only be specified on the MyFalmouth system.
    		
	\item 	\textbf{What should I do to seek help?} \\ 
    		You can email your tutor for informal clarifications. For informal feedback, make a pull request on GitHub. 
    		
    	\item 	\textbf{Is this a mistake?} \\ 	
    		If you have discovered an issue with the brief itself, the source files are available at: \\
    		\url{https://github.com/Falmouth-Games-Academy/bsc-assignment-briefs}.\\
    		 Please make a pull request and comment accordingly.
\end{itemize}

\section*{Additional Resources}

\begin{itemize}
    \item Keith, C. (2010) Agile Game Development with Scrum. Pearson Education.
    \item \url{http://www.gamasutra.com/blogs/RogerPaffrath/20131115/204871/What_NOT_to_do_when_starting_as_an_indie_game_developer.php}
    \item \url{https://www.youtube.com/watch?v=on7endO4lPY}
\end{itemize}

\rubricyeartwo

\rubrictitle{Marking Rubric (Pre-Production Prototype)}
\rubrichead{Criteria marked with a $\ddagger$ are shared by the group. Criteria marked with a $\dagger$ are weighted by individual contribution to a shared deliverable. All other criteria are individual.}
\begin{markingrubric}
    \firstcriterion{Engagement}{40\%  $\dagger$}
        \gradespan{1}{\fail Inadequate participation in PO meetings or sprint reviews and/or breach of academic integrity and/or no working demo.}
        \gradespan{1}{The student (excepting mitigating circumstances) engaged with the project with minimal sufficiency. \par A meaningfully playable build is available.}
        \gradespan{2}{The student (excepting mitigating circumstances) engaged with the project well. \par A meaningfully playable build is available.}
        \gradespan{2}{The student (excepting mitigating circumstances) engaged with the project very well. \par A meaningfully playable build is available.}
%
    \criterion{Conceptual Coherence}{15\% $\ddagger$}
        \grade \fail Few concepts/mechanics evident in build.
        \grade Some conceptual cohesion.
        \grade Prototype (as implemented) corresponds to a somewhat clear core concept.
        \grade Prototype (as implemented) corresponds to a clear core concept.
        \grade Prototype (as implemented) has considerable conceptual cohesion.
        \grade Prototype (as implemented) has significant conceptual cohesion.
%
    \criterion{Player Engagement}{15\% $\ddagger$}
        \grade \fail Few stories have been implemented and/or the build fails to run.
            \par Unplayable and/or serious bugs.
        \grade Some ability to engage, perhaps with several serious bugs.
        \grade Much ability to engage, perhaps with few serious bugs.
        \grade Considerable ability to engage.
        \grade Significant ability to engage.
        \grade Extensive ability to engage.
%
    \criterion{Value Added}{30\%}
        \grade \fail Unable to evidence value-added, or extremely poor prioritisation.
        \grade Some value added.
        \grade Much value added.
        \grade Considerable value added.
            \par A little evidence of play-testing and data analysis to improve prototype.
        \grade Significant value added.
            \par Some evidence of play-testing and data analysis to improve prototype.
        \grade Extensive value added.
            \par Much evidence of play-testing and data analysis to improve prototype. 
%
\end{markingrubric}

\rubrictitle{Marking Rubric (World Pitch)}
\rubrichead{Criteria marked with a $\ddagger$ are shared by the group. All other criteria are individual.}
\begin{markingrubric}
    \firstcriterion{Basic Competency Threshold}{40\%}
        \gradespan{1}{\fail No individual and/or group pitch is delivered, or either pitch is inappropriate.}
        \gradespan{5}{A broadly appropriate `world pitch' and `game concept pitch' are both delivered.}
%
    \criterion{Communication}{20\%}
        \grade \fail The individual pitch was delivered with little enthusiasm, or communication was inappropriate in both pitches. 
        \gradespan{5}{Has the student communicated their World concept clearly? \par Are the game's unique selling points clearly identified?}
%
    \criterion{Supporting Materials}{10\%}
        \grade\fail There were no supporting materials in either pitch.
        \gradespan{5}{Did the student use suitable supporting materials? \par Does the pitch make good use of supporting materials to help communicate the core features of the game? }
%
    \criterion{Planning}{10\%}
        \grade \fail Either pitch was poorly timed.
        \gradespan{5}{Was the pitch content well planned, and also structured to the available time-frame? \par Is the pitch structured to suit the available time-frame?}
%
    \criterion{Professionalism}{10\%}
        \grade\fail Little to no innovation and/or creativity.
            \par Either pitch was unprofessional.
        \gradespan{5}{Was the pitch delivered in a professional way? \par Does the pitch demonstrate a commercial awareness of the market the game will be competing in?}
%
    \criterion{Commercial Feasibility (Game Pitch Only)}{10\%}
        \grade\fail No budget or production plan are proposed, or they are inadequate.
        \gradespan{5}{How confident is the marker that the game concept is viable?}
\end{markingrubric}

\end{document}
