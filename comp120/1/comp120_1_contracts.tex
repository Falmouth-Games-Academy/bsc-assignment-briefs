\documentclass{../../fal_assignment}
\graphicspath{ {../../} }

\usepackage{enumitem}
\setlist{nosep} % Make enumerate / itemize lists more closely spaced
\usepackage[T1]{fontenc} % http://tex.stackexchange.com/a/17858
\usepackage{url}
\usepackage{todonotes}

\title{Tinkering Graphics}
\author{Brian McDonald}
\module{COMP120}

\begin{document}

\section*{2017-18 COMP120 Contracts}

\subsection*{Contract \#1 - Map Generation}
A small Indie Developer is looking for a tool that will allow them to generate fictional maps for their game. These maps will not appear in game but will be used by the team to aid in the development of the Game World. You will provided with a tile set and it is imagined that all of these tiles could be used to generate the map. They would also like names to be generated for features on the map, such as rivers, cities and seas.

\subsection*{Contract \#2 - Unit Reskinning}
As part of a larger project you have been tasked with creating a tool which reskins a number of units to work with a competitive RTS. You will have to remove a colour and then add a new colour for each team. There are four teams which use the following colours; red, green, blue and yellow. Each unit of these teams will have to be saved in a new png file.

\subsection*{Contract \#3 - Unit Generation}
You are creating a side scrolling shooter and you want the ability to create many different types of enemies using various components. These units should be saved to a new png file for use in the game.

\subsection*{Contract \#4 - Colour Blindness Tool}
1 in 12 men and 1 in 200 Women in the World suffer some form of colour blindness. Your studio lead has asked you to create a tool which takes in a screenshot of your game and then displays the image as if viewed by someone with a form of colour blindess.They would like you to save an image to file for every type of colour blindess so that the UI designer can adjust the colours of the in game UI.

\end{document}