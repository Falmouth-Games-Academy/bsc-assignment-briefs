\documentclass{../../fal_assignment}
\graphicspath{ {../../} }

\title{Tinkering Graphics}
\author{Dr Michael Scott}
\module{COMP120}
\version{3.0}

\begin{document}

\maketitle

\begin{marginquote}
``For every complex problem there is an answer that is clear, simple, and wrong.''

--- Henry Mencken

\marginquoterule

``Bad programming is easy. [People] can learn it in 21 days, even if they are dummies… [Good programming requires a] willingness to devote a large portion of one's life to deliberative practice… So go ahead, buy that book; you'll probably get some use out of it. But you won't change your life or your real expertise as a programmer in 21 days… How about working hard to continually improve over 24 months? Well, now you're starting to get somewhere…''
\par --- Peter Norvig

\end{marginquote}
\marginpicture{flavour_pic}{
    @pixelsorter is a Twitter bot written in Ruby that sorts the rows (or optionally columns) of an image according to a specific method like hue, red, brightness, luma, etc.
}

\section*{Introduction}

\paragraph{In this assignment, you are required to write a computer program that will \textit{tinker} with an existing computer graphic in a creative way.}

\paragraph{Creative computing encompasses the broad realms of digital media, computer programming, and human-computer interaction. It is important to draw these areas together in an applied way. You will, therefore, leverage the principles you have learned to exercise your creativity through computer software. This will prepare you to tackle challenges in many creative domains.}

\paragraph{This assignment is formed of several parts:}

\begin{enumerate}
    \item \textbf{Select}, as a \textbf{pair}, \textbf{one} of the contracts provided by your tutor and:
    	\begin{enumerate}
    		\item \textbf{state} which contract you will work on;
    		\item \textbf{list} the requirements implied by the contract;
    		\item \textbf{declare} the terms of your software license;
	\end{enumerate}
    \item \textbf{Write}, as a \textbf{pair}, a draft computer program in Python that will:
    	\begin{enumerate}
    		\item \textbf{address} the requirements implied by the contract and relevant intellectual property law;
    		\item \textbf{implement} \textbf{three} algorithms for tinkering graphics;
	\end{enumerate}
    \item \textbf{Write}, as an \textbf{individual}, a final computer program in Python that will:
    	\begin{enumerate}
    		\item \textbf{revise} any issues raised by your tutor and/or your peers.
	\end{enumerate}
    \item \textbf{Present}, as an \textbf{individual}, a practical demo of the computer program to your tutor that will:
    	\begin{enumerate}
    		\item \textbf{demonstrate} your academic integrity;
    		\item as well as \textbf{demonstrate} your \textbf{individual} programming knowledge.
	\end{enumerate}
\end{enumerate}

\subsection*{Assignment Setup}

\paragraph{This assignment is a \textbf{pair programming task}. Fork the GitHub repository at:}

\url{https://github.com/Falmouth-Games-Academy/comp120-tinkering-graphics}

\paragraph{Use the existing directory structure and, as required, extend this structure with sub-directories. Ensure that you maintain the \texttt{readme.md} file.}

\paragraph{Modify the \texttt{.gitignore} to the defaults for \textbf{Python}. Please, also ensure that you add editor-specific files and folders to \texttt{.gitignore}.}

\subsection*{Part A}

\paragraph{Part A consists of a \textbf{single formative submission}. This work is \textbf{collaborative} and will be assessed on a \textbf{threshold} basis.}

\paragraph{To complete Part A, write about your contract in the \texttt{readme.md} document.  Show this to your tutor in-class.  If acceptable, this will be signed-off.} 

\paragraph{You will receive immediate \textbf{informal feedback} from your \textbf{tutor}.}

\subsection*{Part B}

\paragraph{Part B is a \textbf{single formative submission}. This work is \textbf{collaborative} and will be assessed on a \textbf{threshold} basis. The following criteria are used to determine a pass or fail:}

\begin{itemize}
	\item Submission is timely;
	\item A valid software license is declared;
	\item Enough work is available to conduct a meaningful review;
	\item A broadly appropriate review of a peer's work is submitted.
\end{itemize}

\paragraph{To complete Part B, prepare draft versions of the computer programs. Ensure that the source code and related assets are pushed to GitHub and a pull request is made prior to the scheduled peer-review session. Then, attend the scheduled peer-review session.}

\paragraph{You will receive immediate \textbf{informal feedback} from your \textbf{peers}.}

\subsection*{Part C}

\paragraph{Part C is a \textbf{single summative submission}. This work is \textbf{collaborative} and will be assessed on a \textbf{criterion-referenced} basis. Please refer to the marking rubric at the end of this document for further detail.}

\paragraph{To complete Part C, revise the computer program based on the feedback you have received. Then, upload it to the LearningSpace. Please note, the LearningSpace will only accept a single \texttt{.zip} file.}

\paragraph{You will receive \textbf{formal feedback} from your \textbf{tutor} three weeks after the final submission deadline.}

\subsection*{Part D}

\paragraph{Part D is a \textbf{single summative submission}. This work is \textbf{individual} and will be assessed on a \textbf{threshold} basis.  The following criteria are used to determine a pass or fail:}

\begin{itemize}
	\item Enough work is available to hold a meaningful discussion;
	\item Clear evidence of programming knowledge;
	\item No breaches of academic integrity.
\end{itemize}

\paragraph{To complete Part D, prepare a practical demonstration of the computer programs. Ensure that the source code and related assets are pushed to GitHub and a pull request is made prior to the scheduled viva session. Then, attend the scheduled viva session.}

\paragraph{You will receive immediate \textbf{informal feedback} from your \textbf{tutor}.}

\section*{Additional Guidance}

\paragraph{It is critically important that you do not neglect your individual roles in the development process. Programming in pairs means that you work together on the same computer---switching between driver and navigator. It is a great opportunity to develop your technical communication skills and overcome common misconceptions about programming. It should not, however, be treated as a 'free ride'---you will get to review each others' progress.}

\paragraph{You are being expected to \textit{transform} and \textit{repurpose} encodings (i.e. manipulating existing pictures). However, you may create your own images if desired. When using images you have not authored yourself, the source should be noted in the GitHub README.md file and all relevant rights (e.g., copyright) acknowledged. }

\newpage

\paragraph{You can and should go beyond the techniques introduced in the lectures and the Guzdial book (e.g. researching algorithms for producing or manipulating graphics).}

\paragraph{You are not being assessed on speed or memory performance. Do not worry too much about framerate, etc.}

\paragraph{A common pitfall is poor planning or time management. Often, students underestimate how much work is involved in first learning programming concepts and then actually applying them. Programming is quite unlike other subjects in that it cannot be crammed into a last minute deluge just before a deadline. It is, therefore, very important that you begin work early and sustain a consistent pace: little and often.}

\paragraph{The first deadline is quite close to the start of the course and not much material will have been covered by this point. Please rest assured. This first formative submission is supposed to be a simple analysis of requirements. We expect there to be errors. However, it is very important to make a start on this project so you receive early feedback to give you some direction and to encourage you to practice your programming skills across the entire duration of the course. Ideally, you should be programming every day!}

\paragraph{The peer-review component of this work does sometimes raise alarm. However, the only way to learn how to review code is by reviewing code. Your tutor will guide you through the process and provide advice. With practice, it will become clear what is satisfactory by discussing the quality of work with your peers and your tutor during the peer review sessions.} 

\section*{FAQ}

\begin{itemize}
	\item 	\textbf{What is the deadline for this assignment?} \\ 
    		Falmouth University policy states that deadlines must only be specified on the MyFalmouth system.
    		
	\item 	\textbf{What should I do to seek help?} \\ 
    		You can email your tutor for informal clarifications. For informal feedback, make a pull request on GitHub. 
    		
    	\item 	\textbf{Is this a mistake?} \\ 	
    		If you have discovered an issue with the brief itself, the source files are available at:
    		\url{https://github.com/Falmouth-Games-Academy/bsc-assignment-briefs}
    		 Please raise an issue and comment accordingly.
\end{itemize}

\section*{Additional Resources}

\begin{itemize}
    \item Guzdial, M.J . and Ericson, B. (2015) Introduction to Computing and Programming in Python: A Multimedia Approach, 4th Edition. Pearson: New York.
    \item Martin, R.C. (2008) Clean Code: A Handbook of Agile Software Craftsmanship. Prentice Hall: New York
    \item http://guide.agilealliance.org/guide/pairing.html
    \item http://www.pairprogramming.co.uk/
    \item http://www.pythontutor.com/
\end{itemize}

\rubrichead{All submissions and assessment criteria for this assignment are individual.}
\begin{markingrubric}
%
    \firstcriterion{Basic Competency Threshold}{30\%}
        \grade \fail At least one part is missing or is inadequate.
                   \par Breach of academic integrity.
        \grade Adequate ability to generate ideas, problem solving, concepts, technical competency and proposals in response to set briefs and/or self-initiated activity. \par{}
        \grade  The work demonstrates an adequate, ethically informed, real-world experience of industry/business environments and markets.
        \grade Enough work is available to hold a meaningful discussion.
        \grade  Adequate participation in-class peer-review activities at least at the level of basic competency.
        \grade Clear evidence of programming knowledge.
%
    \criterion{PROCESS: \\Functional Coherence of Code}{5\%}
        \grade\fail 	No algorithm has been implemented successfully.
            \par 		The source code does not compile or there are serious syntax errors.
        \grade 		At least one algorithm has been  implemented successfully.
            \par 		There are many obvious logical errors, more than one of which is significant.   
        \grade 		At least two algorithms have been  implemented successfully.
            \par 		There are several obvious logical errors, no more than one of which is significant. 
        \grade 		At least three algorithms have been implemented successfully.
            \par 		There are some obvious logical errors, which are not significant. 
            \par		The brief has been satisfied.
        \grade 		At least three algorithms have been  implemented successfully.
            \par 		There are few obvious logical errors, which are cosmetic and/or superficial.
            \par		The brief has been satisfied.     
        \grade 		At least three algorithms have been  implemented successfully.
            \par		There are no obvious logical errors.
            \par		The brief has been satisfied.
%
    \criterion{PROCESS: \\Sophistication of Code}{15\%}
        \grade\fail No insight into the appropriate use of programming constructs is evident from the source code.
            \par No attempt to structure the program (e.g. one monolithic function).
        \grade Little insight into the appropriate use of programming constructs is evident from the source code.
            \par The program structure is poor.
        \grade Some insight into the appropriate use of programming constructs is evident from the source code.
            \par The program structure is adequate.
        \grade Much insight into the appropriate use of programming constructs is evident from the source code.
            \par The program structure is appropriate.
        \grade Considerable insight into the appropriate use of programming constructs is evident from the source code.
            \par The program structure is effective. There is high cohesion and low coupling.
        \grade Significant insight into the appropriate use of programming constructs is evident from the source code.
            \par The program structure is very effective. There is high cohesion and low coupling.
%
    \criterion{PROCESS: \\Maintainability of Code}{15\%}
        \grade\fail There are no comments in the source code, or comments are misleading.
            \par Most variable names are unclear or inappropriate.
            \par Code formatting hinders readability.
        \grade The source code is only sporadically commented, or comments are unclear.
            \par Some identifier names are unclear or inappropriate.
            \par Code formatting is inconsistent or does not aid readability.
        \grade The source code is somewhat well commented.
            \par Some identifier names are descriptive and appropriate.
            \par An attempt has been made to adhere to thhe PEP-8 formatting style.
             \par There is little obvious duplication of code or of literal values.           
        \grade The source code is reasonably well commented.
            \par Most identifier names are descriptive and appropriate.
            \par Most code adheres to the PEP-8 formatting style.
             \par There is almost no obvious duplication of code or of literal values.   
        \grade The source code is reasonably well commented, with Python doc-strings.
            \par Almost all identifier names are descriptive and appropriate.
            \par Almost all code adheres to the PEP-8 formatting style.
             \par There is no obvious duplication of code or of literal values. Some literal values can be easily ``tinkered'' in the source code. 
        \grade The source code is very well commented, with Python doc-strings.
            \par All identifier names are descriptive and appropriate.
            \par All source code adheres to the PEP-8 formatting style.
             \par There is no obvious duplication of code or of literal values. Most literal values are, where appropriate, easily ``tinkered'' outside of the source code.  
%
    \criterion{INDUSTRY: \\Creative Response to Brief}{10\%}
        \grade\fail No creativity.
            \par The work is a clone of an existing work with mere cosmetic alterations.
        \grade Little creativity.
            \par The work is derivative of existing works, with only minor alterations.
        \grade Some creativity.
            \par The work is derivative of existing works, demonstrating little divergent and/or subversive thinking.
        \grade Much creativity.
            \par The work is somewhat novel, demonstrating some divergent and/or subversive thinking.
        \grade Considerable creativity.
            \par The work is novel, demonstrating significant divergent and/or subversive thinking.
        \grade Significant creativity.
            \par The work is highly original, with strong evidence of divergent and/or subversive thinking.
%
\end{markingrubric}

\begin{markingrubric}
    \firstcriterion{INDUSTRY: \\Ethically Informed}{10\%}
        \grade\fail There is an inappropriate license and/or misuse of intellectual property.
        \grade There is an minimally appropriate license.
            \par There is implicit recognition of intellectual property rights.
        \grade There is an appropriate license. 
            \par There is explicit recognition of intellectual property rights in the \texttt{readme.md}.
            \par Acknowledgements are clearly demarcated in the source code.
        \grade There is a suitable license.
            \par There is explicit description of intellectual property rights in the \texttt{readme.md}.
            \par Authorship is demarcated in the source code.
            \par Copyright notices are present.
        \grade There chosen license is suitable. 
            \par There is explicit explanation of intellectual property rights in the \texttt{readme.md}.
            \par Authorship is accurately declared in the header using appropriate standards (e.g. PEP257) and demarcated in the source code.
            \par Copyright notices are present and appropriate.
        \grade There chosen license is suitable.
            \par There is explicit justification of intellectual property rights in the \texttt{readme.md}.
            \par Authorship is accurately declared in the header using appropriate standards (e.g. PEP257) and demarcated in the source code.
            \par Copyright notices are accurately declared in the header using appropriate standards (e.g. module-level dunder names, PEP8). There may be reference to a transfer agreement.
%
    \criterion{INDUSTRY: \\Use of Version Control}{15\%}
        \grade\fail Version control (e.g. GitHub) has not been used.
        \grade Source code has been checked into version control (e.g. GitHub).
        \grade Source code has been checked into version control (e.g. GitHub) at least once per week.
            \par Sensible commit messages are present.
        \grade Source code  has been checked into version control (e.g. GitHub) several times per week.
            \par Commit messages are clear, concise and relevant.
            \par There is evidence of somewhat meaningful engagement with peers (e.g.\ code review).
            \par Comments to peers are somewhat constructive and provide some insight.
        \grade Source code has been checked into version control (e.g. GitHub) several times per week.
            \par Commit messages are clear, concise and relevant.
            \par There is evidence of meaningful engagement with peers (e.g.\ code review).
            \par Comments to peers are reasonably constructive and provide much insight.
        \grade Source code has been checked into version control (e.g. GitHub) many times per week.
            \par Commit messages are clear, concise and relevant.
            \par There is evidence of effective engagement with peers (e.g.\ code review).
            \par Comments to peers are reasonably constructive and provide considerable insight.
%
\end{markingrubric}

\end{document}