\RequirePackage{pdfmanagement-testphase}
\DeclareDocumentMetadata{
    pdfversion=1.7,
}
\documentclass{../../fal_assignment}
\graphicspath{ {../../} }

\title{Tinkering Graphics}
\author{Dr Michael Scott}
\module{COMP120}
\version{4.0}

\begin{document}

\maketitle

\begin{marginquote}
``For every complex problem there is an answer that is clear, simple, and wrong.''

--- Henry Mencken

\marginquoterule

``Bad programming is easy. [People] can learn it in 21 days, even if they are dummies… [Good programming requires a] willingness to devote a large portion of one's life to deliberative practice… So go ahead, buy that book; you'll probably get some use out of it. But you won't change your life or your real expertise as a programmer in 21 days… How about working hard to continually improve over 24 months? Well, now you're starting to get somewhere…''
\par --- Peter Norvig

\end{marginquote}
\marginpicture{flavour_pic}{
    @pixelsorter is a Twitter bot written in Ruby that sorts the rows (or optionally columns) of an image according to a specific method like hue, red, brightness, luma, etc.
}

\vspace{-3em}

\section*{Introduction}

\paragraph{Creative computing encompasses the broad realms of digital media, computer programming, and human-computer interaction. It is important to draw these areas together in an applied way to develop the foundational skills required of all computing professionals. You will, therefore, leverage the principle programming constructs of representation, operation, sequence, branching, iteration, and functional decomposition to exercise your creativity. This will prepare you to tackle challenges in many creative domains.}

\paragraph{In this assignment, you are required to complete a series of activities which culminate in a demonstration of a computer program that you have written. This computer program will address a ``contract'' which will require you to \textit{tinker} with a digital graphic in a creative way according to a given context, using the techniques that you have learned through the activities.}

\paragraph{This assignment is formed of several parts:}

\begin{enumerate}
\setlength{\itemindent}{8pt}
    \item \textbf{Complete} the quizzes:
    	\begin{enumerate}
    		\item correctly \textbf{answering} the multiple-choice questions;
    		\item and \textbf{giving} appropriate responses to the code writing exercises.
	\end{enumerate}
    \item \textbf{Select} \textbf{ONE} contract \textbf{and outline} your interpretation of it:
    	\begin{enumerate}
    		\item \textbf{setting up} an appropriate repository to host the project;
    		\item \textbf{documenting} which contract you will work on and \textbf{listing} the requirements implied by the contract in a README;
    		\item as well as \textbf{declaring and justifying} the terms of your software license in a LICENSE;
	\end{enumerate}
    \item \textbf{Write} a draft computer program in \texttt{C\#} that will:
    	\begin{enumerate}
    		\item \textbf{implement} \textbf{FIVE} of the provided algorithms;
     		\item use them to \textbf{address} the requirements implied by the contract and relevant intellectual property law;
    		\item consider future maintenance by \textbf{conforming} with good practice;
    		\item \textbf{version} correctly in the \texttt{git} source control management system.
	\end{enumerate}
    \item \textbf{Engage} in a peer-review exercise:
    	\begin{enumerate}
    		\item \textbf{submit} your draft computer program for review;
    		\item \textbf{review} any issues raised by your tutor and/or your peers.
	\end{enumerate}
    \item \textbf{Update} the computer program:
    	\begin{enumerate}
    		\item \textbf{revise} any issues raised by your tutor and/or your peers.
	\end{enumerate}
    \item \textbf{Demonstrate} your computer program:
    	\begin{enumerate}
    		\item \textbf{showing} your academic integrity;
    		\item \textbf{justifying} decisions made during the project;
    		\item as well as \textbf{evidencing} your \textit{individual} programming knowledge.
	\end{enumerate}
\end{enumerate}

\subsection*{Assignment Setup}

\paragraph{This assignment is a \textbf{programming task}. Create a repository using the template below. Extend the repository with sub-directories and modify the \textt{.gitignore} as appropriate. Create the \texttt{README.md} file. 

\url{https://github.falmouth.ac.uk/Games-Academy/comp120-vscode-template}}

\paragraph{Please remember to create in repository in the correct organisational area on our GitHub Enterprise Server as directed by the technicians and to use the appropriate naming convention which includes the module code and your student identification number.}

\paragraph{Once the repo has been created, please connect the repository to your assignment by pasting the \texttt{.git} link here:}

\url{https://learningspace.falmouth.ac.uk/mod/assign/view.php?id=157229}

\paragraph{You \textbf{MUST NOT} add any other text to the submission field.}

\subsection*{Part A}

\paragraph{Part A consists of \textbf{multiple} activities in the virtual learning environment. This work is \textbf{individual}.

Access the quizzes in COMP120 module area on LearningSpace and complete them. These quizzes are released week by week, beginning in Week 1. }

\paragraph{Your work will be assessed automatically in a near-immediate fashion.}

\subsection*{Part B}

\paragraph{Part B consists of a \textbf{multiple} activities in the virtual learning environment. This work is \textbf{individual} and will be assessed on a \textbf{threshold} basis.}

Firstly, review the contracts in the supplementary materials, and then select the contract you wish to address from the selector activity in the virtual learning environment in Week 2:

\url{https://learningspace.falmouth.ac.uk/mod/choice/view.php?id=157329}

\paragraph{Secondly, research software licensing and decide which software is most appropriate for your project. Make your selection clear and justify your choice in the forum activity in the virtual learning activity, also in Week 2:}

\url{https://learningspace.falmouth.ac.uk/mod/forum/view.php?id=157330}

Then, add the relevant \texttt{LICENSE.md} file to your repository. Show this to your tutor in the next timetabled session. 

\paragraph{You will receive immediate \textbf{informal feedback} from your \textbf{tutor}.}

\subsection*{Part C}

\paragraph{Part C consists of a \textbf{multiple formative submissions}. This work is \textbf{individual}.}

\paragraph{To complete Part C, firstly review the pseudocode available on the virtual learning environment:}

\url{https://learningspace.falmouth.ac.uk/mod/folder/view.php?id=157225}

\paragraph{Then, consider how you might use several of these algorithms into your solution. Implement the relevant algorithms and incorporate them into the code that addresses the contract you have selected. Preparing drafts of your code for your tutor to review.}

\paragraph{Show your work-in-progress to your tutor in your timetabled sessions.}

\paragraph{You will receive immediate \textbf{informal feedback} from your \textbf{tutor}.}

\subsection*{Part D}

\paragraph{Part D is a \textbf{single formative submission}. This work is \textbf{individual} and will be assessed on a \textbf{threshold} basis. The following criteria are used to determine a pass or fail:}

\begin{itemize}
	\item Submission is timely;
	\item A valid software license is declared;
	\item Enough work is available to conduct a meaningful review;
	\item A broadly appropriate review of a peer's work is submitted.
\end{itemize}

\paragraph{To complete Part D, prepare a draft version of the computer program. Ensure that the source code and related assets are pushed to version control and a link is submitted to the peer-review workshop activity on LearningSpace prior to the scheduled peer-review session. You may have to set permissions on your repository to allow other students to review your code. Then, attend the scheduled peer-review session.}

\paragraph{You are expected to translate technical notation into executable code. As such, it is anticipated that you incorporate several of the listed algorithms into your solution. Clearly list these algorithms and state where they are implemented in the \texttt{README.md} file. You are welcome to use third-party libraries and resources in your solution, but these will not `count' as translations for assessment purposes.}

\paragraph{You will receive immediate \textbf{informal feedback} from your \textbf{peers}.}

\subsection*{Part E}

\paragraph{Part E is a \textbf{single summative submission}. This work is \textbf{individual} and will be assessed on a \textbf{criterion-referenced} basis. Please refer to the marking rubric at the end of this document for further detail.}

\paragraph{To complete Part E, revise the computer program based on the feedback you have received. Then, push it into version control. Formal submissions are through the LearningSpace, in this case a link to your repository on Falmouth's GitHub Enterprise server.}

\paragraph{You will receive \textbf{formal feedback} from your \textbf{tutor} three weeks after the final submission deadline.}

\subsection*{Part F}

\paragraph{Part F is a \textbf{single summative submission}. This work is \textbf{individual} and will be assessed on a \textbf{threshold} basis.  The following criteria are used to determine a pass or fail:}

\begin{itemize}
	\item Enough work is available to hold a meaningful discussion;
	\item Clear evidence of programming knowledge;
	\item No breaches of academic integrity.
\end{itemize}

\paragraph{To complete Part F, prepare a practical demonstration of your computer program. Ensure that the source code and related assets are pushed to version control and available to assessors prior to the scheduled viva session. Then, attend the scheduled viva session.}

\paragraph{There is an expectation that you work with other students in your programming tutor group as part of your learning. You have permission to do this. However, you must \textit{understand} the code in your submission and you must \textit{acknowledge} the contributions of peers to your codebase explicitly and in accordance with academic conventions. As such, you should be able to explain \textbf{EACH} section of your solution individually, and place appropriate references in the comments of your source files. In the viva you will discuss the key programming constructs used in your solution, the key design decisions made when structuring the code, and how functionality was achieved. This is your opportunity to demonstrate your academic integrity and individual programming knowledge.}

\paragraph{You will receive immediate \textbf{informal feedback} from your \textbf{tutor}.}

\section*{Additional Guidance}

\paragraph{It is critically important that you do not neglect your individual roles in the development process. Programming in pairs is a good idea if you want to do that, but this means that you work together on the same computer---switching between driver and navigator. It is not an excuse to just share code with each other (that would be collusion). It is a great opportunity to develop your technical communication skills and overcome common misconceptions about programming. It should not, however, be treated as a 'free ride'---you will get to review each others' progress.}

\paragraph{You are being expected to \textit{transform} and \textit{repurpose} encodings (i.e. manipulating existing pictures). However, you may create your own images if desired. When using images you have not authored yourself, the source should be noted in the README.md file and all relevant rights (e.g., copyright) acknowledged. }

\paragraph{You can and should go beyond the techniques introduced in the lectures and the Guzdial book (e.g. researching algorithms for producing or manipulating graphics).}

\paragraph{You are not being assessed on speed or memory performance. Do not worry too much about framerate, etc.}

\paragraph{A common pitfall is poor planning or time management. Often, students underestimate how much work is involved in first learning programming concepts and then actually applying them. Programming is quite unlike other subjects in that it cannot be crammed into a last minute deluge just before a deadline. It is, therefore, very important that you begin work early and sustain a consistent pace: little and often.}

\paragraph{The first deadline is quite close to the start of the course and not much material will have been covered by this point. Please rest assured. This first formative submission is supposed to be a simple analysis of requirements. We expect there to be errors. However, it is very important to make a start on this project so you receive early feedback to give you some direction and to encourage you to practice your programming skills across the entire duration of the course. Ideally, you should be programming every day!}

\paragraph{The peer-review component of this work does sometimes raise alarm. However, the only way to learn how to review code is by reviewing code. Your tutor will guide you through the process and provide advice. With practice, it will become clear what is satisfactory by discussing the quality of work with your peers and your tutor during the peer review sessions.} 

\section*{FAQ}

\begin{itemize}
	\item 	\textbf{What is the deadline for this assignment?} \\ 
    		Falmouth University policy states that deadlines must only be specified on the MyFalmouth system.
    		
	\item 	\textbf{What should I do to seek help?} \\ 
    		You can email your tutor for informal clarifications.
    		
    	\item 	\textbf{Is this a mistake?} \\ 	
    		If you have discovered an issue with the brief itself, the source files are available at:
    		\url{https://github.com/Falmouth-Games-Academy/bsc-assignment-briefs}
    		 Please raise an issue and comment accordingly.
\end{itemize}

\section*{Additional Resources}

Please review the LearningSpace for additional resources.

\rubrichead{All submissions and assessment criteria for this assignment are individual.

To \textbf{pass} this assignment (achieve 40\% or more), you must demonstrate adequate ability to generate ideas, problem solving, concepts, technical competency and proposals in response to the set brief. Your work must reflect an adequate, ethically informed, ``real-world'' experience as if you were in a industry/business environment or market. Enough of your work must be available to demonstrate your programming knowledge and allow a meaningful discussion to take place in the viva. You must also have satisfactory participation in all parts of the assignment and have submitted the final deliverable.
}

\begin{markingrubric}
%
    \firstcriterion{Basic Competency Threshold}{30\%}
        \grade \fail At least one part is missing, incomplete, or inadequate.
                   \par Breach of academic integrity.    
        \grade  All quizzes completed. \par Peer review complete. \par Comments to peers are satisfactory, but may be minimal and/or basic. %x
        \grade  As per quiz performance. \par Peer reviews are at least diagnostic and reassuring.  %
        \grade  As per quiz performance.  \par Peer reviews are instructive and encouraging, integrating appropriate insights. %
        \grade  As per quiz performance.  \par Peer reviews are well-formed and diplomatic, conveying effective advice. %
        \grade  As per quiz performance.  \par Peer reviews are attentive, apt, pithy, and constructive, conveying critical insight that aids improvement. %
%

    \criterion{PROCESS: \\Functional Coherence of Code}{5\%}
        \grade\fail 	There is a reasonable attempt at implementing the algorithms.
            \par 		The source code does not compile, or might return erroneous results, but would work with minor changes.
        \grade 		At least one algorithm has been implemented successfully.
            \par 		There are many obvious logical errors, more than one of which is significant.   
        \grade 		At least two algorithms have been  implemented successfully.
            \par 		There are several obvious logical errors, no more than one of which is significant. 
        \grade 		At least three algorithms have been implemented successfully.
            \par 		There are some obvious logical errors, which are not significant. 
            \par		The brief has been satisfied.
        \grade 		At least four algorithms have been implemented successfully.
            \par 		There are few obvious logical errors, which are cosmetic and/or superficial.
            \par		The brief has been satisfied.     
        \grade 		At least five algorithms have been implemented successfully.
            \par		There are no obvious logical errors.
            \par		The brief has been satisfied.
%
    \criterion{PROCESS: \\Sophistication of Code}{15\%}
        \grade\fail Insight into programming constructs is evident from the source code.
            \par There is an attempt to structure the program (e.g. not just one monolithic function) but it doesn't include key elements (e.g. functions with arguments).
        \grade Some insight into the appropriate use of programming constructs is evident from the source code.
            \par All key elements of program structure (e.g. functions with arguments) are present.
        \grade Much insight into the appropriate use of programming constructs is evident from the source code.
            \par Appropriate use of functions with arguments supports a program structure, minimising redundancy
        \grade Considerable insight into the appropriate use of programming constructs is evident from the source code.
            \par Appropriate use of a rich mixture of structural elements (e.g. classes) supports an extensible program structure.
        \grade Significant insight into the appropriate use of programming constructs is evident from the source code.
            \par The program structure is effective, having a reasonable level of cohesion and coupling.
        \grade Extensive insight into the appropriate use of programming constructs is evident from the source code.
            \par The program structure is very effective, demonstrating high cohesion and low coupling.
%
    \criterion{PROCESS: \\Maintainability of Code}{15\%}
        \grade\fail Insight into maintainability is evident from the source code, with at least one comment.
            \par There is an attempt to make use of sensible variable names.
            \par There is an attempt to format code.
        \grade The source code is commented sporadically.
            \par Many identifier names are clear and appropriate.
            \par Code formatting is mostly consistent, but might not aid readability as well as it could.
        \grade Source code is somewhat well commented.
            \par Most identifier names are descriptive and appropriate.
            \par An attempt has been made to adhere to an appropriate formatting style.
             \par There is little obvious duplication of code or of literal values.           
        \grade Source code is reasonably well commented.
            \par Almost all identifier names are descriptive and appropriate.
            \par Almost all code adheres to an appropriate formatting style.
             \par There is almost no obvious duplication of code or of literal values.   
        \grade Source code is reasonably well commented, with doc-comments.
            \par All identifier names are descriptive and appropriate.
            \par All code adheres to an appropriate formatting style.
             \par There is no obvious duplication of code or of literal values. Some literal values can be easily ``tinkered'' but might still be in the source code. 
        \grade Source code is very well commented, with doc-comments.
            \par All identifier names are descriptive and appropriate.
            \par All source code adheres to an appropriate formatting style.
             \par There is no obvious duplication of code or of literal values. Most literal values are, where appropriate, easily ``tinkered'' outside of the source code.  
%
\end{markingrubric}

\begin{markingrubric}
%
    \firstcriterion{INDUSTRY: \\Creative Response to Brief}{10\%}
        \grade\fail There is something creative, but it is very modest.
            \par The work is a clone of an existing work with mere cosmetic alterations.
        \grade Little creativity.
            \par The work is derivative of existing works, with only minor alterations.
        \grade Some creativity.
            \par The work is derivative of existing works, demonstrating little divergent and/or subversive thinking.
        \grade Much creativity.
            \par The work is somewhat novel, demonstrating some divergent and/or subversive thinking.
        \grade Considerable creativity.
            \par The work is novel, demonstrating significant divergent and/or subversive thinking.
        \grade Significant creativity.
            \par The work is novel, with strong evidence of divergent and/or subversive thinking.
%
    \criterion{INDUSTRY: \\Ethically Informed}{10\%}
        \grade\fail There is a marginally appropriate license and/or at least partial compliance with intellectual property law.
        \grade There is a somewhat appropriate license.
            \par There is implicit recognition of intellectual property rights.
        \grade There is an appropriate license. 
            \par There is explicit recognition of intellectual property rights in the \texttt{readme.md}.
            \par Acknowledgements are clearly demarcated in the source code.
        \grade There is a suitable license.
            \par There is explicit description of intellectual property rights in the \texttt{readme.md}.
            \par Authorship is demarcated in the source code.
            \par Copyright notices are present.
        \grade There chosen license is suitable. 
            \par There is explicit explanation of intellectual property rights in the \texttt{readme.md}.
            \par Authorship is accurately declared in the header using appropriate standards and demarcated in the source code.
            \par Copyright notices are present and appropriate.
        \grade There chosen license is suitable.
            \par There is explicit justification of intellectual property rights in the \texttt{readme.md}.
            \par Authorship is accurately declared in the header using appropriate standards and demarcated in the source code.
            \par Copyright notices are accurately declared in the header using appropriate standards. There may be reference to a transfer agreement.
%
    \criterion{INDUSTRY: \\Use of Version Control}{15\%}
        \grade\fail Version control  has been used only once or twice.
        \grade Source code has seldom been checked into version control .
        \grade Source code has been checked into version control  at least once per week.
            \par Sensible commit messages are present.
        \grade Source code  has been checked into version control  several times per week.
            \par Commit messages are clear, concise and relevant.
            \par There is evidence of somewhat meaningful engagement with peers (e.g. code review).
            \par Comments to peers are somewhat constructive and provide some insight.
        \grade Source code has been checked into version control  several times per week.
            \par Commit messages are clear, concise and relevant.
            \par There is evidence of meaningful engagement with peers (e.g. code review).
            \par Comments to peers are reasonably constructive and provide much insight.
        \grade Source code has been checked into version control  many times per week.
            \par Commit messages are clear, concise and relevant.
            \par There is evidence of effective engagement with peers (e.g. code review).
            \par Comments to peers are reasonably constructive and provide considerable insight.
%
\end{markingrubric}

\end{document}