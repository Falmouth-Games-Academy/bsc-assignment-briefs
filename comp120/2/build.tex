\documentclass{../../fal_assignment}
\graphicspath{ {../../} }

\usepackage{enumitem}
\setlist{nosep} % Make enumerate / itemize lists more closely spaced
\usepackage[T1]{fontenc} % http://tex.stackexchange.com/a/17858
\usepackage{url}
\usepackage{todonotes}

\title{Tinkering Audio}
\author{Dr Michael Scott}
\module{COMP120}
\version{2.0}

\begin{document}

\maketitle

\vspace{-1em}

\section*{Introduction}

\begin{marginquote}
``The programmer, like the poet, works only slightly removed from pure thought-stuff. He builds his castles in the air, from air, creating by exertion of the imagination. Few media of creation are so flexible, so easy to polish and rework, so readily capable of realizing grand conceptual structures''

--- Fred Brookes

\marginquoterule

``Students come into programming classes with a broad range of backgrounds---some have experience in several programming languages, others have never programmed before in their life! Being able to engage with the community and support each other is important.''

--- Robert Macredie 
\end{marginquote}
\marginpicture{flavour_pic}{
    Tidal is an algorave language. It allows you to code musical patterns live using text, describing sequences and ways of transforming and combining them, exploring complex interactions.
}

\paragraph{In this assignment, you are required to write a computer program that will \textit{tinker} with computer generated audio in a creative way.}

\paragraph{Creative computing encompasses the broad realms of digital media, computer programming, and human-computer interaction. It is important to draw these areas together in an applied way. You will, therefore, leverage the principles you have learned to exercise your creativity through computer software. This will prepare you to tackle challenges in many creative domains.}

\paragraph{This assignment is formed of several parts:}

\begin{enumerate}[label=(\Alph*)]
    \item \textbf{Select}, as a \textbf{pair}, \textbf{ONE} of the contracts provided by your tutor and:
    	\begin{enumerate}[label=\roman*.]
    		\item \textbf{state} which contract you will work on;
     		\item \textbf{declare} the terms of your software license;
	\end{enumerate}
    \item \textbf{Write}, as a \textbf{pair}, a draft computer program in C\# that will:
    	\begin{enumerate}[label=\roman*.]
    		\item \textbf{address} the requirements implied by the contract;
    		\item \textbf{comply} with relevant intellectual property law;
    		\item and \textbf{implement SEVEN} algorithms for tinkering audio.
	\end{enumerate}
    \item \textbf{Write}, as an \textbf{pair}, a final computer program in C\# that will:
    	\begin{enumerate}[label=\roman*.]
    		\item \textbf{satisfy ONE} contract;
    		\item and \textbf{revise} any issues raised by your tutor and/or your peers.
	\end{enumerate}
    \item \textbf{Present}, as an \textbf{individual}, a practical demo of the computer program to your tutor that will:
    	\begin{enumerate}[label=\roman*.]
    		\item \textbf{demonstrate} your academic integrity;
    		\item as well as \textbf{showcase} your \textbf{individual} programming knowledge.
	\end{enumerate}
\end{enumerate}

\subsection*{Assignment Setup}

\paragraph{This assignment is a \textbf{pair programming task}. Fork the repository at:}

\url{https://gamesgit.falmouth.ac.uk/scm/comp120/comp120-tinkering-audio.git}

\paragraph{Extend the repository with sub-directories as appropriate. Ensure that you maintain the \texttt{README.md} and \texttt{LICENSE.md} files. Modify the \texttt{.gitignore} to the defaults for \textbf{C\#} and your development environment of choice.}

\subsection*{Part A}

\paragraph{Part A consists of a \textbf{single formative submission}. This work is \textbf{collaborative} and will be assessed on a \textbf{threshold} basis.}

\paragraph{To complete Part A, write about your contract in the relevant LearningSpace activity. Show this to your tutor in your timetabled meeting. You are to select \textbf{ONE} contract to tackle together with your partner and to suggest an appropriate license for your code. If acceptable, this will be signed-off.}

\paragraph{You will receive immediate \textbf{informal feedback} from your \textbf{tutor}.}

\subsection*{Part B}

\paragraph{Part B is a \textbf{single formative submission}. This work is \textbf{collaborative} and will be assessed on a \textbf{threshold} basis. The following criteria are used to determine a pass or fail:}

\begin{itemize}
	\item Submission is timely;
	\item A valid software license is declared;
	\item Enough work is available to conduct a meaningful review;
	\item A broadly appropriate review of a peer's work is submitted.
\end{itemize}

\paragraph{To complete Part B, prepare a draft version of the computer program. Ensure that the source code and related assets are pushed to version control and a link is submitted to the peer-review workshop activity on LearningSpace prior to the scheduled peer-review session. You may have to set permissions on your repository to allow other students to review your code. Then, attend the scheduled peer-review session.}

\paragraph{You are expected to translate technical notation into executable code. Incorporate \textbf{SEVEN} of the listed algorithms into your solution. Clearly list these algorithms and state where they are implemented in the \texttt{README.md} file. You are welcome to use third-party libraries and resources in your solution, but these will not `count' as translations.}

\paragraph{You will receive immediate \textbf{informal feedback} from your \textbf{peers}.}

\subsection*{Part C}

\paragraph{Part C is a \textbf{single summative submission}. This work is \textbf{collaborative} and will be assessed on a \textbf{criterion-referenced} basis. Please refer to the marking rubric at the end of this document for further detail.}

\paragraph{To complete Part C, revise the computer program based on the feedback you have received. Then, push it into version control. Formal submissions are through the LearningSpace, in this case a link to your repository on GamesGIT. Only one of you needs to submit the link to the repository on LearningSpace.}

\paragraph{You will receive \textbf{formal feedback} from your \textbf{tutor} three weeks after the final submission deadline.}

\subsection*{Part D}

\paragraph{Part D is a \textbf{single summative submission}. This work is \textbf{individual} and will be assessed on a \textbf{threshold} basis.  The following criteria are used to determine a pass or fail:}

\begin{itemize}
	\item Enough work is available to hold a meaningful discussion;
	\item Clear evidence of programming knowledge;
	\item No breaches of academic integrity.
\end{itemize}

\paragraph{To complete Part D, prepare a practical demonstration of your computer program. Ensure that the source code and related assets are pushed to version control and available to assessors prior to the scheduled viva session. Then, attend the scheduled viva session.}

\paragraph{There is a clear expectation that you pair program with your partner. As such, you should \textbf{EACH} be able to explain the solution individually. In the viva you will discuss the key programming constructs used in your solution, the key design decisions made when structuring the code, and how functionality was achieved. This is your opportunity to demonstrate your academic integrity and individual programming knowledge. }

\paragraph{You will receive immediate \textbf{informal feedback} from your \textbf{tutor}.}

\section*{Additional Guidance}

\paragraph{It is critically important that you do not neglect your individual roles in the development process. Programming in pairs means that you work together on the same computer---switching between driver and navigator. It is a great opportunity to develop your technical communication skills and overcome common misconceptions about programming. It should not, however, be treated as a 'free ride'---you will get to review each others' progress.}

\paragraph{You are being expected to \textit{transform} and \textit{repurpose} encodings (i.e. manipulating existing pictures). However, you may create your own images if desired. When using images you have not authored yourself, the source should be noted in the README.md file and all relevant rights (e.g., copyright) acknowledged. }

\paragraph{You can and should go beyond the techniques introduced in the lectures and the Guzdial book (e.g. researching algorithms for producing or manipulating graphics).}

\paragraph{You are not being assessed on speed or memory performance. Do not worry too much about framerate, etc.}

\paragraph{A common pitfall is poor planning or time management. Often, students underestimate how much work is involved in first learning programming concepts and then actually applying them. Programming is quite unlike other subjects in that it cannot be crammed into a last minute deluge just before a deadline. It is, therefore, very important that you begin work early and sustain a consistent pace: little and often.}

\paragraph{It is very important to make a start on this project so you receive early feedback to give you some direction and to encourage you to practice your programming skills across the entire duration of the course. Ideally, you should be programming every day!}

\paragraph{The peer-review component of this work does sometimes raise alarm. However, the only way to learn how to review code is by reviewing code. Your tutor will guide you through the process and provide advice. With practice, it will become clear what is satisfactory by discussing the quality of work with your peers and your tutor during the peer review sessions.} 

\section*{FAQ}

\begin{itemize}
	\item 	\textbf{What is the deadline for this assignment?} \\ 
    		Falmouth University policy states that deadlines must only be specified on the MyFalmouth system.
    		
	\item 	\textbf{What should I do to seek help?} \\ 
    		You can email your tutor for informal clarifications. For informal feedback, make a pull request on GitHub. 
    		
    	\item 	\textbf{Is this a mistake?} \\ 	
    		If you have discovered an issue with the brief itself, the source files are available at: \\
    		\url{https://github.com/Falmouth-Games-Academy/bsc-assignment-briefs}.\\
    		 Please raise an issue and comment accordingly.
\end{itemize}

\section*{Additional Resources}

\paragraph{Please review the LearningSpace for additional resources.}

\rubrichead{All submissions and assessment criteria for this assignment are individual.

To \textbf{pass} this assignment (achieve 40\% or more), you must demonstrate adequate ability to generate ideas, problem solving, concepts, technical competency and proposals in response to the set brief. Your work must reflect an adequate, ethically informed, ``real-world'' experience as if you were in a industry/business environment or market. Enough of your work must be available to demonstrate your programming knowledge and allow a meaningful discussion to take place in the viva. You must also have satisfactory participation in all parts of the assignment and have submitted the final deliverable.
}
\begin{markingrubric}
%
    \firstcriterion{Basic Competency Threshold}{30\%}
        \grade \fail At least one part is missing, incomplete, or inadequate.
                   \par Breach of academic integrity.
                   
        \grade  %
        \grade  %
        \grade  %
        \grade  %
        \grade  %
%

    \criterion{PROCESS: \\Functional Coherence of Code}{5\%}
        \grade\fail 	There is a reasonable attempt at implementing the algorithms.
            \par 		The source code does not compile, or might return erroneous results, but would work with minor changes.
        \grade 		At least \textbf{one} algorithm has been implemented successfully.
            \par 		There are many obvious logical errors, more than one of which is significant.   
        \grade 		At least \textbf{three} algorithms have been  implemented successfully.
            \par 		There are several obvious logical errors, no more than one of which is significant. 
        \grade 		At least \textbf{five} algorithms have been implemented successfully.
            \par 		There are some obvious logical errors, which are not significant. 
            \par		The brief has been satisfied.
        \grade 		At least \textbf{six} algorithms have been implemented successfully.
            \par 		There are few obvious logical errors, which are cosmetic and/or superficial.
            \par		The brief has been satisfied.     
        \grade 		At least \textbf{seven} algorithms have been implemented successfully.
            \par		There are no obvious logical errors.
            \par		The brief has been satisfied.
%
    \criterion{PROCESS: \\Sophistication of Code}{15\%}
        \grade\fail Insight into programming constructs is evident from the source code.
            \par There is an attempt to structure the program (e.g. not just one monolithic function) but it doesn't include key elements (e.g. functions with arguments).
        \grade Some insight into the appropriate use of programming constructs is evident from the source code.
            \par All key elements of program structure (e.g. functions with arguments) are present.
        \grade Much insight into the appropriate use of programming constructs is evident from the source code.
            \par Appropriate use of functions with arguments supports a program structure, minimising redundancy
        \grade Considerable insight into the appropriate use of programming constructs is evident from the source code.
            \par Appropriate use of a rich mixture of structural elements (e.g. classes) supports an extensible program structure.
        \grade Significant insight into the appropriate use of programming constructs is evident from the source code.
            \par The program structure is effective, having a reasonable level of cohesion and coupling.
        \grade Extensive insight into the appropriate use of programming constructs is evident from the source code.
            \par The program structure is very effective, demonstrating high cohesion and low coupling.
%
    \criterion{PROCESS: \\Maintainability of Code}{15\%}
        \grade\fail Insight into maintainability is evident from the source code, with at least one comment.
            \par There is an attempt to make use of sensible variable names.
            \par There is an attempt to format code.
        \grade The source code is commented sporadically.
            \par Many identifier names are clear and appropriate.
            \par Code formatting is mostly consistent, but might not aid readability as well as it could.
        \grade Source code is somewhat well commented.
            \par Most identifier names are descriptive and appropriate.
            \par An attempt has been made to adhere to an appropriate formatting style.
             \par There is little obvious duplication of code or of literal values.           
        \grade Source code is reasonably well commented.
            \par Almost all identifier names are descriptive and appropriate.
            \par Almost all code adheres to an appropriate formatting style.
             \par There is almost no obvious duplication of code or of literal values.   
        \grade Source code is reasonably well commented, with doc-comments.
            \par All identifier names are descriptive and appropriate.
            \par All code adheres to an appropriate formatting style.
             \par There is no obvious duplication of code or of literal values. Some literal values can be easily ``tinkered'' but might still be in the source code. 
        \grade Source code is very well commented, with doc-comments.
            \par All identifier names are descriptive and appropriate.
            \par All source code adheres to an appropriate formatting style.
             \par There is no obvious duplication of code or of literal values. Most literal values are, where appropriate, easily ``tinkered'' outside of the source code.  
%
\end{markingrubric}

\begin{markingrubric}
%
    \firstcriterion{INDUSTRY: \\Creative Response to Brief}{10\%}
        \grade\fail There is something creative, but it is very modest.
            \par The work is a clone of an existing work with mere cosmetic alterations.
        \grade Little creativity.
            \par The work is derivative of existing works, with only minor alterations.
        \grade Some creativity.
            \par The work is derivative of existing works, demonstrating little divergent and/or subversive thinking.
        \grade Much creativity.
            \par The work is somewhat novel, demonstrating some divergent and/or subversive thinking.
        \grade Considerable creativity.
            \par The work is novel, demonstrating significant divergent and/or subversive thinking.
        \grade Significant creativity.
            \par The work is novel, with strong evidence of divergent and/or subversive thinking.
%
    \criterion{INDUSTRY: \\Ethically Informed}{10\%}
        \grade\fail There is a marginally appropriate license and/or at least partial compliance with intellectual property law.
        \grade There is a somewhat appropriate license.
            \par There is implicit recognition of intellectual property rights.
        \grade There is an appropriate license. 
            \par There is explicit recognition of intellectual property rights in the \texttt{readme.md}.
            \par Acknowledgements are clearly demarcated in the source code.
        \grade There is a suitable license.
            \par There is explicit description of intellectual property rights in the \texttt{readme.md}.
            \par Authorship is demarcated in the source code.
            \par Copyright notices are present.
        \grade There chosen license is suitable. 
            \par There is explicit explanation of intellectual property rights in the \texttt{readme.md}.
            \par Authorship is accurately declared in the header using appropriate standards and demarcated in the source code.
            \par Copyright notices are present and appropriate.
        \grade There chosen license is suitable.
            \par There is explicit justification of intellectual property rights in the \texttt{readme.md}.
            \par Authorship is accurately declared in the header using appropriate standards and demarcated in the source code.
            \par Copyright notices are accurately declared in the header using appropriate standards. There may be reference to a transfer agreement.
%
    \criterion{INDUSTRY: \\Use of Version Control}{15\%}
        \grade\fail Version control (i.e. GamesGit) has been used only once or twice.
        \grade Source code has seldom been checked into version control (i.e. GamesGit).
        \grade Source code has been checked into version control (i.e. GamesGit) at least once per week.
            \par Sensible commit messages are present.
        \grade Source code  has been checked into version control (i.e. GamesGit) several times per week.
            \par Commit messages are clear, concise and relevant.
            \par There is evidence of somewhat meaningful engagement with peers (e.g. code review).
            \par Comments to peers are somewhat constructive and provide some insight.
        \grade Source code has been checked into version control (i.e. GamesGit) several times per week.
            \par Commit messages are clear, concise and relevant.
            \par There is evidence of meaningful engagement with peers (e.g. code review).
            \par Comments to peers are reasonably constructive and provide much insight.
        \grade Source code has been checked into version control (i.e. GamesGitb) many times per week.
            \par Commit messages are clear, concise and relevant.
            \par There is evidence of effective engagement with peers (e.g. code review).
            \par Comments to peers are reasonably constructive and provide considerable insight.
%
\end{markingrubric}

\end{document}