\documentclass{../../../fal_assignment}
\graphicspath{ {../../../} }

\usepackage{enumitem}
\setlist{nosep} % Make enumerate / itemize lists more closely spaced
\usepackage[T1]{fontenc} % http://tex.stackexchange.com/a/17858
\usepackage{url}
\usepackage{todonotes}

\title{COMP220 Worksheet 3: Prototyping and planning}
\author{Brian McDonald, Kate Bergel}
\module{COMP220}

\begin{document}

\maketitle

\section*{Introduction}

In this worksheet you will build on the implementation from Worksheet 2 and complete a first iteration of your project.

To complete this worksheet, carry out the following steps:
\begin{enumerate}[label=(\alph*)]
	\item \textbf{Identify} the key resources for your project.
	\item \textbf{Implement} an initial prototype of your chosen effect(s).
	\item \textbf{Profile} the performance of the prototype to identify any major bottlenecks
	\item \textbf{Write} a plan for the next iteration, considering potential improvements to appearance/behaviour as well as performance.
\end{enumerate}

\section*{Submission instructions}

Continue using the repository from Worksheet 2; you should consider creating a new branch for this worksheet. Any documents, such as images or reports, should be included in the repository.

You should complete a pull request before the hand-in on \textbf{Monday by 4pm on Week 8}. Feedback will be given in the pull request and in class.

\section*{Marking criteria}

Remember that \textbf{it is better to submit incomplete work than to submit nothing at all}. If you do not manage to finish all assigned tasks, then you can complete them before the submission of Worksheet 4

To demonstrate \textbf{adequate proficiency}, complete the following:
\begin{itemize}
	\item 1 key web resources identified
	\item Basic implementation of your chosen effect(s)
\end{itemize} 

To demonstrate \textbf{competent proficiency}, complete the following:
\begin{itemize}
	\item Achieve \textbf{adequate proficiency}
	\item Additional web resources
	\item Outline plan for next iteration to identify areas for improvement in code and/or effect appearance
\end{itemize}

To demonstrate \textbf{very good proficiency}, complete the following:
\begin{itemize}
	\item Achieve \textbf{competent proficiency}
	\item Additional Conference or Journal source
	\item Basic profiling carried out and evidenced by screenshots or spreadsheets
\end{itemize}

To demonstrate \textbf{excellent proficiency}, complete the following:
\begin{itemize}
	\item Achieve \textbf{very good proficiency}
	\item Additional Conference or Journal sources
	\item Profiling results analysed to identify major bottlenecks
\end{itemize}

To demonstrate \textbf{oustanding proficiency}, complete the following:
\begin{itemize}
	\item Achieve \textbf{very good proficiency}
	\item Detailed plan for next iteration incorporating profiling results
\end{itemize}


\end{document}
