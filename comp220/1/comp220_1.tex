\documentclass{../../fal_assignment}
\graphicspath{ {../../} }

\usepackage{enumitem}
\setlist{nosep} % Make enumerate / itemize lists more closely spaced
\usepackage[T1]{fontenc} % http://tex.stackexchange.com/a/17858
\usepackage[hyphens]{url}
\usepackage{todonotes}

\title{Portfolio of game engine components --- Graphics}
\author{Dr Ed Powley \& Brian McDonald}
\module{COMP220}

\begin{document}

\maketitle

\section*{Introduction}

\begin{marginquote}
``Because of the nature of Moore's law, anything that an extremely clever graphics programmer can do at one point can be replicated by a merely competent programmer some number of years later.''

--- John Carmack

    \marginquoterule

``Currently computer graphics are used a great deal, but it can be excessive.''

--- Hayao Miyazaki
\end{marginquote}
\marginpicture{flavour_pic}{
    A demo of fluid simulation with NVIDIA's PhysX.
    Recent advances in GPU technology have enabled a wide range of high-fidelity real-time rendering and simulation effects.
}

In this assignment, you are required to \textbf{design} and \textbf{implement} a C++ program using SDL and OpenGL
which demonstrates the type of 3D computer graphics techniques that appear in a modern game engine.

Graphics technology is one of the most obvious areas in which innovation has driven gaming technology in recent years.
Modern gaming PCs and consoles contain powerful graphics processing units (GPUs),
and gamers expect modern games to push this hardware to its full potential.
In this assignment you will practice the use of advanced graphical effects.
Your final product will be a portfolio piece, which you can use in future to demonstrate your mastery of these techniques. This portfolio piece will be built up using a series of Worksheets, worksheet A and B will give you a foundational framework for
the rest of the assignment which will be completed in C and D.

This assignment is formed of several parts:
\begin{enumerate}[label=(\Alph*)]
	\item \textbf{Write} a 2-page handout that will:
		\begin{enumerate}[label=(\roman*)]
			\item \textbf{outline} the concept of your demo;
			\item \textbf{explain} how your demo satisfies the requirements of the contract (provided as an appendix at the end of this document);
			\item \textbf{describe} at least \textbf{two} graphical or simulation effects your demo will include;
		\end{enumerate}
	\item Worksheet A - \textbf{Project Framework}
	\item Worksheet B - \textbf{Camera \& Basic Scene}
	\item Worksheet C - \textbf{Demo specific work}
	\item Worksheet D - \textbf{Demo specific work \& Viva}
\end{enumerate}

\subsection*{Assignment Setup}

This assignment is a \textbf{programming} task. Fork the GitHub repository at the following URL:

\indent \url{https://github.com/Falmouth-Games-Academy/comp220-portfolio}

Use the existing directory structure and, as required, extend this structure with sub-directories.
Ensure that you maintain the \texttt{readme.md} file.

Modify the \texttt{.gitignore} to the defaults for \textbf{Visual Studio}.
Please, also ensure that you add editor-specific files and folders to \texttt{.gitignore}. 

\subsection*{Part A}

Part A consists of a \textbf{single formative submission}. This work is \textbf{individual} and will be assessed on a \textbf{threshold} basis. Answer the following questions to pass:

\begin{itemize}
	\item What is the title and high concept of the demo?
	\item What is the intended aesthetic?
	\item For each of two graphical or simulation effects:
	\begin{itemize}
		\item What is the effect?
		\item How will the effect contribute to the aesthetic?
	\end{itemize}
	\item Is the scope appropriate for the product development time-frame?
\end{itemize}

To complete Part A, prepare the handout using any word processing tool.
To help illustrate your intended aesthetic, your handout may include images and/or links to online videos.

Show the handout to your \textbf{tutor} for immediate \textbf{informal feedback} in \textbf{class on Week 3}

Once you have completed Part A, you should move onto the Worksheets. See the worksheets on the Learning Space for more details.  

\section*{Additional Guidance}

As always, avoid underestimating the effort required to implement even simple software; always consider scope.
From the proposal stage, you should consider very carefully what is feasible.

Your code will be assessed on \textbf{functional coherence}:
how well the finished product corresponds to the user stories,
and whether it has any obvious bugs.
Correspondence to user stories runs both ways:
implementing features that were not present in the design (``feature creep'')
is just as bad as neglecting to implement features.

Unlike your previous assignments,
you will be assessed on the \textbf{performance} of your solution.
Real-time graphics and simulation are not just about creating aesthetically pleasing effects,
but doing so whilst maintaining a smooth and consistent framerate free of any lag or glitches that might frustrate the player.
It may be necessary to trade-off the complexity or fidelity of an effect
in order to achieve acceptable performance.

Your code will also be assessed on \textbf{sophistication}.
To succeed on a project of this size and complexity,
you will need to make use of appropriate algorithms, data structures, libraries, and object oriented programming concepts.
Appropriateness to the task at hand is key:
you will \textbf{not} receive credit for complexity  
where something simpler would have sufficed.

\textbf{Maintainability} is important in all programming projects,
but doubly so when working in a team.
Use \textbf{comments} liberally to improve code comprehension,
and carefully choose the \textbf{names} for your files, classes, functions and variables.
Use a well-established commenting convention
for \textbf{high-level documentation}.
The open-source tool Doxygen supports several such conventions.
Also ensure that all code corresponds to a sensible and consistent \textbf{formatting style}:
indentation, whitespace, placement of curly braces, etc.
Hard-coded \textbf{literals} (numbers and strings) within the source should be avoided,
with values instead defined as constants together in a single place.
Consider allowing some literal values, where appropriate, to be ``tinkered'' without changing the source code,
e.g.\ by defining them in an external file read at startup.

As with all assignments on this course, you are expected to display a level of
\textbf{innovation and creative flair} befitting Falmouth University's reputation as a world-leading
arts institution.
One approach to promoting creativity is
\textbf{divergent thinking}: generating ideas by exploring many possible solutions.
Often the most interesting ideas are \textbf{subversive}: they deliberately go against
convention or obvious solutions.

You will \textbf{not} be judged on the quality of your art assets.
It is fine to use meshes and textures found online,
as long as they are available under an appropriate license and are properly attributed.

\section*{FAQ}

\begin{itemize}
	\item 	\textbf{What is the deadline for this assignment?} \\ 
    		Falmouth University policy states that deadlines must only be specified on the MyFalmouth system.
    		
	\item 	\textbf{What should I do to seek help?} \\ 
    		You can email your tutor for informal clarifications. For informal feedback, make a pull request on GitHub. 
    		
    	\item 	\textbf{Is this a mistake?} \\ 	
    		If you have discovered an issue with the brief itself, the source files are available at: \\
    		\url{https://github.com/Falmouth-Games-Academy/bsc-assignment-briefs}.\\
    		 Please make a pull request and comment accordingly.
\end{itemize}

\section*{Additional Resources}

\begin{itemize}
    \item \url{http://www.opengl-tutorial.org}
    \item \url{http://gamedev.stackexchange.com/questions/32876/good-resources-for-learning-modern-opengl-3-0-or-later}
    \item \url{https://google.github.io/styleguide/cppguide.html}
\end{itemize}

\rubricyeartwo
\begin{markingrubric}
	%
    \firstcriterion{Project Ouline}{10\%}
		\grade\fail No or Late submission of project outline
	\grade A poor standard of English and the outline is not clear
	\grade A good standard of English and the outline has some merit
	\grade A good standard of English and the outline is well justified
	\grade A good standard of English and the outline is very good
	\grade A good standard of English and the outline pushes the students skill
	%
	\criterion{Worksheet A}{10\%}
		\grade\fail No or Late submission of worksheet
	\gradespan{5}{See worksheet for details}
	%
	\criterion{Worksheet B}{15\%}
		\grade\fail No or Late submission of worksheet
	\gradespan{5}{See worksheet for details}
	%
	\criterion{Worksheet C}{20\%}
		\grade\fail No or Late submission of worksheet
	\gradespan{5}{See worksheet for details}
	%
	\criterion{Worksheet D}{25\%}
		\grade\fail No or Late submission of worksheet
	\gradespan{5}{See worksheet for details}
	%
    \criterion{Maintainability}{15\% $\ddagger$}
\grade\fail There are no comments in the source code, or comments are misleading.
\par Most variable names are unclear or inappropriate.
\par Code formatting hinders readability.
\grade The source code is only sporadically commented, or comments are unclear.
\par Some identifier names are unclear or inappropriate.
\par Code formatting is inconsistent or does not aid readability.
\grade The source code is somewhat well commented.
\par Some identifier names are descriptive and appropriate.
\par An attempt has been made to adhere to thhe PEP-8 formatting style.
\par There is little obvious duplication of code or of literal values.           
\grade The source code is reasonably well commented.
\par Most identifier names are descriptive and appropriate.
\par Most code adheres to the PEP-8 formatting style.
\par There is almost no obvious duplication of code or of literal values.   
\grade The source code is reasonably well commented, with Python doc-strings.
\par Almost all identifier names are descriptive and appropriate.
\par Almost all code adheres to the PEP-8 formatting style.
\par There is no obvious duplication of code or of literal values. Some literal values can be easily ``tinkered'' in the source code. 
\grade The source code is very well commented, with Python doc-strings.
\par All identifier names are descriptive and appropriate.
\par All source code adheres to the PEP-8 formatting style.
\par There is no obvious duplication of code or of literal values. Most literal values are, where appropriate, easily ``tinkered'' outside of the source code.  	
	%
	\criterion{Use of Version Control}{5\%}
		\grade\fail GitHub has not been used.
		\grade Source code has rarely been checked into GitHub.
		\grade Source code  has been checked into GitHub at least once per week.
			\par Commit messages are present.
			\par There is evidence of engagement with peers (e.g.\ code review).
		\grade Source code  has been checked into GitHub several times per week.
			\par Commit messages are clear, concise and relevant.
			\par There is evidence of somewhat meaningful engagement with peers (e.g.\ code review).
		\grade Source code has been checked into GitHub several times per week.
			\par Commit messages are clear, concise and relevant.
			\par There is evidence of meaningful engagement with peers (e.g.\ code review).
		\grade Source code has been checked into GitHub several times per week.
			\par Commit messages are clear, concise and relevant.
			\par There is evidence of effective engagement with peers (e.g.\ code review).
\end{markingrubric}

\section*{Appendix: Contract}

You must \textbf{design} and \textbf{implement} a technical demo making use of 3D graphics techniques.
The demo must meet the following requirements:

\begin{itemize}
	\item A scene containing \textbf{at least one} textured mesh \textbf{and at least one} light source.
	\item Standard \textbf{first-person movement controls}:
		\begin{itemize}
			\item The player can use the mouse to look up/down and to rotate in place;
			\item The player can use the WASD keys and/or the arrow keys to move around the environment;
			\item Other controls (e.g.\ interact, jump, shoot) should be added
				\textbf{if and only if} they are required for your concept.
			\item Loading of meshes from a standard 3D object file format;
		\end{itemize}
	\item At least \textbf{two} of the following graphics and simulation techniques:
		\begin{itemize}
			\item Procedural generation of complex meshes or terrain;
			\item Rendering of semi-transparent materials;
			\item Realistic rendering of rough surfaces (e.g.\ normal mapping);
			\item Skeletal animation or deformation;
			\item Collision detection or ray-casting;
			\item Integration of a third-party physics engine (e.g.\ Bullet, PhysX);
			\item Particle effects (e.g.\ fire, smoke, fluid);
			\item An advanced real-time lighting effect (e.g.\ shadow casting, reflections, volumetric lighting, bloom);
			\item Non-realistic rendering (e.g.\ cel shading);
			\item Other advanced rendering or simulation techniques of your choice,
				subject to discussion with your tutor
				and demonstration that you have researched the feasibility of your chosen techniques.
		\end{itemize}
	\item Some aspect intended to create \textbf{fun and/or engagement} for the user,
		such as:
		\begin{itemize}
			\item Exploring an environment in a ``walking simulator'' style; or
			\item Interacting with a simulated system; or
			\item Achieving a \textbf{simple} gameplay objective (beware of overscoping!)
		\end{itemize}
\end{itemize}

\end{document}