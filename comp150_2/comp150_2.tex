\documentclass{../fal_assignment}
\graphicspath{ {../} }

\usepackage{enumitem}
\usepackage[T1]{fontenc} % http://tex.stackexchange.com/a/17858
\usepackage{url}
\usepackage{todonotes}

\usepackage{listings}
\lstset{
    basicstyle=\ttfamily,
	frame=single,
	tabsize=4,
	showstringspaces=false,
	breaklines=true,
    language=C++
}

\title{Group Desktop Game}
\author{Dr Edward Powley}

\begin{document}

\maketitle
%\begin{marginquote}
%    ``Students come into programming classes with a broad range of backgrounds ---
%    some have experience in several programming languages, others have never programmed before in their life.
%    
%    Being able to engage with the community and support each other is important.
%    Upload your code to GitHub and receive feedback from experienced peers.
%    Review your peers' work yourself and really consider what `quality' actually means
%    and what `good' source code looks like.
%    Debate, argue, and question others about it ---
%    an open and sustained discourse is an excellent way to learn ---
%    for both beginners and adepts!''
%\end{marginquote}
%\marginpicture{MakeyMakey.jpg}{
    %The \emph{MaKey~MaKey} allows a multitude of materials to be used to create videogame controllers.
%}
\section*{Introduction}

In this assignment, you are required to design and implement a 2D game for Windows PC.
Your game will be written in C++ using the Simple DirectMedia Layer (SDL) library.
You will work in small groups (3--4 students) to design, plan and develop your game.

%Computing for Games combines technical and creative skills in equal parts.
%All of your assignments involve a mixture of the two;
%in this assignment the emphasis is more on creativity.
%You will build upon the technical skills you have learned so far,
%combined with your own creativity and innovation,
%to produce a unique creative artefact.

This assignment is formed of \textbf{four} parts: A, B, C, and D.
Final summative submission is via LearningSpace.

\subsection*{A. \emph{Pitch} your game design}

As an \textbf{individual} effort, prepare and deliver a pitch for a game concept.
You will be given creativity cards to aid in coming up with your concept.
The only hard restriction on your concept is that it must use \textbf{2D graphics},
i.e.\ it must be implementable in SDL without the use of OpenGL or other 3D graphics libraries.

Your pitch should address the following questions:
\begin{itemize}
	\item What is the title and high concept of the game?
	\item What is the intended aesthetic?
	\item What is the core mechanic? 
	\item What makes the game fun?
	\item Is there a market for this type of game? Who is the target audience?
	\item What are the unique selling points?
	\item Is the scope appropriate for the time-frame of this assignment?
\end{itemize}

Prepare a \textbf{10-minute} presentation.
Use appropriate aids to help communicate your design. This may or may not include g a ludic sketch 
and other visual aids. A formal presentation
(e.g.\ PowerPoint slides) is \textbf{neither required nor recommended}, but is permitted. 

\textbf{Formative submission:} Participate in the pitch session in class.
The quality of your pitch is \textbf{not formally assessed},
however you will receive formative feedback from your tutors and peers.
\textbf{Non-participation} will be penalised on a threshold basis.

In the session, a selection process will determine which concepts will be taken forwards to parts~B, C and~D.

\subsection*{B. \emph{Plan} the game}

As a \textbf{group} effort, break your game concept down into user stories.
Create a Trello task board and populate it with your user stories.

\textbf{Formative submission:} Discuss your task board with your tutor.

\subsection*{C. \emph{Implement} the game}

As a \textbf{group} effort, implement your assigned game concept.
You will implement the game across \textbf{four} sprints;
use your Trello board from part~B to plan your sprints.
Each sprint will end with a sprint review, where you should aim to have a playable build of your game.

One member of the group should fork the GitHub project at the following URL, and add the other members as contributors.
Use this forked repository for all source code and other assets.
\begin{center}
\url{https://github.com/Falmouth-Games-Academy/comp150-desktop-game}
\end{center}

\textbf{Formative submission:} Participate in the sprint review sessions.

\subsection*{D. \emph{Demonstrate} the game}

Bring your finished game to the demo session at the end of the semester,
and be prepared to discuss it with tutors and peers.

\textbf{Formative submission:} Participate in the demo session.
    There is no formal submission for this part.

\subsection*{Summative submission}

Create a zip file containing screenshots of your Trello task board from part~B,
and the source code and other assets for your game from part~C.
The recommended way do this is using the ``Download Zip'' button on the GitHub website.
Upload the zip file to the appropriate submission queue on LearningSpace.

You do \textbf{not} need to include a compiled executable in your submission.
Please ensure that the markers will be able to compile and run your code.
In particular you are advised to include any third-party libraries within your repository,
or provide clear and comprehensive instructions on how to set up an appropriate build environment.
If in doubt, please ask your tutor to verify that they can compile your code before you submit it.

\textbf{All students must make an individual submission of the group's work.}
Late or non-submission will be subject to the usual penalties,
even if the other members of your group have submitted on time.

\begin{marginquote}
    ``A game is a series of interesting choices.''
    
    --- Sid Meier
    
    \marginquoterule
    
    ``It seems that perfection is attained not when there is nothing more to add,
        but when there is nothing more to remove.''
    
    --- Antoine de Saint-Exup\'ery
\end{marginquote}
%\marginpicture{guitarhero}{
    %Rhythm games such as \emph{Guitar Hero} and \emph{Rock Band} are excellent examples of games
    %which make use of unique input devices to enhance gameplay.
%}
\section*{Additional Guidance}

A common mistake in this project is poor planning and poor time management.
By now this will be a familiar phrase,
but it is no less true of this project, and the usual guidance applies.
It is particularly common for students to
vastly underestimate the effort required to implement even a simple game,
and thus vastly overscope their games.
From the pitch stage, you should consider carefully what is feasible within the scope of this project.

For the most part, your work will be marked as a group effort.
However we want to avoid the situation where students try to ``coast'' through the assignment
on their fellow group members' work,
and equally the situation where one member of the group takes the lion's share of the work
and prevents the others from contributing effectively.
Marks will be weighted by a multiplier for \textbf{individual contribution},
which aims to penalise both of these behaviours.
We will assess this by several means, including sprint reviews, individual vivas, feedback from your peers,
attribution in the source code, and GitHub commit logs.
Any student who has contributed their \textbf{``fair share'' of effort} to the project will receive 100\% for this,
so any student who is putting in the appropriate level of effort has no need to worry.
Note that ``fair share'' is judged on effort rather than productivity,
so for example you will not be penalised for producing fewer lines of code than a more experienced programmer.

The first step in planning your implementation should be to break your concept down into \textbf{user stories}. 
Your user stories should be \textbf{distinguishable} (i.e.\ there should be little overlap between them)
and \textbf{easily measured} (i.e.\ it should be easy to tell when each user story has been implemented).
They should also be \textbf{comprehensive}, i.e.\ the user stories should completely capture the
desired functionality of the game, with no gaps.
Imagine giving your user stories to a developer who has never seen a game of your target genre.
Would they be able to implement the game correctly, or would they miss key features?

%Wherever possible, you should clearly \textbf{attribute} the author(s) of segments of your code.
%The easiest way to do this is by inserting appropriate comments, for example:
%\begin{lstlisting}
%/* This function was written by student 1511111 */
%\end{lstlisting}
%\begin{lstlisting}
%/* This function was pair-programmed by students 1522222 and 1533333 */
%\end{lstlisting}
%\begin{lstlisting}
%/* This function was adapted by student 1544444 from an example at http://stackoverflow.com/a/1657490 */
%\end{lstlisting}

Your code will be assessed on \textbf{functional coherence}:
how well the finished game corresponds to the user stories,
and whether the game has any obvious bugs.
Correspondence to user stories runs both ways:
implementing features that were not present in the design (``feature creep'')
is just as bad as neglecting to implement features.

Your code will also be assessed on \textbf{sophistication}.
To succeed on a project of this size and complexity,
you will need to make use of appropriate algorithms, data structures, library features, and object oriented programming concepts.
Appropriateness to the task at hand is key, however:
you will \textbf{not} receive credit for shoehorning a complicated solution into your program
where a simpler one would have sufficed.

\textbf{Maintainability} is important in all programming projects,
but doubly so when working in a team.
Use \textbf{comments} liberally to improve comprehension of your code,
and choose carefully the \textbf{names} for your files, classes, functions and variables.
For high marks you should use a well-established commenting convention
for \textbf{high-level documentation} of your files, classes and functions.
The open-source tool Doxygen supports several such conventions.
Also ensure that all code corresponds to a sensible and consistent \textbf{formatting style}:
indentation, whitespace, placement of curly braces, etc.
Hard-coded \textbf{literals} (numbers and strings) within the source should be avoided,
with values instead defined as constants together in a single place.
Consider allowing some literal values, where appropriate, to be ``tinkered'' without changing the source code,
e.g.\ by defining them in an external file read by the game on startup.

As with all assignments on this course, you are expected to display a level of
\textbf{innovation and creative flair} befitting Falmouth University's reputation as a world-leading
arts institution.
We are looking for creativity in the design of your \textbf{game mechanics};
you will \textbf{not} be judged on the quality of your art assets.
One approach to promoting creativity is
\textbf{divergent thinking}: generating ideas by exploring many possible solutions.
Often the most interesting ideas are \textbf{subversive}: they deliberately go against
convention or obvious solutions.

\begin{marginquote}
    ``The first 90 percent of the code accounts for the first 90 percent of the development time.
    
    ``The remaining 10 percent of the code accounts for the other 90 percent of the development time.''
    
    --- Tom Cargill
    
    \marginquoterule
    
    ``Hofstadter's Law:
    
    ``It always takes longer than you expect, even when you take into account Hofstadter's Law.''
    
    --- Douglas Hofstadter
\end{marginquote}
%\marginpicture{fishing}{
    %The \emph{Dreamcast Fishing Controller}, released as a peripheral for the game \emph{Sega Bass Fishing}.
    %Even peripherals which appeal to only a small audience can enjoy moderate commercial success.
%}
\section*{Additional Resources}

\begin{itemize}
    \item Mitchell, S. (2013) SDL Game Development. Packt Publishing.
    \item Keith, C. (2010) Agile Game Development with Scrum. Pearson Education.
    \item Sims, C. and Johnson, H.L. (2012) SCRUM: A Breathtakingly Brief and Agile Introduction. Dymaxicon.
    \item \url{https://www.mountaingoatsoftware.com/agile/user-stories}
    \item \url{http://www.doxygen.org}
    \item \url{https://travis-ci.org}
\end{itemize}

\begin{markingrubric}
    \firstcriterion{Individual Pitch}{Threshold \par 5\%}
        \gradespan{5}{\fail No individual pitch is delivered, or the pitch is inappropriate.}
        \grade An appropriate individual pitch is delivered.
%
    \criterion{Sprint Reviews}{Threshold \par 2.5\% + 2.5\%}
        \gradespan{2}{\fail The student fails to participate in at least one sprint review.}
        \gradespan{2}{The student participates in all four sprint reviews.
             \par At least one sprint does not result in a playable build.}
        \gradespan{2}{The student participates in all four sprint reviews.
             \par A playable build is delivered at the end of every sprint.}
%
    \criterion{Appropriateness of User Stories and Sprint Plans}{10\% $\dagger$}
        \grade\fail No user stories and/or sprint plans are provided.
        \grade Few user stories are distinguishable and easily measured.
            \par Sprint plans provide little support for the project.
        \grade Some user stories are distinguishable and easily measured.
            \par Sprint plans provide some support for the project.
        \grade Most user stories are distinguishable and easily measured.
            \par User stories correspond to the game design.
            \par Sprint plans provide much support for the project.
        \grade Nearly all user stories are distinguishable and easily measured.
            \par User stories clearly correspond to the game design.
            \par Sprint plans provide effective support for the project.
        \grade All user stories are distinguishable and easily measured.
            \par User stories clearly and comprehensively correspond to the game design.
            \par Sprint plans provide exemplary support for the project.
%
    \criterion{Functional Coherence}{10\% $\dagger$}
        \grade\fail No gameplay elements have been implemented and/or the code fails to compile or run.
        \grade Few gameplay elements have been implemented.
            \par There are many obvious and serious bugs.
        \grade Some gameplay elements have been implemented.
            \par There are some obvious bugs.
        \grade Many gameplay elements have been implemented.
            \par There is some evidence of feature creep.
            \par There are few obvious bugs.
        \grade Almost all gameplay elements have been implemented.
            \par There is little evidence of feature creep.
            \par There are some minor bugs.
        \grade All gameplay elements have been implemented.
            \par There is no evidence of feature creep.
            \par Bugs, if any, are purely cosmetic and/or superficial.
%
    \criterion{Sophistication}{20\% $\dagger$}
        \grade\fail No insight into the appropriate use of programming constructs is evident from the source code.
            \par No attempt to structure the program is evident (e.g. one monolithic source file).
        \grade Little insight into the appropriate use of programming constructs is evident from the source code.
            \par The program structure is poor.
        \grade Some insight into the appropriate use of programming constructs is evident from the source code.
            \par The program structure is adequate.
        \grade Much insight into the appropriate use of programming constructs is evident from the source code.
            \par The program structure is appropriate.
        \grade Significant insight into the appropriate use of programming constructs is evident from the source code.
            \par The program structure is effective. There is high cohesion and low coupling.
        \grade Exemplary insight into the appropriate use of programming constructs is evident from the source code.
            \par The program structure is very effective. There is high cohesion and low coupling.
%
    \criterion{Maintainability}{20\% $\dagger$}
        \grade\fail There are no comments, or comments are misleading.
            \par Most variable names are unclear or inappropriate.
            \par Code formatting hinders readability.
        \grade The code is only sporadically commented, or comments are unclear.
            \par Some identifier names are unclear or inappropriate.
            \par Code formatting is inconsistent or does not aid readability.
        \grade The code is well commented.
            \par Some identifier names are descriptive and appropriate.
            \par An attempt has been made to adhere to a consistent formatting style.
             \par There is little obvious duplication of code or of literal values.           
        \grade The code is reasonably well commented.
            \par Most identifier names are descriptive and appropriate.
            \par Most code adheres to a consistent formatting style.
             \par There is almost no obvious duplication of code or of literal values.   
        \grade The code is reasonably well commented, with some Doxygen-compatible module documentation.
            \par Almost all identifier names are descriptive and appropriate.
            \par Almost all code adheres to a consistent formatting style.
             \par There is no obvious duplication of code or of literal values. Some literal values can be easily ``tinkered''. 
        \grade The code is very well commented, with comprehensive Doxygen-compatible module documentation.
            \par All identifier names are descriptive and appropriate.
            \par All code adheres to a consistent formatting style.
             \par There is no obvious duplication of code or of literal values. Most literal values are, where appropriate, easily ``tinkered'' outside of the source.  
%
    \criterion{Innovation and Creative Flair}{5\% $\dagger$}
        \grade\fail No innovation and/or creativity.
            \par The game concept is a clone of existing works with only cosmetic alterations.
        \grade Little innovation and/or creativity.
            \par The game concept is derivative of existing works, with only minor gameplay alterations.
        \grade Some innovation and/or creativity.
            \par The game concept is derivative of existing works, but shows emerging divergent and/or subversive thinking in terms of gameplay.
        \grade Much innovation and/or creativity.
            \par The game concept is somewhat original, with an attempt at divergent and/or subversive thinking in terms of gameplay.
            \par The gameplay shows promise of fun and engagement.
        \grade Significant innovation and/or creativity.
            \par The game concept is original, with evidence of divergent and/or subversive thinking in terms of gameplay.
            \par The gameplay is somewhat fun and engaging.
        \grade Exemplary innovation and/or creativity.
            \par The game concept is highly original, with strong evidence of divergent and/or subversive thinking in terms of gameplay.
            \par The gameplay is fun and engaging.
%
    \criterion{Commercial Awareness}{5\% $\dagger$}
        \grade\fail No evidence of commercial awareness.
        \grade Little commercial awareness is evident in the design.
            \par Game has few, if any, unique selling points.
            \par Target audience is unclear and/or is not channelled into the game design.
        \grade Some commercial awareness is evident in the design.
            \par Game has some unique selling points.
            \par Target audience is clear, but its relation to the game design is weak.
        \grade Much commercial awareness is evident in the design.
            \par Game has several unique selling points.
            \par Target audience and its relation to the game design is clear.
            \par There is a clear market for the game.
        \grade Significant commercial awareness is evident in the design.
            \par Game has several unique selling points.
            \par Target audience and its relation to the game design is clear.
            \par There is a clear market gap which the game could fill. There is a possibility the game will attract interest.
        \grade Exemplary commercial awareness is evident in the design.
            \par Game has several unique selling points.
            \par Target audience and its relation to the game design is clear.
            \par There is a clear and commercially attractive market gap which the game could fill. There is a possibility the game will attract investment.
%
    \criterion{Portability and Navigability}{5\% $\dagger$}
        \grade\fail Game will not execute at all on another machine for reasons related to code portability which cannot be fixed easily due to its poor structure.
            \par The provided template has not been followed.
        \grade There were challenges executing the game, but these were resolvable.
            \par The directory structure inside the submitted zip file is unclear.
            \par The provided template has not been followed.
        \grade Several portability issues are present.
            \par The directory structure inside the submitted zip file is somewhat confusing.
            \par The provided template has mostly been followed.
        \grade Some portability issues are present.
            \par The directory structure inside the submitted zip file is adequate.
            \par The provided template has been followed.
        \grade Few portability issues are present.
            \par The directory structure inside the submitted zip file is mostly sensible.
            \par The provided template has been followed.
        \grade Almost no portability issues are present.
            \par The directory structure inside the submitted zip file is sensible.
            \par The provided template has been followed.
%
    \criterion{Professional Practice}{5\% $\dagger$}
        \grade\fail The group's professional conduct has been unacceptable,
            and/or the group has failed to function at all as a team.
            \par Agile working practices have not been used.
        \grade The group has demonstrated an emerging level of professionalism.
            \par Agile working practices have provided little support for the project.
        \grade The group has demonstrated a progressing level of professionalism,
            functioning adequately as a team.
            \par Agile working practices have provided some support for the project.
        \grade The group has demonstrated an appropriate level of professionalism,
            functioning somewhat effectively as a team.
            \par Agile working practices have provided much support for the project.
        \grade The group has demonstrated a high level of professionalism,
            functioning effectively as a cohesive team.
            \par Agile working practices have provided significant support for the project.
            \par There is evidence of some use of Travis CI to support a continuous integration approach.
        \grade The group has demonstrated an exemplary level of professionalism,
            functioning highly effectively as a cohesive team.
            \par Agile working practices have provided exemplary support for the project.
            \par Travis CI has been used to effectively support a continuous integration approach.
%
    \criterion{Use of Version Control}{10\%}
        \grade\fail GitHub has not been used.
        \grade Material has been checked into GitHub less frequently than once per sprint.
            \par All code has been checked into the Master branch.
        \grade Code has been checked into GitHub at least once per sprint.
            \par An attempt has been made to use branches.
        \grade Code has been checked into GitHub several times per sprint.
            \par Commit messages are clear, concise and relevant.
            \par Branches are used sensibly.
        \grade Code has been checked into GitHub several times per sprint.
            \par Commit messages are clear, concise and relevant.
            \par Branches are used somewhat effectively.
            \par There is evidence of engagement with peers (e.g.\ code review).
        \grade Code has been checked into GitHub several times per sprint.
            \par Commit messages are clear, concise and relevant.
            \par Branches are used effectively.
            \par There is significant evidence of engagement with peers (e.g.\ code review).
%
    \criterion{Individual Contribution}{Multiplier for criteria marked $\dagger$}
        \gradespan{5}{\fail The student has failed to contribute their ``fair share'' to the project,
            or has actively prevented others from doing so.}
        \grade The student has contributed their ``fair share'' to the project,
            and has facilitated others in doing so.
\end{markingrubric}

\end{document}
